%https://www.overleaf.com/learn/latex/Subscripts_and_superscripts
\RequirePackage[2020-02-02]{latexrelease}
\documentclass[14pt,fleqn]{extbook} % Default font size and left-justified equations
\usepackage{graphicx}
\usepackage{setspace}
\input{structure.tex} % Insert the commands.tex file which contains the majority of the structure behind the template
\usepackage{color}
\definecolor{light}{rgb}{0.5, 0.5, 0.5}
\definecolor{back}{RGB}{245, 214, 235}
\definecolor{sborder}{RGB}{255, 230, 234}
\definecolor{scontent}{RGB}{255, 191, 191}
\definecolor{oborder}{RGB}{214, 245, 214}
\definecolor{ocontent}{RGB}{148, 201, 115}
\definecolor{eborder}{RGB}{255, 153, 153}
\definecolor{econtent}{RGB}{255, 0, 0}
\definecolor{ans}{RGB}{242, 242, 242}
\def\light#1{{\color{light}#1}}
\usepackage{multicol}
\usepackage{parskip}
\usepackage{fancyhdr}
\usepackage{tabulary}
\usepackage{adjustbox}
\usepackage{subfig}
\usepackage{xcolor}
\usepackage{tcolorbox}
\usepackage{amssymb}
\usepackage{afterpage}
\tcbuselibrary{breakable}
\usepackage[printwatermark]{xwatermark}
\usepackage{multirow}
% tickmarks and wrong mark
\usepackage{amssymb}% http://ctan.org/pkg/amssymb
\usepackage{pifont}% http://ctan.org/pkg/pifont
\newcommand{\cmark}{\ding{51}}%
\newcommand{\xmark}{\ding{55}}%
\definecolor{ncontent}{RGB}{255, 242, 0}
\definecolor{nborder}{RGB}{255, 255, 179}

\definecolor{border}{RGB}{245,245,245}

\newcommand\blankpage{%
	\null
	\thispagestyle{empty}%
	\addtocounter{page}{-1}%
	\newpage}

\fancyfoot[R]
{
	\includegraphics[scale=0.6]{content/logo.png}%
}


\usepackage{framed}


\newcommand{\codeblockfull}[2] {
	\begin{tcolorbox}[
		space to upper,
		colback=white,
		collower=white,
		title=\textbf{#1},
		coltitle = white,]
		\color{black}
		\fontdimen2\font=8pt
		#2
		\fontdimen2\font=4pt
	\end{tcolorbox}
}

\newcommand{\codeblock}[1] {
	\begin{tcolorbox}[
		space to upper,
		colback=white,
		collower=white,
		title=\textbf{Code:},
		coltitle = white,]
		\color{black}
		\fontdimen2\font=8pt
		#1
		\fontdimen2\font=4pt
	\end{tcolorbox}
}



\newcommand{\codecontinue}[1] {
	\begin{tcolorbox}[
		space to upper,
		colback=white,
		collower=white,
		coltitle = white,]
		\color{black}
		\fontdimen2\font=8pt
		#1
		\fontdimen2\font=4pt
	\end{tcolorbox}
}



\newcommand{\datablock}[1] {
	\begin{tcolorbox}[
		space to upper,
		colback=white,
		collower=white,
		title=\textbf{Dataset:},
		coltitle = white,]
		\color{black}
		\fontdimen2\font=8pt
		#1
		\fontdimen2\font=4pt
	\end{tcolorbox}
}


\newcommand{\syntaxblock}[1] {
	\begin{tcolorbox}[
		space to upper,
		colback=scontent,
		title=\textbf{Syntax:},
		colframe=sborder,
		coltitle = black,]
		\color{black}
		\fontdimen2\font=8pt
		#1
		\fontdimen2\font=4pt
	\end{tcolorbox}
}


\newcommand{\noteblock}[1] {
	\begin{tcolorbox}[
		space to upper,
		colback=ncontent,
		title=\textbf{Note:},
		colframe=nborder,
		coltitle = black,]
		\color{black}
		\fontdimen2\font=8pt
		#1
		\fontdimen2\font=4pt
	\end{tcolorbox}
}


\newcommand{\outputblock}[1] {
	\begin{tcolorbox}[
		space to upper,
		colback=ocontent,
		title=\textbf{Output:},
		colframe=oborder,
		coltitle = black,]
		\color{black}
		\fontdimen2\font=8pt
		#1
		\fontdimen2\font=4pt
	\end{tcolorbox}
}

\newcommand{\errorblock}[1] {
	\begin{tcolorbox}[
		space to upper,
		colback=econtent,
		title=\textbf{Error:},
		colframe=eborder,
		coltitle = black,]
		\color{black}
		\fontdimen2\font=8pt
		#1
		\fontdimen2\font=4pt
	\end{tcolorbox}
}

\newcommand{\n} {
	\newline
}

\newcommand{\boximage}[2] {
	\includegraphics[width=#1\textwidth]{#2}
}


\newcommand{\s} {
	\hphantom{} \hphantom{} \hphantom{}
}

\newcommand{\bspace} {
	\bigskip
}

\newcommand{\tabletwo}[1] {
	\bspace
	\begin{tabular}{ |p{7cm}|p{7cm}| }
		\hline
		#1
		\hline
	\end{tabular}
}

\newcommand{\tableinsidebox}[1] {
	\bspace
	\begin{tabular}{ |p{6cm}|p{6cm}| }
		\hline
		#1
		\hline
	\end{tabular}
}

\newcommand{\commandblock}[1] {
	\begin{tcolorbox}[
		space to upper,
		colback=white,
		collower=white,
		title=\textbf{Command:},
		coltitle = white,]
		\color{black}
		\fontdimen2\font=8pt
		#1
		\fontdimen2\font=4pt
	\end{tcolorbox}
}


\newcommand{\tablethree}[1] {
	\bspace
	\begin{tabular}{ |p{4.5cm}|p{4.5cm}|p{4.5cm}| }
		\hline
		#1
		\hline
	\end{tabular}
}

\newcommand{\tablefour}[1] {
	\bspace
	\begin{tabular}{ |p{3.5cm}|p{3.5cm}|p{3.5cm}|p{3.5cm}| }
		\hline
		#1
		\hline
	\end{tabular}
}
6
\newcommand{\error}[1] {
	\color{red}\# Error: \newline
	#1 \color{black}
}

\newcommand{\correct}[1] {
	\color{correctans}\# Output: \newline
	#1 \color{black}
}

\newcommand{\answer}[1] {
	
	\begin{tcolorbox}[space to upper,
		colback=ans,
		collower=white,
		coltitle = black, 
		colframe=border,
		title={Answer:},]
		#1
	\end{tcolorbox}
}

\newcommand{\quest}[2] {
	\begin{tcolorbox}[space to upper,
		colback=ans,
		collower=white,
		coltitle = white, 
		title={\textbf{#1}},]
		Ans: #2
	\end{tcolorbox}
}

\newcommand{\anscontinue}[1] {
	\begin{tcolorbox}
		#1
	\end{tcolorbox}
}

\newcommand{\newimage}[2] {
	\begin{figure}[h!]
		\centering
		\includegraphics[scale=#1]{#2}
	\end{figure}			
}

\thispagestyle{plain}
%\hypersetup{pdftitle={Title},pdfauthor={Author}} % Uncomment and fill out to include PDF metadata for the author and title of the book

%----------------------------------------------------------------------------------------
\setstretch{1.25}
%\definecolor[new][h=9A957A, a=1, t=.3]
\definecolor{code}{RGB}{248,248,248}
\definecolor{output}{RGB}{148, 201, 115}
\definecolor{error}{RGB}{255, 77, 77}
\definecolor{trick}{RGB}{242, 204, 255}
%%%%%%%%%%%%%%%%%%%%%%%   This block adds watermark  %%%%%%%%%%%%%%%%%
%\newsavebox\mybox
%\savebox\mybox{\tikz[color=mycolor,opacity=0.2]\node{lavatechtechnology.com};}
%\newwatermark*[
%allpages,
%angle=45,
%scale=2.5,
%xpos=-20,
%ypos=15
%]{\usebox\mybox}
%%%%%%%%%%%%%%%%%%%%%%%   This block adds watermark  %%%%%%%%%%%%%%%%%



\begin{document}

%----------------------------------------------------------------------------------------
%	TITLE PAGE
%----------------------------------------------------------------------------------------

\begingroup
\thispagestyle{empty} % Suppress headers and footers on the title page
\begin{tikzpicture}[remember picture,overlay]
\node[inner sep=0pt] (background) at (current page.center) {\includegraphics[width=\paperwidth, height=\paperheight]{cover_new.pdf}};
\end{tikzpicture}
\vfill
\endgroup

%----------------------------------------------------------------------------------------
%	COPYRIGHT PAGE
%----------------------------------------------------------------------------------------

\newpage



~\vfill
\thispagestyle{empty}

\noindent Copyright \copyright\ 2022 Lavatech Technology\\ % Copyright notice

The contents of this course and all its modules and related materials, including handouts are
Copyright ©

No part of this publication may be stored in a retrieval system, transmitted or reproduced in any way, including, but not limited to, photocopy, photograph, magnetic, electronic or other record, without the prior written permission of Lavatech Technology.

If you believe Lavatech Technology training materials are being used, copied, or otherwise improperly distributed please e-mail: 
\newline
\textbf{info@lavatechtechnology.com}

\noindent \textsc{Published by Lavatech Technology}\\ % Publisher

\noindent \textit{lavatechtechnology.com}\\ % URL

%\noindent Licensed under the Creative Commons Attribution-NonCommercial 3.0 Unported License (the ``License''). You may not use this file except in compliance with the License. You may obtain a copy of the License at \url{http://creativecommons.org/licenses/by-nc/3.0}. Unless required by applicable law or agreed to in writing, software distributed under the License is distributed on an \textsc{``as is'' basis, without warranties or conditions of any kind}, either express or implied. See the License for the specific language governing permissions and limitations under the License.\\  License information, replace this with your own license (if any)
%
\noindent \textit{January 2022} % Printing/edition date

\afterpage{\blankpage}
\afterpage{\blankpage}


%----------------------------------------------------------------------------------------
%	TABLE OF CONTENTS
%----------------------------------------------------------------------------------------


\usechapterimagetrue % If you don't want to include a chapter image, use this to toggle images off - it can be enabled later with \usechapterimagetrue

\chapterimage{image1.png} % Table of contents heading image

\pagestyle{empty} % Disable headers and footers for the following pages


\tableofcontents



\cleardoublepage % Forces the first chapter to start on an odd page so it's on the right side of the book

\pagestyle{fancy} % Enable headers and footers again

%----------------------------------------------------------------------------------------
%	PART One
%----------------------------------------------------------------------------------------

%\part{System Admin Level I}

%----------------------------------------------------------------------------------------
%	CHAPTER 0
%----------------------------------------------------------------------------------------
%\input{content/chapter1/1.0.tex}
\chapterimage{index2.png}
\chapter{Introduction to Java}
\section{Getting started with Java}

\setlength{\columnsep}{3pt}
\begin{flushleft}
	\bigskip
	\bigskip
	\begin{tcolorbox}[breakable,notitle,boxrule=1pt,colback=black,colframe=black]
	\color{white}
	\bigskip
	In this section, you are going to learn:
	\begin{enumerate}
		\item \textbf{What is Java?}
		\item \textbf{Uses of Java}
		\item \textbf{Problem with C and C++}
		\item \textbf{Advantages of Java over C++}
		\item \textbf{How does program execution works?}
		\item \textbf{The way Java works}
		\item \textbf{Compiler V/S JVM}
		\item \textbf{JVM components}
		\item \textbf{History of Java}
	\end{enumerate}	
	\end{tcolorbox}

	
	\bigskip
	\bigskip
	
	\begin{multicols}{2}
		\vspace*{\fill}
		\vspace*{\fill}
		\vspace*{\fill}
		\vspace*{\fill}
		\vspace*{\fill}
		\vspace*{\fill}
		\vspace*{\fill}
		\vspace*{\fill}
		\vspace*{\fill}
		
	\end{multicols}	
	
\end{flushleft}

\newpage






\subsection{What is Java?}

\begin{flushleft}

	\begin{itemize}
		\item Java is a popular \textbf{high-level programming language} that is \textbf{platform-independent}, \textbf{object-oriented} and \textbf{open source}.
		\item Let’s understand each bold words in detail:
		\begin{itemize}
			\item \textbf{High level programming langauge}: Language that are in english and can be understood by humans.
			
			\newimage{0.2}{content/chapter0/images/img1.png}
			
			\item \textbf{Platform-independent}: Java code can run on any platform (i.e OS) using Java Virtual Machine (JVM). 
			
			\newimage{0.2}{content/chapter0/images/img2.jpg}
						
			\item \textbf{Object-Oriented}: Java is built around objects.
			
			\newimage{0.2}{content/chapter0/images/img3.png}
			
			\item \textbf{Open source}: Java source code is available for anyone to view and modify.
			
			\newimage{0.05}{content/chapter0/images/img4.png}
			
		\end{itemize}
	\end{itemize}
		
\end{flushleft}

\newpage


\subsection{Uses of Java}
\begin{flushleft}
	
	\begin{itemize}
		\item \textbf{Enterprise-level applications} - Eg: SAP, IBM websphere, Salesforce, Oracle E-Business suite
		\newimage{0.3}{content/chapter0/images/1.png}
		\bigskip	
		
		\item \textbf{Web applications} - Eg: LinkedIn, Netflix, Twitter, Amazon, Airbnb
		\newimage{0.3}{content/chapter0/images/2.png}
		
		\item \textbf{Mobile applications} - Eg: Instagram, WhatsApp, Google Maps, Uber
		\newimage{0.3}{content/chapter0/images/3.png}
		\bigskip	
		
		\item \textbf{Games} - Eg: Minecraft, RuneScape, Puzzle Pirates		
		\newimage{0.3}{content/chapter0/images/4.png}
		\bigskip	
		
		\item \textbf{Financial Applications} - Eg: Quicken, Bloomberg Terminal		
		\newimage{0.3}{content/chapter0/images/5.png}
		\bigskip	
		
		\item \textbf{Scientific Applications} - Eg: BioJava, ImageJ		
		\newimage{0.3}{content/chapter0/images/6.png}
		
	\end{itemize}
	
\end{flushleft}

\newpage
\subsection{Problem with C and C++}

\begin{flushleft}
	
	\begin{itemize}
		\item \textbf{Original idea for Java was not the Internet!}
		\item Java was created to be \textbf{platform-independent language} that can be embedded in various consumer electronic devices, eg: \textbf{microwave ovens and remote controls}. 
		\item The trouble with C and C++  is that they are \textbf{compiled for a specific target}. 
		\item A full C++ compiler targeted for a specific CPU is needed to compile a C++ program for different CPU.
		\item The problem is that \textbf{compilers are expensive and time-consuming to create}. 
		\item James Gosling (founder of Java) began work on a portable, platform-independent language that could be used to produce code that would run on any CPUs.
		\item This led to the creation of Java.
	\end{itemize}
	
	
\end{flushleft}



\subsection{Advantages of Java over C++}
\begin{flushleft}
	
	\begin{itemize}
		\item \textbf{Security}: No danger of reading bogus data when accidentally going over the size of an array.
		\item \textbf{Automatic memory management}: 
		\begin{itemize}
			\item Garbage collector allocate and deallocate memory for objects. 
			\item Pointers are not necessary.
		\end{itemize}
		
		\item \textbf{Simplicity}: 
		\begin{itemize}
			\item No pointers, unions, templates, structures, multiple inheritance. 
			\item Anything in Java can be declared as a class.
		\end{itemize}
		
		\item \textbf{Support for multithreaded execution}: Support development of multithreaded software.
		\item \textbf{Portability}: Support WORA (Write it once, run it anywhere), using Java virtual machine (JVM).
		
	\end{itemize}

\end{flushleft}

\newpage
\subsection{How does program execution works?}
\begin{flushleft}
	
	\begin{itemize}
		\item \textbf{Compiler: }
		\begin{itemize}
			\item A compiler is a \textbf{software program that converts source code written in a high-level programming language into machine code}, which can be executed by a computer. 
			\item The compiler performs:
			\begin{itemize}
				\item Syntax analysis
				\item Semantic analysis
				\item Bytecode generation
			\end{itemize}
		\end{itemize}
		
		
		\item \textbf{Interpreter:}
		\begin{itemize}
			\item An interpreter is a program that reads and executes code \textbf{line-by-line}, without the need to compile the entire program beforehand. 
			\item Interpreters run code in a virtual environment, where each line of code is executed as soon as it is read. 
		\end{itemize}
		
		\newimage{0.4}{content/chapter0/images/new03.png}
		
		\item \textbf{Java Virtual Machine (JVM):}
		\begin{itemize}
			\item The JVM is a virtual machine that is responsible for executing Java bytecode. 
			\item The JVM is an example of an interpreter, as it reads Java bytecode and executes it line-by-line. 
			\item The JVM performs:
			\begin{itemize}
				\item Memory management
				\item Security
				\item Garbage collection
			\end{itemize}  
		\end{itemize}
		
	\end{itemize}
	
\end{flushleft}

\newpage
\subsection{The way Java works}
\begin{flushleft}
	
	\textbf{So, what exactly happens to developer-written Java code until the actual execution?}
	
	\newimage{0.35}{content/chapter0/images/new01.png}
	\newimage{0.35}{content/chapter0/images/new02.png}
\end{flushleft}

\newpage
\subsection{Compiler V/S JVM}
\begin{flushleft}
	
	\quest{Compiler and JVM battle over the question, “Who’s more important?”}{
		Go through below discussion to find the answer:\\ \\
		\textbf {JVM:} I am Java. I’m the guy who actually makes a program run. The compiler just gives you a file in bytecode after checking it’s syntax. I'm the one who run it.
		\\ \\
		\textbf{Compiler:} Excuse me? Without me, you would have to translate everything from source code and be very very slow!
		\\ \\	
		\textbf{JVM:} Your work is not important. A programmer could just write bytecode by hand. You might be out of a job soon, buddy.
		\\ \\
		\textbf{Compiler:} That’s arrogant. A programmer writing bytecode by hand is next to possible, some scholars might write, not everyone!
		\\ \\
		\textbf{JVM:} But you still didn’t answer my question, what you actually do?
		\\ \\
		\textbf{Compiler:} Remember that Java is a strongly-typed language, I can’t allow variables to hold data of the wrong type. This is a security feature, implement by ME!
		\\ \\
		\textbf{JVM:} Your type checking is not very strict! Sometimes people put the wrong type of data in an array of different type.
		\\ \\
		\textbf{Compiler:} Yes, that can emerge at runtime that only you can catch to allow dynamic binding. But my job is to stop anything that would never succeed at runtime. 
	}
	
	\newpage
	
	\anscontinue{
	
	\textbf{JVM:} OK. Sure. But what about security? Look at all the security stuff I do! You just perform silly syntax checking.
	\\ \\
	\textbf{Compiler:} Listen, I'm the first line of defense. I also prevents access violations, such as code trying to invoke a private method. I stop people from touching code they’re not meant to see.
	\\ \\
	\textbf{JVM:} Whatever. I have to do that same stuff too!
		
	}
\end{flushleft}

\subsection{JVM components}
\begin{flushleft}
	
	\newimage{0.5}{content/chapter0/images/new05.png}
	\begin{itemize}
		\item \textbf{Class Loader:} Responsible for loading the class files into the memory of the JVM.
		\item \textbf{Execution Engine:} Responsible for executing the bytecode that is loaded into the memory. It includes:
		\begin{itemize}
			\item \textbf{Interpreter:} Reads and executes the bytecode one instruction at a time. 
			\item \textbf{JIT compiler:} Compiles the bytecode into machine code for fast execution.
		\end{itemize}
		
		\item \textbf{Garbage Collector:} It periodically frees up the memory that is not used by the Java application.
		
		\item \textbf{Runtime Data Area:} It is memory space allocated by the JVM for the execution of the Java application. It includes:
		\begin{itemize}
			\item Method area
			\item Heap
			\item Stack
			\item PC registers
		\end{itemize}
		\item \textbf{Native Method Interface (JNI):} Allows Java code to call code written in other programming languages like C and C++. The JNI allows Java applications to interact with OS and hardware.
		
	\end{itemize}
	
	
\end{flushleft}

\newpage
\subsection{History of Java}
\setlength{\columnsep}{20pt}
\begin{flushleft}

	\item \quest{What year Java was invented?}{1995}

	\bigskip
	\quest{What company invented Java?}{Sun Microsystems \\ \\
		\boximage{0.3}{content/chapter0/images/sun.png}
	}

	\bigskip
	\quest{Who is founder of Java?} {
	James Gosling  \\ \\
	\boximage{0.3}{content/chapter0/images/james.jpg}
	}
  
  	\bigskip
  	\quest{What is Java mascot?}{
  	A cartoon character named Duke \\ \\
  	\boximage{0.3}{content/chapter0/images/mascot.png}
  	}
  
  	\bigskip
  	\quest{What is the original name of Java ?}{
  	"Oak" after the oak tree that was outside Gosling's office. 
  	}

	\bigskip
	\quest{What was the reason for changing original name?}{
	"Oak" was already trademarked	
	}
	
	\bigskip
	\quest{What is the inspiration behind Java's name?}{
	Java language is named after coffee grown on the Indonesian island
	}
	
	\bigskip
	\quest{What is original Java logo?}{
		Original logo: \\ \\
	\boximage{0.3}{content/chapter0/images/jlogo.png}
	}

	\bigskip
	\quest{Who has the current ownership of Java?}{Oracle acquired Java in 2009}
	
	\newpage
	
	\quest{What are the Java versions?}{
		Below are the Java version details:
		
		\tableinsidebox{
		\hline
		Version &  Year \\
		\hline
		JDK Alpha and Beta & 1995  \\
		\hline
		JDK 1.0 & Jan, 1996 \\
		\hline
		JDK 1.1 & Feb, 1997 \\
		\hline
		J2SE 1.2 or \textbf{Java2} (codename: Playground) & Dec, 1998 \\
		\hline
		J2SE 1.3 or \textbf{Java2} (codename: Kestrel) &  May, 2000 \\
		\hline
		J2SE 1.4 or \textbf{Java2} (codename: Merlin) & Feb, 2002 \\
		\hline
		J2SE 1. 5 or \textbf{Java5} (codename: Tiger) & Sep, 2004 \\
		\hline
		Java SE 1.6 or \textbf{Java6} (codename: Mustang) &  Dec, 2006 \\
		\hline
		Java SE 1.7 or \textbf{Java7} (codename: Dolphin) & July, 2011 \\
		\hline
		Java SE 1.8 or \textbf{Java8} (codename: Spider) & (18th March, 2014) \\
		\hline
		Java SE 1.9 or \textbf{Java9} & September, 2017 \\
		\hline
		\textbf{Java 10} & March, 2018 \\
		\hline
		\textbf{Java SE 11} & September 2018 \\
		\hline
		\textbf{Java SE 12} & March 2019 \\
		\hline
		\textbf{Java SE 13} & September 2019 \\
		\hline
		\textbf{Java SE 14} & March 2020 \\
		\hline
		\textbf{Java SE 15} & September 2020 \\
		\hline
		\textbf{Java SE 16} & March 2021 \\
		\hline
		\textbf{Java SE 17} & September 2021 \\
		\hline
		}
		
	}

	\newpage
	\quest{Why is Java 2 consider very significant in history of Java?}{		
		Starting \textbf{Java 2}, it is composed of three parts:
		\begin{itemize}
			\item \textbf{J2SE (Java 2 Platform, Standard Edition) or JSE}, a computing platform for the development and deployment of portable code for \textbf{desktop and server environments}.
			\item \textbf{J2EE (Java 2 Platform, Enterprise Edition) or JEE}, extending Java SE with enterprise features such as \textbf{distributed computing and web services}.
			\item \textbf{J2ME (Java 2 Platform, Micro Edition) or JME}, a computing platform for \textbf{embedded and mobile devices}.
		\end{itemize}
		Other major highlights of this release:
		\begin{itemize}
			\item \textbf{JIT compiler} became part of JVM (means turning code into executable code became a faster operation).
			\item \textbf{Swing graphical API} was introduced as alternative to AWT.
			\item Java collections framework (for working with sets of data) was introduced.
		\end{itemize}
	}

	\quest{I see Java 2 and Java 5.0, but was there a Java 3 and 4? And why is it Java 5.0 but not Java 2.0?}{
		The joys of marketing... 
		\begin{itemize}
			\item When the version of Java shifted from 1.1 to 1.2, the changes to Java were so many that the marketers decided a whole new “name”, so they started calling it Java 2, even though the actual version of Java was 1.2.
			\item But versions 1.3 and 1.4 were still considered Java 2.
			\item There never was a Java 3 or 4. 
			\item Beginning with Java version 1.5, the marketers decided a new name was needed. 
			\item The next number in the name sequence would be “3”, but calling Java 1.5 Java 3 seemed more confusing, so they decided to name it Java 5.0 to match the “5” in version “1.5”. 	
		\end{itemize}
	}

\end{flushleft}



\section{Installing Java and IDE}
\input{content/chapter1/1.2.tex}
\subsection{Java installation}
\input{content/chapter1/1.2.1.tex}
\subsection{IDE installation}
\setlength{\columnsep}{5pt}
\begin{flushleft}

	\begin{itemize}
	
	\end{itemize}

\end{flushleft}

\newpage
\subsection{Our first Java program}
\input{content/chapter1/1.2.3.tex}
\subsection{Executing our first Java program}
\setlength{\columnsep}{5pt}
\begin{flushleft}

		\begin{itemize}
			\item Using simple text editor:
			\bigskip
			\codeblockfull{MyFirstApp.java}{
				public class MyFirstApp \{ \\
				\s	public static void main(String[] args) \{ \\
				\s \s		System.out.println("I Rule!"); \\
				\s \s		System.out.println("The World"); \\
				\s	\} \\
				\}	
		
			}
			\bigskip
			\commandblock{
				javac MyFirstApp.java \\
				java MyFirstApp 
			}
			\bigskip
			\outputblock{
				I Rule! \\
				The World
			}
		
			\newpage
			\item Using Eclipse:		
			\begin{enumerate}
				\item Create New Project:
				\\
				\textbf{File -> New -> Java Project -> Add “Starter”} as project name
				
				\newimage{0.35}{content/chapter0/images/new11.png}
				
				\item This will create directory structure as shown below:
				
				\newimage{0.8}{content/chapter0/images/new12.png}
				
				\newpage
				\item Right click the src -> Select “Package” -> Add “Start1” as package name as shown below:
				
				
				\newimage{0.35}{content/chapter0/images/new13.png}
				
				\item Right click the “Start1” -> Select New -> Class -> Add “MyFirstClass” as classname as shown below:
				
				\newimage{0.4}{content/chapter0/images/new14.png}
				
				\newpage
				\item This will create a class with below structure:
				\bigskip
				\codeblock{
					package Start1; \\
					public class MyFirstClass \{ \\
					\s	public static void main(String[] args) \{ \\
					\s	\s	// TODO Auto-generated method stub \\
					\s \s	System.out.println("Welcome Back!"); \\
					\s	\} \\
					\}
				}
			
				\item Execute the code by pressing the run button:
				
				\newimage{0.65}{content/chapter0/images/new15.png}
				
			\end{enumerate}
			
			
		\end{itemize}
		
\end{flushleft}

\newpage

\subsection{Using JShell}
\input{content/chapter1/1.2.5.tex}


%--------------------------------------------------------------------------
%	CHAPTER 2
%-------------------------------------------------------------------------

\chapterimage{index3.png} % Table of contents heading image
\chapter{Java Language Fundamentals}
%-----------------------
\section{Identifiers, variables and more}
\setlength{\columnsep}{3pt}
\begin{flushleft}
	\bigskip
	\bigskip
	\begin{tcolorbox}[breakable,notitle,boxrule=1pt,colback=black,colframe=black]
		\color{white}
		\bigskip
		In this section, you are going to learn:
		\begin{enumerate}
			\item \textbf{Java identifiers}
			\item \textbf{Java grammar}
			\item \textbf{Java variable}
			\item \textbf{Literals}
			\item \textbf{Java reserved words or keywords}
			\item \textbf{Java comments}
			\item \textbf{How objects can change your life?}
		\end{enumerate}	

	\end{tcolorbox}
		
\end{flushleft}

\newpage






\subsection{Java identifiers}

\begin{flushleft}
	
	A Java identifier is a name of a variable, function, class, module or other object.
	\newline
	Eg:
	\begin{tcolorbox}[breakable,notitle,boxrule=-0pt,colback=code,colframe=code]
		\color{black}
		\fontdimen2\font=8pt
		package \textbf{Starter}; \newline
		\newline
		public class \textbf{Test} \{ \newline
		\hphantom{} \hphantom{}	public static void main(\textbf{String}[] \textbf{args}) \{ \newline
		\hphantom{} \hphantom{} \hphantom{} \hphantom{}	int \textbf{x}=999; \newline
		\hphantom{} \hphantom{} \hphantom{} \hphantom{}	System.out.println(x); \newline
		\hphantom{} \hphantom{}	\} \newline
		\}
		\fontdimen2\font=4pt
	\end{tcolorbox}
	
	\bigskip
	In above code, there are total 6 identifiers:
	\begin{itemize}
		\item Starter - name of package
		\item Test - name of class
		\item main - name of function
		\item String - name of class 
		\item args - name of object
		\item x - name of integer variable
	\end{itemize}
	
	\textbf{Rules of identifiers:}
	\begin{itemize}
		\item Names can contain A-Z, a-z, 0-9, \_, and \$ signs.
		\item Names cannot begin with number.
		\item Names are case sensitive ("myVar" and "myvar" are different variables).
		\item Reserved words (like Java keywords, such as int or boolean) cannot be used as names.
		\item Names can be of any length, but it's not recommended to have big names.
		\item Developers should declare identifiers using the \textbf{Camel case} writing style (e.g., StringBuilder, isAdult)
	\end{itemize}
	
	\bigskip
	\begin{figure}[h!]
		\centering
		\includegraphics[scale=.45]{content/chapter2/images/java.png}
	\end{figure}
	
	
\end{flushleft}



\subsection{Java grammar}
\input{content/chapter2/2.1.8.tex}
\subsection{Java variable}
\input{content/chapter2/2.1.2.tex}
\subsection{Literals}
\input{content/chapter2/2.2.9.tex}
\subsection{Java reserved words or keywords}
\input{content/chapter2/2.1.3.tex}
\subsection{Java Comments}
\input{content/chapter2/2.1.9.tex}
\subsection{How Objects Can Change Your Life?}
\input{content/chapter2/2.1.7.tex}
\section{Introductions to OOPs}
\input{content/chapter2/2.1.11.tex}
\subsection{Java source file}
\begin{flushleft}
	
	\begin{itemize}
		\item A Java program can contain any number of classes in a single program. 
		\item However, only 1 class can be declared as \textbf{public}.
		\item If there is a public class, then \textbf{name of the program and name of public class must be matched}, otherwise the program will result in compile time error.
		\item Below are some use-cases on this:
		
		\begin{itemize}
			\item \textbf{Case I}: If there is no public class and the program contains multiple class, then the program name can be anything. 
			\bigskip
			\codeblockfull{Lava.java}{
					class A \{\} \\
					class B \{\} \\
					class C \{\} 
			}			
			\bigskip
			\commandblock{
			\$ javac Lava.java \\
			 // This will create A.class, B.class, C.class
			}
			
			\bigskip
			\item \textbf{Case II}: If \textbf{class B is public}, then name of the program should be \textbf{B.java}, otherwise we will get compile-time error saying: \textbf{class B is public should be declared in file named B.java}:
			\bigskip
			\codeblockfull{Lava.java}{
				class A \{\} \\
				public class B \{\} \xmark // Lava.java is incorrect name \\
				class C \{\}
			}
			\bigskip
			\codeblockfull{B.java}{
				class A \{\} \\
				public class B \{\} \cmark \\
				class C \{\}
			}
			
		
			\bigskip
			
			\item \textbf{Case III}: If class B and C are declared as public and name of program is “B.java”, then we will get compile time error saying: class C is public, should be declared in a file named C.java
			
			\bigskip
			\codeblockfull{B.java}{
				class A \{\} \\
				public class B \{\} \\
				public class C \{\} \xmark \s // Two class cannot be public
			}

			\item \textbf{Class IV:} If a program contains main() for multiple class, then executing those specific class will execute their respective main()
			\bigskip
			
			\codeblockfull{Lava.java}{
				class A \{ \\ 
				\s public static void main(String[] args) \{ \\
				\s \s  System.out.println("A class main"); \\
				\s	\} 			\\
				class B \{ \\ 
				\s public static void main(String[] args) \{ \\
				\s \s  System.out.println("B class main"); \\
				\s	\} \\
				class C \{ \\ 
				\s public static void main(String[] args) \{ 
			}	
			\newpage
			
			\codecontinue{
				\s \s  System.out.println("C class main"); \\
				\s	\} \\
				class D \{\} \\
				\}
			}
		
			\bigskip
			\commandblock{
			\$ javac Lava.java \\
			A.class B.class C.class D.class \\
			\\
			\$ java A \\
			A class main \\
			\\
			\$ java B \\
			B class main \\
			\\
			\$ java C \\
			C class main \\
			\\
			\$ java D \\ 
			RuntimeError: NoSuchMethodError: main() \\
			\\
			\$ java Lava \\
			RuntimeError: NoClassDefFoundError: Lava \\
			}
		\end{itemize}
	\end{itemize}

	\textbf{Conclusion:}
	\begin{itemize}
		\item While executing a java program, for every class present in that program, a separate \textbf{“.class”} will be generated.
		\item You can compile a java program (Java source file), but you can run a java \textbf{".class"} file.
		\item On executing a java class, the corresponding class main() will be executed.
		\item If the class doesn’t contain main(), then you will get runtime exception.
		\item If the corresponding .class file is not available, then you will get runtime exception.
		\item It is not recommended to declare multiple classes in a single source file. 
		\item It is recommended to declare only one class per source file and name of the program to be same as class name.
	\end{itemize}
	
\end{flushleft}
\newpage



 
\subsection{main() method}


\begin{flushleft}
	
	\begin{itemize}
		\item \textbf{main()} serves as the entry point for a Java program. 
		\item When a Java program is executed, the JVM starts by looking for the main() method in the class specified in the command line arguments, and then executes the code inside it.
		\bigskip
		\syntaxblock{
			public static void main(String[] args)
		}
		\bigskip
		\item At runtime, JVM always searches for main method with the above prototype:
		\begin{itemize}
			\item \textbf{public:} To call main() from anywhere
			\item \textbf{static:} without existing object also, JVM has to call this method
			\item \textbf{void:} main() method wont return anything to JVM
			\item \textbf{main:} This is the name which is configured inside JVM
			\item \textbf{String[] args:} command line argument
		\end{itemize}
		\bigskip
		\noteblock{
			The main() syntax is very strict and if we perform any change then we will get runtime error from JVM saying “NoSuchMethodError: main”.
		}
		\bigskip
		\item Changes allowed in main():
		\begin{itemize}
			\item Order of modifier can be changed:
			\newline
			Eg: static public void main(String[] args)
			\item The command line argument’s string array can have different syntax:
			\newline
			Eg: public static void main(String args[])
			\item Identifier of the string array can change:
			\newline
			Eg: public static void main(String name[])
			\item String array can be taken as var\_arg parameter
			\newline
			Eg: public static void main(String… args)
			\item main() method can be declared with following modifiers:
			\begin{itemize}
				\item final
				\item synchroised
				\item strictfop
			\end{itemize}
			Eg:
			\codeblockfull{New.java}{
				class New \{ \\
				\s static final synchronized strictfp public void main(String... name)\{ \\
				\s \s System.out.println("Valid main method"); \\
				\s \} \\
				\} 
			}
		\end{itemize}
		\bigskip
		\item There can be multiple main() methods (i.e main() method over-loading is possible!). However JVm will always call String[] argument main method only.
		\bigskip
		\codeblockfull{New.java}{
			class New \{ \\
			\s	public static void main(String[] args)\{ \\
			\s	\s	System.out.println("Starting"); \\
			\s	\}
			\s	public static void main(int[] args)\{ \\
			\s \s		System.out.println("Sample 2"); \\
			\s	\}
			\}
		}
		\bigskip
		\outputblock{
			Starting
		}
		\newpage
		\item \textbf{Inheritance:} While executing child class, if child does not contain main(), then parent class main() will be executed.
		
		\codeblockfull{New.java}{
		class New \{ \\
		\s public static void main(String[] args)\{ \\
		\s \s System.out.println("Starting"); \\
		\s \} \\
		\} \\
		class C extends New \{\}
		}
		\bigskip
		\commandblock{
			\$ javac New.java  <- Creates C.class New.clas New.java  \\ 
			\$ java New \\ 
			Starting \\ 
			\$ java C \\
			Starting
		}
		\bigskip
		\item \textbf{Method hiding}: Child class can override parent class’s main(). This is not method overriding but it is method hiding.
		\newline
		Eg:
		\codeblockfull{New.java}{
			class New \{ \\
			\s	public static void main(String[] args)\{ \\
			\s \s		System.out.println("Starting New"); \\
			\s	\} \\
			\} \\
			class C extends New \{ \\
			\s	public static void main(String[] args)\{ \\
			\s \s		System.out.println("Starting C"); \\
			\s	\} \\
			\} 
		}
		\newpage
		\commandblock{
			\$ javac New.java  <- Creates C.class New.clas New.java  \\ 
			\$ java New \\ 
			Starting New \\ 
			\$ java C \\
			Starting C
		}
	\end{itemize}
	
	
	
	\noteblock{
		\begin{itemize}
			\item Whether class contains main() method or not, and whether main() method is declared according to requirement or not, \textbf{these things are won’t be checked by compiler}.
			\item At runtime, \textbf{JVM is responsible to check these things}.
			\item If JVM unable to find main() method, then will throw runtime exception!
		\end{itemize}
	}
	
\end{flushleft}
\newpage



 
\subsection{Command-line argument}


\begin{flushleft}
	
	\begin{itemize}
			\item Command line arguments are values passed to Java program when it is run from the command line. 
			\item With these command line arguments, JVM creates an array and pass it to main().
			\item Command line arguments can be accessed using the args parameter of the main(). 
			\item Args parameter is an array of String objects.
			\item You can customise behaviour of main() using command-line argument:
			
			\codecontinue{
			public static void main(String[] args) //  ← Here, \textbf{String[] args} contains command line args
			}
		
			\newimage{0.5}{content/chapter2/images/ans.png}
		
			\item Command line argument are always String[]
			
			\codeblockfull{New.java}{
				class New \{ \\
				\s	public static void main(String... args) \{  \\
				\s \s		for(int i = 0; i < args.length; i++) \{ \\
				\s \s			System.out.println(args[i]); \\
				\s \s		\} \\
				\s	\} \\
				\}
			}
		
			\commandblock{
			\$ javac New.java  \\
			\$ java New 1 23 3 \\
			1  \\
			23  \\
			3 \\
			}
			\bigskip
			\item Command line arguments are separated by space. To give one argument with space character, using "" :
			\commandblock{
				\$ java New "Note Book"
			}
			
							
	\end{itemize}
	
	
\end{flushleft}
\newpage



 
\subsection{Pillars of OOPs}
\input{content/chapter2/2.1.13.tex}


\chapterimage{index3.png} % Table of contents heading image
\chapter{Data types in Java}
%-----------------------
\section{Getting started with data type}
\input{content/chapter2/2.2.tex}
\subsection{Java is strongly typed}



\begin{flushleft}
	
	This means:
	\begin{itemize}
		\item Every variable has a type and every type is strictly defined
		\item All assignments are checked for type compatibility
		\item There are no automatic conversions of conflicting types
		\item Compiler checks all variables to ensure that the types are compatible. 
		\item Any type mismatches errors must be corrected before the compiler will finish compiling the class.
		
	\end{itemize}
	
	
\end{flushleft}
	





\subsection{Types of data types}
\setlength{\columnsep}{5pt}

\begin{flushleft}
	
	There are two types of data types in Java:
	
	\begin{itemize}
		\item \textbf{Primitive data types}: Includes boolean, char, byte, short, int, long, float and double.
		\item \textbf{Reference data types}: 
		\begin{itemize}
			\item These are not predefined by the language.
			\item They are instead created by the programmer using class definitions. \item Examples of reference data types include:
			\begin{itemize}
				\item Classes
				\item Interfaces
				\item Arrays
				\item Strings
				\item Enumerations
			\end{itemize}
		\end{itemize}
	
	\end{itemize}
	
	We shall see primitive data type in detail in this chapter.
	
\end{flushleft}

\newpage
\section{Integer}

\setlength{\columnsep}{5pt}

\begin{flushleft}
	
	

	\begin{figure}[h!]
		\centering
		\includegraphics[scale=.6]{content/chapter2/images/primitive.png}
	\end{figure}
	
	There are 4 types of integer in Java:
	\begin{itemize}
		\item byte
		\item short
		\item int
		\item long
	\end{itemize}	
	
	
\end{flushleft}

\newpage
\subsection{byte}
\input{content/chapter2/2.2.1.tex}
\subsection{short}
\input{content/chapter2/2.2.2.tex}
\subsection{int}
\input{content/chapter2/2.2.3.tex}
\subsection{long}
\input{content/chapter2/2.2.4.tex}
\subsection{Integer literals}
\input{content/chapter2/2.2.10.tex}
\section{Floating-point}
\input{content/chapter2/2.2.5.tex}
\subsection{Float}
\input{content/chapter2/2.2.18.tex}
\subsection{Double}
\input{content/chapter2/2.2.19.tex}
\subsection{Floating-point literals}

\begin{flushleft}

	\begin{itemize}
		
		\item By default, floating-point numbers are represented in double form.
		\item So below declaration will result in error: 
		\bigskip		
		\codeblock{
				float f = 1.0 \xmark
		}
		\bigskip		
		
		\item Correct way to represent float data type is by suffixing \textbf{"F"} or \textbf{"f"} to the floating-point number as shown:
		\bigskip
		\codeblock{
			float f = 1.6F; \cmark \newline
			float f = 7.8f;		\cmark
		}

		\item Double data type can be represented using suffix \textbf{"D"} or \textbf{"d"} or no suffix as below:
		\bigskip
		\codeblock{
		    double a = 12.67; \cmark \newline
			double b = 13.7d; \cmark \newline
			double c = 123.456D; \cmark \newline
			double d = 0123.456; \cmark \newline
			double e = 0789.9; \cmark
		}
	
		\item Floating-point literal are only in decimal form, not in octal and hexa decimal forms. Below are \textbf{invalid} declarations:
		\bigskip
		\codeblock{
			float a = 045.8; \xmark \newline
			float b = 0X56.9; \xmark \newline
			double c = 0X56.9; \xmark
		}
		\newpage
		\item We can assign integral literal directly to floating-point variables like double and float. Below are valid declarations:
		\bigskip
		\codeblock{
			double a = 0456; \cmark \newline
			double b = 0XFace; \cmark \newline
			double c = 10; \cmark \newline
			float a = 0456; \cmark \newline
			float b = 0XFace; \cmark \newline
			float c = 10; \cmark
		}
			
		\item \textbf{Expontential format:} This is scientific notation to represent very large or small floating-point values. Use the letter “e” or “E” to indicate the exponent:
				\bigskip
		\codeblock{
			double a = 1.2e3; \cmark \newline
			float b = 1.3e4F; \cmark
		}
		
		\item \textbf{Hexadecimal floating-point literals:} You can represent double and float in hexadecimal form using the letter “p” or “P”:
		\bigskip		
		\codeblock{
			double d = 0x12.2P2; \cmark \\
			float e = 0x12.2P2f; \cmark
		}
	
		\bigskip
		\item \textbf{Usage of \_ in floating literal}:
		\begin{itemize}
			\item From Java 1.7 version, we can use "\_" in middle of big numbers to increase floating-point's readability.
			\item At the time of compilation, these "\_" symbols will be removed automatically.
			\item Eg:
					\bigskip
			\codeblock{
				float x = 78\_3.2\_34\_23f; \cmark \newline
				double y = 12\_45\_23\_\_23\_2323.90;  \cmark
			}
			
			\item "\_" symbol cannot be used in the starting or end of integer or decimal point. Below are \textbf{invalid} declarations:
			\item Eg:
					\bigskip
			\codeblock{
				float x = 78\_3.2\_34\_23f\_; \cmark \newline
				double y = \_12\_45\_23\_\_23\_2323.90; \cmark  \newline
				double z = 12\_45\_2\_.3\_\_23\_2323.90;  \cmark
			}
						
		\end{itemize}	
		
	\end{itemize}
	
\end{flushleft}

\newpage


\section{Character}
\input{content/chapter2/2.2.20.tex}
\subsection{What is ascii \& unicode?}
\input{content/chapter2/2.2.21.tex}
\subsection{char datatype}
\input{content/chapter2/2.2.7.tex}
\subsection{Character literals}
\input{content/chapter2/2.2.13.tex}
\subsection{Escape character}
\input{content/chapter2/2.2.14.tex}
\section{Boolean}
\input{content/chapter2/2.2.6.tex}
\section{Type conversion}
\input{content/chapter2/2.2.16.tex}

%-----------------------

%-------------------------------------------------------------------------------------
%	CHAPTER 4
%-------------------------------------------------------------------------------------
\chapterimage{index5.png} % Table of contents heading image
\chapter{Arrays}
%-----------------------
\section{Arrays in detail}
\input{content/chapter4/4.1.tex}
\subsection{Array introduction}
\input{content/chapter4/4.1.1.tex}
\subsection{Array declaration}
\setlength{\columnsep}{3pt}
\begin{flushleft}
	\begin{itemize}
		\item \textbf{One dimensional array declaration}:
		\bigskip
		\syntaxblock{
			int[] x; (Recommended as name of variable is clearly separated from type) \newline
			int []x; \newline
			int x[];
		}
		\bigskip
		\noteblock{
			Array declaration \textbf{cannot define size} of array.
			\codeblock{
				int[6] x; \xmark
			}
		}
		
		\bigskip
		\item \textbf{Two-dimensional array declaration}:
		\bigskip
		\syntaxblock{
			int[][] x;  (Recommended)
			\newline
			int [][]x;
			\newline
			int x[][];
			\newline
			int[] []x;
		}
		
		
		
		\item \textbf{3-dimensional array declaration:}
		\bigskip
		\syntaxblock{
			int[][][] x;  (Recommended)
			\newline
			int [][][]x;
			\newline
			int x[][][];
			\newline
			int[] [][]x;
			\newline
			int[] x[][];
			\newline
			int[] []x[];
		}
		
		\bigskip		
		\item \textbf{More combinations}:
		\begin{itemize}
			\item Declaring variable \textbf{"a"} and \textbf{"b"} with \textbf{1} dimension:
			\bigskip
			\codeblock{
				int[] a,b;
			}
			\bigskip
			\item Declaring variable \textbf{"a"} with 2 dimension and variable \textbf{"b"} with 1 dimension
			\bigskip
			\codeblock{
				int[] a[],b;
			}
			
			\item Declaring variable \textbf{"a"} and \textbf{"b"} with 2 dimensions.
			\bigskip
			\codeblock{
				int[] a[],b[]; \\
				int[] \s a[],b;
			}
			
			\item Declaring variable \textbf{"a"} with 2 dimension and variable \textbf{"b"} with 3 dimension:
			\codeblock{
				int[] \s []a,b[];
			}		
			
			\bigskip
			\noteblock{
				\textbf{"[]" is allowed only in front of first variable.} 
				\codeblock{
					int[] \hphantom{} \hphantom{} []a,[]b; \xmark  \\
					int[] \hphantom{} \hphantom{} []a,[]b,[]c; \xmark
				}
			}
			
		\end{itemize}
		
	\end{itemize}
	
	
	
\end{flushleft}

\newpage





\subsection{Array creation}
\setlength{\columnsep}{3pt}
\begin{flushleft}
	\bigskip
	
	Things to note about array:
	\begin{itemize}
		\item In Java, every array is an Object.
		\item \textbf{"new"} operator is used to create an object.
		\item Hence, we can create array by using \textbf{new} operator.
	\end{itemize}
	
	
	\begin{itemize}
		
		\item \textbf{One dimensional array creation}:
		
		\begin{tcolorbox}[breakable,notitle,boxrule=1pt,colback=pink,colframe=pink]
			\color{black}
			\fontdimen2\font=8pt
			Syntax:  \newline
			int[] a = new int[4];
			\fontdimen2\font=4pt
		\end{tcolorbox}
		
		
		\begin{figure}[h!]
			\centering
			\includegraphics[scale=.45]{content/chapter4/images/array.png}
		\end{figure}	
		
		Important:
		\begin{itemize}
			\item At the time of array creation, \textbf{size should be mentioned compulsorily.}
			\item An array can be of zero size.
			\begin{tcolorbox}[breakable,notitle,boxrule=-0pt,colback=code,colframe=code]
				\color{black}
				\fontdimen2\font=8pt
				int[] x = new int[]; \xmark \par
				int[] x = new int[6]; \cmark  \par
				int[] x = new int[0]; \cmark 
				\fontdimen2\font=4pt
			\end{tcolorbox}
			
			\item Java compiler will never throw error for negative size of array. However, Java Virtual Machine will throw runtime error: \textbf{NegativeArraySizeException}.
			\begin{tcolorbox}[breakable,notitle,boxrule=-0pt,colback=code,colframe=code]
				\color{black}
				\fontdimen2\font=8pt
				int[] x = new int[-3]; \xmark
				\fontdimen2\font=4pt
			\end{tcolorbox}
			
			\item Allowed data types for mentioning array size are:
			\begin{itemize}
				\item integer
				\item byte
				\item short
				\item char
			\end{itemize}
			
			\begin{figure}[h!]
				\centering
				\includegraphics[scale=.45]{content/chapter4/images/allow.png}
			\end{figure}	
			
			\begin{tcolorbox}[breakable,notitle,boxrule=-0pt,colback=code,colframe=code]
				\color{black}
				\fontdimen2\font=8pt
				int[] x = new int[10]; \cmark \par
				int[] x = new int['a']; \cmark \par
				byte b = 20; \par
				int[] x = new int[b]; \cmark \par
				short s = 30; \par
				int[] x = new int[s]; \cmark \par
				\fontdimen2\font=4pt
			\end{tcolorbox}
			
			\newpage
			Below array creation will result in error:
			\begin{tcolorbox}[breakable,notitle,boxrule=-0pt,colback=code,colframe=code]
				\color{black}
				\fontdimen2\font=8pt
				int[] x = new int[10l]; \xmark \par
				int[] x = new int[3.5]; \xmark 
				\fontdimen2\font=4pt
			\end{tcolorbox}
			
			\item Maximum size of array can be 2147483647:
			\begin{tcolorbox}[breakable,notitle,boxrule=-0pt,colback=code,colframe=code]
				\color{black}
				\fontdimen2\font=8pt
				int[] x = new int[2147483647]; \cmark \par
				int[] x = new int[2147483648]; \xmark 
				\fontdimen2\font=4pt
			\end{tcolorbox}
			
			\item For every array type, corresponding classes are available and these classes are part of Java language and not available to the programmer level.
			\bigskip
			\bigskip
			\begin{tabulary}{1.0\textwidth}{|p{12em}|p{12em}|}
				\toprule
				\textbf{Array type} & \textbf{Corresponding class name} \\
				\midrule
				int[] & [I \\
				\hline
				int[][] & [[I \\
				\hline
				double[] & [D \\
				\hline
				short[] & [S \\
				\hline
				byte[] & [B \\
				\hline
				boolean[] & [Z \\
				\bottomrule
			\end{tabulary}
			\bigskip
			Eg: You can find name of class for different array type:
			\begin{tcolorbox}[breakable,notitle,boxrule=-0pt,colback=code,colframe=code]
				\color{black}
				\fontdimen2\font=8pt
				int[] a = new int[3]; \newline
				System.out.println(a.getClass().getName());
				\fontdimen2\font=4pt
			\end{tcolorbox}
			
			Output:
			\begin{tcolorbox}[breakable,notitle,boxrule=-0pt,colback=output,colframe=output]
				\color{black}
				\fontdimen2\font=8pt
				[I			
				\fontdimen2\font=4pt
			\end{tcolorbox}	
			
			
		\end{itemize}
		
		\newpage
		\item \textbf{Two-dimensional array creation}:
		\begin{itemize}
			\item In Java, two dimensional array is not implmeneted using matrix approach.
			\item Array of arrays approach is followed for multi-dimensional array creation.
			\item Advantage of array of arrays approach is improved memory utilisation.
			
			
		\end{itemize}
		\bigskip
		
		There are different ways of creating two-dimensional array.
		\begin{itemize}
			\item \textbf{Base size}: In this we specifiy the size of first dimension at the time of array creation.
			\begin{tcolorbox}[breakable,notitle,boxrule=-0pt,colback=code,colframe=code]
				\color{black}
				\fontdimen2\font=8pt
				int[][] x = new int[2][]; \par
				x[0] = new int[2]; \par
				x[1] = new int[3];
				\fontdimen2\font=4pt
			\end{tcolorbox}
			
			\begin{figure}[h!]
				\centering
				\includegraphics[scale=.45]{content/chapter4/images/two.png}
			\end{figure}	
			
			\newpage
			
		\end{itemize}
		
		\item Three-dimensional array creation:
		\begin{tcolorbox}[breakable,notitle,boxrule=-0pt,colback=code,colframe=code]
			\color{black}
			\fontdimen2\font=8pt
			int[][][] x = new int[2][][]; \par
			x[0] = new int[3][]; \par
			x[0][0] = new int[1]; \par
			x[0][1] = new int[2]; \par
			x[0][2] = new int[3]; \par
			x[1] = new int[2][2];
			\fontdimen2\font=4pt
		\end{tcolorbox}
		
		\begin{figure}[h!]
			\centering
			\includegraphics[scale=.45]{content/chapter4/images/three.png}
		\end{figure}
		
		
	\end{itemize}
	
	
	
\end{flushleft}

\newpage





%\subsection{Practice}
%\setlength{\columnsep}{3pt}
\begin{flushleft}
	\paragraph{}
	
	\bigskip
	
	\begin{figure}[h!]
		\centering
		\includegraphics[scale=.2]{content/practise.jpg}
	\end{figure}	
	
	
	
\end{flushleft}
\newpage



\subsection{Array initialisation}
\setlength{\columnsep}{3pt}
\begin{flushleft}
	\bigskip
	\begin{itemize}
		\item \textbf{One dimensional array}:\par
		
		Once we create an array, every array element is by default initialized with default values.
		
		\begin{figure}[h!]
			\centering
			\includegraphics[scale=.45]{content/chapter4/images/new1.png}
		\end{figure}	
		
		
		\begin{tcolorbox}[breakable,notitle,boxrule=1pt,colback=code,colframe=code]
			\color{black}
			\fontdimen2\font=8pt
			int[] a = new int[3]; \par
			System.out.println(a); \par
			System.out.println(a[0]);
			\fontdimen2\font=4pt
		\end{tcolorbox}
		
		Output:
		\begin{tcolorbox}[breakable,notitle,boxrule=-0pt,colback=output,colframe=output]
			\color{black}
			\fontdimen2\font=8pt
			[I@422a8473 \par
			0
			\fontdimen2\font=4pt
		\end{tcolorbox}	
		
		Whenever we are trying to print any reference variable, internally two string method will be called, which is implemented by default to return the string in the following form:
		\par
		\textbf{class\_name@hexadecimal\_form}
		
		\bigskip
		\item \textbf{Two-dimensional array}:  \par
		
		Example 1:
		\begin{figure}[h!]
			\centering
			\includegraphics[scale=.45]{content/chapter4/images/new2.png}
		\end{figure}	
		
		\begin{tcolorbox}[breakable,notitle,boxrule=1pt,colback=code,colframe=code]
			\color{black}
			\fontdimen2\font=8pt
			int[][] a = new int[2][3]; \par
			System.out.println(a);  \par
			System.out.println(a[0]);  \par
			System.out.println(a[0][0]);			
			\fontdimen2\font=4pt
		\end{tcolorbox}
		
		Output:
		\begin{tcolorbox}[breakable,notitle,boxrule=-0pt,colback=output,colframe=output]
			\color{black}
			\fontdimen2\font=8pt
			[[I@5a39699c \par
			[I@129a8472  \par
			0
			\fontdimen2\font=4pt
		\end{tcolorbox}	
		
		Example 2:
		
		\begin{figure}[h!]
			\centering
			\includegraphics[scale=.45]{content/chapter4/images/new3.png}
		\end{figure}	
		
		\begin{tcolorbox}[breakable,notitle,boxrule=1pt,colback=code,colframe=code]
			\color{black}
			\fontdimen2\font=8pt
			int[][] a = new int[2][]; \par
			System.out.println(a);  \par
			System.out.println(a[0]);  \par
			System.out.println(a[0][0]);			
			\fontdimen2\font=4pt
		\end{tcolorbox}
		
		Output:
		\begin{tcolorbox}[breakable,notitle,boxrule=-0pt,colback=output,colframe=output]
			\color{black}
			\fontdimen2\font=8pt
			[[I@5a39699c \par
			null  \par
			Exception in thread "main" java.lang.NullPointerException: 
			\fontdimen2\font=4pt
		\end{tcolorbox}	
		
		\bigskip
		
		\item \textbf{Over-riding array value:} \par
		Once we create an array, every array element by default initialised with default values. \par
		We can over-ride default values with custom values.
		
		\begin{figure}[h!]
			\centering
			\includegraphics[scale=.45]{content/chapter4/images/image2.png}
		\end{figure}	
		
		\begin{tcolorbox}[breakable,notitle,boxrule=1pt,colback=code,colframe=code]
			\color{black}
			\fontdimen2\font=8pt
			int[] a = new int[3]; \par
			a[0]=10; \par
			a[1]=20; \par
			a[2]=30; \par
			System.out.println(a[0]); \par
			System.out.println(a[1]); \par
			System.out.println(a[2]);
			\fontdimen2\font=4pt
		\end{tcolorbox}
		
		Output:
		\begin{tcolorbox}[breakable,notitle,boxrule=-0pt,colback=output,colframe=output]
			\color{black}
			\fontdimen2\font=8pt
			10 \par
			20 \par
			30 
			\fontdimen2\font=4pt
		\end{tcolorbox}	
		
		Note: Trying to access array element with out of range index (either positive or negative integer value) will result in runtime exception:  \textbf{"ArrayINdexOutOfBoundException"}
		
	\end{itemize}
	
	
\end{flushleft}
\newpage


\subsection{Array declaration, creation and initialisation in one line}
\input{content/chapter4/4.1.6.tex}
\subsection{length variable}
\input{content/chapter4/4.2.1.tex}
\subsection{Anonymous Arrays}
\setlength{\columnsep}{3pt}
\begin{flushleft}
	
	\begin{itemize}
		\item Anonymous arrays are nameless arrays.
		\item These arrays are used for instant one-time purpose.
		\bigskip
		\syntaxblock{
			Single dimension array: \textbf{new datatype[]\{\}} \\
			Multi-dimension array: \textbf{new datatype[][]\{\{\},\{\}\}}
		}

		\item While creating anonymous arrays, you cannot mention it's size:
		\bigskip
		\codeblock{
			new int[3]\{10,20,30\}	\xmark \\
			new int[]\{10,20,30\} \cmark \\
			new int[][]\{\{10,20,30\},\{40,50,60\}\} \cmark	
		}
				
		\item In below example, main() is calling sum() using an anonymous arrays:
		\bigskip
		\codeblockfull{Test.java}{
			\{ \\
			\s	public static void main (String[] args) \\
			\s 	\{ \\
			\s \s		sum(new int[]\{10,20,30,40,50\}); \\
			\s	\} \\
			\s	public static void sum(int[] x) \\
			\s	\{ \\
			\s \s		int total = 0; \\
			\s \s		for(int x1 :x) \\
			\s \s	\{ \\
			\s \s \s	total = total+x1; 		\\
				\s \s		\} \\
			\s \s		System.out.println("The sum is : "+total); 
		}
	\newpage
		\codecontinue{
		
			\s	\} \\
			\} 
		}
		\bigskip
		\outputblock{
		The sum is : 150
		}
		
	\end{itemize}
	
\end{flushleft}
\subsection{Array element assignments}
\input{content/chapter4/4.2.3.tex}
\subsection{Array variable assignments}
\input{content/chapter4/4.2.4.tex}
%-----------------------


%----------------------------------------------------------------------------------------
%	CHAPTER 3
%----------------------------------------------------------------------------------------
\chapterimage{index4.png} % Table of contents heading image
\chapter{Operators}
%-----------------------
\section{Operators in Java}
\setlength{\columnsep}{3pt}
\begin{flushleft}
	\bigskip
	\bigskip
	\begin{tcolorbox}[breakable,notitle,boxrule=1pt,colback=black,colframe=black]
		\color{white}
		\bigskip
		In this section, you are going to learn:
		\begin{enumerate}
			\item \textbf{Arithematic Operator}
			\item \textbf{String Concatenation Operator}
			\item \textbf{Increment/Decrement Operator}
			\item \textbf{Relational Operator}
			\item \textbf{Equality Operator}
			\item \textbf{Bitwise Operator}
			\item \textbf{Boolean complement Operator}
			\item \textbf{Short circuit Operator}
			\item \textbf{Type cast Operator}
			\item \textbf{Assignment Operator}
			\item \textbf{Conditional Operator}
			\item \textbf{new Operator}
			\item \textbf{ $\left[ \right]$ Operator}
			\item \textbf{Operator Precedence}
		\end{enumerate}	
	
	\end{tcolorbox}
	
\end{flushleft}

\newpage


\subsection{Arithematic Operator}
\setlength{\columnsep}{3pt}
\begin{flushleft}
	
	\tabletwo{
		Operator &  Example \\
		\hline
		Addition(+) & 
		\bigskip
		\codecontinue{
			int a = 5; \\
			int b = 10; \\
			int c = a + b; // c will be 15 
			}  \\
		\hline
		
		Subtraction(-) & 
		\bigskip
		\codecontinue{
			int a = 10; \\
			int b = 5; \\
			int c = a - b; // c will be 5
		}  \\
	}
	
	\tabletwo{
		
		Multiplication(*) & 
		\bigskip
		\codecontinue{
			int a = 2; \\
			int b = 3; \\
			int c = a * b; // c will be 6
		}  \\
		\hline
		
		Modulus (\%) &
		\bigskip
		\codecontinue{
			int a = 10;
			int b = 3;
			int c = a \% b; // c will be 1
		}\\
		
		\hline
		Division(/) & 
		\bigskip
		\codecontinue{
			int a = 10; \\
			int b = 3; \\
			int c = a / b; // c will be 3 (the remainder is discarded) \\
			\\
			double d = 10.0; \\
			double e = 3.0; \\
			double f = d / e; // f will be 3.33333
		}  \\
		
	}
	\newpage
	\textbf{Important Points:}
	\begin{itemize}
		\item \textbf{Implicit type casting:} If we apply any arithmetic operator between 2 variables “a” and “b”, the result type is always:
		\codecontinue{
			maximum(int, type of a, type of b)
		}
		
		Using above formula,
		\codecontinue{
			\begin{itemize}
				\item Byte + byte = int
				\item Byte + short = int
				\item Short + short = int
				\item Byte + long = long
				\item Long + double = double
				\item Float + long = float
				\item Char + char = int
				\item Char + double = double
			\end{itemize}
		}

		Eg:
		\codeblock{
			System.out.println('a'+'b'); // output: 195 \\
			System.out.println('a'+3.29); // output: 100.29
		}
	
		\bigskip
		
		\item \textbf{Infinity}: 
		\begin{itemize}
			\item In integral arithmetic \textbf{(byte short int long)}\textbf{, infinity cannot be represented} and JVM will return runtime error.
			\bigskip
			\codeblock{
				System.out.println(10/0); \xmark
			}
			
			\item But in \textbf{floating point arithmetic (float, double), infinity can be represented}.  For this, Float and Double classes contains below 2 constants:
			\begin{itemize}
				\item POSITIVE\_INFINITY;
				\item NEGATIVE\_INFINITY;
			\end{itemize}
			\bigskip		
			\codeblock{
				System.out.println(10/0.0); // output: Infinity\\
				System.out.println(-10/0.0); // output: -Infinity
			}
		\end{itemize}
		
		\bigskip
		\item \textbf{NaN (not a number):}
		\begin{itemize}
			\item In \textbf{integer arithmetic (byte, short, int, long) , undefined results cannot be represented} and JVM will return runtime error.
			\bigskip
			\codeblock{
			System.out.println(0/0);	\xmark				
			}
		
			\item But, in floating point arithmetic \textbf{(float,double), undefined results can be represented as NaN constant}.
			\bigskip
			\codeblock{
				System.out.println(0.0/0); // output: NaN \\
				System.out.println(-0/0.0); // output: NaN	
			}
				
		\end{itemize} 
		
	\end{itemize}
	
\end{flushleft}

\newpage






\subsection{String Concatenation Operator}
\setlength{\columnsep}{3pt}
\begin{flushleft}
	
	\begin{itemize}
		\item The only overloaded operator in Java is \textbf{"+"} operator. 
		\item It can act as arithmetic addition operator as well as string concatecation operator.
		\bigskip
		\codeblock{
			System.out.println(10+20); // output: 30  \\
			System.out.println("ab"+"cd"); // output: abcd
		}
		\bigskip
		\noteblock{
			Apart from "+", Java does not support \textbf{operator overloading}!	
		}
	
		\item Working of \textbf{"+"} with string:
		
		\begin{itemize}
			\item If atleast one argument is string type, + operator acts as concatenation operator.
			\item If both arguments are number type, + operator acts as arithmetic addition operator.
		\end{itemize}
	
		Eg:
		\codeblock{
			String a = "lavatech";   \\
			int b=10, c=20, d=30;   		\\
			System.out.println(a+b+c+d);  // output: lavatech102030 \\
			System.out.println(b+c+d+a);  // output: 60lavatech		\\
			System.out.println(b+c+a+d);  // output: 30lavatech30		\\
			System.out.println(b+a+c+d);  // output: 10lavatech2030
		}
		
	\end{itemize}
	

\end{flushleft}

\newpage






\subsection{Increment/Decrement Operator}

\begin{flushleft}

	
	\begin{itemize}
		\item The increment(++) and decrement(--) operators are unary operators.
		\item They are used to increment or decrement the value of a variable by 1.
		\item \textbf{Increment Operator (++):} Used in two ways:
		\begin{itemize}
			\item \textbf{Prefix (++var)}: Variable is incremented first and then used in the expression.
			\bigskip
			\codeblock{
				int a = 5; \\
				int b = ++a; // b will be 6, a will be 6
			}
			
			\item \textbf{Postfix (var++)}: Variable is used in the expression and then incremented. 
			\bigskip
			\codeblock{
				int a = 5; \\
				int b = a++; // b will be 5, a will be 6
			}			
		\end{itemize}
		
		\item \textbf{Decrement Operator (--):} Works in a similar way and can be used in prefix and postfix forms:
		\bigskip
		\codeblock{
			int a = 5; \\
			int b = --a; // b will be 4, a will be 4  \\
			int c = a--; // c will be 4, a will be 3
		}
	\end{itemize}
	
	Summary:
	\tablefour{
		\hline
		Expression & Initial value of x & Value of y & Final value of x \\
		\hline
		y=++x; & 10 & 11 & 11 \\
		\hline
		y=++x; & 10 & 10 & 11 \\
		\hline
		y=++x; & 10 & 9 & 9 \\
		\hline
		y=++x; & 10 & 10 & 9 \\
		\hline
	}
	
	Important Points:
	\begin{itemize}
		\item Increment/decrement is \textbf{applicable only on variable and not on constant}.
		\bigskip
		\codeblock{
			System.out.println(++10); \xmark
		}
		\item \textbf{Listing} of increment/decrement operators \textbf{not allowed}.
		\bigskip
		\codeblock{
			int x=10; \\
			int y = ++(++x); \xmark
		}
	
		\item \textbf{For final variables}, increment/decrement operators \textbf{cannot} be used:
		\bigskip
		\codeblock{
			final int x=10; \\
			System.out.println(x++); \xmark
		}
	
		\item Increment/decrement is applicable on all primitive type, \textbf{except boolean datatype}:
		\begin{itemize}
			\item Integer example:
			\bigskip
			\codeblock{
				int x=10; \\
				x++; \\
				System.out.println(x);   // output: 11
			}
			\item Character example:
			\bigskip
			\codeblock{
				char ch = 'a'; \\
				ch++; \\
				System.out.println(ch); // output: 'b'
			}
			\newpage
			\item Boolean example:
			\bigskip
			\codeblock{
				boolean b=true;  \\
				b++;  \xmark  
			}
		
		
		\end{itemize}
		
	\end{itemize}
		
	\textbf{Difference between “x++” and “x=x+1”}
	
	\begin{itemize}
		\item We know that: If we apply any arithmetic operator between 2 variables “a” and “b”, the result type is always:
		\bigskip
		\codecontinue{
			maximum(int, type of a, type of b)
		}
		
		\item This is the reason why using "x=x+1" can result in compile-time error:
		\bigskip
		\codeblock{
			byte b=10; \\
			b = b+1;        \xmark 
		}
	
		\item But in case of increment/decrement operators, internal type casting will be performed automatically:
		\bigskip
		\codeblock{
			byte b=10;  \\
			b++;   \cmark
		}
		
	\end{itemize}
	
	
		
\end{flushleft}
\newpage
\subsection{Relational Operator}

\begin{flushleft}
	
	\begin{itemize}
		\item Below are available relational operator in Java:
		\newimage{0.5}{content/chapter3/images/relational.png}
		\item We can apply relational operator for every primitive datatype, except boolean datatype.
		\item Eg:
		\bigskip
		\codeblock{
			System.out.println(10>20);  // output: false  \\
			System.out.println('a'>20);  // output: false  \\ 
			System.out.println('b'>2.0); // output: false  \\
			//System.out.println(true > false); \xmark
		}
	
		\item We can’t apply relational operators for object types
		\bigskip
		\codeblock{
			System.out.println('lava' > 'lavatech');  \xmark
		}
	
		\item Chaining of relational operators is not allowed.
		\bigskip
		\codeblock{
			System.out.println(10>20>30);  \xmark
		}
		
	\end{itemize}
	
\end{flushleft}
\newpage
\subsection{Equality Operator}

\begin{flushleft}
	
	\begin{itemize}
		\item Equality operators can be used for every primitive type including boolean.
		\bigskip
		\codeblock{
			System.out.println(10==20);  // output: fasle \\
			System.out.println('a' == 'b'); // output: false \\
			System.out.println('a' == 97.0); // output: true \\
			System.out.println(false == false); // output: true
		}
		
		\item Equality operators can be applied for object types also. For object references, "r1" \& "r2", "r1==r2" returns true, if both reference pointing to the same object (reference comparison or address comparison)
		\bigskip
		\codeblock{
			Thread t1 = new Thread(); \\
			Thread t2 = new Thread(); \\
			Thread t3 = t1; \\
			System.out.println(t1 == t2);   // output: false \\
			System.out.println(t1 == t3);   // output: true
		}
		
		There should be some relation between argument types(either child to parent or parent to child or same type). Otherwise, it will result in compile-time error.
		\bigskip
		\codeblock{
			Thread t1 = new Thread();   \\
			Object o = new Object();   \\
			String s = new String("lava");   \\
			System.out.println(t1==o) ; // output: false    \\
			System.out.println(o==s); // output: false    \\
			//System.out.println(s==t1); \xmark
		}
		
		For any object reference "r": "r==null"  ← is always False
		\bigskip
		\codeblock{
			String s1 = new String("lava"); \\
			System.out.println(s1 == null) ; // output: false \\
			\\
			String s2 = new String(); \\
			System.out.println(s2==null); // output: false \\
			\\
			String s3 = null; \\
			System.out.println(s3==null); // output: true
		}
		
	\end{itemize}


	
\end{flushleft}
\newpage
\subsection{Bitwise Operator}

\begin{flushleft}

	\bigskip
	\begin{figure}[h!]
		\centering
		\includegraphics[scale=0.5]{content/chapter3/images/bitwise2.png}
	\end{figure}
	
	\begin{tabular}{|p{10em}|p{8em}|p{6em}|}
		
		\multicolumn{3}{l}{} \\
		\multicolumn{3}{l}{\textbf{Bitwise \&} - If both bits are 1, then only 1 otherwise 0}                  
		\\ \hline
		\multicolumn{1}{|l|}{Sample Code} & \multicolumn{1}{l|}{Output}  & \multicolumn{1}{l|}{Explaination}   
		\\ \hline
		\multicolumn{1}{|l|}{\begin{tabular}[c]{@{}l@{}}int a=4; \\
				int b=5; \\
				System.out.println(a \& b)	 \end{tabular}}  & 
		\multicolumn{1}{l|}{4}   &  
		\multicolumn{1}{|l|}{\begin{tabular}[c]{@{}l@{}}100 \\
				101 \\
				----- \\
				100	 \end{tabular}}  
		\\ 
		\hline
	
		\multicolumn{3}{l}{} \\
		\multicolumn{3}{l}{\textbf{Bitwise |} - If atleast one bit is 1, then only 1 otherwise 0}                  
		\\ \hline
		\multicolumn{1}{|l|}{Sample Code} & \multicolumn{1}{l|}{Output}  & \multicolumn{1}{l|}{Explaination}   
		\\ \hline
		\multicolumn{1}{|l|}{\begin{tabular}[c]{@{}l@{}}int a=4; \\
				int b=5; \\
				System.out.println(a | b)	 \end{tabular}}  & 
		\multicolumn{1}{l|}{5}   &  
		\multicolumn{1}{|l|}{\begin{tabular}[c]{@{}l@{}}100 \\
				101 \\
				----- \\
				101	 \end{tabular}}  
		\\ 
		\hline
	
		
	\end{tabular}
	
	\newpage
	
	\begin{tabular}{|p{10em}|p{8em}|p{8em}|}
		
		\multicolumn{3}{l}{} \\
		\multicolumn{3}{l}{\textbf{Bitwise \^} - Also called x-or. If both bits are different, then 1, otherwise 0}                  
		\\ \hline
		\multicolumn{1}{|l|}{Sample Code} & \multicolumn{1}{l|}{Output}  & \multicolumn{1}{l|}{Explaination}   
		\\ \hline
		\multicolumn{1}{|l|}{\begin{tabular}[c]{@{}l@{}}int a=4; \\
				int b=5; \\
				System.out.println(a \textbf{\^} b)	 \end{tabular}}  & 
		\multicolumn{1}{l|}{1}   &  
		\multicolumn{1}{|l|}{\begin{tabular}[c]{@{}l@{}}100 \\
				101 \\
				----- \\
				001	 \end{tabular}}  
		\\ 
		\hline
		
		\multicolumn{3}{l}{} \\
		\multicolumn{3}{l}{\textbf{Bitwise \~} - Bitwise complement operator, 1 becomes 0 and 0
becomes 1}                  
		\\ \hline
		\multicolumn{1}{|l|}{Sample Code} & \multicolumn{1}{l|}{Output}  & \multicolumn{1}{l|}{Explaination}   
		\\ \hline
		\multicolumn{1}{|l|}{\begin{tabular}[c]{@{}l@{}}int a=4; \\
				System.out.println( \textbf{\~} a)	 \end{tabular}}  & 
		\multicolumn{1}{l|}{-5}   &  
		\multicolumn{1}{|l|}{\begin{tabular}[c]{@{}l@{}} 32-bit os represents number with	\\
				total 32 bits as 000...000100 \\
				 \textbf{\~} 000...000100 = 111...111011 \\
				Left most bit is sign bit where, \\
				1 is negative and 0 is positive \\
				Since left most bit is now 1, number is negative \\
				Negative number is represented as \\
				1's complement + 1 \\
				ie. 100..000100 + 1 = 100...000101	
		 \end{tabular}}  
		\\ 
		\hline
	

		\multicolumn{3}{l}{} \\
		\multicolumn{3}{l}{\begin{tabular}[c]{@{}l@{}}\textbf{Bitwise >>} - Bitwise right shift. \\
		Remove "x" bit from right side and add "x" 0 to left side.	 \end{tabular}}                  
		\\ \hline
		\multicolumn{1}{|l|}{Sample Code} & \multicolumn{1}{l|}{Output}  & \multicolumn{1}{l|}{Explaination}   
		\\ \hline
		\multicolumn{1}{|l|}{\begin{tabular}[c]{@{}l@{}}int a=10; \\
				int x=2; \\
				System.out.println(a >> x)	 \end{tabular}}  & 
		\multicolumn{1}{l|}{2}   &  
		\multicolumn{1}{|l|}{\begin{tabular}[c]{@{}l@{}}000...1010 >> 2 \\
				000...0010 \\
					 \end{tabular}}  
		\\ 
		\hline
	\end{tabular}

	\newpage
	
	\begin{tabular}{|p{10em}|p{8em}|p{6em}|}
		
		
	\multicolumn{3}{l}{} \\
	\multicolumn{3}{l}{\begin{tabular}[c]{@{}l@{}}\textbf{Bitwise <<} - Bitwise left shift. \\	
	Remove "x" bit from left side and add "x" 0 to right side.	 \end{tabular}}                  
	\\ \hline
	\multicolumn{1}{|l|}{Sample Code} & \multicolumn{1}{l|}{Output}  & \multicolumn{1}{l|}{Explaination}   
	\\ \hline
	\multicolumn{1}{|l|}{\begin{tabular}[c]{@{}l@{}}int a=10; \\
			int x=2; \\
			System.out.println(a << x)	 \end{tabular}}  & 
	\multicolumn{1}{l|}{40}   &  
	\multicolumn{1}{|l|}{\begin{tabular}[c]{@{}l@{}}000...1010 << 2 \\
			000...101000 \\
	\end{tabular}}  
	\\ 
	\hline
		
	\end{tabular}
		
	\noteblock{
		\begin{itemize}
			\item \&, | , \textbf{\^}  are applicable for both boolean and integral type.
			\item >> , << are applicable for integral type only.
			\item \textbf{\~} applicable for only integral type but not for boolean type.
		\end{itemize}	
	}
		
		
\end{flushleft}

\subsection{Boolean complement operator}

\begin{flushleft}
	
	\begin{itemize}
		\item The Boolean complement operator (!) is a unary operator that negates the value of a Boolean expression. 
		\bigskip
		\codeblock{
			boolean a = true; \\
			boolean b = !a; // output: b is false 
		}
		\bigskip
		\noteblock{
			! operator can be applied only on boolean data-type.
		}
		
	\end{itemize}
	
\end{flushleft}
\newpage
\subsection{Short circuit Operator}

\begin{flushleft}
	
	 	In Java, the short-circuit operators are:
		\begin{itemize}
			\item \textbf{\&\& (logical AND)}: Same as bitwise \&
			\item \textbf{|| (logical OR)}: Same as bitwise |
		\end{itemize}

		\tabletwo{
		\hline
		\& , |  & \&\& , || \\
		\hline
		Both arguments should be evaluated always & Second argument evaluation is optional \\
		\hline
		Relatively performance is low & Relatively performance is high \\
		\hline
		Applicable for both boolean and integral types & Applicable only for boolean but not for integral types \\
		\hline
		Eg: \newline
		For x \& y, both x and y will be evaluated. \newline
		Similarly, \newline
		For x | y, both x and y will be evaluated. 
		& 
		Eg: \newline
		For x \&\& y, y will be evaluated if and only if x is true i.e if x is false then y wont be evaluated. \newline
		Similarly, \newline
		For x || y,  y will be evaluated if and only if x is false i.e if x is true then y wont be evaluated. \\
		
		\hline
		
		}
		
		\bigskip
		Consider below code showing different behaviour of \&, \&\&, |, ||:
		\codeblock{
			int x = 10, y = 15;   \\
			if(  ++x < 10  \& ++y > 15) \{ \\
			\s	x++;   \\
			\}  \\
			else \{  \\
			\s	y++;   \\
			\}  \\
			System.out.println(x + "..." + y);  //   output: 11...17
		}
		\bigskip
		\codeblock{
			int x = 10, y = 15;   \\
			if(  ++x < 10  | ++y > 15) \{ \\
			\s	x++;   \\
			\}  \\
			else \{  \\
			\s	y++;   \\
			\}  \\
			System.out.println(x + "..." + y);  //   output: 12...16
		}
		\bigskip
		\codeblock{
			int x = 10, y = 15;   \\
			if(  ++x < 10  \&\& ++y > 15) \{ \\
			\s	x++;   \\
			\}  \\
			else \{  \\
			\s	y++;   \\
			\}  \\
			System.out.println(x + "..." + y);  //   output: 11...16
		}
		\bigskip
				\codeblock{
			int x = 10, y = 15;   \\
			if(  ++x < 10  || ++y > 15) \{ \\
			\s	x++;   \\
			\}  \\
			else \{  \\
			\s	y++;   \\
			\}  \\
			System.out.println(x + "..." + y);  //   output: 12...16
		}
		\bigskip

	
\end{flushleft}
\newpage
\subsection{Assignment Operator}
\input{content/chapter3/3.1.10.tex}
\subsection{Conditional Operator}
\input{content/chapter3/3.1.11.tex}
%\subsection{instanceOf Operator}
%\setlength{\columnsep}{3pt}
\begin{flushleft}
	\bigskip
	\begin{figure}[h!]
		\centering
		\includegraphics[scale=0.5]{content/chapter3/images/relational.png}
	\end{figure}

	\begin{tcolorbox}[breakable,notitle,boxrule=1pt,colback=yellow,colframe=yellow]
		\color{black}
		\fontdimen2\font=8pt
		\textbf{Note}: 
		\begin{itemize}
			\item \textbf{"<" , "<=" , ">" , ">="} are applicable on integers, strings \& boolean , but not on cross datatypes.	
			\item \textbf{"==" , "!="} are applicable on all datatypes, including cross datatypes.
			\item \textbf{"==" , "!="} never returns error.
		\end{itemize}
		\fontdimen2\font=4pt
	\end{tcolorbox}
	
	\newpage

	Sample code 1:
	\begin{tcolorbox}[breakable,notitle,boxrule=-0pt,colback=code,colframe=code]
			\color{white}
			\fontdimen2\font=8pt
			a=6 \newline
			b=2 \newline
			print(a>b, a<b) \newline
			a=6 \newline
			b=6 \newline
			print(a>=b, a<=b) \newline
			a="apple" \newline
			b="mango" \newline
			print(a>b, a<b)
			\fontdimen2\font=4pt
	\end{tcolorbox}
		
	Output:
	\begin{tcolorbox}[breakable,notitle,boxrule=-0pt,colback=output,colframe=output]
			\color{black}
			True False \newline
			True True \newline
			False True
			\fontdimen2\font=4pt
	\end{tcolorbox}
	\bigskip \bigskip
	Sample code 2:
		\begin{tcolorbox}[breakable,notitle,boxrule=-0pt,colback=code,colframe=code]
		\color{white}
		\fontdimen2\font=8pt
		a=6 \newline
		b="6" \newline
		print(a==b, a!=b) \newline
		a=6 \newline
		b=6 \newline
		print(a==b, a!=b)
		\fontdimen2\font=4pt
	\end{tcolorbox}
	
	Output:
	\begin{tcolorbox}[breakable,notitle,boxrule=-0pt,colback=output,colframe=output]
		\color{black}
		False True \newline
		True False
		\fontdimen2\font=4pt
	\end{tcolorbox}
\end{flushleft}

\newpage






\subsection{new Operator}
\input{content/chapter3/3.1.14.tex}
\subsection{ $\left[ \s \right]$ Operator}
\input{content/chapter3/3.1.12.tex}
\subsection{Operator Precedence}
\input{content/chapter3/3.1.13.tex}
%-----------------------

%----------------------------------------------------------------------------------------
%	CHAPTER 6
%----------------------------------------------------------------------------------------
\chapterimage{index4.png} % Table of contents heading image
\chapter{Flow Control}
%-----------------------
\section{Flow control statement}
\input{content/chapter8/8.1.tex}
\section{Selection Statements}
\setlength{\columnsep}{3pt}
\begin{flushleft}
	
	Below are 2 selection statements:
	\begin{itemize}
		\item \textbf{if..else}
		\item \textbf{switch case}
	\end{itemize}
	Let's see each of these in detail.
	
	
\end{flushleft}






\subsection{if...else}
\setlength{\columnsep}{3pt}
\begin{flushleft}
	
	\begin{itemize}
		\item The \textbf{if} statement allows you to execute a block of code if a certain condition is true.
		\bigskip
		\syntaxblock{
			if (condition) \\
			\s Action if is true
		}		
		\bigskip
		\syntaxblock{
			if (condition) \{ \\
			\s Action if is true \\
			\} \\
			else \{ \\
			\s Action if is false \\
			\} 
		}
		
		
		\item The argument to if statement should be boolean type only, else there will be compile-time error.
		
		\item \textbf{Else part and curly braces are optional.}
		\item Without curly braces only one statement is allowed under if statement, which should \textbf{not be declarative statement}.
		\newpage
		\item Eg 1:
		\codeblock{
			if (true) \\
			\s System.out.println("Hello");
		}
		
		\item Eg 2:
		\codeblock{
			if (true);		\cmark // Note: Semicolon is also valid statement.
			
		}
		
		\item Eg 3:
		\codeblock{
			if (true) \\
			\s int x = 10;  \xmark // Compile-time error for declarative statement
		}
		
		\item Eg 4:
		\codeblock{
			if (true) \{ \\
			\s	int x = 10; \\ 
			\}
		}
		
		\item Eg 5:
		\codeblock{
			int x = 0; \\
			if (x) \{          \xmark // Compile-time error as not a boolean     \\
			\s System.out.println("Hello"); \\
			\} \\
			else \{ \\
			\s System.out.println("Hi"); \\
			\} 
		}
		\newpage
		\item Eg 6:
		\codeblock{
			int x = 0; \\
			if (x=20) \{          \xmark // Compile-time error as not a boolean     \\
			\s System.out.println("Hello"); \\
			\} \\
			else \{ \\
			\s System.out.println("Hi"); \\
			\} 
		}
		
		\item Eg 7:
		\codeblock{
			int x = 0; \\
			if (x==20) \{   \\
			\s System.out.println("Hello"); \\
			\} \\
			else \{ \\
			\s System.out.println("Hi"); // output: Hi \\
			\} 
		}
		
		\item Eg 8:
		\codeblock{
			boolean b = false; \\
			if (b == false) \{   \\
			\s System.out.println("Hello"); // output: Hello \\
			\} \\
			else \{ \\
			\s System.out.println("Hi");  \\
			\} 
		}
	\end{itemize}
	
	\newpage
	
	\textbf{Dangling else}
	
	\begin{itemize}
		\item There is no dangling else problem in Java.
		\item \textbf{Every else is mapped to the nearest if statement.}
		
		\codeblock{
			if (true)  \\
			if (true) \\
			\s System.out.println("Hello"); // output: Hello \\
			else \\
			\s System.out.println("Hi");  \\
		}	
	\end{itemize}
		
\end{flushleft}


\subsection{switch case}
\setlength{\columnsep}{3pt}
\begin{flushleft}
	
	\begin{itemize}
		\item If several options ar available, then it is not recommended to use nested if..else statement, as it reduces readability.
		
		\item Solution: switch statement
		\bigskip
		\syntaxblock{
			switch(argument) \{ \\
			\s	case arg-1:  	\\
			\s \s	action1;	\\
			\s \s	break;		\\
			\s	case arg-2:		\\
			\s \s	action2;		\\
			\s \s	break;		\\
			\s	case n:		\\
			\s \s	action-n;	\\
			\s \s	break		\\
			\s	default:		\\
			\s \s	Default action	\\
			\}
		}
		\bigskip
		\item Curly braces are mandatory.
		\item \textbf{Case and default are optional}, i.e an empty switch statement is a valid Java syntax.
		\codeblock{
			int x = 10; \\
			switch(x) \{\}  \cmark
		}
		\bigskip
		\item Allowed argument types in switch statement:
		\begin{itemize}
			\item Upto Java 1.4 version -> \textbf{byte, short, char, int}
			\item From Java 1.5 version -> byte, short, char, int, \textbf{wrapper classes (Byte, Short, Character, Integer), enum}
			\item From Java 1.7 versionbyte, short, char, int, wrapper classes (Byte, Short, Character, Integer), enum, \textbf{string}
		\end{itemize}
		
		\bigskip	
		\item Inside switch, every statement should be under some case or default.
		
		\codeblock{
			int x = 10; \\
			switch(x)\{   \\
			\s System.out.println();   \xmark \s // Compile-time error! \\
			\}
		}
		
		\item Case argument should be compile-time constant (i.e constant expression).
		\bigskip
		\codeblock{
			int x=10; \\
			int y=20; \\
			switch(x) \{ \\
			\s case 10:  \\
			\s \s System.out.println(10); \\
			\s \s	break; \\
			\s case y;    \xmark \s // Compile-time error!   \\         
			\s \s System.out.println(); \\
			\s \s	break; \\
			\} 
		}
		\bigskip
		\noteblock{
			If y is declared as "final", then there will be no compile-time error.
		}
		
		\newpage
		
		\item Switch arguments and case label can be expressions. But, case label should be constant expression.
		
		\codeblock{
			int x = 10; \\
			switch(x+1) \{ \\
			\s case 10: \\
			\s \s System.out.println(10); \\
			\s \s break; \\
			\s case 10+20+30: \\
			\s \s System.out.println(60); \\
			\s \s break; \\
			\}
		}
		\bigskip
		\item \textbf{Case label should be in range of switch arg type}, else it will result in compile-time error.
		\newline
		Eg 1:
		\codeblock{
			byte b = 10; \\
			switch(b) \{ \\			
			\s case 10: \\
			\s \s System.out.println(10); \\
			\s \s break; \\
			\s case 100: \\
			\s \s System.out.println(100); \\
			\s \s break; \\
			\s case 1000:   \xmark \s // Compille-time error! \\
			\s \s System.out.println(1000); \\
			\s \s break; \\				
			\}
		}
		\newpage
		
		Eg 2:
		\codeblock{
			byte b = 10; \\
			switch(b+1) \{  // Implicit type-caste to integer \\			 
			\s case 10: \\
			\s \s System.out.println(10); \\
			\s \s break; \\
			\s case 100: \\
			\s \s System.out.println(100); \\
			\s \s break; \\
			\s case 1000:  \\
			\s \s System.out.println(1000); \\
			\s \s break; \\				
			\}
		}
		
		\item Duplicate case labels are not allowed.
		\bigskip
		
		\codeblock{
			int x = 10; \\
			switch(x) \{  \\			 
			\s case 12: \\
			\s \s System.out.println(12); \\
			\s \s break; \\
			\s case 97: \\
			\s \s System.out.println(97); \\
			\s \s break; \\
			\s case 'a':   \xmark \s // Duplicate labels error    \\
			\s \s System.out.println(1000); \\
			\s \s break; \\				
			\}
		}
	\end{itemize}
	
	\newpage
	\textbf{Summary for case-label argument}
	\begin{itemize}
		\item It should  be constant expression.
		\item The value should be in the range of switch argument type.
		\item Duplicate case label are not allowed
	\end{itemize}
	
	\bigskip
	\bigskip
	\bigskip
	\textbf{Fall-through inside switch}
	\begin{itemize}
		\item Within the switch, if any case is matched, from that case onwards all statements will be executed until break or end of the switch.
		\item This is called fall-through inside switch.
		\item The main advantage of fall inside the switch is, we can define common action for multiple cases (code-reuseablilty).
		\bigskip
		\syntaxblock{
			switch(argument) \{  \\
			\s	case arg-1: \\
			\s	case arg-2: \\
			\s	case arg-3: \\
			\s \s		action; \\
			\s \s		break; \\
			\s	case arg-4: \\
			\s	case arg-5: \\
			\s	case arg-6: \\
			\s \s		action; \\
			\s \s		break; \\
			\}
		}
		\newpage
		Eg:
		\codeblock{
			int x = 0;
			switch(x) \{ \\
			\s case 0: \\
			\s \s System.out.println(0); \\
			\s case 1: \\
			\s \s System.out.println(1); \\
			\s \s break; \\
			\s case 2: \\
			\s \s System.out.println(2); \\
			\s default: \\
			\s \s System.out.print("default"); \\
			\}
		}
		\bigskip
		\outputblock{
			0 \\
			1
		}
		\item \textbf{Default case:}
		\begin{itemize}
			\item Within the switch, \textbf{you can take default case at most once}.
			\item Default case will be executed if and only if, there is no case matched.
			\item Within the switch, you can write default case anywhere but it is recommended to write as last case. \newline
			Eg:
			\codeblock{
				int x = 3;
				switch(x) \{ \\
				\s	default:  \\
				\s \s	System.out.println("default") \\
				\s case 0:  \\
				\s \s 	System.out.println(0) \\
				\s case 1: 
			}
			\newpage
			\codecontinue{
				\s \s	System.out.println(1) \\
				\s case 2: \\
				\s \s	System.out.println(2) \\
				\}
			}
			\bigskip
			\outputblock{
				default \\
				0
			}
		\end{itemize}
	\end{itemize}
	
	
\end{flushleft}

\newpage






\section{Iteration}
\input{content/chapter8/8.2.tex}
\subsection{while()}
\input{content/chapter8/8.2.1.tex}
\subsection{do-while()}
\setlength{\columnsep}{3pt}
\begin{flushleft}
	
	\begin{itemize}
		\item If we want to execute loop body atleast once, then we should go for do-while loop.
		\bigskip
		\syntaxblock{
			do \{ \\
			\s action \\
			\} while(condition); 
		}
		\bigskip
		\item The ";" after while is compulsory.
		\item The condition should be of boolean type.
		\item Curly braces are optional
		\item Without curly braces only one statement is allowed which should not be declarative statement.
		\item Below are some valid and invlaid example of do-while:
		\bigskip
		\codeblock{
			do \\
			\s System.out.println("Hello"); \cmark \\
			while(true);
		}
		\bigskip
		\codeblock{
			do;   \cmark \\
			while(true);
		}
		\bigskip
		\codeblock{
			do \\
			\s int x = 10; \xmark \\
			while(true);
		}
		\bigskip
		\codeblock{
			do \{ \\
				int x = 10; \\
			\} while(true);  \cmark
		}
		
		\bigskip
		\codeblock{
			do \\
			while(true); \xmark
		}
		\bigskip
		\codeblock{
			do \\
			\s while(true) \cmark \\ 
			\s \s System.out.println("Hello"); \\
			while(false);
		}
		\item Unreachable statement in do-while loop also results in compile-time error. Below are some examples showing unreachable statement:
		\bigskip
		\codeblock{
			do \{\\
			\s System.out.println("Hello"); \\
			\} while(true); \\
			System.out.println("Hi");   \xmark // Unreachable statement error
		}
	
		\bigskip
		\codeblock{
			do \{ \\
			\s System.out.println("Hello"); \\
			\} while(false); \\
			System.out.println("Hi"); \cmark 
		}
		\bigskip
		\outputblock{
		Hello \\
		Hi
		}
	
		\bigskip
		\item Every final variable will be replaced by value at compile-time only. If any variable is final in while's condition, then the condition will be evaluated at compile-time only.
		
		\bigskip
		\codeblock{
			final int a = 10, b = 20; \\
			do \{ \\
				\s System.out.println("Hello"); \\
			\} while(a < b); \\
			System.out.println("Hi");  \xmark 
			\s // Unreachable statement
		}
		\bigskip
		\codeblock{
			final int a = 10, b = 20; \\
			do \{ \\
			\s System.out.println("Hello"); \\
			\} while(a > b); \\
			System.out.println("Hi");  \cmark
		}
	
	
	\end{itemize}		
	
\end{flushleft}

\newpage


\subsection{for()}
\setlength{\columnsep}{3pt}
\begin{flushleft}

	\begin{itemize}
		\item If you know number of iterations in advance then for loop is the best choice. 
		\newimage{0.6}{content/chapter8/images/for.png}
		
		\item Curly braces are optional, without curly braces only one statement is allowed, which should not be declarative statement.
		
		\item Egs:
		\bigskip
		\codeblock{
			for(int i = 0; true; i++) \\
			\s System.out.println("Hello");   \cmark
		}
		\bigskip
		\codeblock{
			for(int i = 0; i < 10; i++) ; \cmark
		}
		\bigskip
		\codeblock{
			for(int i = 0; i < 10; i++) \\
			\s int x = 10; \xmark
		}
		\item Let's see each section of for loop in detail:
		\begin{itemize}
			\item \textbf{Initialisation section:}
			\begin{itemize}
				\item This section will be executed only once in for loop lifecycle.
				\item Use to declare and initialise local variables.
				\item You can declare any number of variables, but should be of the same type. \item If you are trying to declare different datatype variables, then you'll get compile time error.
				\bigskip
				\codeblock{
					for(\textbf{int i = 0}; i < 10; i++) \{\} \cmark \\
					for(\textbf{int i = 0, j=0}; i < 10; i++) \{\} \cmark \\
					for(\textbf{int i = 0, String = "a"}; i < 10; i++) \{\} \xmark \\
					for(\textbf{int i = 0, int j = 0}; i < 10; i++) \{\} \xmark 
				}
				
				\item You can take any valid Java statement in this section.
				\item Eg:
				\bigskip
				\codeblock{
				int i = 0; \\
				for( \textbf{System.out.println("Hi")}; i < 3; i++ ) \{ \\
				\s System.out.println("Hello"); \\
				\}
				}
				\bigskip
				\outputblock{
					Hi \\
					Hello \\
					Hello \\
					Hello
				}
			\end{itemize}
		
			\item \textbf{Conditional section:}
			\begin{itemize}
				\item In this section you can take any valid Java expression, but it should be of the type boolean.
				\item Eg:
				\bigskip
				\codeblock{
					for(int i = 0; \textbf{true} ; i++)
				}
				\item This section is optional, if nothing is added here, the compiler will always place true.
			\end{itemize}
		
			\item \textbf{Increment/decrement section:}
			\begin{itemize}
				\item In this section, you can can take any valid Java statement.
				\item Eg:
				\bigskip
				\codeblock{
					int i = 0; \\
					for(\textbf{System.out.println("Hello")}; i < 3; System.out.println("Hi")) \{ \\
					\s i++; \\
					\}
				}
			\bigskip
			\outputblock{
				Hello
				Hi
				Hi
				Hi
			}	
			\end{itemize}
		
		\end{itemize}
		
		\bigskip
		\noteblock{
			All 3 parts of for loop are independent of each other and optional.	
		}

		\item Infinite loop examples:
		\bigskip
		\codeblock{
			for(;;) \{ \\ 
			\s System.out.println("Hello"); \\
			\}
		}
		\bigskip
		\codeblock{
			for(;;);
		}
		
		\item Unreachable statement in for loop, results in compile-time error. 
		\newline
		Eg:
		\bigskip
		\codeblock{
			for(int i = 0; \textbf{true} ; i++) \{ \\
			\s System.out.println("Hello"); \\
			\} \\
			System.out.println("Hello");  \xmark // Unreachable statement
		}
		
		
	\end{itemize}

		
	
\end{flushleft}

\newpage


\subsection{for-each loop}
\setlength{\columnsep}{3pt}
\begin{flushleft}

	\begin{itemize}
		\item This is enhanced for loop introduced in Java 1.5 version.
		\item It is used to retrieve elements of arrays and collections.
		\bigskip
		\syntaxblock{
			for (type var : array) \{  \\
			\s	statements using var; \\
			\}	
		}
		\item Eg: Print elements of 1-dimensional array -
		\end{itemize}
		\tabletwo{
			\hline
			Normal for loop & Enhanced for loop \\
			\hline
			\codeblock{
				int[] x = \{10,20,30\}; \\
				for(int i=0;i<x.length;i++) \{ \\
				\s	System.out.println(x[i]); \\
				\}
			} &
			\codeblock{
				int[] x = \{10,20,30\}; \\
				for(int x1; x) \{ \\
				\s	System.out.println(x1); \\
				\}
			} \\
		}
	

	
	\begin{itemize}
		\item Eg: Print elements of 2-dimensional array -
	\end{itemize}

	\tabletwo{
		\hline
		Normal for loop & Enhanced for loop \\
		\hline

		\codeblock{
			int[][] x=\{\{10,20\},\{40,50\}\}; \\
			for(int i=0;i<x.length; i++)\{ \\
			for(int j=0;j<x[i].length;j++)\{ \\
			\s	System.out.println(x1); 	\\
			\s \} \\	
			\}
		} &
		\codeblock{
			int[][] x=\{\{10,20\},\{40,50\}\}; \\
			for(int[] x1: x) \{  \\
			for(int x2: x1) \{    \\
			\s	System.out.println(x1);  \\
			\s \}  \\
			\} 
		}  \\
	}
	
	\newpage
	
	\begin{itemize}
		
		\item Eg3: Print 3-dimensional array using for-each loop -
		\bigskip
		\codeblock{
			
			int[][][] x = \{ \\
			\s	\{ \{1, 2\}, \{3, 4\}, \{5, 6\}, \{7, 8\} \}, \\
			\s	\{ \{9, 10\}, \{11, 12\}, \{13, 14\}, \{15, 16\} \}, \\
			\s	\{ \{17, 18\}, \{19, 20\}, \{21, 22\}, \{23, 24\} \} \\
			\}; \\
			\\
			for(int[]][] x1:x) \{ \\
			\s for(int[] x2: x1 ) \{  \\
			\s \s for(int x3: x2) \{ \\
			\s \s \s System.out.println(x3); \\
			\s \s \} \\
			\s \}	 \\
			\} 
		}
		
		\item Drawback of for-each loop:
		\begin{itemize}
			\item Applicable only for arrays and collections.
			\item Using for-each loop, you can print array elements in original order but not in reverse order.	
		\end{itemize}
		
		
	\end{itemize}


	
\end{flushleft}

\newpage


\section{Transfer Statements}
\setlength{\columnsep}{3pt}
\begin{flushleft}

	Transfer statements are used to alter the flow of control in a program. They allow you to jump from one part of the code to another based on certain conditions or requirements.
	Below are transfer statement:
	\begin{itemize}
		\item break
		\item continue
		\item return
		\item try..catch..finally
		\item assert
	\end{itemize}
	
\end{flushleft}



\subsection{break}
\setlength{\columnsep}{3pt}
\begin{flushleft}
	
	Use break statement in the following places:
	\begin{itemize}
		\item Inside switch to stop fall-through:
		\bigskip
		\codeblock{
			int x = 0; \\
			switch(x) \{ \\
			\s case 0: \\
			\s \s System.out.print(0); \\
			\s case 1: \\
			\s \s System.out.print(1); \\
			\s case 2: \\
			\s \s System.out.print(2); \\
			\}
		}
		
		\newpage
		\item Inside loop to break loop execution based on some condition.
		\bigskip
		\codeblock{
			for(int i=0; i < 10; i++) \{ \\
			\s if(i==5) \\
			\s \s break;
			\s 	System.out.print(i); \\
			\}
		}
		
		\item Inside labeled blocks to break block execution based on some condition:
		\bigskip
		\codeblockfull{Test.java}{
			class Test \{ \\
			\s public static void main(String[] args) \{ \\
			\s int x = 10;  \\
			\s l1: \{ \\
			\s \s 	System.out.print("begin"); \\ 
			\s \s if(x==10) \\
			\s \s \s	break l1; \\
			\s \s 	System.out.print("end"); \\
			\s	\} \\
			\s \s 	System.out.print("Hello"); \\
			\s	\} \\
			\}
		}
		
		
	\end{itemize}

\end{flushleft}

\newpage


\subsection{continue}
\setlength{\columnsep}{3pt}
\begin{flushleft}

	\begin{itemize}
		\item Use continue statement inside loops to skip current iteration and continue for the next iteration.
		
		\codeblock{
			for(int i=0; i<10; i++) \{ \\
			\s	if(i \% 2 == 0) \\
			\s \s		continue; \\
			\s	System.out.println(i); \\
			\}
		}
		\bigskip
		\outputblock{
		1 \\
		3 \\
		5 \\
		7 \\
		9
		}
		
	\end{itemize}

\end{flushleft}
\newpage


\subsection{Labeled break \& continue}
\setlength{\columnsep}{3pt}
\begin{flushleft}
	
	\begin{itemize}
		\item Use labeled break and continue to break/continue a particular loop in nested loops.
		\bigskip
		\syntaxblock{
			label-1: \\
			for(...) \{ \\
			\s label-2: \\
			\s for(...) \{ \\
			\s ..\\
			\s ..\\
			\s \s for(...) \{ \\
			\s \s \s break label-1; \\
			\s \s \s break label-2; \\
			\s \s \s break; \\
			\s \s \} \\
			\s \} \\
			\} \\
		}
		
	\end{itemize}
	
	Eg 1:
	\codeblock{
		l1: \\
		for(int i=0; i<j; i++) \{ \\
		\s	for(int j=0; j<3; j++) \{ \\
		\s \s		if(i==j) \\
		\s \s \s			continue l1; \\
		\s \s		System.out.println(i+"..."+j); \\
		\s	\} \\
		\} \\
	}
	\bigskip
	\outputblock{
		1...0 \\
		2...0 \\
		2...1
	}

	\bigskip
	Eg 2:
	\codeblock{
		l1: \\
		for(int i=0; i<j; i++) \{ \\
		\s	for(int j=0; j<3; j++) \{ \\
		\s \s		if(i==j) \\
		\s \s \s			break l1; \\
		\s \s		System.out.println(i+"..."+j); \\
		\s	\} \\
		\} \\
	}
	\bigskip
	\outputblock{
		---No output---
	}
	
	
\end{flushleft}

\newpage


\subsection{return}
%\setlength{\columnsep}{3pt}
\begin{flushleft}
	
	\tabletwo{
		Operator &  Example \\
		\hline
		Addition(+) & 
		\bigskip
		\codecontinue{
			int a = 5; \\
			int b = 10; \\
			int c = a + b; // c will be 15 
			}  \\
		\hline
		
		Subtraction(-) & 
		\bigskip
		\codecontinue{
			int a = 10; \\
			int b = 5; \\
			int c = a - b; // c will be 5
		}  \\
	}
	
	\tabletwo{
		
		Multiplication(*) & 
		\bigskip
		\codecontinue{
			int a = 2; \\
			int b = 3; \\
			int c = a * b; // c will be 6
		}  \\
		\hline
		
		Modulus (\%) &
		\bigskip
		\codecontinue{
			int a = 10;
			int b = 3;
			int c = a \% b; // c will be 1
		}\\
		
		\hline
		Division(/) & 
		\bigskip
		\codecontinue{
			int a = 10; \\
			int b = 3; \\
			int c = a / b; // c will be 3 (the remainder is discarded) \\
			\\
			double d = 10.0; \\
			double e = 3.0; \\
			double f = d / e; // f will be 3.33333
		}  \\
		
	}
	\newpage
	\textbf{Important Points:}
	\begin{itemize}
		\item \textbf{Implicit type casting:} If we apply any arithmetic operator between 2 variables “a” and “b”, the result type is always:
		\codecontinue{
			maximum(int, type of a, type of b)
		}
		
		Using above formula,
		\codecontinue{
			\begin{itemize}
				\item Byte + byte = int
				\item Byte + short = int
				\item Short + short = int
				\item Byte + long = long
				\item Long + double = double
				\item Float + long = float
				\item Char + char = int
				\item Char + double = double
			\end{itemize}
		}

		Eg:
		\codeblock{
			System.out.println('a'+'b'); // output: 195 \\
			System.out.println('a'+3.29); // output: 100.29
		}
	
		\bigskip
		
		\item \textbf{Infinity}: 
		\begin{itemize}
			\item In integral arithmetic \textbf{(byte short int long)}\textbf{, infinity cannot be represented} and JVM will return runtime error.
			\bigskip
			\codeblock{
				System.out.println(10/0); \xmark
			}
			
			\item But in \textbf{floating point arithmetic (float, double), infinity can be represented}.  For this, Float and Double classes contains below 2 constants:
			\begin{itemize}
				\item POSITIVE\_INFINITY;
				\item NEGATIVE\_INFINITY;
			\end{itemize}
			\bigskip		
			\codeblock{
				System.out.println(10/0.0); // output: Infinity\\
				System.out.println(-10/0.0); // output: -Infinity
			}
		\end{itemize}
		
		\bigskip
		\item \textbf{NaN (not a number):}
		\begin{itemize}
			\item In \textbf{integer arithmetic (byte, short, int, long) , undefined results cannot be represented} and JVM will return runtime error.
			\bigskip
			\codeblock{
			System.out.println(0/0);	\xmark				
			}
		
			\item But, in floating point arithmetic \textbf{(float,double), undefined results can be represented as NaN constant}.
			\bigskip
			\codeblock{
				System.out.println(0.0/0); // output: NaN \\
				System.out.println(-0/0.0); // output: NaN	
			}
				
		\end{itemize} 
		
	\end{itemize}
	
\end{flushleft}

\newpage






\subsection{try..catch..finally}
%\setlength{\columnsep}{3pt}
\begin{flushleft}
	
	\tabletwo{
		Operator &  Example \\
		\hline
		Addition(+) & 
		\bigskip
		\codecontinue{
			int a = 5; \\
			int b = 10; \\
			int c = a + b; // c will be 15 
			}  \\
		\hline
		
		Subtraction(-) & 
		\bigskip
		\codecontinue{
			int a = 10; \\
			int b = 5; \\
			int c = a - b; // c will be 5
		}  \\
	}
	
	\tabletwo{
		
		Multiplication(*) & 
		\bigskip
		\codecontinue{
			int a = 2; \\
			int b = 3; \\
			int c = a * b; // c will be 6
		}  \\
		\hline
		
		Modulus (\%) &
		\bigskip
		\codecontinue{
			int a = 10;
			int b = 3;
			int c = a \% b; // c will be 1
		}\\
		
		\hline
		Division(/) & 
		\bigskip
		\codecontinue{
			int a = 10; \\
			int b = 3; \\
			int c = a / b; // c will be 3 (the remainder is discarded) \\
			\\
			double d = 10.0; \\
			double e = 3.0; \\
			double f = d / e; // f will be 3.33333
		}  \\
		
	}
	\newpage
	\textbf{Important Points:}
	\begin{itemize}
		\item \textbf{Implicit type casting:} If we apply any arithmetic operator between 2 variables “a” and “b”, the result type is always:
		\codecontinue{
			maximum(int, type of a, type of b)
		}
		
		Using above formula,
		\codecontinue{
			\begin{itemize}
				\item Byte + byte = int
				\item Byte + short = int
				\item Short + short = int
				\item Byte + long = long
				\item Long + double = double
				\item Float + long = float
				\item Char + char = int
				\item Char + double = double
			\end{itemize}
		}

		Eg:
		\codeblock{
			System.out.println('a'+'b'); // output: 195 \\
			System.out.println('a'+3.29); // output: 100.29
		}
	
		\bigskip
		
		\item \textbf{Infinity}: 
		\begin{itemize}
			\item In integral arithmetic \textbf{(byte short int long)}\textbf{, infinity cannot be represented} and JVM will return runtime error.
			\bigskip
			\codeblock{
				System.out.println(10/0); \xmark
			}
			
			\item But in \textbf{floating point arithmetic (float, double), infinity can be represented}.  For this, Float and Double classes contains below 2 constants:
			\begin{itemize}
				\item POSITIVE\_INFINITY;
				\item NEGATIVE\_INFINITY;
			\end{itemize}
			\bigskip		
			\codeblock{
				System.out.println(10/0.0); // output: Infinity\\
				System.out.println(-10/0.0); // output: -Infinity
			}
		\end{itemize}
		
		\bigskip
		\item \textbf{NaN (not a number):}
		\begin{itemize}
			\item In \textbf{integer arithmetic (byte, short, int, long) , undefined results cannot be represented} and JVM will return runtime error.
			\bigskip
			\codeblock{
			System.out.println(0/0);	\xmark				
			}
		
			\item But, in floating point arithmetic \textbf{(float,double), undefined results can be represented as NaN constant}.
			\bigskip
			\codeblock{
				System.out.println(0.0/0); // output: NaN \\
				System.out.println(-0/0.0); // output: NaN	
			}
				
		\end{itemize} 
		
	\end{itemize}
	
\end{flushleft}

\newpage






\subsection{assert}
%\setlength{\columnsep}{3pt}
\begin{flushleft}
	
	\tabletwo{
		Operator &  Example \\
		\hline
		Addition(+) & 
		\bigskip
		\codecontinue{
			int a = 5; \\
			int b = 10; \\
			int c = a + b; // c will be 15 
			}  \\
		\hline
		
		Subtraction(-) & 
		\bigskip
		\codecontinue{
			int a = 10; \\
			int b = 5; \\
			int c = a - b; // c will be 5
		}  \\
	}
	
	\tabletwo{
		
		Multiplication(*) & 
		\bigskip
		\codecontinue{
			int a = 2; \\
			int b = 3; \\
			int c = a * b; // c will be 6
		}  \\
		\hline
		
		Modulus (\%) &
		\bigskip
		\codecontinue{
			int a = 10;
			int b = 3;
			int c = a \% b; // c will be 1
		}\\
		
		\hline
		Division(/) & 
		\bigskip
		\codecontinue{
			int a = 10; \\
			int b = 3; \\
			int c = a / b; // c will be 3 (the remainder is discarded) \\
			\\
			double d = 10.0; \\
			double e = 3.0; \\
			double f = d / e; // f will be 3.33333
		}  \\
		
	}
	\newpage
	\textbf{Important Points:}
	\begin{itemize}
		\item \textbf{Implicit type casting:} If we apply any arithmetic operator between 2 variables “a” and “b”, the result type is always:
		\codecontinue{
			maximum(int, type of a, type of b)
		}
		
		Using above formula,
		\codecontinue{
			\begin{itemize}
				\item Byte + byte = int
				\item Byte + short = int
				\item Short + short = int
				\item Byte + long = long
				\item Long + double = double
				\item Float + long = float
				\item Char + char = int
				\item Char + double = double
			\end{itemize}
		}

		Eg:
		\codeblock{
			System.out.println('a'+'b'); // output: 195 \\
			System.out.println('a'+3.29); // output: 100.29
		}
	
		\bigskip
		
		\item \textbf{Infinity}: 
		\begin{itemize}
			\item In integral arithmetic \textbf{(byte short int long)}\textbf{, infinity cannot be represented} and JVM will return runtime error.
			\bigskip
			\codeblock{
				System.out.println(10/0); \xmark
			}
			
			\item But in \textbf{floating point arithmetic (float, double), infinity can be represented}.  For this, Float and Double classes contains below 2 constants:
			\begin{itemize}
				\item POSITIVE\_INFINITY;
				\item NEGATIVE\_INFINITY;
			\end{itemize}
			\bigskip		
			\codeblock{
				System.out.println(10/0.0); // output: Infinity\\
				System.out.println(-10/0.0); // output: -Infinity
			}
		\end{itemize}
		
		\bigskip
		\item \textbf{NaN (not a number):}
		\begin{itemize}
			\item In \textbf{integer arithmetic (byte, short, int, long) , undefined results cannot be represented} and JVM will return runtime error.
			\bigskip
			\codeblock{
			System.out.println(0/0);	\xmark				
			}
		
			\item But, in floating point arithmetic \textbf{(float,double), undefined results can be represented as NaN constant}.
			\bigskip
			\codeblock{
				System.out.println(0.0/0); // output: NaN \\
				System.out.println(-0/0.0); // output: NaN	
			}
				
		\end{itemize} 
		
	\end{itemize}
	
\end{flushleft}

\newpage






%-----------------------

%----------------------------------------------------------------------------------------
%	CHAPTER 6
%----------------------------------------------------------------------------------------
\chapterimage{index4.png} % Table of contents heading image
\chapter{Functions}
%-----------------------
\section{var\_arg methods}
\input{content/chapter8/8.1.tex}
\section{Selection Statements}
\setlength{\columnsep}{3pt}
\begin{flushleft}
	
	Below are 2 selection statements:
	\begin{itemize}
		\item \textbf{if..else}
		\item \textbf{switch case}
	\end{itemize}
	Let's see each of these in detail.
	
	
\end{flushleft}








%----------------------------------------------------------------------------------------
%	CHAPTER 6
%----------------------------------------------------------------------------------------
%\chapterimage{index4.png} % Table of contents heading image
%\chapter{Scanner}
%-----------------------
%\section{Flow control statement}
%\input{content/chapter8/8.1.tex}
%\section{Selection Statements}
%\setlength{\columnsep}{3pt}
\begin{flushleft}
	
	Below are 2 selection statements:
	\begin{itemize}
		\item \textbf{if..else}
		\item \textbf{switch case}
	\end{itemize}
	Let's see each of these in detail.
	
	
\end{flushleft}






%\subsection{if...else}
%\setlength{\columnsep}{3pt}
\begin{flushleft}
	
	\begin{itemize}
		\item The \textbf{if} statement allows you to execute a block of code if a certain condition is true.
		\bigskip
		\syntaxblock{
			if (condition) \\
			\s Action if is true
		}		
		\bigskip
		\syntaxblock{
			if (condition) \{ \\
			\s Action if is true \\
			\} \\
			else \{ \\
			\s Action if is false \\
			\} 
		}
		
		
		\item The argument to if statement should be boolean type only, else there will be compile-time error.
		
		\item \textbf{Else part and curly braces are optional.}
		\item Without curly braces only one statement is allowed under if statement, which should \textbf{not be declarative statement}.
		\newpage
		\item Eg 1:
		\codeblock{
			if (true) \\
			\s System.out.println("Hello");
		}
		
		\item Eg 2:
		\codeblock{
			if (true);		\cmark // Note: Semicolon is also valid statement.
			
		}
		
		\item Eg 3:
		\codeblock{
			if (true) \\
			\s int x = 10;  \xmark // Compile-time error for declarative statement
		}
		
		\item Eg 4:
		\codeblock{
			if (true) \{ \\
			\s	int x = 10; \\ 
			\}
		}
		
		\item Eg 5:
		\codeblock{
			int x = 0; \\
			if (x) \{          \xmark // Compile-time error as not a boolean     \\
			\s System.out.println("Hello"); \\
			\} \\
			else \{ \\
			\s System.out.println("Hi"); \\
			\} 
		}
		\newpage
		\item Eg 6:
		\codeblock{
			int x = 0; \\
			if (x=20) \{          \xmark // Compile-time error as not a boolean     \\
			\s System.out.println("Hello"); \\
			\} \\
			else \{ \\
			\s System.out.println("Hi"); \\
			\} 
		}
		
		\item Eg 7:
		\codeblock{
			int x = 0; \\
			if (x==20) \{   \\
			\s System.out.println("Hello"); \\
			\} \\
			else \{ \\
			\s System.out.println("Hi"); // output: Hi \\
			\} 
		}
		
		\item Eg 8:
		\codeblock{
			boolean b = false; \\
			if (b == false) \{   \\
			\s System.out.println("Hello"); // output: Hello \\
			\} \\
			else \{ \\
			\s System.out.println("Hi");  \\
			\} 
		}
	\end{itemize}
	
	\newpage
	
	\textbf{Dangling else}
	
	\begin{itemize}
		\item There is no dangling else problem in Java.
		\item \textbf{Every else is mapped to the nearest if statement.}
		
		\codeblock{
			if (true)  \\
			if (true) \\
			\s System.out.println("Hello"); // output: Hello \\
			else \\
			\s System.out.println("Hi");  \\
		}	
	\end{itemize}
		
\end{flushleft}



%----------------------------------------------------------------------------------------
%	CHAPTER 6
%----------------------------------------------------------------------------------------
\chapterimage{index4.png} % Table of contents heading image
\chapter{Packages in Java}
%-----------------------
\section{Packages in detail}
\input{content/chapter10/10.1.tex}
\subsection{import keyword}
\setlength{\columnsep}{3pt}
\begin{flushleft}
	
	\begin{itemize}
		\item import keyword is used to access classes, interfaces, and other types from external packages or libraries. 

		\item Consider an example where you want to use \textbf{ArrayList} class from \textbf{java.util} package. For this, you will need to use fully qualified name:
		\bigskip
		\codeblock{
			class New \{ \\
			\s public static void main(String[] args) \{ \\
		    \s \textbf{java.util.ArrayList l1 = new java.util.ArrayList();}	 \\
		   	\s	\} \\
		   \} 
		}
	
		With \textbf{import} keyword, the code would look like:
		\bigskip
		\codeblock{
		 \textbf{import java.util.ArrayList;} \\
		 class New \{ \\
		 \s public static void main(String[] args) \{ \\
		 \s \s	\textbf{ArrayList l1 = new ArrayList();} \\
		 \s	\} \\
		 \}
		}
		
		\item Using import statement, it is not required to use fully qualified name everytime.
		\bigskip
		\noteblock{
			All classes and interfaces present in the following packages are by default available to every Java program. Hence we are not required to write import statement:
			\begin{itemize}
				\item java.lang
				\item default package (current working directory)
			\end{itemize}
		}
		\item 		There are 2 types of \textbf{import}. Let's see each of these in detail.	
		\end{itemize}		

		\newpage

		\textbf{Explicit class import:} 
		\begin{itemize}
			\item Used to import a single class.
			\item It is highly recommended as improves readability of the code.
			\bigskip
			\syntaxblock{
				import packageName.ClassName;	
			}
			\item Eg:
			\codeblock{
				import java.util.ArrayList; \cmark \\
				import java.util; \xmark
			}
		\end{itemize}
		
		
		\textbf{Implicit class import:} 
		\begin{itemize}
			\item Used tp import all classes from a package.
			\item Not recommended to use as reduces readability of the code.
			\bigskip
			\syntaxblock{
				import packageName.*;
			}
			\item Eg:
			\codeblock{
				import java.util.ArrayList.*; \xmark  \\
				import java.util.*; \cmark
			}
		
			
			\item Implicit declaration results in \textbf{ambiguity problem}. Eg:

			
			\codeblock{
				\textbf{import java.util.*;} \\
				\textbf{import java.sql.*;} \\
				class New \{ \\
				\s public static void main(String[] args) \{ \\
				\s \s \textbf{Date d = new Date();  \xmark \s Ambiguous error } \\
				\s	\} \\
				\} 
			}

			\newpage			
			In above code, \textbf{Date} is available in both \textbf{java.util} as well as \textbf{java.sql} package.
			
			\item While resolving class names, compiler will give precedence in the following order:
			\begin{itemize}
				\item Explicit class import
				\item Classes present in current working directory (default package)
				\item Implicit class import
			\end{itemize}
			\bigskip
			\codeblock{
				import java.util.*; \\
				import java.sql.*; \\
				class New \{ \\
				\s public static void main(String[] args) \{ \\
				\s \s \textbf{Date d = new Date();}  \\
				\s \s  System.out.println(d.getClass().getName()); \\
				\s	\} \\
				\} 
			}
			\bigskip
			\outputblock{
				java.util.Date
			}	
		\end{itemize}
		
		
		
			
		\newpage
		
		\textbf{Some more things to note about import statement:}
		\begin{itemize}
			\item \textbf{No subpackage import}: By importing a Java package, all classes and interfaces present in that package are available to Java program, but not the subpackage classes.
			\bigskip
			\item \textbf{No effect on execution time:} Import statements is totally compile-time 
			related concept. \textbf{If more number of imports, then more will be the compile-time.} But, there is no effect on execution time (runtime).
			\bigskip
			\item Difference between C language \textbf{#include} and Java \textbf{import} statement:
			\bigskip
			\bigskip
			\tabletwo{
				\hline
				\#include & import statement \\
				\hline
				All I/O header files will be loaded at the beginning (at translation time) &
				No \textbf{".class"} file will be loaded at the beginning. Whenever we are using a particular class, then only corresponding \textbf{".class"} file will be loaded. \\
				\hline
				It is static include & It is "dynamic include" or "load on fly" or "load on demand" \\
				\hline
			}	
		\end{itemize}
		
		
		
		
		

		
		
		
				

	
\end{flushleft}

\newpage


%\subsection{static import keyword}
%\setlength{\columnsep}{3pt}
\begin{flushleft}
	\begin{itemize}
		\item Introduced in Java1.5 version.
		\item According to Sun microsystem, usage of static import reduces length of the code and improves readability.
		\item But according to world-wide programming experts, usage of static import \textbf{creates confusion and reduces readability of code}. 
		\item Hence it is not recommended to use static import.
		\item Using static import, we can access static members directly without classname.
		
		
	\end{itemize}
\end{flushleft}
\newpage



\subsection{Packages}
\setlength{\columnsep}{3pt}
\begin{flushleft}

	\begin{itemize}
		\item Java package groups related classes and method can be grouped into a single unit.
		\item Eg: All classes and interfaces for File I/O operations are group into \textbf{java.io} package.
		\item Advantage of package:
		\begin{itemize}
			\item \textbf{Resolve naming conflict}
			\item \textbf{Modularity:} Modularises the application
			\item \textbf{Maintainability:} Improves application maintainability.
			\item \textbf{Security:} Provides security for our components
		\end{itemize}

	\end{itemize}

	\textbf{Naming for Package}
	\begin{itemize}
		\item There is one universally accepted naming convention for packages, i.e to use internet domain name in reverse:
		\newline
		Eg:
		\codeblockfull{Test.java}{
			package com.lavatech.www; \\
			class Test \{ \\
			\s	public static void main(String[] args) \{ \\
			\s \s		int a=10; \\
			\s \s 		System.out.println(a); \\
			\s	\} \\
			\}
		}
		\bigskip
		
 		\item Below command can be used to directory structure at valid location.
 		\bigskip
 		\syntaxblock{
 			\$ javac -d <location> <filename.java>
 		}
 		\newpage
 		Eg:
 		\commandblock{
 			\# To create directory structure in current location \\
 			\$ javac -d . Test.java    \\
 			\\
 			\# To create directory structure under "D:" location \\
 			\$ javac -d D: Test.java   
 		}
 		
		Above command will create below directory structure:
		\newimage{0.8}{content/chapter10/images/mew.png}
		
		\item While running, make sure that you provide fully qualified name:
		\bigskip
		\commandblock{
		\$ java com.lavatech.www.Test  \\
		10	
		}
		
	\end{itemize}

\newpage

\textbf{Important pointers:}
\begin{itemize}
	\item There can be utmost one package statement in any Java source file.
	\newline
	Eg 1:
	\bigskip
	\codeblock{
		package pack1; \\
		package pack2;  \xmark \s \# Result in compile-time error \\
		public class A {} 	
	}
	Eg 2:
	\bigskip
	\codeblock{
		package pack1; \cmark \\
		public class A \{\} 	
	}

	\item The first non comment statement should be package statement (if it is available), otherwise we will get compile-time error.
	\bigskip
	\codeblock{
		import java.util.*; \xmark \s \# Result in compile-time error  \\
		package pack1; \\
		package pack2; \\
		public class A \{\}
	}	
	
	\item The following is valid order in any Java source file:
	\bigskip
	\codecontinue{
		package statement;   $\leftarrow$  Atmost one\\
		import statements;  $\leftarrow$ Any number		\\
		class | interface | enum declarations   $\leftarrow$ Any number
	}
	
	\item An empty source file is a valid java program. Hence the following are valid java source files:
	\bigskip
	\codeblockfull{Test.java}{
	\# Empty file	
	}

	\bigskip
	\codeblockfull{Test.java}{
	pacakge pack1;
	}	

	\bigskip
	\codeblockfull{Test.java}{
		import java.util.*;
	}

	\bigskip
	\codeblockfull{Test.java}{
		package pack1; \\
		import java.util.*;
	}

	\bigskip
	\codeblockfull{Test.java}{
		class Test \{\}
	}
\end{itemize}

	
\end{flushleft}

\newpage



%----------------------------------------------------------------------------------------
%	CHAPTER 6
%----------------------------------------------------------------------------------------
\chapterimage{index4.png} % Table of contents heading image
\chapter{Polymorphism \& Inheritance}
%-----------------------
\section{Class}
\setlength{\columnsep}{3pt}
\begin{flushleft}
	
	\begin{itemize}
		\item A class is a blueprint to create an object.
		\item A class consists of instance variable and methods.
	\end{itemize}	
	In this section, we shall see class in detail.
	
\end{flushleft}

\newpage


\subsection{Attributes in detail}
\setlength{\columnsep}{3pt}
\begin{flushleft}

	\begin{itemize}
		\item Represents the \textbf{state or characteristics} of an object.
		\item Declared \textbf{within a class but outside any methods}.
		\item Each object of the class has its own copy of instance variables.
		\bigskip
		\syntaxblock{
			modifier type identifier;
		}
		where,
		\begin{itemize}
			\item \textbf{access modifiers} can be public, private, protected, default, static, final, transient, volatile.
			\item \textbf{type} can be data-type, classname
			\item \textbf{identifier} is name of attribute
		\end{itemize}
		
	\end{itemize}	
	
	Let see each effect of the access modifier on instance variable in detail.
	\begin{itemize}
		\item \textbf{public attributes:}
		\begin{itemize}
			\item Can be accessed from any other class or package.
			\item Eg 1: Accessing from other class -
			\bigskip
			\codeblock{
				class A \{ \\
				\s	\textbf{public int no;} \\
				\} \\
				\\
				public class Test3 \{ \\
				\s	public static void main(String[] args) \{  \\
				\s \s		A a1 = new A(); \\
				\s \s		a1.no = 1; \\
				\s \s		System.out.println(a1.no); \\
				\s	\} \\
				\}
			}

			\newpage
			\outputblock{
				1
			}	
		
			\bigskip
			\item Eg 2: Accessing from other package -
			\bigskip
			\codeblockfull{Test1.java}{
				package com.lavatech.www; \\
				public class Test1 \{ \\
				\s	\textbf{public int code;} \\
				\} 
			}
			
			\bigskip
			\codeblockfull{Test2.java}{
				package com.lavatech.info; \\
				import com.lavatech.www.Test1;			 \\
				class Test2 \{ \\
				\s	public static void main(String[] args) \{  \\
				\s \s		Test1 test = new Test1(); \\
				\s \s		test.code = 12345678; \\
				\s \s		System.out.println(test.code); \\
				\s	\} \\
				\}	
			}
			
			\bigskip
			\commandblock{
				\$ javac -d . Test1.java  \\
				\$ javac -d . Test2.java  \\
				\$ java com.lavatech.info.Test2  \\
				12345678
			}		
		\end{itemize}
	
		\newpage
		\item \textbf{private attributes}:
		\begin{itemize}
			\item  Can only be accessed within the same class. 
			\item Not visible to other classes or packages.
			\item Eg: Accessing with the same class:
			\bigskip
			\codeblockfull{Test.java}{
			class A \{ \\
			\s	\textbf{private int no;} \\
			\s 	public void display()\{ \\
			\s \s		no=100; \\
			\s \s		System.out.println(no); \\
			\s	\} \} \\
			public class Test \{ \\
			\s	public static void main(String[] args) \{  \\
			\s \s		A a1 = new A(); \\
			\s \s		a1.display(); \\
			\s	\} \}
			}
		\end{itemize}
	
		\item \textbf{protected attribute}:
		\begin{itemize}
			\item Accessible within the same package or subclass. 
			\item Accessed from subclasses even in different package.
			
			\item Eg 1: Accessing from within subclass: 
			\bigskip
			\codeblockfull{Test.java}{
				class A \{ \\
				\s	protected int no; \\
				\} \\
				public class Test extends A \{  \\
				\s	public static void main(String[] args) \{ \\
				\s \s		Test3 t1 = new Test3(); \\
				\s \s		t1.no = 120; \\
				\s \s		System.out.println(t1.no); \\
				\s	\} \}
			}
		\newpage
			\outputblock{
				120
			}
			
			\item Eg 2: Accessing from other package:
			\bigskip
			\codeblockfull{Test1.java}{
				package com.lavatech.www; \\
				public class Test1 \{  \\
				\s	protected int no; \\
				\}	
			}
			\bigskip
			\codeblockfull{Test2.java}{
				package com.lavatech.info; \\
				import com.lavatech.www.Test1; \\
				class Test2 extends Test1 \{  \\
				\s	public static void main(String[] args) \{ \\
				\s \s		Test2 test = new Test2(); \\
				\s \s		test.no = 150; \\
				\s \s		System.out.println(test.no); \\
				\s	\} \\
				\}
			}
			
			\bigskip
			\commandblock{
				\$ javac -d . Test1.java  \\
				\$ javac -d . Test2.java  \\
				\$ java com.lavatech.info.Test2  \\
				150
			}	
		\end{itemize}
	
		\bigskip
		\item \textbf{default (package-private):} 
		\begin{itemize}
			\item No access modifier specified is considered as the default attribute. 
			\item Can be accessed within the same package but not from other packages.
			\item Eg: Accessing from same class -
			\bigskip
			\codeblockfull{Test.java}{ 
				class A \{ \s	\textbf{int no;}  \} \\
				public class Test3 \{  \\
				\s public static void main(String[] args) \{  \\
				\s \s		A a1 = new A(); \\
				\s \s		System.out.println(a1.no); \\
				\s	\} \}	
			}
			\bigskip
			\outputblock{
			0
			}
		\end{itemize}
	
		\item \textbf{static attribute}:
		\begin{itemize}
			\item A static instance variable belongs to the class rather than an instance of the class. 
			\item It is shared among all instances of the class.
			\item Eg: Accessing varaible using class name -
			\bigskip
			\codeblockfull{Test.java}{
				class A \{ \\
				\s static int count; \\
				\} \\
				public class Test3 \{ \\
				\s	public static void main(String[] args) \{ \\
				\s \s		System.out.println(A.count); \\
				\s	\} \\
				\}
			}
			\bigskip
			\outputblock{
			0
			}
		\end{itemize}
		\newpage
		
		\item \textbf{final attribute}:
		\begin{itemize}
			\item Can only be assigned a value once, and its value cannot be changed thereafter. \item Onec value is assigned it becomes a constant and cannot be modified.			
			\item Key points:
			\begin{itemize}
				\item \textbf{Initialization:} Must be initialized when it is declared or within the \textbf{constructor} of the class.
				\item \textbf{Naming convention}: Final variable names are written in \textbf{uppercase} letters with underscores separating words (e.g., FINAL\_VARIABLE).
				\item \textbf{Primitive types:} For final variables of primitive types (e.g., int, double, boolean), the value assigned at initialization cannot be modified.
				\newline
				Eg: Assigning final variable value using constructor:
				\bigskip
				\codeblockfull{Test.java}{
					class A \{ \\
					\s final int COUNT; \\
					\s final int NO = 40; \\
					\s	public A(int count) \{ \\
					\s \s		this.COUNT = count; \\
					\s	\} \\
					\} \\
					public class Test \{ \\
					\s	public static void main(String[] args) \{ \\
					\s \s		A a1 = new A(50); \\
					\s \s		System.out.println(a1.COUNT); \\
					\s \s		System.out.println(a1.NO); \\
					\s	\} \\
					\}	
				}
				\bigskip
				\outputblock{
					50 \\
					40
				}
				\bigskip
				\item \textbf{Reference types:} For final variables that are references to objects, the reference itself cannot be changed, but the state of the object it refers to can be modified.
				\newline
				Eg:
				\bigskip
				\codeblockfull{Test.java}{
					class A \{ \\
					\s	final StringBuilder NAME = new StringBuilder("Raman"); \\
					\} \\
					public class Test \{ \\
					\s	public static void main(String[] args) \{ \\
					\s \s		A a1 = new A(); \\
					\s \s		a1.NAME.append(" Verma");  \\
					\s \s		System.out.println(a1.NAME); \\
					\s	\} \\
					\}	
				}
				\bigskip
				\outputblock{
					Raman verma
				}
				\bigskip
				\item \textbf{Final and static:} It is also possible to declare a final variable as static, making it a class-level constant accessible without creating an instance of the class.
				\newline
				Eg:
				\newpage
				\codeblockfull{Test.java}{
					class Constant \{ \\
					\s	public static final int MAX\_VALUE = 500; \\
					\} \\
					public class Test3 \{  \\
					\s	public static void main(String[] args) \{  \\
					\s \s		System.out.println(Constant.MAX\_VALUE); \\
					\s	\} \\
					\}	
				}
				\bigskip
				\outputblock{
					500
				}
			
				\bigskip
				
				\item \textbf{protected final:} Provides a read-only variable accessible within the same package or subclass.
				
				\bigskip
				\codeblockfull{Test.java}{
					class Constant \{ \\
					\s	protected final int MAX\_VALUE = 500; \\
					\} \\
					public class Test3 \{ \\
					\s	public static void main(String[] args) \{ \\
					\s \s		Constant c1 = new Constant(); \\
					\s \s		System.out.println(c1.MAX\_VALUE); \\
					\s	\} \\
					\} 
				}
				\bigskip
				\outputblock{
					500
				}
				
			\end{itemize}
			
		\end{itemize}
	
		\newpage
		\item \textbf{transient}: 
		\begin{itemize}
			\item Transient keyword is applicable only on attributes.
			\item Use transient keyword in serialisation context.
			\item At the time of serialisation if we dont want to save the value of a particular variable to meet security constraint then we should declare that variable as trainsent.
			\item At the time of serialsation JVM ignores original value of trainsent variables and save default value to the file. 
			\item Hence, trainsent means not to serialise.
			\item More on this is in serialisation chapter
		\end{itemize}
		\bigskip
		\item \textbf{volatile attributes}: 
		\begin{itemize}
			\item A volatile instance variable is used in multithreaded programs to ensure that changes to the variable are visible to all threads. 
			\item It guarantees that reads and writes to the variable are atomic and consistent.
			\item More on this in multi-threading chapter.		
		\end{itemize}
		
	\end{itemize}
	
	
	

	
	
\end{flushleft}

\newpage


\subsection{Methods in detail}
\setlength{\columnsep}{3pt}
\begin{flushleft}
	
	\begin{itemize}
		\item Defines the \textbf{behavior or actions} that objects.
		\item Declared within a class and can access class attributes.
		\item Invoked using object of the class.
		\bigskip
		\syntaxblock{
			modifier returnType methodName(type1 arg1, type2 arg2, ...)
		}
		
		where,
		\begin{itemize}
			\item \textbf{access modifier} can be public, private, protected, default, static, final, abstract, synchronized, native or strictfp.
			\item \textbf{returnType:} Specifies the type of value returned. Can be a \textbf{primitive type}, an \textbf{object type}, or \textbf{void} if the method does not return any value.
			\item \textbf{methodName:} Method name that follow the Java naming conventions.
			\item \textbf{type:} Data type of parameter passed to the method.
			\item arg: Name given to each parameter.
		\end{itemize}
		
	\end{itemize}

	Let's see each part of method syntax in detail.
	
	\newpage
	
	\textbf{Access modifiers applicable on methods}
	\begin{itemize}
		\item \textbf{public:}
		\begin{itemize}
			\item Can be accessed from any other class or package.
			\item Eg:
			\bigskip
			\codeblockfull{Test.java}{
			class A \{ \\
			\s 		public void message() \{ \\
			\s \s			System.out.println("Time is money"); \\
			\s		\} \\
			\} \\
			public class Test \{ \\
			\s public static void main(String[] args) \{ \\
			\s \s			A a1 = new A(); \\
			\s \s			a1.message(); \\
			\s 		\} \\
				\}	
			}
			\bigskip
			\outputblock{
				Time is money
			}
		\end{itemize}
	
		\bigskip
		\item \textbf{private}:
		\begin{itemize}
			\item Can only be accessed within the same class. 
			\item It is not visible to other classes or packages.
			\item Eg:
			\bigskip
			\codeblockfull{Test.java}{
				class A \{ \\
				\s 	String msg; \\
				\s	\textbf{private void set()} \{ \\
				\s \s		\textbf{msg  = "Time is money";} \\
				\s	\textbf{\}} 
			}
			\newpage		
			\codecontinue{
				\s	public void get() \{ \\
				\s \s		\textbf{set();} \\
				\s \s		System.out.println(msg); \\
				\s	\} \\
				\} \\
				public class Test3 \{ \\
				\s public static void main(String[] args) \{ \\
				\s \s		A a1 = new A(); \\
				\s \s		a1.get(); \\
				\s	\} \\
				\}	
			}
			\bigskip
			
			\outputblock{
			Time is money
			}
		\end{itemize}
		\bigskip
		\textbf{protected}:
		\begin{itemize}
			\item Accessible within the same package or subclass. 
			\item It can be accessed from subclasses even if they are in a different package.
			
			\item Eg 1: Accessing from within subclass: 
			\bigskip
			\codeblockfull{Test.java}{
				class A \{ \\
				\s	protected void msg() \{ \\
				\s \s		System.out.println("Time is money"); \\
				\s	\} \\
				\s	public static void main(String[] args) \{ \\
				\s \s		A a1 = new A(); \\
				\s \s		a1.msg(); \\
				\s	\} \\
				\}
			}
			\newpage
			\outputblock{
				Time is money
			}
			
			
		\end{itemize}
		\bigskip
		\item \textbf{default (package-private):} 
		\begin{itemize}
			\item No access modifier specified is considered as the default attribute. 
			\item Can be accessed within the same package but not from other packages.
			\item Eg: Accessing from same class -
			\bigskip
			\codeblockfull{Test.java}{ 
				class A \{ \\
				\s \textbf{void msg()} \{ \\
				\s \s		System.out.println("Time is money"); \\
				\s	\} \\
				\} \\
				public class Test3 \{ \\
				\s	public static void main(String[] args) \{ \\
				\s \s		A a1 = new A(); \\
				\s \s		a1.msg(); \\
				\s	\} \\
				\}
			}
			\bigskip
			\outputblock{
				Time is money
			}
		\end{itemize}
		
		\bigskip
		\item \textbf{static}:
		\begin{itemize}
			\item A static method belongs to the class rather than an instance of the class. 
			\item  Can be called directly using the class name without creating an object of the class.
			\item Eg: Accessing method using class name -
			\bigskip
			\codeblockfull{Test.java}{
				class A \{ \\
				\s	static void msg() \{ \\
				\s \s		System.out.println("Time is money"); \\
				\s	\} \\
				\} \\
				public class Test3 \{ \\
				\s	public static void main(String[] args) \{ \\
				\s \s		A.msg(); \\
				\s	\}
				\}
			}
			\bigskip
			\outputblock{
				Time is money
			}
		\end{itemize}
		
		\bigskip
		\item \textbf{final}:
		\begin{itemize}
			\item Cannot be overridden by subclasses. 
			\item It provides the implementation that cannot be changed.
			\item Eg:
			\bigskip
			\codeblockfull{Test3.java}{
				class Parent \{ \\
				\s \textbf{public final void display()} \{ \\
				\s	System.out.println("Parent class final method"); \\
				\s	\} \\
				\}  \\
				class Child extends Parent \{ \\
				\s	// public void display() \{ \}   \\
				\} \\
				class FinalMethodExample \{ \\
				\s	public static void main(String[] args) \{ 
				
			}
			\newpage
			\codecontinue{
				\s \s		Parent parent = new Parent();  \\
				\s \s		parent.display();  \\
				\s \s		Child child = new Child(); \\
				\s \s		child.display();  \\
				\s	\} \\
				\}
			}
			\bigskip
			\outputblock{
				Parent class final method \\
				Parent class final method
			}
		\end{itemize}
	
		\bigskip
		\item \textbf{abstract}:
		\begin{itemize}
			\item An abstract method does not have an implementation and must be overridden by any concrete subclass.
			\item More on this in abstract class chapter
		\end{itemize}
		\bigskip
		\item \textbf{synchronized:}
		\begin{itemize}
			\item A synchronized method can be accessed by only one thread at a time, ensuring thread safety.
			\item More on this in threading chapter
		\end{itemize}
		\bigskip
		\item \textbf{native:} A native method is implemented in a language other than Java, typically using JNI (Java Native Interface). It provides a bridge between Java and other languages like C or C++.
		\bigskip
		\item \textbf{strictfp:} 
		\begin{itemize}
			\item Enforces strict floating-point precision for floating-point calculations. 
			\item It ensures consistent results across different platforms.
			\item Eg:
			\bigskip
			\codeblockfull{A.java}{
			strictfp class A \{ \\
			\s	public strictfp double cal(double a, double b) \{ \\
			\s \s		return a * b / Math.sqrt(a + b); \\
			\s	\}
				\\
			\s	public static void main(String[] args) \{ \\
			\s \s		A example = new A(); \\
			\s \s		double result = example.cal(10.5, 5.3); \\
			\s \s		System.out.println("Result: " + result); \\
			\s	\}
			}
			\bigskip
			\outputblock{
				Result: 14.000276895995912
			}
			
		\end{itemize}
	
	\end{itemize}
	
	\newpage
	
	\textbf{Method argument types}
	\begin{itemize}
		\item Method arguments specify the types of values that can be passed to a method when it is invoked. 
		\item Method arguments define the parameters that a method expects to receive in order to perform its functionality. 
		i\item Here are some common types of method arguments in Java:
		
		\begin{itemize}
			\item \textbf{Primitive types}: Includes int, double, long, short, byte, char, float, 
			boolean.

			\codeblock{
				class Person \{ \\
				\s	String name; \\
				\s	int age; \\
				\s	long phno; \\
				\s	public void setDetails(String n, int a, long p) \{ \\
				\s \s		name = n; \\
				\s \s		age = a; \\
				\s \s		phno = p; \\
				\s	\} \\
				\s	public static void main(String[] args) \{ \\
				\s \s		Person p = new Person(); \\
				\s \s		p.setDetails("Ravi",23,987123546L); \\
				\s \s		System.out.println("Name : " + p.name); \\
				\s \s		System.out.println("Age : " + p.age); \\
				\s 	System.out.println("Phone number : " + p.phno); \\
				\s	\} \\
				\}
			}
		\bigskip
			\outputblock{
			Name : Ravi \\
			Age : 23 \\
			Phone number : 987123546
			}
			\bigskip
			\item \textbf{Reference types}: Refer to objects or classes or arrays or interfaces or enums. Eg:
			\bigskip
			\codeblock{
				class Person \{ \\
				\s	String name="Raman"; \\
				\s	int age=56; \\
				\s	long phno=785642312L; \\
				\} \\
				class Test \{ \\
				\s	public static void displayPerson(Person person)\{ \\ 
				\s \s		System.out.println(person.name); \\
				\s \s		System.out.println(person.age); \\
				\s \s		System.out.println(person.phno); \\
				\s	\} \\
				\\
				\s	public static void main(String[] args) \{ \\
				\s \s		Person p = new Person(); \\
				\s \s		displayPerson(p); \\
				\s	\} \\
				\}
			}
			\bigskip
			\outputblock{
				Raman \\
				56 \\
				785642312
			}
			\bigskip
			\item \textbf{Varargs}: 
			\begin{itemize}
				\item Allows a method to accept a variable number of arguments of the same type. \item It is denoted by an ellipsis (...) after the parameter type.
				\item Eg:
				\newpage
				\codeblockfull{Test.java}{
					class Test \{ \\
					\s \textbf{public static void nos(int... numbers)} \{ \\
							\textbf{System.out.println("Arguments: " + numbers.length);}  \\
					\s \s		System.out.print("Numbers: "); \\
					\s \s		for (int number : numbers) \{ \\
					\s \s	\s		System.out.print(number + " "); \\
					\s	\s	\} \\
					\s	\s	System.out.println(); \\
					\s	\} \\
						\\
						public static void main(String[] args) \{ \\
					\s \s		nos(1, 2, 3); \\
					\s \s		nos(10, 20, 30, 40, 50);  \\
					\s \s		nos(); \\
					\s	\} \\
					\}
				}
				\bigskip
				\outputblock{
					Arguments: 3 \\
					Numbers: 1 2 3  \\
					Arguments: 5 \\
					Numbers: 10 20 30 40 50  \\
					Arguments: 0 \\
					Numbers: 	
				}
			\end{itemize}
			
			
		\end{itemize}
	\end{itemize}

	\newpage
	
	\textbf{Method return types}
	\begin{itemize}
		\item The return type of a method specifies the type of value that the method will return when it is executed. 
		\item The return type is declared in the method syntax, immediately before the method name. 
		\item Here are some common return types in Java:
		\begin{itemize}
			\item \textbf{Primitive types}: Includes int, double, long, short, byte, char, float, 
			boolean.
			
			\codeblock{
				class Person \{ \\
				\s	String name; \\
				\s	int age; \\
				\s	long phno; \\
				\s	public void setDetails(String n, int a, long p) \{ \\
				\s \s		name = n; \\
				\s \s		age = a; \\
				\s \s		phno = p; \\
				\s	\} \\
				\s	public static void main(String[] args) \{ \\
				\s \s		Person p = new Person(); \\
				\s \s		p.setDetails("Ravi",23,987123546L); \\
				\s \s		System.out.println("Name : " + p.name); \\
				\s \s		System.out.println("Age : " + p.age); \\
				\s 	System.out.println("Phone number : " + p.phno); \\
				\s	\} \\
				\}
			}
			\bigskip
			\outputblock{
				Name : Ravi \\
				Age : 23 \\
				Phone number : 987123546
			}
			\bigskip
			\item \textbf{Reference types}: Refer to objects or classes or arrays (int[] or String[]) or interfaces or enums. Eg:
			\bigskip
			\codeblock{
				class Person \{ \\
				\s	String name; \\
				\s	int age; \\
				\s	long phno; \\
				\}
				class Test \{ \\
				\s	public static Person setPerson(Person person)\{  \\
				\s \s		person.name="Raman"; \\
				\s \s		person.age=67; \\
				\s \s		person.phno=785642312L; \\
				\s \s		return person; \\
				\s	\}
				\s	public static void main(String[] args) \{ \\
				\s \s		Person p; \\
				\s \s		p = setPerson(new Person()); \\
				\s \s		System.out.println("Name:"+p.name); \\
				\s \s		System.out.println("Age:"+p.age); \\
				\s \s		System.out.println("Phone no:"+p.phno); \\
				\s	\} \\
				\}
			}
			\bigskip
			\outputblock{
				Name:Raman \\
				Age:67 \\
				Phone no:785642312
			}
			\bigskip
			
		\end{itemize}
		
		\newpage
		\item \textbf{void}:
		\begin{itemize}
			\item Indicates that the method does not return any value. 
			\item Eg:
			\bigskip
			\codeblockfull{Test.java}{
				class Person \{ \\
				\s	String name="Raman"; \\
				\s	int age=56; \\
				\s	long phno=785642312L; \\
				\} \\
				class Test \{ \\
				\s	public static void displayPerson(Person person)\{ \\
				\s \s		System.out.println(person.name); \\
				\s \s		System.out.println(person.age); \\
				\s \s		System.out.println(person.phno); \\
				\s	\} \\
				\\
				\s	public static void main(String[] args) \{ \\
				\s \s		Person p = new Person(); \\
				\s \s		displayPerson(p); \\
				\s	\} \\
				\}
			}
		
			\bigskip
			\outputblock{
				Raman \\
				56 \\
				785642312
			}
		\end{itemize}
		
	\end{itemize}	
	
\end{flushleft}

\newpage


\subsection{Class level access modifier}
\setlength{\columnsep}{3pt}
\begin{flushleft}
	
	\begin{itemize}
		\item Using access modifier, class can provide more information to the JVM like:
		\begin{itemize}
			\item Whether the class is accessible from anywhere or not
			\item Whether child class creation is possible or not
			\item Whether object creation is possible or not
		\end{itemize}
		\item The only applicable modifiers for top-level classes are:
		\begin{itemize}
			\item public
			\item default
			\item final
			\item abstract
			\item strictfp
		\end{itemize}
		\item But, for inner classes, the applicable modifiers are:
		\begin{itemize}
			\item private
			\item protected 
			\item static
		\end{itemize}
	
	\end{itemize}

	\newimage{0.4}{content/chapter11/images/class.png}
	
	\noteblock{
	In Java, there are only access modifiers, there is no word like access specifier.
	
	}
	
	Let's see these class level access modifiers in detail.
	
	\newpage
	
	\textbf{public classes}:
	\begin{itemize}
		\item  If a class is declared as public then we can access that class from anywhere.
		\item Eg:
		\bigskip
		\codeblockfull{Test1.java}{
			package com.lavatech.www; \\
			public class Test1 \{  \\
			\s	public void message()\{ \\
			\s \s		System.out.println("Hard work pays off!"); \\
			\s	\} \\
			\}
		}
		\bigskip
		\codeblockfull{Test2.java}{
			package com.lavatech.info; \\
			import com.lavatech.www.Test1; \\
			public class Test2 \{ \\
			\s public static void main(String[] args) \{ \\
			\s \s		Test1 t1 = new Test1(); \\
			\s \s		t1.message(); \\
			\s	\} \\
			\}	
		}
		\bigskip
		\commandblock{
		\$ javac -d . Test1.java  \\
		\$ javac -d . Test2.java  \\
		\$ java com.lavatech.info.Test2  \\
		Hard work pays off!
		}
	\end{itemize}
	
	\newpage
	\textbf{default classes}:
	
	\begin{itemize}
		\item A default class \textbf{does not have an access modifier} specified. 
		\item Also known as a \textbf{package-private class} 
		\item It is accessible \textbf{only within the same package}.
		\item If no access modifier (such as public, private, or protected) is specified, the class is considered to have default access.
		\item Eg:
		\bigskip
		\codeblockfull{Test1.java}{
			package com.lavatech.www; \\
			class Test1 \{ \\
			\s public void message()\{ \\
			\s \s		System.out.println("Hard work pays off!"); \\
			\s	\} \\
			\}
		}
		\bigskip
		\codeblockfull{Test2.java}{
			package com.lavatech.info; \\
			import com.lavatech.www.Test1;
		}
		\bigskip
		\commandblock{
			\$ javac -d . Test2.java  \\
			\color{red}
			Test2.java:2: error: Test1 is not public in com.lavatech.www; cannot be accessed from outside package \\
			import com.lavatech.www.Test1; \\
			1 error 
		}
	\end{itemize}

	\newpage
	\textbf{final class}
	\begin{itemize}
		\item Final class can’t be inherited 
		\item You can’t create child class for final class.
		\item More on final class is mentioned in inheritance section.
	\end{itemize}

	\textbf{abstract class}
	\begin{itemize}
		\item Abstract classes can be instantiated.
		\item You cannot create object of abstract class.
		\item More on abstract class is mentioned in Abstract class section.
	\end{itemize}

	\textbf{strictfp class}
	\begin{itemize}
		\item strictfp was introduced in Java1.2 version.
		\item Strictfp class ensures all methods in the class \& its subclasses, adhere to strict floating-point precision rules.
		\item strictfp is used to ensure that floating points operations give the same result on any platform. 
		\item As floating points precision may vary from one platform to another. strictfp keyword ensures the consistency across the platforms.
		\item Eg:
		\codeblockfull{Test.java}{
		\textbf{strictfp class A} \{ \\
		\s	double num1 = 10e+102; \\
		\s	double num2 = 6e+08; \\
		\s	double calculate() \{ \\
		\s \s		return num1 + num2; \\
		\s	\} \} \\
		public class Test \{  \\
		\s	public static void main(String[] args) \{  \\
		\s \s		A a1 = new A();      \\
		\s \s		System.out.println("Result: " + a1.calculate()); \\
		\s	\} \}
		}
		\newpage
		
		\bigskip
		\outputblock{
			Result: 1.0E103	
		}
		
	\end{itemize}
	
\end{flushleft}

\newpage

\section{Constructors and garbage collection}
\setlength{\columnsep}{3pt}
\begin{flushleft}
	\bigskip
	\bigskip
	\begin{tcolorbox}[breakable,notitle,boxrule=1pt,colback=black,colframe=black]
		\color{white}
		\bigskip
		In this section, you are going to learn:
		\begin{enumerate}
			\item \textbf{What is YUM \& DNF?}
			\item \textbf{Advantages of YUM over RPM}
			\item \textbf{Configuration of YUM server and client}
			\item \textbf{Subscribing RHEL8 server}
			\item \textbf{YUM commands}
			\item \textbf{DNF commands}
		\end{enumerate}	
		\bigskip
		Finally, there will be a \textbf{small excerise} on these topics to check your knowledge.
		\bigskip
	\end{tcolorbox}
	
	
	\begin{multicols}{2}
		\vspace*{\fill}
		\vspace*{\fill}
		\vspace*{\fill}
		\vspace*{\fill}
		\vspace*{\fill}
		\vspace*{\fill}
		\vspace*{\fill}
		\vspace*{\fill}
		\vspace*{\fill}
		
		\vfill \null
		\columnbreak
		So let's get started....
		\includegraphics[scale=0.08]{content/linux_section.png}
	\end{multicols}	
	
\end{flushleft}

\newpage


\subsection{new operator}
\setlength{\columnsep}{3pt}
\begin{flushleft}
	\bigskip
	\paragraph{What is YUM?}
	\begin{itemize}
		\item \textbf{YUM} stands for \textbf{Yellow dog Updater, Modified}.
		\item YUM is a command-line as well as graphical-based package management tool for RPM.
		\item YUM allows to install, update, remove \& search packages easily.
	\end{itemize}

	\paragraph{What is DNF?}
	\begin{itemize}
		\item The DNF command (Dandified yum) is the next version of the YUM.
		\item It is the default package manager for Fedora 22, CentOS8, and RHEL8.
		\item It is intended to be a replacement for YUM.
	\end{itemize}

	\paragraph{Advantages of YUM over RPM}
	\begin{enumerate}
		\item Dependency resolution:
		\begin{itemize}
			\item If a package installation requires additional packages, YUM can list these dependencies and prompt the user to install them.
		\end{itemize}
		\item Locate package from multiple locations \& install them: 
		\begin{itemize}
			\item YUM can be configured to search for software packages in more than one location.
		\end{itemize}
		\item YUM allows to specify particular software versions or architectures while installing or removing the package.
		
	\end{enumerate}
\end{flushleft}
\newpage




\subsection{Constructor}
\setlength{\columnsep}{3pt}
\begin{flushleft}
	\bigskip
	You can create a private software repository that can be used everytime you want to install a package by configuring \textbf{yum server}.
	
	Below are the steps to configure a \textbf{yum server} using \textbf{FTP (file transfer protocol)}:
	
	\begin{enumerate}
		\item First attach the RHEL ISO image to the virtual machine(VM) while the machine is up and running.
		
		\begin{figure}[h!]
			\centering
			\includegraphics[scale=.3]{content/chapter11/images/ISO.png}
			\caption{Upload RHEL7 ISO}
			\label{fig:iso}
		\end{figure}	
		
		\item Unmount \textbf{/dev/sr0} \& mount the iso image attach on device \textbf{/dev/sr0} to \textbf{"/mnt"} folder.
		\begin{tcolorbox}[breakable,notitle,boxrule=-0pt,colback=black,colframe=black]
			\color{green}
			\fontdimen2\font=9pt
			\# umount	/dev/sr0
			\newline
			\# mount	/dev/sr0	/mnt
			\fontdimen2\font=4pt
		\end{tcolorbox}

		\bigskip
		\item Install following packages from attached ISO image.
		\begin{tcolorbox}[breakable,notitle,boxrule=-0pt,colback=black,colframe=black]
			\color{green}
			\fontdimen2\font=9pt
			\# rpm	-ivh	/mnt/Packages/ftp-xxxx
			\newline
			\# rpm	-ivh 	/mnt/Packages/vsftpd-xxxx
			\newline
			\# rpm	-ivh	/mnt/Packages/createrepo-xxxx
			\fontdimen2\font=4pt
		\end{tcolorbox}
		
		\bigskip
		\item Copy all packages from attached ISO image to pub directory:
		\begin{tcolorbox}[breakable,notitle,boxrule=-0pt,colback=black,colframe=black]
			\color{green}
			\fontdimen2\font=9pt
			\# cp	-rv	/mnt/Packages/*	/var/ftp/pub
			\fontdimen2\font=4pt
		\end{tcolorbox}
		
		\bigskip
		\item Create a repo in pub directory.
		\begin{tcolorbox}[breakable,notitle,boxrule=-0pt,colback=black,colframe=black]
			\color{green}
			\fontdimen2\font=9pt
			\# createrepo	-v	/var/ftp/pub/
			\fontdimen2\font=4pt
		\end{tcolorbox}

		\bigskip
		\item Start and enable the vsftpd service.
		\begin{tcolorbox}[breakable,notitle,boxrule=-0pt,colback=black,colframe=black]
			\color{green}
			\fontdimen2\font=9pt
			\# systemctl   restart   vsftpd
			\newline
			\# systemctl   enable    vsftpd
			\fontdimen2\font=4pt
		\end{tcolorbox}

		\bigskip
		\item Browse the pub directory using browser(like firefox or google chrome) with URL \textbf{ftp://(server IP address)/pub}.
		\newline
		Eg: "ftp://192.168.10.111/pub"

		\bigskip
		\item Create a repository file named \textbf{/etc/yum.repos.d/new.repo}:
		\begin{tcolorbox}[breakable,notitle,boxrule=-0pt,colback=black,colframe=black]
			\color{white}
			\fontdimen2\font=9pt
			[rhcsarepo] \color{yellow}  \# Add title name for this repository 
			\newline
			\color{white}
			name = RHCSA\_REPO
			\color{yellow}
			  \# Add repository name
			\newline
			\color{white}
			enabled = 1
			\newline
			gpgcheck = 0
			\newline
			baseurl = ftp://(server-IP-address)/pub
			\fontdimen2\font=4pt
		\end{tcolorbox}
		\bigskip
		Let's see the parameters of this file in detail.
		
		\newpage
		File explaination:
		
		
		\begin{tabulary}{1.0\textwidth}{|p{10em}|p{14em}|}
			\toprule
			\textbf{Linux OS} & \textbf{Windows OS}\\
			\midrule
			name = RHCSA\_REPO & Set the name of repository \\
			\hline
			enabled = 1 & Defines the state of repository. If value is set to 1 then repository is enabled. If value is set to 0 then repository is disabled. \\
			\hline
			gpgcheck = 0 & Defines whether the integrity of package should be check or not. If value is set to 1, integrity will be checked. If value is set to 0, integrity will not be checked.\\
			\hline
			baseurl = ftp://(server-IP-address)/pub & Defines the location of rpm files \\
			\bottomrule
		\end{tabulary}
		
		
		

		\begin{tcolorbox}[breakable,notitle,boxrule=-0pt,colback=yellow,colframe=yellow]
			\color{black}
			\textbf{Note:} 
			\begin{itemize}
				\item Location of repository file should be \textbf{/etc/yum.repos.d}
				\item Name of the file can be anything
				\item Extension of the file of should be \textbf{".repo"}
			\end{itemize}
		\end{tcolorbox}
		
		


		\item Clean all the cached files from any enabled repository.
		\begin{tcolorbox}[breakable,notitle,boxrule=-0pt,colback=black,colframe=black]
			\color{green}
			\fontdimen2\font=9pt
			\# yum	clean	all
			\fontdimen2\font=4pt
		\end{tcolorbox}
		


		\item Refresh all the enabled repository:
		\begin{tcolorbox}[breakable,notitle,boxrule=-0pt,colback=black,colframe=black]
			\color{green}
			\fontdimen2\font=9pt
			\# yum 	update 	all
			\fontdimen2\font=4pt
		\end{tcolorbox}



		\item Now you can install any packages using the YUM server.
		\newline
		Eg:
		\begin{tcolorbox}[breakable,notitle,boxrule=-0pt,colback=black,colframe=black]
			\color{green}
			\fontdimen2\font=9pt
			\# yum	install	httpd -y		
			\fontdimen2\font=4pt
		\end{tcolorbox}
		

		
		
		
	\end{enumerate}

	

\end{flushleft}
\newpage




\subsection{this keyword}
\setlength{\columnsep}{3pt}
\begin{flushleft}
	\bigskip
	You can start using yum server for package installation on any client, by following below steps:
	
	\begin{enumerate}
		\item Create a repository file named \textbf{/etc/yum.repos.d/new.repo}:
		\begin{tcolorbox}[breakable,notitle,boxrule=-0pt,colback=black,colframe=black]
			\color{white}
			\fontdimen2\font=9pt
			[Title\_name] \color{yellow}  \# Add title name for this repository 
			\newline
			\color{white}
			name = Repo\_name
			\color{yellow}
			\# Add repository name
			\newline
			\color{white}
			enabled = 1
			\newline
			gpgcheck = 0
			\newline
			baseurl = ftp://(server-IP-address)/pub
			\fontdimen2\font=4pt
		\end{tcolorbox}
		\bigskip
		\begin{tcolorbox}[breakable,notitle,boxrule=-0pt,colback=yellow,colframe=yellow]
			\color{black}
			\textbf{Note:} 
			\begin{itemize}
				\item Location of repository file should be \textbf{/etc/yum.repos.d}
				\item Name of the file can be anything
				\item Extension of the file of should be \textbf{".repo"}
			\end{itemize}
		\end{tcolorbox}

		
		
		\item Clean all the cached files from any enabled repository.
		\begin{tcolorbox}[breakable,notitle,boxrule=-0pt,colback=black,colframe=black]
			\color{green}
			\fontdimen2\font=9pt
			\# yum	clean	all
			\fontdimen2\font=4pt
		\end{tcolorbox}
		
		
		\item Refresh all the enabled repository:
		\begin{tcolorbox}[breakable,notitle,boxrule=-0pt,colback=black,colframe=black]
			\color{green}
			\fontdimen2\font=9pt
			\# yum 	update 	all
			\fontdimen2\font=4pt
		\end{tcolorbox}
		
		
		\item Now you can install any packages using the YUM server.
		\newline
		Eg:
		\begin{tcolorbox}[breakable,notitle,boxrule=-0pt,colback=black,colframe=black]
			\color{green}
			\fontdimen2\font=9pt
			\# yum	install	httpd -y		
			\fontdimen2\font=4pt
		\end{tcolorbox}

		
	\end{enumerate}
	
	
	
\end{flushleft}
\newpage




\subsection{Constructor Chaining.}
\setlength{\columnsep}{3pt}
\begin{flushleft}
	\bigskip
	
	\textbf{yum}: Used to install, update, remove \& search package.
	\begin{tcolorbox}[breakable,notitle,boxrule=-0pt,colback=pink,colframe=pink]
		\color{black}
		\fontdimen2\font=9pt
		Syntax: yum [options] command [package\_name]
		\fontdimen2\font=4pt
	\end{tcolorbox}
	Command and options with \textbf{yum} command:
	
	\begin{itemize}
		\item \textbf{install}: Install package along with it's dependency.
		\newline
		\textbf{-y}: Automatically answer yes for all questions
		\begin{tcolorbox}[breakable,notitle,boxrule=-0pt,colback=pink,colframe=pink]
			\color{black}
			\fontdimen2\font=9pt
			Syntax: yum install package\_name -y
			\fontdimen2\font=4pt
		\end{tcolorbox}
		Eg:
		\begin{tcolorbox}[breakable,notitle,boxrule=-0pt,colback=black,colframe=black]
			\color{green}
			\fontdimen2\font=9pt
			\# yum install httpd -y
			\fontdimen2\font=4pt
		\end{tcolorbox}
		\bigskip
		\bigskip
		\item \textbf{remove}: Uninstall a package along with this dependency.
		\begin{tcolorbox}[breakable,notitle,boxrule=-0pt,colback=pink,colframe=pink]
			\color{black}
			\fontdimen2\font=9pt
			Syntax: yum remove package\_name 
			\fontdimen2\font=4pt
		\end{tcolorbox}
		Eg:
		\begin{tcolorbox}[breakable,notitle,boxrule=-0pt,colback=black,colframe=black]
			\color{green}
			\fontdimen2\font=9pt
			\# yum remove httpd
			\fontdimen2\font=4pt
		\end{tcolorbox}
		\bigskip
		\bigskip		
		
		\item \textbf{info}: Display the information of a package.
		\begin{tcolorbox}[breakable,notitle,boxrule=-0pt,colback=pink,colframe=pink]
			\color{black}
			\fontdimen2\font=9pt
			Syntax: yum info package\_name 
			\fontdimen2\font=4pt
		\end{tcolorbox}
		Eg:
		\begin{tcolorbox}[breakable,notitle,boxrule=-0pt,colback=black,colframe=black]
			\color{green}
			\fontdimen2\font=9pt
			\# yum info httpd
			\fontdimen2\font=4pt
		\end{tcolorbox}
		\bigskip
		\bigskip		

		\item \textbf{groupinfo}: List all packages provided by the group.
		\begin{tcolorbox}[breakable,notitle,boxrule=-0pt,colback=pink,colframe=pink]
			\color{black}
			\fontdimen2\font=9pt
			Syntax: yum groupinfo group\_name 
			\fontdimen2\font=4pt
		\end{tcolorbox}
		Eg:
		\begin{tcolorbox}[breakable,notitle,boxrule=-0pt,colback=black,colframe=black]
			\color{green}
			\fontdimen2\font=9pt
			\# yum groupinfo "Development Tools"
			\fontdimen2\font=4pt
		\end{tcolorbox}
		\bigskip
		\bigskip	

		\item \textbf{groupinstall}: Install all packages provided by the group.
		\begin{tcolorbox}[breakable,notitle,boxrule=-0pt,colback=pink,colframe=pink]
			\color{black}
			\fontdimen2\font=9pt
			Syntax: yum groupinstall group\_name 
			\fontdimen2\font=4pt
		\end{tcolorbox}
		Eg:
		\begin{tcolorbox}[breakable,notitle,boxrule=-0pt,colback=black,colframe=black]
			\color{green}
			\fontdimen2\font=9pt
			\# yum groupinstall "Development Tools"
			\fontdimen2\font=4pt
		\end{tcolorbox}
		\bigskip
		\bigskip	

		\item \textbf{groupremove}: Remove all packages provided by the group.
		\begin{tcolorbox}[breakable,notitle,boxrule=-0pt,colback=pink,colframe=pink]
			\color{black}
			\fontdimen2\font=9pt
			Syntax: yum groupremove group\_name 
			\fontdimen2\font=4pt
		\end{tcolorbox}
		Eg:
		\begin{tcolorbox}[breakable,notitle,boxrule=-0pt,colback=black,colframe=black]
			\color{green}
			\fontdimen2\font=9pt
			\# yum groupremove "Development Tools"
			\fontdimen2\font=4pt
		\end{tcolorbox}
		\bigskip
		\bigskip	


		\item \textbf{update}: Update all packages and their dependencies.
		\begin{tcolorbox}[breakable,notitle,boxrule=-0pt,colback=pink,colframe=pink]
			\color{black}
			\fontdimen2\font=9pt
			Syntax: yum update
			\fontdimen2\font=4pt
		\end{tcolorbox}
		\bigskip
		\bigskip	

		\item \textbf{history}: Display a list of all the latest yum transactions.
		\begin{tcolorbox}[breakable,notitle,boxrule=-0pt,colback=pink,colframe=pink]
			\color{black}
			\fontdimen2\font=9pt
			Syntax: yum history
			\fontdimen2\font=4pt
		\end{tcolorbox}
		\bigskip
		\bigskip			

		\item \textbf{history info}: Examine a particular yum transaction.
		\begin{tcolorbox}[breakable,notitle,boxrule=-0pt,colback=pink,colframe=pink]
			\color{black}
			\fontdimen2\font=9pt
			Syntax: yum history info transactionID
			\fontdimen2\font=4pt
		\end{tcolorbox}
		Eg:
		\begin{tcolorbox}[breakable,notitle,boxrule=-0pt,colback=black,colframe=black]
			\color{green}
			\fontdimen2\font=9pt
			\# yum history info 8
			\fontdimen2\font=4pt
		\end{tcolorbox}
		\bigskip
		\bigskip			

		\item \textbf{history undo}: Revert a particular transaction.
		\begin{tcolorbox}[breakable,notitle,boxrule=-0pt,colback=pink,colframe=pink]
			\color{black}
			\fontdimen2\font=9pt
			Syntax: yum history undo transactionID
			\fontdimen2\font=4pt
		\end{tcolorbox}
		Eg:
		\begin{tcolorbox}[breakable,notitle,boxrule=-0pt,colback=black,colframe=black]
			\color{green}
			\fontdimen2\font=9pt
			\# yum history undo 8
			\fontdimen2\font=4pt
		\end{tcolorbox}
		\bigskip
		\bigskip			

		\item \textbf{history redo}: Repeat a particular transaction.
		\begin{tcolorbox}[breakable,notitle,boxrule=-0pt,colback=pink,colframe=pink]
			\color{black}
			\fontdimen2\font=9pt
			Syntax: yum history redo transactionID
			\fontdimen2\font=4pt
		\end{tcolorbox}
		Eg:
		\begin{tcolorbox}[breakable,notitle,boxrule=-0pt,colback=black,colframe=black]
			\color{green}
			\fontdimen2\font=9pt
			\# yum history redo 8
			\fontdimen2\font=4pt
		\end{tcolorbox}
		\bigskip
		\bigskip					
		
		\item \textbf{list all}: List all installed packages.
		\begin{tcolorbox}[breakable,notitle,boxrule=-0pt,colback=pink,colframe=pink]
			\color{black}
			\fontdimen2\font=9pt
			Syntax: yum list all
			\fontdimen2\font=4pt
		\end{tcolorbox}
		Eg: List all installed pacakges associated with httpd.
		\begin{tcolorbox}[breakable,notitle,boxrule=-0pt,colback=black,colframe=black]
			\color{green}
			\fontdimen2\font=9pt
			\# yum list all | grep httpd
			\fontdimen2\font=4pt
		\end{tcolorbox}
		\bigskip
		\bigskip					

		\item \textbf{localinstall}: Install locally downloaded package.
		\begin{tcolorbox}[breakable,notitle,boxrule=-0pt,colback=pink,colframe=pink]
			\color{black}
			\fontdimen2\font=9pt
			Syntax: yum localinstall package.rpm
			\fontdimen2\font=4pt
		\end{tcolorbox}
		Eg: Download cvs package using command \textbf{"wget https://rpmfind.net/linux/epel/8/Everything/x86\_64/Packages/c/cvs-1.11.23-52.el8.x86\_64.rpm"}
		\newline
		 Install downloaded package using command:
		\begin{tcolorbox}[breakable,notitle,boxrule=-0pt,colback=black,colframe=black]
			\color{green}
			\fontdimen2\font=9pt
			\# yum localinstall cvs-1.11.23-52.el8.x86\_64.rpm -y
			\fontdimen2\font=4pt
		\end{tcolorbox}
		\bigskip
		\bigskip					

		
	\end{itemize}
	
	
\end{flushleft}
\newpage




\subsection{super()}
\setlength{\columnsep}{3pt}
\begin{flushleft}
	
	\paragraph{}
	\bigskip
	
	\begin{figure}[h!]
		\centering
		\includegraphics[scale=.2]{content/practise.jpg}
	\end{figure}	
	\begin{enumerate}
		
		\item \textbf{Which of the following statement is true about yum \& rpm command? (Select all that applies.)}
		\begin{enumerate}[label=(\alph*)]
			\item The \textbf{rpm} command can install a package, but cannot resolve the dependency issue of the package. %correct
			\item The \textbf{yum} command can install a package while resolving its dependency issue. %correct
			\item The \textbf{yum} command cannot resolve the dependency issue of a package.
			\item The \textbf{rpm} command can resolve the dependency issue of a package.
		\end{enumerate}
		\bigskip
		\bigskip
		
		\item \textbf{Which of the following is the correct location for yum ".repo" configuration?}
		\begin{enumerate}[label=(\alph*)]
			\item /etc/yum/repos.d
			\item /etc/yum/
			\item /etc/yum.repos.d/  %correct
			\item /etc/repos.d
		\end{enumerate}
		\bigskip
		\bigskip	
		
		\item \textbf{Select the correct YUM command to install a package.}
		\begin{enumerate}[label=(\alph*)]
			\item yum install package-name -y %correct
			\item yum -install package-name -y 
			\item yum -i package-name -y 
			\item yum --install package-name -y 
		\end{enumerate}
		\bigskip
		\bigskip	

		\item \textbf{Select the correct YUM command to uninstall a package.}
		\begin{enumerate}[label=(\alph*)]
			\item yum --remove package-name -y 
			\item yum remove package-name -y %correct
			\item yum uninstall package-name -y
			\item yum -e package-name -y
		\end{enumerate}
		\bigskip
		\bigskip	
		
		
		\item \textbf{Select the correct DNF command display package details.}
		\begin{enumerate}[label=(\alph*)]
			\item dnf display package\_name
			\item dnf show package\_name
			\item dnf info package\_name   %correct
			\item dnf list package\_name
		\end{enumerate}
		\bigskip
		\bigskip	

		\item \textbf{Select the correct DNF command to list all installed package.}
		\begin{enumerate}[label=(\alph*)]
			\item dnf list all   %correct
			\item dnf report all
			\item dnf show all
			\item dnf list
		\end{enumerate}
		\bigskip
		\bigskip	

		
	\end{enumerate}
	
	
\end{flushleft}


\subsection{super keyword}
\input{content/chapter11/11.2.9.tex}
\subsection{Constructor overloading}
\setlength{\columnsep}{3pt}
\begin{flushleft}
	\bigskip
	
	\textbf{dnf}: Used to install, update, remove \& search package.
	\begin{tcolorbox}[breakable,notitle,boxrule=-0pt,colback=pink,colframe=pink]
		\color{black}
		\fontdimen2\font=9pt
		Syntax: dnf [options] command [package\_name]
		\fontdimen2\font=4pt
	\end{tcolorbox}
	Command and options with \textbf{dnf} command:
	
	\begin{itemize}
		\item \textbf{install}: Install package along with it's dependency.
		\newline
		\textbf{-y}: Automatically answer yes for all questions.
		\begin{tcolorbox}[breakable,notitle,boxrule=-0pt,colback=pink,colframe=pink]
			\color{black}
			\fontdimen2\font=9pt
			Syntax: dnf install package\_name -y
			\fontdimen2\font=4pt
		\end{tcolorbox}
		Eg:
		\begin{tcolorbox}[breakable,notitle,boxrule=-0pt,colback=black,colframe=black]
			\color{green}
			\fontdimen2\font=9pt
			\# dnf install httpd -y
			\fontdimen2\font=4pt
		\end{tcolorbox}
		\bigskip
		\bigskip
		\item \textbf{remove}: Uninstall a package along with this dependency.
		\begin{tcolorbox}[breakable,notitle,boxrule=-0pt,colback=pink,colframe=pink]
			\color{black}
			\fontdimen2\font=9pt
			Syntax: dnf remove package\_name 
			\fontdimen2\font=4pt
		\end{tcolorbox}
		Eg:
		\begin{tcolorbox}[breakable,notitle,boxrule=-0pt,colback=black,colframe=black]
			\color{green}
			\fontdimen2\font=9pt
			\# dnf remove httpd
			\fontdimen2\font=4pt
		\end{tcolorbox}
		\bigskip
		\bigskip		
		
		\item \textbf{info}: Display the information of a package.
		\begin{tcolorbox}[breakable,notitle,boxrule=-0pt,colback=pink,colframe=pink]
			\color{black}
			\fontdimen2\font=9pt
			Syntax: dnf info package\_name 
			\fontdimen2\font=4pt
		\end{tcolorbox}
		Eg:
		\begin{tcolorbox}[breakable,notitle,boxrule=-0pt,colback=black,colframe=black]
			\color{green}
			\fontdimen2\font=9pt
			\# dnf info httpd
			\fontdimen2\font=4pt
		\end{tcolorbox}
		\bigskip
		\bigskip		

		\item \textbf{groupinfo}: List all packages provided by the group.
		\begin{tcolorbox}[breakable,notitle,boxrule=-0pt,colback=pink,colframe=pink]
			\color{black}
			\fontdimen2\font=9pt
			Syntax: dnf groupinfo group\_name 
			\fontdimen2\font=4pt
		\end{tcolorbox}
		Eg:
		\begin{tcolorbox}[breakable,notitle,boxrule=-0pt,colback=black,colframe=black]
			\color{green}
			\fontdimen2\font=9pt
			\# dnf groupinfo "Development Tools"
			\fontdimen2\font=4pt
		\end{tcolorbox}
		\bigskip
		\bigskip	
		
		\item \textbf{groupinstall}: Install all packages provided by the group.
		\begin{tcolorbox}[breakable,notitle,boxrule=-0pt,colback=pink,colframe=pink]
			\color{black}
			\fontdimen2\font=9pt
			Syntax: dnf groupinstall group\_name 
			\fontdimen2\font=4pt
		\end{tcolorbox}
		Eg:
		\begin{tcolorbox}[breakable,notitle,boxrule=-0pt,colback=black,colframe=black]
			\color{green}
			\fontdimen2\font=9pt
			\# dnf groupinstall "Development Tools"
			\fontdimen2\font=4pt
		\end{tcolorbox}
		\bigskip
		\bigskip	
		
		\item \textbf{groupremove}: Remove all packages provided by the group.
		\begin{tcolorbox}[breakable,notitle,boxrule=-0pt,colback=pink,colframe=pink]
			\color{black}
			\fontdimen2\font=9pt
			Syntax: dnf groupremove group\_name 
			\fontdimen2\font=4pt
		\end{tcolorbox}
		Eg:
		\begin{tcolorbox}[breakable,notitle,boxrule=-0pt,colback=black,colframe=black]
			\color{green}
			\fontdimen2\font=9pt
			\# dnf groupremove "Development Tools"
			\fontdimen2\font=4pt
		\end{tcolorbox}
		\bigskip
		\bigskip	



		\item \textbf{update}: Update all packages and their dependencies.
		\begin{tcolorbox}[breakable,notitle,boxrule=-0pt,colback=pink,colframe=pink]
			\color{black}
			\fontdimen2\font=9pt
			Syntax: dnf update
			\fontdimen2\font=4pt
		\end{tcolorbox}
		\bigskip
		\bigskip	

		\item \textbf{history}: Display a list of all the latest dnf transactions
		\begin{tcolorbox}[breakable,notitle,boxrule=-0pt,colback=pink,colframe=pink]
			\color{black}
			\fontdimen2\font=9pt
			Syntax: dnf history
			\fontdimen2\font=4pt
		\end{tcolorbox}
		\bigskip
		\bigskip			

		\item \textbf{history info}: Examine a particular dnf transaction.
		\begin{tcolorbox}[breakable,notitle,boxrule=-0pt,colback=pink,colframe=pink]
			\color{black}
			\fontdimen2\font=9pt
			Syntax: dnf history info transactionID
			\fontdimen2\font=4pt
		\end{tcolorbox}
		Eg:
		\begin{tcolorbox}[breakable,notitle,boxrule=-0pt,colback=black,colframe=black]
			\color{green}
			\fontdimen2\font=9pt
			\# dnf history info 8
			\fontdimen2\font=4pt
		\end{tcolorbox}
		\bigskip
		\bigskip			

		\item \textbf{history undo}: Revert a particular transaction.
		\begin{tcolorbox}[breakable,notitle,boxrule=-0pt,colback=pink,colframe=pink]
			\color{black}
			\fontdimen2\font=9pt
			Syntax: dnf history undo transactionID
			\fontdimen2\font=4pt
		\end{tcolorbox}
		Eg:
		\begin{tcolorbox}[breakable,notitle,boxrule=-0pt,colback=black,colframe=black]
			\color{green}
			\fontdimen2\font=9pt
			\# dnf history undo 8
			\fontdimen2\font=4pt
		\end{tcolorbox}
		\bigskip
		\bigskip			
		\newpage
		\item \textbf{history redo}: Repeat a particular transaction.
		\begin{tcolorbox}[breakable,notitle,boxrule=-0pt,colback=pink,colframe=pink]
			\color{black}
			\fontdimen2\font=9pt
			Syntax: dnf history redo transactionID
			\fontdimen2\font=4pt
		\end{tcolorbox}
		Eg:
		\begin{tcolorbox}[breakable,notitle,boxrule=-0pt,colback=black,colframe=black]
			\color{green}
			\fontdimen2\font=9pt
			\# dnf history redo 8
			\fontdimen2\font=4pt
		\end{tcolorbox}
		\bigskip
		\bigskip		
		
		
		\item \textbf{list all}: List all installed packages.
		\begin{tcolorbox}[breakable,notitle,boxrule=-0pt,colback=pink,colframe=pink]
			\color{black}
			\fontdimen2\font=9pt
			Syntax: dnf list all
			\fontdimen2\font=4pt
		\end{tcolorbox}
		Eg: List all installed pacakges associated with httpd.
		\begin{tcolorbox}[breakable,notitle,boxrule=-0pt,colback=black,colframe=black]
			\color{green}
			\fontdimen2\font=9pt
			\# dnf list all | grep httpd
			\fontdimen2\font=4pt
		\end{tcolorbox}
		\bigskip
		\bigskip					

		\item \textbf{localinstall}: Install locally downloaded package.
		\begin{tcolorbox}[breakable,notitle,boxrule=-0pt,colback=pink,colframe=pink]
			\color{black}
			\fontdimen2\font=9pt
			Syntax: dnf localinstall package.rpm
			\fontdimen2\font=4pt
		\end{tcolorbox}
		Eg: Download cvs package using command \textbf{"wget https://rpmfind.net/linux/epel/8/Everything/x86\_64/Packages/c/cvs-1.11.23-52.el8.x86\_64.rpm"}
		\newline
		Install downloaded package using command:		
		\begin{tcolorbox}[breakable,notitle,boxrule=-0pt,colback=black,colframe=black]
			\color{green}
			\fontdimen2\font=9pt
			\# dnf localinstall cvs-1.11.23-52.el8.x86\_64.rpm -y
			\fontdimen2\font=4pt
		\end{tcolorbox}
		\bigskip
		\bigskip					
		
					
	\end{itemize}
	
	
\end{flushleft}
\newpage




\subsection{this()}
\setlength{\columnsep}{3pt}
\begin{flushleft}

Starting with RHEL8, you can directly install package from Red Hat Network. 
\newline
For this, you need join a free Red Hat Developer program. 
\newline
Follow along for step-by-step guide:

\begin{itemize}
			\item Browse to \textbf{https://developers.redhat.com/register}.
			\item Create an account as shown in the image:
			\begin{figure}[h!]
				\centering
				\includegraphics[scale=.4]{content/chapter11/images/account.png}
				\caption{Sample output}
				\label{fig:iso3}
			\end{figure}
		\newpage
			\item Once you have created your account, Red Hat should send a verification email as shown. Click on the link provided in email to confirm your email address.
			\begin{figure}[h!]
				\centering
				\includegraphics[scale=.25]{content/chapter11/images/verify.png}
				\caption{Sample output of verification email}
				\label{fig:iso4}
			\end{figure}
			\item Now register your RHEL8 server to Red Hat Network. Enter valid username and password that you created during account creation.
				\begin{tcolorbox}[breakable,notitle,boxrule=-0pt,colback=pink,colframe=pink]
				\color{black}
				\fontdimen2\font=9pt
				Syntax: subscription-manager register
				\fontdimen2\font=4pt
			\end{tcolorbox}
			Eg:
			\begin{figure}[h!]
				\centering
				\includegraphics[scale=.2]{content/chapter11/images/2.png}
				\caption{Sample output}
				\label{fig:iso5}
			\end{figure}
			
			\bigskip
			\bigskip
			\item Confirm whether the system is successfully registered.
			\begin{tcolorbox}[breakable,notitle,boxrule=-0pt,colback=pink,colframe=pink]
				\color{black}
				\fontdimen2\font=9pt
				Syntax: subscription-manager status
				\fontdimen2\font=4pt
			\end{tcolorbox}
			Eg:
			\begin{figure}[h!]
				\centering
				\includegraphics[scale=.3]{content/chapter11/images/3.png}
				\caption{Sample output}
				\label{fig:iso6}
			\end{figure}
			
			\newpage
			\item Finally, attach the subscriptions to your RHEL8 server.
			\begin{tcolorbox}[breakable,notitle,boxrule=-0pt,colback=pink,colframe=pink]
				\color{black}
				\fontdimen2\font=9pt
				Syntax: subscription-manager attach ---auto
				\fontdimen2\font=4pt
			\end{tcolorbox}
			Eg:
			\begin{figure}[h!]
				\centering
				\includegraphics[scale=.3]{content/chapter11/images/4.png}
				\caption{Sample output}
				\label{fig:iso7}
			\end{figure}		
			\bigskip
						\bigskip
			\item Clear the yum cache.
			\begin{tcolorbox}[breakable,notitle,boxrule=-0pt,colback=pink,colframe=pink]
				\color{black}
				\fontdimen2\font=9pt
				Syntax: yum clean all
				\fontdimen2\font=4pt
			\end{tcolorbox}		
			\bigskip
						\bigskip
			\item Refresh yum repository.
			\begin{tcolorbox}[breakable,notitle,boxrule=-0pt,colback=pink,colframe=pink]
				\color{black}
				\fontdimen2\font=9pt
				Syntax: yum update all
				\fontdimen2\font=4pt
			\end{tcolorbox}		
			Eg:
			\begin{figure}[h!]
				\centering
				\includegraphics[scale=.25]{content/chapter11/images/update.png}
				\caption{Sample output}
				\label{fig:iso7}
			\end{figure}		
			\bigskip
			\bigskip
			\item Install httpd package to confirm you are connected to Red Hat Network.
			\bigskip
			\begin{tcolorbox}[breakable,notitle,boxrule=-0pt,colback=black,colframe=black]
				\color{white}
				\fontdimen2\font=9pt
				\color{green}
				\# yum install httpd -y
				\fontdimen2\font=4pt
			\end{tcolorbox}
\end{itemize}
	
		
		
	
	
\end{flushleft}

\newpage


\subsection{super(),this() V/S super,this}
\input{content/chapter11/11.2.8.tex}
\subsection{Access modifier and constructor}
\setlength{\columnsep}{3pt}
\begin{flushleft}

	\textbf{private constructor}
	
	\tablethree{
		\hline
		Syntax & Accessibility & Uses \\
		\hline
		private Classname \{\} & Only accessible within the same class & Used to \textbf{prevent direct instantiation} of the class \\
		\hline
	}
	\begin{itemize}
			\item Eg 1:
			\codeblockfull{A.java}{
				class A \{ \\
				\s \textbf{private A()} \{ \\
				\s \s		System.out.println("A constructor called"); \\
				\s	\} \\
				\s 	public static void main(String[] args) \{ \\
				\s \s 		\textbf{A a1 = new A(); \cmark} \\
				\s	\} \\
				\}
			}
			\bigskip
			\outputblock{
				A constructor called
			}
		
			\item Eg 2:
			\codeblockfull{B.java}{
			class A \{ \\
			\s	\textbf{private A()} \{ \\
			\s \s		System.out.println("A constructor called"); \\
			\s	\} \} \\	 
			public class B extends A \{ \\
			\s	public static void main(String[] args) \{ \\
			\s \s		\textbf{B b1 = new B();} \xmark \s // Cannot instantiate \\
			\s	\} \} 
			}
		\end{itemize}

		\newpage
		\textbf{protected constructor}
		
		\tablethree{
			\hline
			Syntax & Accessibility & Uses \\
			\hline
			protected Classname \{\} & Accessible within \begin{itemize}
				\item Same class
				\item Subclasses
				\item Other classes within the same package
			\end{itemize} 
			Cannot be accessed from different package where class is not subclass. & 
			Used to allow subclasses to access the constructor but restrict direct access from unrelated classes. \\
			\hline
		}
	
		\begin{itemize}
			\item Eg 1:
			\codeblockfull{B.java}{
				class A \{ \\
				\s \textbf{protected A()} \{ \\
				\s \s		System.out.println("A constructor called"); \\
				\s	\} \\
				\} \\
				public class B extends A \{ \\
				\s	public B() \{ \\
				\s \s		System.out.println("B constructor called"); \\
				\s	\} \\
				\s	public static void main(String[] args) \{ \\
				\s \s		\textbf{B b1 = new B();} \cmark \\
				\s	\} 	\}
			}
		
			\bigskip
			\outputblock{
				A constructor called \\
				B constructor called
			}
		\end{itemize}
		
		\newpage
		
		\textbf{public constructor}

		\tablethree{
			\hline
			Syntax & Accessibility & Uses \\
			\hline
			public Classname \{\} & Accessible from anywhere like: \begin{itemize}
				\item Other classes
				\item Subclasses
				\item Different packages
			\end{itemize} 
			& 
			Used to create objects of the class from any context \\
			\hline
		}
			
		\begin{itemize}
			\item Eg 1:
			\codeblockfull{B.java}{
				class A \{ \\
				\s \textbf{public A()} \{ \\
				\s \s		System.out.println("A constructor called"); \\
				\s	\} \\
				\} \\
				public class B extends A \{ \\
				\s	public B() \{ \\
				\s \s		System.out.println("B constructor called"); \\
				\s	\} \\
				\s	public static void main(String[] args) \{ \\
				\s \s		\textbf{B b1 = new B();} \cmark \\
				\s	\} 	\}
			}
			\bigskip
			\outputblock{
				A constructor called \\
				B constructor called
			}
		\end{itemize}
	
		\newpage
		\textbf{default constructor}: Has no explicit access modifier specified.
		\newline
		\tablethree{
			\hline
			Syntax & Accessibility & Uses \\
			\hline
			Classname \{\} & Accessible within the same package but not accessible from classes outside the package.
			& 
			If no constructor is defined, a default constructor is automatically provided by the compiler. \\
			\hline
		}
		
		\begin{itemize}
			\item Eg 1:
			\codeblockfull{B.java}{
				class A \{ \\
				\s \textbf{A()} \{ \\
				\s \s		System.out.println("A constructor called"); \\
				\s	\} \\
				\} \\
				public class B extends A \{ \\
				\s	public B() \{ \\
				\s \s		System.out.println("B constructor called"); \\
				\s	\} \\
				\s	public static void main(String[] args) \{ \\
				\s \s		\textbf{B b1 = new B();} \cmark \\
				\s	\} 	\}
			}
			\bigskip
			\outputblock{
				A constructor called \\
				B constructor called
			}
		\end{itemize}
	
\end{flushleft}

\newpage

\subsection{Instance block}
\setlength{\columnsep}{3pt}
\begin{flushleft}
	
	\begin{itemize}
		\item An instance initializer block (or instance block) is a block of code defined within a class, executed when an instance of the class is created. 
		\item It initialises instance variables or perform other initialization tasks for each object of the class.
		\item It is \textbf{executed before the constructor of the class}. 
		\item It is used when you have multiple constructors or when you want to perform common initialization logic.
		\item Eg:
		\codeblockfull{Demo.java}{
			public class Demo \{ \\
			\s	int value;  \\
			\s \{	 	\\
			\s \s value = 8080;	\\
			\s \}	\\
			\s public Demo() \{\}	\\
			\s public Demo(int value) \{	\\
			\s \s this.value = value;		\\
			\s \}	\\
			\s public static void main(String[] args) \{	\\
			\s \s Demo d1 = new Demo();	\\
			\s \s Demo d2 = new Demo(10);		\\
			\s \s System.out.println(d1.value);	\\
			\s \s System.out.println(d2.value);	\\
			\s \}	\\
			\}
		}
		\bigskip
		\outputblock{
			8080 \\
			10
		}
				
	\end{itemize}	
	
	
\end{flushleft}

\newpage

\subsection{Static initializer}
\setlength{\columnsep}{3pt}
\begin{flushleft}
	
	\begin{itemize}
		\item Also known as a static initialization block
		\item Executed only once when the class is loaded into memory. 
		\item It initialize static variables or perform other one-time initialization tasks.
		\item It is defined within a class and is marked with the static keyword. 
		\item Within a class, you can declare any number of static blocks but all these static blocks will be executed from top to bottom
		\item It does not have a method name and is enclosed within curly braces {}. 
		\syntaxblock{
			class Classname \{ \\
			\s static \{ \\
			\s \s // codehere \\
			\s\} \\
			\}
		}
		\item Eg:
		\bigskip
		\codeblockfull{Test.java}{
			class Test \{ \\
			\s	static int count; \\
			\s	static \{  \\
			\s \s		count = 10; \\
			\s \s		System.out.println("Count: " + count); \\
			\s	\} \\
			\\
			\s	public static void main(String[] args) \{ \\	
			\s	\s   System.out.println("Executed after static block"); \\
			\s	\}
			\}
		}
		\bigskip
		\outputblock{
			Count: 10 \\
			Executed after static block
		}
		
	\end{itemize}	
	
	\quest{Without main method is it possible to print some statements to console?}{
		\begin{itemize}
			\item Before Java 1.7, yes by using static blocks.
			\item Eg:
			\codeblock{
				class Test \{ \\
				\s static int count;  \\
				\s	static \{ \\
				\s \s	count = 10; \\
				\s \s		System.out.println("Count: " + count); \\
				\s	\} \\
				\}
			}
			\item After Java 1.7, this code will result in runtime error.
		\end{itemize}
		
		
	}
	
\end{flushleft}

\newpage

\subsection{Constructor V/S Instance block V/S Static block}
\input{content/chapter11/11.2.14.tex}
\subsection{Destroying object}
\input{content/chapter11/11.2.12.tex}
\subsection{The Stack and the Heap: where things live}
\setlength{\columnsep}{3pt}
\begin{flushleft}
	\bigskip
	\bigskip
	\begin{tcolorbox}[breakable,notitle,boxrule=1pt,colback=black,colframe=black]
		\color{white}
		\bigskip
		In this section, you are going to learn:
		\begin{enumerate}
			\item \textbf{What is YUM \& DNF?}
			\item \textbf{Advantages of YUM over RPM}
			\item \textbf{Configuration of YUM server and client}
			\item \textbf{Subscribing RHEL8 server}
			\item \textbf{YUM commands}
			\item \textbf{DNF commands}
		\end{enumerate}	
		\bigskip
		Finally, there will be a \textbf{small excerise} on these topics to check your knowledge.
		\bigskip
	\end{tcolorbox}
	
	
	\begin{multicols}{2}
		\vspace*{\fill}
		\vspace*{\fill}
		\vspace*{\fill}
		\vspace*{\fill}
		\vspace*{\fill}
		\vspace*{\fill}
		\vspace*{\fill}
		\vspace*{\fill}
		\vspace*{\fill}
		
		\vfill \null
		\columnbreak
		So let's get started....
		\includegraphics[scale=0.08]{content/linux_section.png}
	\end{multicols}	
	
\end{flushleft}

\newpage


\section{Inheritance}
\setlength{\columnsep}{3pt}
\begin{flushleft}
	\bigskip
	\bigskip
	\begin{tcolorbox}[breakable,notitle,boxrule=1pt,colback=black,colframe=black]
		\color{white}
		\bigskip
		In this section, you are going to learn:
		\begin{enumerate}
			\item \textbf{What is a package?}
			\item \textbf{What is an RPM?}
			\item \textbf{RPM commands}
			\item \textbf{Drawback of RPM}
		\end{enumerate}	
		\bigskip
		Finally, there will be a \textbf{small excerise} on these topics to check your knowledge.
		\bigskip
	\end{tcolorbox}
	
	
	\begin{multicols}{2}
		\vspace*{\fill}
		\vspace*{\fill}
		\vspace*{\fill}
		\vspace*{\fill}
		\vspace*{\fill}
		\vspace*{\fill}
		\vspace*{\fill}
		\vspace*{\fill}
		\vspace*{\fill}
		
		\vfill \null
		\columnbreak
		So let's get started....
		\includegraphics[scale=0.08]{content/linux_section.png}
	\end{multicols}	
	
\end{flushleft}

\newpage


\subsection{Types of inheritance}
\setlength{\columnsep}{3pt}
\begin{flushleft}

	\newimage{0.6}{content/chapter11/images/type.png}
	
	\begin{itemize}
		\item \textbf{Single inheritance:} In single inheritance, a class inherits from a single superclass.
		\bigskip
		\codeblock{
			class Superclass \{ \\
			\s	// superclass members \\
			\} \\
			\\
			class Subclass extends Superclass \{ \\
			\s	// subclass members \\
			\}
		}
	\bigskip
	
		\item \textbf{Multilevel inheritance:} Multilevel inheritance refers to a situation where one class inherits from another class, and then a third class inherits from the second class.
		\newpage
		\codeblock{
			class Grandparent \{ \\
			\s	// grandparent members \\
			\} \\
			\\
			class Parent extends Grandparent \{ \\
			\s	// parent members \\
			\} \\
			\\
			class Child extends Parent \{ \\
			\s	// child members \\
			\}
		}
		\bigskip
		\item \textbf{Hierarchical inheritance:} Hierarchical inheritance occurs when multiple classes inherit from a single superclass.
		\bigskip
		\codeblock{
			class Superclass \{ \\
			\s	// superclass members \\
			\} \\
			\\
			class Subclass1 extends Superclass \{ \\
			\s	// subclass1 members \\
			\} \\
			\\			
			class Subclass2 extends Superclass \{ \\
			\s	// subclass2 members \\
			\}
		}
		
		\newpage
		
		\item \textbf{Multiple inheritance (through interfaces)}: 
		\begin{itemize}
			\item A java class cant extend more than one class at a time.
			\item Multiple inheritance is not directly supported.
			\item It can be achieved through interfaces. 
			\bigskip
			\codeblock{
				interface Interface1 \{  \\
				\s	// interface1 methods \\
				\} \\
				\\
				interface Interface2 \{ \\
				\s	// interface2 methods \\
				\} \\
				\\
				class MyClass implements Interface1, Interface2 \{ \\
				\s	// class members \\
				\}	
			}
			\bigskip
			\item Why Java does not provide support for multiple-inheritance?
			\begin{itemize}
				\item There maybe a chance of ambiguity problem. 
				\newimage{0.5}{content/chapter11/images/prob.png}
			\end{itemize}
		
			\item Why multiple-inheritance is supported by interface?
			\begin{itemize}
				\item We shall see this in the interface chapter.
			\end{itemize}
			
		\end{itemize}
		\newpage
		\item \textbf{Cyclic inheritance:}
		\begin{itemize}
			\item This is not allowed in java. 
			\item Eg 1:
			\bigskip
			\codeblock{
				class A extends B \{\} \xmark \\
				\\
				class B extends A \{\} \xmark
			}
			\item Eg 2:
			\bigskip
			\codeblock{
				class A extends A \{\} \xmark
			}
		\end{itemize}	
		
	\end{itemize}
	
\end{flushleft}

\newpage



\subsection{Final class}
\setlength{\columnsep}{3pt}
\begin{flushleft}

	\begin{itemize}
		\item Final class can’t be inherited 
		\item You can’t create child class for final class.
		\item Eg:
		\bigskip
		\codeblockfull{Test.java}{
			final class A \{\} \\
			class Test \textbf{extends} A \{\}  \xmark
		}
		\item Every method present inside final class is always final by default(i.e method cannot be overridden).
		\item But every variable present inside final class need not be final.
		
		\item Advantage of final keyword:
		\begin{itemize}
			\item Security
			\item Unique implementation
		\end{itemize}
	
		\item Disadvantage of final keyword:
		\begin{itemize}
			\item Missing inheritance (due to final classes)
			\item Polymorphism (due to final methods)
		\end{itemize}
		
		\item Hence, if there is no specific requirement, it is not recommended to use final keyword.
		
		
	\end{itemize}	
	
\end{flushleft}

\newpage


\subsection{Has-A relationship}
\setlength{\columnsep}{3pt}
\begin{flushleft}

	\begin{itemize}
		\item Advantage of has-a relationship is \textbf{reuseability} of the code.
		\item There are 2 types of Has-A relationship:
		\begin{itemize}
					\item \textbf{Aggregation:}
			\begin{itemize}
				\item Aggregation is a weaker form of the "has-a" relationship.
				\item It is a one-way relationship and called unidirectional association. 
				\item For example, Bank can have employees but vice versa is not possible.
				\bigskip
				\codeblock{
					class Bank \{ \\\
					\s String nameOfBank; \\
					\s	Bank(String nameOfBank)	\{ \\
					\s \s		this.nameOfBank = nameOfBank; \\
					\s \} \\
					\s public void displayAllDetails(Customer customer) \{ \\
					\s \s		System.out.println("Bank = "+ nameOfBank); \\
					\s \s		System.out.println("Customer = "+ customer.nameOfCustomer); \\
					\s \} \\
					\} \\
					class Customer \{ \\
					\s	String nameOfCustomer; \\
						\s	Customer(String nameOfCustomer) \{ \\
					\s \s		this.nameOfCustomer = nameOfCustomer; \\
					\s	\} \\
					\} 
				}
				\newpage
				\codecontinue{
					class Branch \{ \\
					\s	public static void main(String arg[]) \\
					\s	\{ \\
					\s \s		Bank bank = new Bank("AXIS"); \\
					\s \s		Customer customer = new Customer("Ram"); \\
					\s \s 		bank.displayAllDetails(customer); \\
					\s	\} \\
					\}
				}
			\end{itemize}
			\bigskip
			\item \textbf{Composition:}
			\begin{itemize}
				\item Composition is a strong form of the "has-a" relationship.
				\item In composition two entities are highly dependent on each other. 
				\item One entity cannot exist without the other.
				\item It represents a \textbf{part-of} relationship.
				\item Eg: a car has an engine
				\bigskip
				\codeblock{
					class Car \{ \\
					\s	private final Engine engine;   \\
					\s	String nameOfCar; \\
					\s	String model; \\
					\s	public Car(String nameOfCar, String model) \{ \\
					\s \s		engine  = new Engine("POWERHIGH", "ABC"); \\
					\s \s		this.nameOfCar = nameOfCar; \\
					\s \s		this.model = model; \\
					\s	\}	 
				}	
			
			\newpage
			
			\codecontinue{
					\s	public void showAlldetails() \{ \\
			\s 	System.out.println("Car ="+nameOfCar); \\
			\s 		System.out.println("Model ="+model); \\
			\s 		System.out.println("Engine used ="+engine.typeOfEngine); \\
			\s 		System.out.println("Engine name ="+engine.nameOfEngine); \\
			\s	\} \\
			\} \\
			\\
			class Engine  \{ \\
			\s	String typeOfEngine; \\
			\s	String nameOfEngine; \\
			\s	Engine(String typeOfEngine, String nameOfEngine) \{ \\
			\s \s		this.typeOfEngine = typeOfEngine; \\
			\s \s		this.nameOfEngine = nameOfEngine; \\
			\s	\} \\
			\} \\
			\\
			public class Test \{ \\
			\s	public static void main(String arg[]) \{ \\
			\s \s		Car car = new Car("BMW", "5X"); \\
			\s \s		car.showAlldetails(); \\
			\s 	\} \\
			\}
			}	
				
			\end{itemize}
			
		\end{itemize}
	
	\end{itemize}


\end{flushleft}
\newpage



\subsection{Has-A V/S Is-A relationship}
\setlength{\columnsep}{3pt}
\begin{flushleft}
	\textbf{rpm}: Used to install packages with \textbf{.rpm} extension.
	\begin{tcolorbox}[breakable,notitle,boxrule=-0pt,colback=pink,colframe=pink]
		\color{black}
		\fontdimen2\font=9pt
		Syntax: rpm option package\_name
		\fontdimen2\font=4pt
	\end{tcolorbox}

	Options with \textbf{"rpm"} command:
	\begin{itemize}
		\item \textbf{-i}: Installs the package, if it isn't already installed
		\item \textbf{-v}: Provide more detailed output
		\item \textbf{-h}: Print hash marks to display progress
		\begin{tcolorbox}[breakable,notitle,boxrule=-0pt,colback=pink,colframe=pink]
			\color{black}
			\fontdimen2\font=9pt
			Syntax: rpm -ivh package\_name
			\fontdimen2\font=4pt
		\end{tcolorbox}
		Eg: Download package named \textbf{"cvs"} and install it:
		\bigskip
		\begin{tcolorbox}[breakable,notitle,boxrule=-0pt,colback=black,colframe=black]
			\color{yellow}
			\fontdimen2\font=9pt
			\# For rhel/centos7, download \& install cvs package
			\newline
			\color{green}
			\# wget https://rpmfind.net/linux/centos/7.9.2009/os/x86\_64
			/Packages/cvs-1.11.23-35.el7.x86\_64.rpm
			\newline
			\# rpm -ivh cvs-1.11.23-35.el7.x86\_64.rpm
			\newline
			\newline
			\color{yellow}
			\fontdimen2\font=9pt
			\# For rhel/centos8, download \& install cvs package
			\color{green}
			\newline
			\# wget https://rpmfind.net/linux/epel/8/Everything/x86\_64/
			Packages/c/cvs-1.11.23-52.el8.x86\_64.rpm
			\newline
			\# rpm -ivh cvs-1.11.23-52.el8.x86\_64.rpm 
			\fontdimen2\font=4pt
		\end{tcolorbox}
		\bigskip
		\bigskip
		\item \textbf{-U}: Upgrades any existing package or installs it if an earlier version
		isn't already installed.
		\begin{tcolorbox}[breakable,notitle,boxrule=-0pt,colback=pink,colframe=pink]
			\color{black}
			\fontdimen2\font=9pt
			Syntax: rpm -U package\_name
			\fontdimen2\font=4pt
		\end{tcolorbox}
		\bigskip
		\bigskip
		\item \textbf{-F}: Upgrades only existing packages. It does not install a package if it wasn't previously installed.
		\begin{tcolorbox}[breakable,notitle,boxrule=-0pt,colback=pink,colframe=pink]
			\color{black}
			\fontdimen2\font=9pt
			Syntax: rpm -F package\_name
			\fontdimen2\font=4pt
		\end{tcolorbox}
		\bigskip
		\bigskip
		\item \textbf{-e}: Un-install a package.
		\begin{tcolorbox}[breakable,notitle,boxrule=-0pt,colback=pink,colframe=pink]
			\color{black}
			\fontdimen2\font=9pt
			Syntax: rpm -e package\_name
			\fontdimen2\font=4pt
		\end{tcolorbox}
		Eg: Un-install package named \textbf{"cvs"}.
		\bigskip
		\begin{tcolorbox}[breakable,notitle,boxrule=-0pt,colback=black,colframe=black]
			\color{white}
			\fontdimen2\font=9pt
			\color{green}
			\# rpm -e cvs
			\fontdimen2\font=4pt
		\end{tcolorbox}
		\bigskip
		\begin{tcolorbox}[breakable,notitle,boxrule=-0pt,colback=yellow,colframe=yellow]
			\color{black}
			\fontdimen2\font=9pt
			Note: \textbf{rpm -e} fails with an error if there's some dependent package issues.
			\fontdimen2\font=4pt
		\end{tcolorbox}
	
		\bigskip
		\bigskip
		\item \textbf{--nodeps}: Ignore dependency error and uninstall the package (which may break the package dependent on it).
		\bigskip
		\begin{tcolorbox}[breakable,notitle,boxrule=-0pt,colback=pink,colframe=pink]
			\color{black}
			\fontdimen2\font=9pt
			Syntax: rpm -e ---nodeps package\_name
			\fontdimen2\font=4pt
		\end{tcolorbox}
		
		Eg: 
		\bigskip
		\begin{tcolorbox}[breakable,notitle,boxrule=-0pt,colback=black,colframe=black]
			\color{white}
			\fontdimen2\font=9pt
			\color{green}
			\# rpm -e ---nodeps cvs
			\fontdimen2\font=4pt
		\end{tcolorbox}
		
		\bigskip
		\bigskip
		\item \textbf{-qa}: Lists all installed packages. Here \textbf{-q} stands for \textbf{query} and \textbf{-a} stands for all.
		\bigskip
		\begin{tcolorbox}[breakable,notitle,boxrule=-0pt,colback=pink,colframe=pink]
			\color{black}
			\fontdimen2\font=9pt
			Syntax: rpm -qa
			\fontdimen2\font=4pt
		\end{tcolorbox}
		
		
		\bigskip
		\bigskip
		\item \textbf{-qf}: Identifies the package associated with a specifc file/directory.
		\bigskip
		\begin{tcolorbox}[breakable,notitle,boxrule=-0pt,colback=pink,colframe=pink]
			\color{black}
			\fontdimen2\font=9pt
			Syntax: rpm -qf /path/to/file
			\fontdimen2\font=4pt
		\end{tcolorbox}
		Eg: Find package associated with \textbf{/etc/ssh} file or \textbf{/etc} folder:
		\bigskip
		\begin{tcolorbox}[breakable,notitle,boxrule=-0pt,colback=black,colframe=black]
			\color{white}
			\fontdimen2\font=9pt
			\color{green}
			\# rpm -qf /etc/ssh
			\newline
			\# rpm -qf /etc
			\fontdimen2\font=4pt
		\end{tcolorbox}
		

		\bigskip
		\bigskip
		\item \textbf{-qc}: Lists configuration files that comes with the package installation.
		\bigskip
		\begin{tcolorbox}[breakable,notitle,boxrule=-0pt,colback=pink,colframe=pink]
			\color{black}
			\fontdimen2\font=9pt
			Syntax: rpm -qc packagename
			\fontdimen2\font=4pt
		\end{tcolorbox}
		Eg: Find all configuration files associated with package \textbf{openssh}:
		\bigskip
		\begin{tcolorbox}[breakable,notitle,boxrule=-0pt,colback=black,colframe=black]
			\color{white}
			\fontdimen2\font=9pt
			\color{green}
			\# rpm -qc openssh
			\fontdimen2\font=4pt
		\end{tcolorbox}

		\bigskip
		\bigskip
		\item \textbf{-qi}: Displays basic information for package name.
		\bigskip
		\begin{tcolorbox}[breakable,notitle,boxrule=-0pt,colback=pink,colframe=pink]
			\color{black}
			\fontdimen2\font=9pt
			Syntax: rpm -qi packagename
			\fontdimen2\font=4pt
		\end{tcolorbox}
		Eg: Find basic information associated with package \textbf{openssh}:
		\bigskip
		\begin{tcolorbox}[breakable,notitle,boxrule=-0pt,colback=black,colframe=black]
			\color{white}
			\fontdimen2\font=9pt
			\color{green}
			\# rpm -qi openssh
			\fontdimen2\font=4pt
		\end{tcolorbox}

		\bigskip
		\bigskip
		\item \textbf{-qR}: Display all package dependencies.
		\bigskip
		\begin{tcolorbox}[breakable,notitle,boxrule=-0pt,colback=pink,colframe=pink]
			\color{black}
			\fontdimen2\font=9pt
			Syntax: rpm -qR packagename
			\fontdimen2\font=4pt
		\end{tcolorbox}
		Eg: Display all dependency associated with package \textbf{openssh}:
		\bigskip
		\begin{tcolorbox}[breakable,notitle,boxrule=-0pt,colback=black,colframe=black]
			\color{white}
			\fontdimen2\font=9pt
			\color{green}
			\# rpm -qR openssh
			\fontdimen2\font=4pt
		\end{tcolorbox}


		
	
	\end{itemize}
\end{flushleft}
\newpage



\section{Polymorphism}
\setlength{\columnsep}{3pt}
\begin{flushleft}
	
	\begin{itemize}
		\item Polymorphism refers to the ability of objects to take on multiple forms or behaviors. 
		\item It allows objects of different classes to be treated as objects of a common superclass or interface.
		\item Eg:
		\codeblockfull{Test.java}{
			class Animal \{ \\
			\s	String sound; \\
			\s	public void makenoise() \{ \\
			\s \s		System.out.println(sound); \\
			\s	\} \} \\
			class Dog extends Animal \{ \\
			\s	public Dog(String sound) \{ \\
			\s \s		this.sound = sound; \\
			\s	\} \} \\
			class Cat extends Animal \{ \\
			\s	public Cat(String sound) \{ \\
			\s \s		this.sound = sound; \\
			\s	\} \} \\
			\\
			class Test \{ \\
			\s	public static void main(String[] args) \{ \\
			\s \s		\textbf{Animal[] animals = new Animal[2];} \\
			\s \s		\textbf{animals[0] = new Dog("Baw Baw!");} \\
			\s \s		\textbf{animals[1] = new Dog("Meow!");}  \\
			\s \s		for(int i=0; i<animals.length; i++) \{ \\
			\s \s \s			\textbf{animals[i].makenoise();} \\
			\s \s		\} \\
			\s	\} 		\\
			\}
		}
		\newpage
		
		\outputblock{
			Baw Baw! \\
			Meow!
		}
		
		\bigskip
		\item Polymorphism in Java can be achieved through two main mechanisms:
		
		\newimage{0.5}{content/chapter11/images/poly.png}
		
		
		Let's see each of these type of polymorphism in detail.
		
		
	\end{itemize}
	
\end{flushleft}

\newpage

\subsection{Method overloading}
\setlength{\columnsep}{3pt}
\begin{flushleft}
	
	\begin{itemize}
		\item Allows multiple methods with the same name but different parameters to coexist within the same class. 
		\item When an overloaded method is called, the compiler determines the appropriate method to execute based on the arguments provided.
		
		\item Key aspects of method overloading:
		\bigskip
		\begin{itemize}
			\item \textbf{Method signature:} 
			\begin{itemize}
				\item Consists of the method name and the parameter list. 
				\bigskip
				\syntaxblock{
					modifier returnType \textbf{methodName(type1 arg1, type2 arg2, ..)}
				}
				\bigskip
				\item Overloaded methods must have the same name but different parameter lists. \item The parameter lists can differ in terms of the number of parameters, their types, or both.
				\item Compiler will use method signature while resolving method calls
			\end{itemize}
			\bigskip
			\noteblock{
			\begin{itemize}
				\item Method \textbf{returnType} and \textbf{modifier} are not considered as part of method signature and won't affect method overloading.
				\item Within a class 2 methods with same signature not allowed
			\end{itemize}
			}			
						
		\end{itemize}
		\newpage
			\item Eg:
			
			\codeblockfull{Calculate.java}{
				class Calculate \{ \\
				\s	public int add(int a, int b) \{ \\
				\s \s		return a + b; \\
				\s	\} \\
				\s	public double add(double a, double b) \{ \\
				\s \s		return a + b; \\
				\s	\} \\
				\s	public int add(int a, int b, int c) \{ \\
				\s \s		return a + b + c; \\
				\s	\} \\
				\s	public String add(String str1, String str2) \{ \\
				\s \s		return str1 + str2; \\
				\s	\} \\
				\\
				\s	public static void main(String[] args) \{ \\
				\s \s		Calculate math = new Calculate(); \\
				\s \s		int sum1 = math.add(5, 10); \\
				\s \s		double sum2 = math.add(2.5, 3.7); \\
				\s \s		int sum3 = math.add(1, 2, 3); \\
				\s \s		String concatenated = math.add("Hello", "World"); \\
				\s \s		System.out.println("Sum1: " + sum1); \\
				\s \s		System.out.println("Sum2: " + sum2);  \\
				\s \s		System.out.println("Sum3: " + sum3);  \\
				\s \s		System.out.println("Concatenated: " + concatenated);  \\
				\s	\} \\
				\}
			}
			\newpage
			\outputblock{
				Sum1: 15 \\
				Sum2: 6.2 \\
				Sum3: 6 \\
				Concatenated: HelloWorld	
			}
	\end{itemize}
	
\end{flushleft}


\subsection{Method overriding}
\setlength{\columnsep}{3pt}
\begin{flushleft}
	
	\begin{itemize}
		\item Allows a subclass to provide its own implementation of a method that is already defined in its superclass. 
		\item When a method in the subclass has the \textbf{same name, return type, and parameter list} as a method in the superclass, it is said to override the superclass method.
		
		\item Key points about method overriding:
		
		\begin{itemize}
			\item \textbf{Inheritance:} Based on the concept of inheritance, where a subclass inherits methods and fields from its superclass.
			
			\item \textbf{Signature:} The overriding method must have the same method signature (name, return type, and parameter list) as the method it is overriding.
			
			\item \textbf{Access modifier:} The overriding method cannot have a more restrictive access modifier than the method it is overriding. It can have the same or a more permissive access modifier.
			
			\item \textbf{@Override annotation:} It is a good practice to use the @Override annotation when overriding a method.
			
		\end{itemize}
		\newpage
		\item Eg:
		\bigskip
		\codeblockfull{Test.java}{
			class Animal \{ \\
			\s	public void makeSound() \{ \\
			\s \s		System.out.println("Animal makes a sound"); \\
			\s	\} \\
			\} \\
			\\
			class Cat extends Animal \{ \\
			\s	\textbf{@Override} \\
			\s 	\textbf{public void makeSound()} \{ \\
			\s \s		System.out.println("Cat meows"); \\
			\s	\} \\
			\} \\
			\\
			class Dog extends Animal \{ \\
			\s 	\textbf{@Override} \\
			\s	\textbf{public void makeSound()} \{ \\
			\s \s		System.out.println("Dog barks"); \\
			\s	\} \\
			\} \\
			\\
			class Test \{ \\
			\s	public static void main(String[] args) \{ \\
			\s \s		Animal animal1 = new Cat(); \\
			\s \s		Animal animal2 = new Dog(); \\
					\\
			\s \s		\textbf{animal1.makeSound();}  \\
			\s \s		\textbf{animal2.makeSound();}  \\
			\s	\} \\
			\}
		}
		\outputblock{
			Cat meows \\
			Dog barks
		}
	\end{itemize}
	
	\textbf{Method overriding with respect to var\_arg}
	
	\begin{itemize}
		\item For overriding wrt var\_arg method, the overriding method must have the \textbf{same method name, return type, and parameter types} as the overridden method.
		\item Eg:
		\bigskip
		\codeblockfull{Main.java}{
			class BaseClass \{ \\
			\s public void printValues(String... values) \{ \\
			\s \s		System.out.println("BaseClass - printValues:"); \\
			\s \s		for (String value : values) \{ \\
			\s \s \s			System.out.println(value); \\
			\s \s		\} \\
			\s	\} \} \\
			class SubClass extends BaseClass \{ \\
			\s	@Override \\
			\s	public void printValues(String... values) \{ \\
			\s \s		System.out.println("SubClass - printValues:"); \\
			\s \s		for (String value : values) \{ \\
			\s \s \s			System.out.println(value); \\
			\s \s		\} \\
			\s	\} \} \\
			\\
			class Main \{ \\
			\s	public static void main(String[] args) \{ \\
			\s \s		BaseClass base = new BaseClass(); \\
			\s \s		base.printValues("Hello", "World");  
		}
		
		\codecontinue{
			\s \s		SubClass sub = new SubClass(); \\
			\s \s		sub.printValues("Hello", "World");   \\
			\s \s		BaseClass polymorphic = new SubClass(); \\
			\s \s		polymorphic.printValues("Hello", "World"); \\  
			\s	\} \}
		}
		\bigskip
		\outputblock{
			BaseClass - printValues: \\
			Hello \\
			World \\
			SubClass - printValues: \\
			Hello \\
			World \\
			SubClass - printValues: \\
			Hello \\
			World 
		}
	\end{itemize}
	
\end{flushleft}

\newpage

\subsection{Overloading V/S Overriding}
\setlength{\columnsep}{3pt}
\begin{flushleft}
	
	\tablethree{
	\hline
	Property & Overloading &  Overriding \\
	\hline
	Method names & Must be same & Must be same \\
	\hlnie
	Argument types & Must be different (atleast order) & Must be same (including order) \\
	\hline
	Method signature & Must be different & Must be same \\
	\hline
	Return types & No restrictions & Must be same until Java1.4 version , from Java1.5 version co-varient return types also allowed \\
	\hline
	Private, static, final methods & Can be overloaded & Cannot be overloaded \\
	\hline
	Access modifiers & No restrictions & The scope of access modifiers cannot be reduced but we can increase \\
	\hline
	Method resolution & Always takes care by compiler based on reference types & Always takes care by JVM based on runtime object \\
	\hline
	It is also known as  & Compile time polymorphism, static polymorpohim or early binding & Runtime polymorphism, dynamic polymorphism, or late binding \\
	\hline
	}
	
\end{flushleft}

\newpage

\subsection{Method hiding}
\setlength{\columnsep}{3pt}
\begin{flushleft}
	
	All rules of method hiding are exactly same overriding except the following difference:
	
	\tabletwo{
		\hline
		Method hiding & Method overriding \\
		\hline
		Both parent and child class method should be static & Both parent and child class method should be non-static \\
		\hline
		Compiler is responsible for method resolution based on reference type & JVM is always responsible for method resolution based on runtime object \\
		\hline
		It is also known as compile-time polymorphism or static polymorphism or early binding & It is also known as runtime polymorphism or dynamic polymorphism or late binding \\
		\hline
	}
	
	Eg:
	\codeblockfull{Test.java}{
		class Parent \{ \\
		\s public static void m1() \{ \\
		\s \s	System.out.println("Parent"); \\
		\s	\} \\
		\} \\
		class Child extends Parent \{ \\
		\s	public static void m1() \{ \\
		\s \s		System.out.println("Child"); \\
		\s 	\} \\
		\}  \\
		class Test \{ \\
		\s	public static void main(String[] args) \{ \\
		\s \s		Parent p1 = new Parent(); \\
			\s \s		p1.m1(); \\
		\s \s		Child c1 = new Child(); 
	}	
	\newpage
	\codecontinue{
	\s \s		c1.m1(); \\
	\s \s		Parent p2 = new Child(); \\
	\s \s		p2.m1(); \\
	\s	\} \\
	\}
	}
	
	\bigskip
	\outputblock{
		Parent \\
		Child \\
		Parent
	}

	\bigskip
	\noteblock{
	Method hiding is like sticking poster on black board, while method overriding is like erasing board and writing new content.
	}
\end{flushleft}

\newpage

\section{Data Encapsulation}
\setlength{\columnsep}{3pt}
\begin{flushleft}

	\begin{itemize}
		\item The process of binding data and corresponding methods (behavior) together into a single unit is called encapsulation in Java.
		
		\newimage{0.6}{content/chapter11/images/encap.png}
		
		\item It makes variables and methods safe from outside interference and misuse.
	
		\item Data encapsulation = \textbf{data hiding} + \textbf{data abstraction}.
		\item Let see each of these in detail.
	\end{itemize}	

\end{flushleft}
\newpage
\subsection{Data Hiding}
\setlength{\columnsep}{3pt}
\begin{flushleft}
	
	\begin{itemize}
		\item Data hiding refers to encapsulating data within a class and controlling its visibility and accessibility from outside the class. 
		\item Below code shows data being directly accessible to the objects and it's not data hidiing:
		
		\codeblockfull{Test.java}{
			class Sample \{ \\
			\s int no; \\
			\s String name; \\
			\} \\
			class Test \{ \\
			\s public static void main(String[] args) \{ \\
			\s \s Sample s1 = new Sample(); \\
			\s \s s1.no = 1;   // \color{red} No Data hiding \\ \color{black}
			\s \s s1.name = "Apples"; // \color{red} No Data hiding \\
			\s	\} \\
			\}
		}
		\bigskip
		\item In order to hide the data, the instance variable should be declared as \textbf{private}.
		
		\item Eg: You can create public methods \textbf{setter()} and \textbf{getter()} to get and set private instance variables as shown below:
		
		\codeblockfull{Test.java}{
			class Sample \{ \\
			\s private int no; \\
			\s private String name; \\
			\s 	public void setter(int no, String name) \{ \\
			\s \s		this.no = no; \\
			\s \s		this.name = name; \\
			\s	\} 
		}
		\newpage
		\codecontinue{
			\s	public void getter() \{ \\
			\s \s		System.out.println(no + ": " + name); \\
			\s	\} \\
			\} \\
			class Test \{ \\
			\s	public static void main(String[] args) \{ \\
			\s \s		Sample s1 = new Sample(); \\
			\s \s		s1.setter(0,"Apple");   // Data hidden \\
			\s \s		s1.getter(); \\
			\s	\} \\
			\}	
		}
		\bigskip
		\outputblock{
		0: Apple
		}
	
		
		\item It makes variables and methods safe from outside interference and misuse.
	
		\item Data encapsulation is closely related to \textbf{data hiding}.
			
	\end{itemize}	
	
\end{flushleft}

\subsection{Data Abstraction}
\setlength{\columnsep}{3pt}
\begin{flushleft}
	
	\begin{itemize}
		\item Data abstraction is the process of hiding certain details and showing only essential information to the user. 
		\item Abstraction can be achieved with either abstract classes or interfaces (which you will learn more about in the next chapter).
		
		\newimage{0.4}{content/chapter11/images/data.png}
	\end{itemize}	
	
\end{flushleft}

\newpage
\subsection{Tightly encapsulated class}
\setlength{\columnsep}{3pt}
\begin{flushleft}
	
	\begin{itemize}
		\item A class is said to be tightly encapsulated if and only if each variable declared is \textbf{private}.
		\item Tightly encapsulate class may or may not contain corresponding getter and setter method or these methods may or may not be declared as public.
		\item Eg:
		\codeblock{
		//Below is tightly encapsulated class \\
		class A \{ \\
			\s private int a;  \\
		\} \\
		\\
		//Below is not tightly encapsulated class \\
		class B extends A \{ \\
			\s int b; \\
		\} \\
		}
	
	\end{itemize}	
	
\end{flushleft}

\subsection{Coupling}
\setlength{\columnsep}{3pt}
\begin{flushleft}
	
	\begin{itemize}
		\item The degree of dependency between the components is called \textbf{coupling}.
		\item If dependency is more, then it is considered as \textbf{tightly coupling}.
		\item If dependency is less, then it is considered as \textbf{lossly coupling}.
		\item Eg:
		\newpage
		\codeblockfull{A.java}{
			class A \{ \\
			\s	static int i = B.j; \\ 
			\s	public static void main(String[] args) \{ \\
			\s \s		A a = new A(); \\
			\s \s		System.out.println(a.i); \\
			\s	\} \\
			\} \\
			class B \{ \\
			\s	static int j = C.k; \\
			\} \\
			class C \{ \\
			\s	static int k = D.m1(); \\
			\} \\
			class D \{ \\
			\s	public static int m1() \{ \\
			\s \s		return 10; \\
			\s	\} \\
			\}	
		}
		\bigskip
		\outputblock{
		10
		}
		\bigskip
		\item \textbf{Tightly coupling is not good} programming practise.
		\item Problem with tighly coupling:
		\begin{itemize}
			\item Without affecting remaining components we can modify any component and hence enhaacement will become difficult.
			\item It suppresses reuseability. 
			\item It reduces mainitainabilty of the application.
			\item Hence we have to maintain dependency between the components as less as possible i.e lossly coupling is a good programming practice.
		\end{itemize}
		
	\end{itemize}	
	
\end{flushleft}

\subsection{Cohesion}
\input{content/chapter11/11.5.6.tex}

%----------------------------------------------------------------------------------------
%	CHAPTER 6
%----------------------------------------------------------------------------------------
\chapterimage{index4.png} % Table of contents heading image
\chapter{Abstract Class \& Interface}
%-----------------------
\section{Abstract class}
\setlength{\columnsep}{3pt}
\begin{flushleft}
	\begin{itemize}
		\item \textbf{A process is a program running} in the OS \textbf{using memory \& CPU}.
		\item A Linux process is also called \textbf{service or daemon}.
		\item Every process has many details associated with it, some of them are:
		\begin{itemize}
			\item Process ID (PID)
			\item Process name
			\item A program associated with it
			\item Process state
			\item User owning the process
			\item Parent Process ID (PPID)
		\end{itemize}
		
		\begin{figure}[h!]
			\centering
			\includegraphics[scale=.55]{content/chapter12/images/process.png}
			\caption{Process/Service/Daemon}
			\label{fig:process}
		\end{figure}
		
		\item Keeping unused Linux process running in the system is \textbf{a waste of RAM \& CPU}.
		\item Unused process can expose your system to \color{red}security threat.
	\end{itemize}
\end{flushleft}

\newpage



\subsection{What is abstract class?}
\setlength{\columnsep}{3pt}
\begin{flushleft}

	\begin{itemize}
		\item An \textbf{abstract class cannot be instantiated directly}.
		\item It \textbf{serves as a blueprint} or template for its subclasses. 
		\item It defines \textbf{common attributes and behaviors} that can be shared among multiple related classes. 
		
		\bigskip
		\item \textbf{What is the need of abstract class?}
		\newimage{0.62}{content/chapter12/images/img1.png}
		
		\newpage
		\item Problem is -
		\newimage{0.5}{content/chapter12/images/img2.png}	
		
		\item Similarly, there can be situations where there is no need to create parent class object.
		
		\item The \textbf{compiler} won’t let you instantiate an abstract class.

		\bigskip
		\syntaxblock{
			\textbf{abstract} public class Classname \{\}
		}
		\bigskip
		
		\item Eg:
		\bigskip
		\codeblockfull{Test.java}{
			\textbf{abstract public class Canine extends Animal} \{ \\
			\s public void roam() \{\} \\
			\} \\
			\\
			public class MakeCanine \{ \\
			\s public void go() \{ \\
			\s \s	Canine c; \\
			\s \s	c = new Dog(); \cmark \\
			\s \s	c = new Canine(); \xmark \\
			\s \} \\
			\}
		}
		
	\end{itemize}
	
		
\end{flushleft}

\newpage



\subsection{What is concrete class?}
\setlength{\columnsep}{3pt}
\begin{flushleft}
	
	\begin{itemize}
		\item A class that’s not abstract is called a \textbf{concrete} class. 
		\item Eg:
		\bigskip
		\newimage{0.9}{content/chapter12/images/img3.png}
		
	\end{itemize}

\end{flushleft}

\newpage



\subsection{What is abstract method?}
\setlength{\columnsep}{3pt}
\begin{flushleft}

	\begin{itemize}
		\item An abstract method has no body!
		\item If you declare an abstract method, you MUST mark the class abstract as well. \item You can’t have an abstract method in a non-abstract class.
		\item Abstract methods are declared using the abstract keyword, and they end with a semicolon instead of a method body.
		\bigskip
		\syntaxblock{
			abstract class Classname() \{ \\
			\s public abstract void methodname1();  \\
			\s public abstract void methodname2();  \\
			\}
		}
		\bigskip
		
		\item These methods are intended to be overridden by the subclasses. 
		\item Subclasses \textbf{must provide the implementation} for all the abstract methods. 
		\item Eg:
		\bigskip
		\codeblockfull{Main.java}{
		abstract class Animal \{ \\
		\s	private String name; \\
		\s	public Animal(String name) \{ \\
		\s\s		this.name = name; \\
		\s	\}    \\
		\s	public String getName() \{ \\
		\s\s		return name; \\
		\s	\} \\
		\s	public abstract void sound(); \\
		\} 
		}
		\newpage
		\codecontinue{
		class Dog extends Animal \{ \\
			\s	public Dog(String name) \{ \\
			\s\s		super(name); \\
			\s	\}  \\
		\s	public void sound() \{ \\
		\s\s		System.out.println("Woof!"); \\
		\s	\} \\
		\} \\
		class Cat extends Animal \{ \\
		\s	public Cat(String name) \{ \\
		\s\s		super(name); \\
		\s	\} \\
		\s	public void sound() \{ \\
		\s\s		System.out.println("Meow!"); \\
		\s	\} \\
		\} \\
		\\
		class Main \{ \\
		\s	public static void main(String[] args) \{ \\
		\s\s		Animal dog = new Dog("Buddy"); \\
		\s\s		Animal cat = new Cat("Whiskers"); \\
		\\
		\s\s		System.out.println(dog.getName()); \\
		\s\s		dog.sound(); \\
		\\	
		\s\s		System.out.println(cat.getName()); \\
		\s\s		cat.sound(); \\
		\s	\} \\
		\}
		}
		\bigskip
		\outputblock{
			Buddy  \\
			Woof!  \\
			Whiskers \\
			Meow!
		}
	\end{itemize}

\end{flushleft}



\subsection{Illegal modifier: final abstract}
\setlength{\columnsep}{3pt}
\begin{flushleft}
	
	\paragraph{}
	\bigskip
	
	\begin{figure}[h!]
		\centering
		\includegraphics[scale=.2]{content/practise.jpg}
	\end{figure}	
	\begin{enumerate}
		
		\item \textbf{Which of the following statement is true about Linux process? (Select all that applies.)}
		\begin{enumerate}[label=(\alph*)]
			\item A process is a program running in RAM and utilizing CPU.  %correct
			\item A daemon is Linux process that runs in background. %correct
			\item A process has process name, PID, process state, PPID. %correct
			\item A process that is dead but still utilizing computer memory and CPU is a zombie process. %correct
		\end{enumerate}
		\bigskip
		\bigskip
		
		\item \textbf{Which of the following command will be executed in the background?}
		\begin{enumerate}[label=(\alph*)]
			\item sleep 4000 \&  %correct
			\item cat /etc/passwd  \&\& 
			\item ifconfig 
			\item ls -ld /etc/repos.d
		\end{enumerate}
		\bigskip
		\bigskip	
		
		\item \textbf{Which of the following command is used to check status of a daemon or service?}
		\begin{enumerate}[label=(\alph*)]
			\item systemctl daemon\_name  status
			\item systemctl daemon\_name report
			\item systemctl report daemon\_name  
			\item systemctl status daemon\_name  %correct
		\end{enumerate}
		\bigskip
		\bigskip	


		\item \textbf{Select the correct symbol representing running process.}
		\begin{enumerate}[label=(\alph*)]
			\item R  %correct
			\item D
			\item T
			\item W
		\end{enumerate}
		\bigskip
		\bigskip	
		
		
		\item \textbf{Select the correct symbol representing stopped process.}
		\begin{enumerate}[label=(\alph*)]
			\item S  %correct
			\item D
			\item T
			\item W
		\end{enumerate}
		\bigskip
		\bigskip	


		\item \textbf{Select the correct symbol represeting zombie process.}
		\begin{enumerate}[label=(\alph*)]
			\item R  
			\item D
			\item Z  %correct
			\item W
		\end{enumerate}
		\bigskip
		\bigskip	


		\item \textbf{Which of the following command provides real-time process details with CPU load, memory usage, etc.?}
		\begin{enumerate}[label=(\alph*)]
			\item iostat
			\item netstat
			\item top    %correct
			\item ps
		\end{enumerate}
		\bigskip
		\bigskip	

		\item \textbf{Which of the following command displays all process executed by user named "ravi"? (Select all that applies.)}
		\begin{enumerate}[label=(\alph*)]
			\item iostat -u ravi
			\item netstat -u ravi
			\item top -u ravi    %correct
			\item ps -u ravi   %correct
		\end{enumerate}
		\bigskip
		\bigskip	
		
		\item \textbf{Which of the following command kills a process using process ID?}
		\begin{enumerate}[label=(\alph*)]
			\item kill  %correct
			\item killall
			\item terminate
			\item stop
		\end{enumerate}
		\bigskip
		\bigskip	

		\item \textbf{Which of the following command kills a process using process name?}
		\begin{enumerate}[label=(\alph*)]
			\item kill  
			\item killall  %correct
			\item terminate
			\item stop
		\end{enumerate}
		\bigskip
		\bigskip

		\item \textbf{Which of the following command display all kill signals?}
		\begin{enumerate}[label=(\alph*)]
			\item kill -l  %correct
			\item kill -s
			\item kill -p
			\item kill -t
		\end{enumerate}
		
	\end{enumerate}
	
	
\end{flushleft}

\newpage


\section{The ultimate superclass: Object}
\setlength{\columnsep}{3pt}
\begin{flushleft}
	
	\begin{itemize}
		\item Every class in Java extends class \textbf{Object}.
		\item Class Object is the mother of all classes; it’s the superclass of everything.
		\item Every class you write extends Object, without your ever having to say it.
	\end{itemize}	
	\bigskip
	\textbf{So what’s in this Object class?}
	\begin{itemize}
		\item Class Object does indeed have methods for below four things:
		
		\newimage{0.98}{content/chapter12/images/img5.png}
		
		\newpage
		\begin{itemize}
			\item \textbf{equals(Object o)}: Used to check if 2 objects are considered "equal".
			\bigskip
			\codeblockfull{Test.java}{
			class Dog \{\} \\
			class Cat \{\} \\
			\\
			class Test \{ \\
			\s public static void main(String[] args) \{ \\
			\s \s		Dog a = new Dog(); \\
			\s \s		Cat b = new Cat(); \\
			\s \s		if (a.equals(b)) \{ \\
			\s \s \s			System.out.println("true"); \\
			\s \s		\} else \{ \\
			\s \s \s			System.out.println("false");   \\
			\s \s		\} \\
			\s	\} \\
			\}
			}
			\bigskip
			\outputblock{
			false
			}
			\newpage
			\item \textbf{getClass()}: Used to display the class that object was instantiated from.
			\bigskip
			\codeblockfull{Test.java}{
			class Cat \{\} \\
			\\
			class Test \{ \\
			\s	public static void main(String[] args) \{ \\
			\s \s		Cat a = new Cat(); \\
			\s \s		System.out.println(a.getClass()); \\
			\s	\} \\
			\}
			}
			\bigskip
			\outputblock{
				class Cat
			}
			\bigskip
	
			\item \textbf{hashCode()}: Prints out a hashcode for the object (think of it as a unique code).
			\bigskip
			\codeblockfull{Test.java}{
				class Cat \{\} \\
				class Test \{ \\
				\s	public static void main(String[] args) \{ \\
					\s \s		Cat a = new Cat(); \\
					\s \s		System.out.println(a.hashCode()); \\
					\s	\} \\
					\}	
			}
			\bigskip
			\outputblock{
			1178945
			}
			\newpage
			\item \textbf{toString()}: Prints out a string message with the name of the class and random number.
			\bigskip
			\codeblockfull{Test.java}{
				class Cat \{\} \\
				\\
				class Test \{ \\
				\s	public static void main(String[] args) \{ \\
				\s \s		Cat a = new Cat(); \\
				\s \s		System.out.println(a.toString()); \\
				\s	\} \\
				\}
			}
			\bigskip
			\outputblock{
				Cat@515f550a
			}
		\end{itemize}
	\end{itemize}
	
	
\end{flushleft}

\newpage


\subsection{Object type-casting}
\setlength{\columnsep}{3pt}
\begin{flushleft}
	\begin{itemize}
		\item \textbf{A process is a program running} in the OS \textbf{using memory \& CPU}.
		\item A Linux process is also called \textbf{service or daemon}.
		\item Every process has many details associated with it, some of them are:
		\begin{itemize}
			\item Process ID (PID)
			\item Process name
			\item A program associated with it
			\item Process state
			\item User owning the process
			\item Parent Process ID (PPID)
		\end{itemize}
		
		\begin{figure}[h!]
			\centering
			\includegraphics[scale=.55]{content/chapter12/images/process.png}
			\caption{Process/Service/Daemon}
			\label{fig:process}
		\end{figure}
		
		\item Keeping unused Linux process running in the system is \textbf{a waste of RAM \& CPU}.
		\item Unused process can expose your system to \color{red}security threat.
	\end{itemize}
\end{flushleft}

\newpage



\section{Interface}
\setlength{\columnsep}{3pt}
\begin{flushleft}
	
	\begin{itemize}
		\item An interface is like a 100\% pure abstract class.
		\item Inside interface, every method is always abstract whether we are declaring it or not.
		\item Any service requirement specification(SRS) or any contract between client and service provider are 100\% pure abstract class, is nothing but interface.
		
		\item To declare an interface, use the interface keyword followed by the name of the interface.
		\bigskip
		\syntaxblock{
		public interface InterfaceName \{\}
		}

		\item To implement an interface, a class must use the implements keyword followed by the name of the interface(s) it is implementing.
		\bigskip
		\syntaxblock{
		public class ClassName implements InterfaceName \{\}
		}
				
		\item A class can implement any number of interfaces solving the multiple inheritance problem.
				
	\end{itemize}	
	
\end{flushleft}

\newpage


\subsection{Interface methods}
\setlength{\columnsep}{3pt}
\begin{flushleft}
	
	\begin{itemize}
		\item Implementing an interface requires implementing each method of interface.
		\item Interface methods should always by declared as \textbf{public}.
		\item Interface methods are abstract whether we are declaring or not.			
		\item Hence inside interface the ofllowing method declaration are equal:
		\bigskip
		\codeblock{
			void m1(); \\
			public void m1(); \\
			abstract void m1(); \\
			public abstract void m1();
		}
		
		\item Eg:
		\codeblockfull{Circle.java}{
			public interface Drawable \{ \\
			\s void draw(); \\
			\s	double calculateArea(); \\
			\} \\
			public class Circle implements Drawable \{ \\
			\s	private double radius; \\
				\\
			\s	public Circle(double radius) \{ \\
			\s \s		this.radius = radius; \\
			\s	\}  \\
					\s	@Override \\
			\s	public void draw() \{ \\
			\s \s		System.out.println("Drawing a circle."); \\
			\s	\} \\
									\s	@Override \\
			\s	public double calculateArea() \{ 
		}
		
		\newpage
		\codecontinue{
			\s \s		return Math.PI * radius * radius; \\
			\s	\} \\
		\s public static void main(String[] args) \{ \\
		\s \s	Circle circle = new Circle(4); \\
		\s \s	circle.draw(); \\
		\s \s	System.out.println(circle.calculateArea()); \\
		\s \} \\
		\}
		}
	
		\bigskip
		\outputblock{
			Drawing a circle. \\
			50.26548245743669
		}
	\end{itemize}

	
\end{flushleft}
\newpage


\subsection{extends V/S implements}
\input{content/chapter12/12.3.3.tex}
\subsection{Interface variables}
\input{content/chapter12/12.3.4.tex}
\subsection{Interface method naming conflicts}
\input{content/chapter12/12.3.5.tex}
\subsection{Interface variable naming conflicts}
\input{content/chapter12/12.3.6.tex}
\subsection{Marker interface}
\setlength{\columnsep}{3pt}
\begin{flushleft}
	
	\begin{itemize}
		\item A marker interface is an interface that does not declare any methods. 
		\item Its sole purpose is to mark or tag a class, indicating that the class has a certain property or behavior.
		
		\item A marker interface is essentially an empty interface, but by implementing that interface, a class can convey some additional information to the compiler.
		
		\item Here's an example of a marker interface named Serializable in Java:
		\bigskip
		\codeblock{
			public interface Serializable \{ \\
			\s	// Empty interface \\
			\}
		}
		\item Below are some markers interfaces, these are marked for some ability:
		\begin{itemize}
			\item Serialisable (I)
			\item Cloneable (I)
			\item RandomAccess (I)
		\end{itemize}
	\end{itemize}
	\quest{Without having any methods, how the objects will get some ability in marker interfaces?}{
		Internally JVM is responsible to provide required ability.	
	}
	
	\quest{Why JVM is providing required ability in marker interfaces?}{
		\begin{itemize}
			\item To reduce complexity of programming.
			\item And to make java language as simple.
		\end{itemize}
	}
	
	\quest{Is it possible to create our own marker interface?}{
		\begin{itemize}
			\item Yes, but customisation of JVM  is required.
			\item For this we will have to design our own JVM.
		\end{itemize}
	}
	
\end{flushleft}


\subsection{Adapter Classes}
\setlength{\columnsep}{3pt}
\begin{flushleft}
	
	\begin{itemize}
		\item Adapter class is a simple Java class that implements an interface with only empty implementation.
		\item Using adapter class, you can extend the adapter class and override only the methods they need, instead of implementing all the methods of the interface.
		\item Eg:
		\bigskip
		\codeblockfull{AudioPlayer.java}{
			public interface MediaPlayer \{ \\
			\s void play(); \\
			\s	void pause(); \\
			\s	void stop(); \\
			\} \\
			\\
			abstract class MediaPlayerAdapter implements  MediaPlayer \{ \\
			\s @Override \\
			\s	public void play() \{ \\
			\s \s		// Default implementation \\
			\s	\} \\
				\\
			\s	@Override \\
			\s	public void pause() \{ \\
			\s \s		// Default implementation \\
			\s	\} 
		}
	
		\newpage
		\codecontinue{
		\s 	@Override \\
		\s	public void stop() \{ \\
		\s \s		// Default implementation \\
		\s	\} \\
		\} \\
		\\
		public class AudioPlayer extends MediaPlayerAdapter \{ \\
		\s	@Override \\
		\s	public void play() \{ \\
		\s \s		// Implementation specific to AudioPlayer  \\
		\s \s	System.out.println("Playing audio."); \\
		\s	\} \\
		\s	public static void main(String[] args) \{ \\
		\s \s		AudioPlayer player = new AudioPlayer(); \\
		\s \s		player.play(); \\
		\s	\} \\
		\}
		}
		
		\bigskip
		\outputblock{
			Playing audio.	
		}
		
	\end{itemize}
\end{flushleft}
\newpage


\subsection{Interface V/S Abstract class}
\input{content/chapter12/12.3.9.tex}


%----------------------------------------------------------------------------------------
%	CHAPTER 7
%----------------------------------------------------------------------------------------
\chapterimage{index8.png} % Table of contents heading image
\chapter{Java Built-in packages and API}
%%%%%-----------------------
\section{Overview of java API}
\input{content/chapter7/7.2.tex}
\subsection{String, String Buffer and String Builder}
\input{content/chapter7/7.2.1.tex}
\subsection{Exception Handling}
\input{content/chapter7/7.2.1.tex}
\subsection{Threads and multithreading}
\input{content/chapter7/7.2.1.tex}
\subsection{Wrapper Classes}
\input{content/chapter7/7.2.1.tex}
\subsection{Data Structures}
\input{content/chapter7/7.2.1.tex}
\subsection{JAVA COLLECTION FRAMEWORKS}
\input{content/chapter7/7.2.1.tex}
\subsection{File Handling}
\input{content/chapter7/7.2.1.tex}
\subsection{Serialization}
\input{content/chapter7/7.2.1.tex}



%----------------------------------------------------------------------------------------
%	CHAPTER 7
%----------------------------------------------------------------------------------------
%\chapterimage{index8.png} % Table of contents heading image
%\chapter{Pandas}
%%%%%-----------------------
%\section{Getting started with Pandas}
%\input{content/chapter7/7.2.tex}
%\subsection{Pandas Introduction}
%\input{content/chapter7/7.2.1.tex}
%\subsection{Pandas Series}
%\input{content/chapter7/7.2.2.tex}
%\subsection{Pandas DataFrames}
%\input{content/chapter7/7.2.3.tex}
%\subsection{Pandas Read CSV}
%\input{content/chapter7/7.2.4.tex}
%\subsection{Pandas Analyzing Data}
%\input{content/chapter7/7.2.1.tex}
%\section{Pandas Deep Dive}
%\input{content/chapter7/7.1.tex}
%\subsection{Cleaning Data}
%\input{content/chapter7/7.1.1.tex}
%\subsection{Pandas Plotting}
%\input{content/chapter7/7.2.1.tex}
%%%%%-----------------------




%\chapterimage{appen.png} % Table of contents heading image
%\chapter{Appendix}
%\section{An A-Z Index of the Linux command line}
%\input{content/chapter19/19.1.9.tex}
%\section{Technical books from Lavatech Technology}
%\input{content/chapter19/19.1.10.tex}
%\input{page.tex}

%%%%-----------------------





%%%%
%%%%
%%%%
%%%%
%%%%%----------------------------------------------------------------------------------------
%%%%%	PART
%%%%%----------------------------------------------------------------------------------------
%%%%
%%%%\part{System Admin Level II}
%%%%
%%%%%----------------------------------------------------------------------------------------%%%%%	CHAPTER 3
%%%%%----------------------------------------------------------------------------------------
%%%%
%%%%\chapterimage{index4.pdf} % Chapter heading image
%%%%
%%%%\chapter{Presenting Information}
%%%%
%%%%\section{Table}\index{Table}
%%%%
%%%%\begin{table}[h]
%%%%\centering
%%%%\begin{tabular}{l l l}
%%%%\toprule
%%%%\textbf{Treatments} & \textbf{Response 1} & \textbf{Response 2}\\
%%%%\midrule
%%%%Treatment 1 & 0.0003262 & 0.562 \\
%%%%Treatment 2 & 0.0015681 & 0.910 \\
%%%%Treatment 3 & 0.0009271 & 0.296 \\
%%%%\bottomrule
%%%%\end{tabular}
%%%%\caption{Table caption}
%%%%\label{tab:example} % Unique label used for referencing the table in-text
%%%%%\addcontentsline{toc}{table}{Table \ref{tab:example}} % Uncomment to add the table to the table of contents
%%%%\end{table}
%%%%
%%%%Referencing Table \ref{tab:example} in-text automatically.
%%%%
%%%%%------------------------------------------------
%%%%
%%%%\section{Figure}\index{Figure}
%%%%
%%%%\begin{figure}[h]
%%%%\centering\includegraphics[scale=0.5]{placeholder.jpg}
%%%%\caption{Figure caption}
%%%%\label{fig:placeholder} % Unique label used for referencing the figure in-text
%%%%%\addcontentsline{toc}{figure}{Figure \ref{fig:placeholder}} % Uncomment to add the figure to the table of contents
%%%%\end{figure}
%%%%
%%%%Referencing Figure \ref{fig:placeholder} in-text automatically.

%----------------------------------------------------------------------------------------
%	BIBLIOGRAPHY
%----------------------------------------------------------------------------------------

%\chapter*{Bibliography}
%\addcontentsline{toc}{chapter}{\textcolor{ocre}{Bibliography}} % Add a Bibliography heading to the table of contents
%
%%------------------------------------------------
%
%\section*{Articles}
%\addcontentsline{toc}{section}{Articles}
%\printbibliography[heading=bibempty,type=article]
%
%%------------------------------------------------
%
%\section*{Books}
%\addcontentsline{toc}{section}{Books}
%\printbibliography[heading=bibempty,type=book]
%
%%----------------------------------------------------------------------------------------
%%	INDEX
%%----------------------------------------------------------------------------------------
%
%\cleardoublepage % Make sure the index starts on an odd (right side) page
%\phantomsection
%\setlength{\columnsep}{0.75cm} % Space between the 2 columns of the index
%\addcontentsline{toc}{chapter}{\textcolor{ocre}{Index}} % Add an Index heading to the table of contents
%\printindex % Output the index

%----------------------------------------------------------------------------------------

\end{document}


