\setlength{\columnsep}{3pt}
\begin{flushleft}
	
	\begin{itemize}
		\item Data sets in Pandas are usually multi-dimensional tables, called \textbf{DataFrames}.
		\item \textbf{Series is like a column}, a DataFrame is the whole table.
		\item Eg:
		\bigskip
		\codeblock{
			import pandas as pd \\
			a = \{ \\
			\s	'Skill': ['Python','Ruby','Perl'], \\
			\s	'Candidates': ['Ravi','Ram','Raman'] \\
			\} \\
			myvar = pd.DataFrame(a) \\
			print(myvar)  
		}
		\bigskip
		\outputblock{
			\s \s    Skill \s \s Candidates \\
			0 \s \s  Python  \s \s     Ravi \\
			1 \s \s   Ruby   \s \s     Ram \\
			2 \s \s   Perl   \s \s   Raman
		}	
	\end{itemize}	
	\newpage
	\textbf{Locate Row}
	\begin{itemize}
		\item Pandas use the \textbf{"loc"} attribute to return one or more specified row(s).
		\item To locate single row:
		\codeblock{
			print(myvar.loc[0])
		}
		\item To locate multiple row:
		\codeblock{
			print(myvar.loc[[0,1]]) 
		}
		Example:
		\bigskip
		\codeblock{
			import pandas as pd \\
			a = \{ \\
			\s	'Skill': ['Python','Ruby','Perl'], \\
			\s	'Candidates': ['Ravi','Ram','Raman'] \\
			\} \\
			myvar = pd.DataFrame(a) \\
			print(myvar.loc[0]) \\
			print(myvar.loc[1])
		}
		\bigskip
		\outputblock{
			Skill   \s \s      Python  \\
			Candidates  \s \s     Ravi  \\
			Name: 0, dtype: object  \\
			Skill    \s \s     Ruby  \\
			Candidates  \s \s   Ram  \\
			Name: 1, dtype: object
		}

		\newpage		
		\item \textbf{Named Indexes}
		\bigskip
		\codeblock{
			import pandas as pd  \\
			a = \{ \\
				'Skill': ['Python','Ruby','Perl'], \\
				'Candidates': ['Ravi','Ram','Raman'] \\
			\} \\
			myvar = pd.DataFrame(a,index=['a','b','c']) \\
			print(myvar) \\
			print(myvar.loc['a'])
		}
		\bigskip
		\outputblock{
			\s \s    Skill Candidates  \\
			a \s \s Python  \s \s     Ravi  \\
			b \s \s   Ruby  \s \s      Ram  \\
			c \s \s   Perl  \s \s    Raman  \\
			Skill   \s \s      Python  \\
			Candidates   \s \s   Ravi  \\
			Name: a, dtype: object
		}
		
	\end{itemize}
	
	
	
	
	
	
	
	
\end{flushleft}

\newpage

