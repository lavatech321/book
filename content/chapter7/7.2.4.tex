\setlength{\columnsep}{3pt}
\begin{flushleft}
	
		In this chapter, we shall use below dataset to analyse.
	
	\datablock{
		Series reference,Period,Data value,Suppressed,Group,Stage
		BDCQ.SEA1AA,2011.06,80078,,Agriculture,Filled jobs
		BDCQ.SEA1AA,2011.09,78324,,Agriculture,Filled jobs
		BDCQ.SEA1AA,2011.12,85850,,Agriculture,Not Filled
		BDCQ.SEA1AA,2012.03,90743,,Agriculture,Filled jobs
		BDCQ.SEA1AA,2012.06,81780,,Agriculture,Filled jobs
		BDCQ.SEA1AA,2012.09,79261,,Agriculture,Filled jobs
		BDCQ.SEA1AA,2012.12,87793,,Agriculture,Filled jobs
		BDCQ.SEA1AA,2013.03,91571,,Agriculture,Filled jobs
		BDCQ.SEA1AA,2013.06,81687,,Agriculture,Filled jobs
		BDCQ.SEA1AA,2013.09,81471,,Agriculture,Filled jobs
		BDCQ.SEA1AA,2013.12,93950,,Industry,Filled jobs
		BDCQ.SEA1AA,2014.03,97208,,Agriculture,Filled jobs
		BDCQ.SEA1AA,2014.06,85879,,Industry,Not Filled
		BDCQ.SEA1AA,2014.09,84447,,Industry,Filled jobs
		BDCQ.SEA1AA,2014.12,95075,,Industry,Filled jobs
		BDCQ.SEA1AA,2015.03,98202,,Fishing,Filled jobs
		BDCQ.SEA1AA,2015.06,87987,,Fishing,Filled jobs
		BDCQ.SEA1AA,2015.09,84529,,Fishing,Filled jobs
		BDCQ.SEA1AA,2015.12,96848,,Fishing,Filled jobs
		BDCQ.SEA1AA,2016.03,99291,,Fishing,Filled jobs
		BDCQ.SEA1AA,2016.06,88716,,Industry,Not Filled
		BDCQ.SEA1AA,2016.09,85933,,Industry,Not Filled
		BDCQ.SEA1AA,2016.12,96540,,Industry,Not Filled
		BDCQ.SEA1AA,2017.03,98994,,Fishing,Not Filled
	}
	
	
	\begin{itemize}
		\item If your data sets are stored in a file, Pandas can load them into a DataFrame.
		\item CSV files contains plain text and is a well know format that can be read by everyone including Pandas.
		\item Eg:
		\bigskip
		\codeblock{
			import pandas as pd \\
			df = pd.read\_csv('data.csv') \\
			print(df) 
		}
		\bigskip
		\noteblock{
			\begin{itemize}
				\item By default, if dataset is too large, DataFrame will display first 5 and last 5 lines.
				\item Use \textbf{to\_string()} to print the entire DataFrame.		
			\end{itemize}
			 
		}
	
	\end{itemize}	

	\textbf{Viewing the Data}
	
	\begin{itemize}
		\item The \textbf{head(n)} method returns the headers and a specified number of rows specified using "n", starting from the top.
		\bigskip
		\codeblock{
			import pandas as pd \\
			df = pd.read\_csv('data.csv') \\
			print(df.head(3)) 
		}
		\bigskip
		\textbf{Output:}
		\newimage{0.4}{content/chapter7/images/output1.png}
		
		\item The \textbf{tail(n)} method returns the headers and a specified number of rows, starting from the bottom.
		\bigskip
		\codeblock{
			import pandas as pd \\
			df = pd.read\_csv('data.csv') \\
			print(df.tail(3)) 
		}
		
	\end{itemize}
	
	\textbf{Info About the Data}
	\begin{itemize}
		\item The DataFrames object has a method called \textbf{info()}, that gives you more information about the data set.
		
		\bigskip
		\codeblock{
			import pandas as pd \\
			df = pd.read\_csv('data.csv') \\
			print(df.info()) 			
		}
	
		\bigskip
		\outputblock{
			<class 'pandas.core.frame.DataFrame'> \\
			RangeIndex: 31 entries, 0 to 30 \\
			Data columns (total 6 columns): \\
			\# \s  Column    \s        Non-Null Count \s Dtype   \\
			--- \s ------    \s        --------------  \s  -----   \\
			0 \s  Series reference  \s 31 non-null  \s   object   \\
			1 \s  Period            \s 31 non-null  \s   float64 \\
			2 \s  Data value        \s 31 non-null  \s   int64    \\
			3 \s  Suppressed        \s 0 non-null   \s   float64  \\
			4 \s  Group             \s 31 non-null  \s   object   \\
			5 \s  Stage             \s 31 non-null  \s   object   \\
			dtypes: float64(2), int64(1), object(3)  \\
			memory usage: 1.6+ KB \\
			None  \\
		}
	
		\noteblock{
			The info() method also tells us how many Non-Null values there are present in each column, and in our data set it seems like there are 31 Non-Null values in the "Series reference","Period","Data value","Suppressed","Group","Stage" column.
		}

	\end{itemize}
	
	\textbf{Save data in csv format}
	
	\begin{itemize}
		\item Use \textbf{"to\_csv()"} to save a dataframe to a ".csv" file.
		\item Example
		\bigskip
		\codeblock{
			import pandas as pd \\
			df = pd.read\_csv('data3.csv') \\
			df = df.head(20) \\
			df.to\_csv('analysis.csv')
		}
	\end{itemize}
	
	
	
	
	
	
	
\end{flushleft}

\newpage

