\setlength{\columnsep}{3pt}
\begin{flushleft}
	\begin{itemize}
		\item Append operator is similar to output redirection.
		\item Append redirection is done using the "\textbf{{$\textgreater$}}\textbf{{$\textgreater$}}".
		\bigskip
		\begin{tcolorbox}[breakable,notitle,boxrule=-0pt,colback=pink,colframe=pink]
			\color{black}
			\fontdimen2\font=9pt
			Syntax: command \textbf{{$\textgreater$}}\textbf{{$\textgreater$}} output\_file
			\fontdimen2\font=4pt
		\end{tcolorbox}
	
		\item "\textbf{{$\textgreater$}}\textbf{{$\textgreater$}}" performs below operations:
		\begin{itemize}
			\item If the file exists, it will \textbf{append} new content to the existing content.
			\item If the file does not exist, "\textbf{{$\textgreater$}}\textbf{{$\textgreater$}}" operator will create it.
		\end{itemize}
	
		\item 	Eg:
		\begin{tcolorbox}[breakable,notitle,boxrule=-0pt,colback=black,colframe=black]
			\color{green}
			\fontdimen2\font=9pt
			\$ uname > my\_files 
			\newline
			\$ echo "Hello World!" \textbf{{$\textgreater$}}\textbf{{$\textgreater$}} my\_files 			
			\newline
			\newline
			\color{yellow}
			\# Read content of my\_files
			\color{green}
			\newline
			\color{green}
			\$ cat my\_files    
			\newline
			\color{white}
			Linux 
			\newline
			\color{white}
			Hello World!
			\fontdimen2\font=4pt
		\end{tcolorbox}
		
		
	\end{itemize}

	
\end{flushleft}

