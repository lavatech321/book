\setlength{\columnsep}{3pt}
\begin{flushleft}
	
	\begin{itemize}
		\item Polymorphism refers to the ability of objects to take on multiple forms or behaviors. 
		\item It allows objects of different classes to be treated as objects of a common superclass or interface.
		\item Eg:
		\codeblockfull{Test.java}{
			class Animal \{ \\
			\s	String sound; \\
			\s	public void makenoise() \{ \\
			\s \s		System.out.println(sound); \\
			\s	\} \} \\
			class Dog extends Animal \{ \\
			\s	public Dog(String sound) \{ \\
			\s \s		this.sound = sound; \\
			\s	\} \} \\
			class Cat extends Animal \{ \\
			\s	public Cat(String sound) \{ \\
			\s \s		this.sound = sound; \\
			\s	\} \} \\
			\\
			class Test \{ \\
			\s	public static void main(String[] args) \{ \\
			\s \s		\textbf{Animal[] animals = new Animal[2];} \\
			\s \s		\textbf{animals[0] = new Dog("Baw Baw!");} \\
			\s \s		\textbf{animals[1] = new Dog("Meow!");}  \\
			\s \s		for(int i=0; i<animals.length; i++) \{ \\
			\s \s \s			\textbf{animals[i].makenoise();} \\
			\s \s		\} \\
			\s	\} 		\\
			\}
		}
		\newpage
		
		\outputblock{
			Baw Baw! \\
			Meow!
		}
		
		\bigskip
		\item Polymorphism in Java can be achieved through two main mechanisms:
		
		\newimage{0.5}{content/chapter11/images/poly.png}
		
		
		Let's see each of these type of polymorphism in detail.
		
		
	\end{itemize}
	
\end{flushleft}

\newpage
