\setlength{\columnsep}{3pt}
\begin{flushleft}
	\bigskip
	You can start using yum server for package installation on any client, by following below steps:
	
	\begin{enumerate}
		\item Create a repository file named \textbf{/etc/yum.repos.d/new.repo}:
		\begin{tcolorbox}[breakable,notitle,boxrule=-0pt,colback=black,colframe=black]
			\color{white}
			\fontdimen2\font=9pt
			[Title\_name] \color{yellow}  \# Add title name for this repository 
			\newline
			\color{white}
			name = Repo\_name
			\color{yellow}
			\# Add repository name
			\newline
			\color{white}
			enabled = 1
			\newline
			gpgcheck = 0
			\newline
			baseurl = ftp://(server-IP-address)/pub
			\fontdimen2\font=4pt
		\end{tcolorbox}
		\bigskip
		\begin{tcolorbox}[breakable,notitle,boxrule=-0pt,colback=yellow,colframe=yellow]
			\color{black}
			\textbf{Note:} 
			\begin{itemize}
				\item Location of repository file should be \textbf{/etc/yum.repos.d}
				\item Name of the file can be anything
				\item Extension of the file of should be \textbf{".repo"}
			\end{itemize}
		\end{tcolorbox}

		
		
		\item Clean all the cached files from any enabled repository.
		\begin{tcolorbox}[breakable,notitle,boxrule=-0pt,colback=black,colframe=black]
			\color{green}
			\fontdimen2\font=9pt
			\# yum	clean	all
			\fontdimen2\font=4pt
		\end{tcolorbox}
		
		
		\item Refresh all the enabled repository:
		\begin{tcolorbox}[breakable,notitle,boxrule=-0pt,colback=black,colframe=black]
			\color{green}
			\fontdimen2\font=9pt
			\# yum 	update 	all
			\fontdimen2\font=4pt
		\end{tcolorbox}
		
		
		\item Now you can install any packages using the YUM server.
		\newline
		Eg:
		\begin{tcolorbox}[breakable,notitle,boxrule=-0pt,colback=black,colframe=black]
			\color{green}
			\fontdimen2\font=9pt
			\# yum	install	httpd -y		
			\fontdimen2\font=4pt
		\end{tcolorbox}

		
	\end{enumerate}
	
	
	
\end{flushleft}
\newpage



