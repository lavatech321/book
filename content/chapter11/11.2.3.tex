\setlength{\columnsep}{3pt}
\begin{flushleft}

	\begin{itemize}
		\item \textbf{this} keyword refers to the current object instance within a class.
		\item It is used to differentiate between class members and local variables.
		\item It is used to access or modify instance variables or invoke instance methods of the current object.
		\item \textbf{this} keyword can be used within constructors.
	\item Eg:
	\bigskip
		\codeblockfull{Test.java}{
			class Rectangle \{ \\
			\s	float l, w, h; \\
			\s	public Rectangle(int l, int w, int h) \{ \\
			\s \s		\textbf{this.l = l;} \\
			\s \s		\textbf{this.w = w;} \\
			\s \s		\textbf{this.h = h;} \\
			\s	\} \\
			\s	public float getArea() \{ \\
			\s \s		float area; \\
			\s \s		area = 2*(l*w) + 2*(l*h) + 2*(h*w); \\
			\s \s		return area; \\
			\s	\} \\
			\} \\
			public class Test3 \{ \\
			\s	public static void main(String[] args) \{ \\
			\s \s		Rectangle r1 = new Rectangle(3,4,5); \\
			\s \s		float area; \\
			\s \s		area = r1.getArea(); \\
			\s \s		System.out.println(area); \\
			\s	\} \\
			\} 
		}		
		
	\end{itemize}	
	
	
\end{flushleft}
\newpage



