\setlength{\columnsep}{3pt}
\begin{flushleft}
	
	\begin{itemize}
		\item Defines the \textbf{behavior or actions} that objects.
		\item Declared within a class and can access class attributes.
		\item Invoked using object of the class.
		\bigskip
		\syntaxblock{
			modifier returnType methodName(type1 arg1, type2 arg2, ...)
		}
		
		where,
		\begin{itemize}
			\item \textbf{access modifier} can be public, private, protected, default, static, final, abstract, synchronized, native or strictfp.
			\item \textbf{returnType:} Specifies the type of value returned. Can be a \textbf{primitive type}, an \textbf{object type}, or \textbf{void} if the method does not return any value.
			\item \textbf{methodName:} Method name that follow the Java naming conventions.
			\item \textbf{type:} Data type of parameter passed to the method.
			\item arg: Name given to each parameter.
		\end{itemize}
		
	\end{itemize}

	Let's see each part of method syntax in detail.
	
	\newpage
	
	\textbf{Access modifiers applicable on methods}
	\begin{itemize}
		\item \textbf{public:}
		\begin{itemize}
			\item Can be accessed from any other class or package.
			\item Eg:
			\bigskip
			\codeblockfull{Test.java}{
			class A \{ \\
			\s 		public void message() \{ \\
			\s \s			System.out.println("Time is money"); \\
			\s		\} \\
			\} \\
			public class Test \{ \\
			\s public static void main(String[] args) \{ \\
			\s \s			A a1 = new A(); \\
			\s \s			a1.message(); \\
			\s 		\} \\
				\}	
			}
			\bigskip
			\outputblock{
				Time is money
			}
		\end{itemize}
	
		\bigskip
		\item \textbf{private}:
		\begin{itemize}
			\item Can only be accessed within the same class. 
			\item It is not visible to other classes or packages.
			\item Eg:
			\bigskip
			\codeblockfull{Test.java}{
				class A \{ \\
				\s 	String msg; \\
				\s	\textbf{private void set()} \{ \\
				\s \s		\textbf{msg  = "Time is money";} \\
				\s	\textbf{\}} 
			}
			\newpage		
			\codecontinue{
				\s	public void get() \{ \\
				\s \s		\textbf{set();} \\
				\s \s		System.out.println(msg); \\
				\s	\} \\
				\} \\
				public class Test3 \{ \\
				\s public static void main(String[] args) \{ \\
				\s \s		A a1 = new A(); \\
				\s \s		a1.get(); \\
				\s	\} \\
				\}	
			}
			\bigskip
			
			\outputblock{
			Time is money
			}
		\end{itemize}
		\bigskip
		\textbf{protected}:
		\begin{itemize}
			\item Accessible within the same package or subclass. 
			\item It can be accessed from subclasses even if they are in a different package.
			
			\item Eg 1: Accessing from within subclass: 
			\bigskip
			\codeblockfull{Test.java}{
				class A \{ \\
				\s	protected void msg() \{ \\
				\s \s		System.out.println("Time is money"); \\
				\s	\} \\
				\s	public static void main(String[] args) \{ \\
				\s \s		A a1 = new A(); \\
				\s \s		a1.msg(); \\
				\s	\} \\
				\}
			}
			\newpage
			\outputblock{
				Time is money
			}
			
			
		\end{itemize}
		\bigskip
		\item \textbf{default (package-private):} 
		\begin{itemize}
			\item No access modifier specified is considered as the default attribute. 
			\item Can be accessed within the same package but not from other packages.
			\item Eg: Accessing from same class -
			\bigskip
			\codeblockfull{Test.java}{ 
				class A \{ \\
				\s \textbf{void msg()} \{ \\
				\s \s		System.out.println("Time is money"); \\
				\s	\} \\
				\} \\
				public class Test3 \{ \\
				\s	public static void main(String[] args) \{ \\
				\s \s		A a1 = new A(); \\
				\s \s		a1.msg(); \\
				\s	\} \\
				\}
			}
			\bigskip
			\outputblock{
				Time is money
			}
		\end{itemize}
		
		\bigskip
		\item \textbf{static}:
		\begin{itemize}
			\item A static method belongs to the class rather than an instance of the class. 
			\item  Can be called directly using the class name without creating an object of the class.
			\item Eg: Accessing method using class name -
			\bigskip
			\codeblockfull{Test.java}{
				class A \{ \\
				\s	static void msg() \{ \\
				\s \s		System.out.println("Time is money"); \\
				\s	\} \\
				\} \\
				public class Test3 \{ \\
				\s	public static void main(String[] args) \{ \\
				\s \s		A.msg(); \\
				\s	\}
				\}
			}
			\bigskip
			\outputblock{
				Time is money
			}
		\end{itemize}
		
		\bigskip
		\item \textbf{final}:
		\begin{itemize}
			\item Cannot be overridden by subclasses. 
			\item It provides the implementation that cannot be changed.
			\item Eg:
			\bigskip
			\codeblockfull{Test3.java}{
				class Parent \{ \\
				\s \textbf{public final void display()} \{ \\
				\s	System.out.println("Parent class final method"); \\
				\s	\} \\
				\}  \\
				class Child extends Parent \{ \\
				\s	// public void display() \{ \}   \\
				\} \\
				class FinalMethodExample \{ \\
				\s	public static void main(String[] args) \{ 
				
			}
			\newpage
			\codecontinue{
				\s \s		Parent parent = new Parent();  \\
				\s \s		parent.display();  \\
				\s \s		Child child = new Child(); \\
				\s \s		child.display();  \\
				\s	\} \\
				\}
			}
			\bigskip
			\outputblock{
				Parent class final method \\
				Parent class final method
			}
		\end{itemize}
	
		\bigskip
		\item \textbf{abstract}:
		\begin{itemize}
			\item An abstract method does not have an implementation and must be overridden by any concrete subclass.
			\item More on this in abstract class chapter
		\end{itemize}
		\bigskip
		\item \textbf{synchronized:}
		\begin{itemize}
			\item A synchronized method can be accessed by only one thread at a time, ensuring thread safety.
			\item More on this in threading chapter
		\end{itemize}
		\bigskip
		\item \textbf{native:} A native method is implemented in a language other than Java, typically using JNI (Java Native Interface). It provides a bridge between Java and other languages like C or C++.
		\bigskip
		\item \textbf{strictfp:} 
		\begin{itemize}
			\item Enforces strict floating-point precision for floating-point calculations. 
			\item It ensures consistent results across different platforms.
			\item Eg:
			\bigskip
			\codeblockfull{A.java}{
			strictfp class A \{ \\
			\s	public strictfp double cal(double a, double b) \{ \\
			\s \s		return a * b / Math.sqrt(a + b); \\
			\s	\}
				\\
			\s	public static void main(String[] args) \{ \\
			\s \s		A example = new A(); \\
			\s \s		double result = example.cal(10.5, 5.3); \\
			\s \s		System.out.println("Result: " + result); \\
			\s	\}
			}
			\bigskip
			\outputblock{
				Result: 14.000276895995912
			}
			
		\end{itemize}
	
	\end{itemize}
	
	\newpage
	
	\textbf{Method argument types}
	\begin{itemize}
		\item Method arguments specify the types of values that can be passed to a method when it is invoked. 
		\item Method arguments define the parameters that a method expects to receive in order to perform its functionality. 
		i\item Here are some common types of method arguments in Java:
		
		\begin{itemize}
			\item \textbf{Primitive types}: Includes int, double, long, short, byte, char, float, 
			boolean.

			\codeblock{
				class Person \{ \\
				\s	String name; \\
				\s	int age; \\
				\s	long phno; \\
				\s	public void setDetails(String n, int a, long p) \{ \\
				\s \s		name = n; \\
				\s \s		age = a; \\
				\s \s		phno = p; \\
				\s	\} \\
				\s	public static void main(String[] args) \{ \\
				\s \s		Person p = new Person(); \\
				\s \s		p.setDetails("Ravi",23,987123546L); \\
				\s \s		System.out.println("Name : " + p.name); \\
				\s \s		System.out.println("Age : " + p.age); \\
				\s 	System.out.println("Phone number : " + p.phno); \\
				\s	\} \\
				\}
			}
		\bigskip
			\outputblock{
			Name : Ravi \\
			Age : 23 \\
			Phone number : 987123546
			}
			\bigskip
			\item \textbf{Reference types}: Refer to objects or classes or arrays or interfaces or enums. Eg:
			\bigskip
			\codeblock{
				class Person \{ \\
				\s	String name="Raman"; \\
				\s	int age=56; \\
				\s	long phno=785642312L; \\
				\} \\
				class Test \{ \\
				\s	public static void displayPerson(Person person)\{ \\ 
				\s \s		System.out.println(person.name); \\
				\s \s		System.out.println(person.age); \\
				\s \s		System.out.println(person.phno); \\
				\s	\} \\
				\\
				\s	public static void main(String[] args) \{ \\
				\s \s		Person p = new Person(); \\
				\s \s		displayPerson(p); \\
				\s	\} \\
				\}
			}
			\bigskip
			\outputblock{
				Raman \\
				56 \\
				785642312
			}
			\bigskip
			\item \textbf{Varargs}: 
			\begin{itemize}
				\item Allows a method to accept a variable number of arguments of the same type. \item It is denoted by an ellipsis (...) after the parameter type.
				\item Eg:
				\newpage
				\codeblockfull{Test.java}{
					class Test \{ \\
					\s \textbf{public static void nos(int... numbers)} \{ \\
							\textbf{System.out.println("Arguments: " + numbers.length);}  \\
					\s \s		System.out.print("Numbers: "); \\
					\s \s		for (int number : numbers) \{ \\
					\s \s	\s		System.out.print(number + " "); \\
					\s	\s	\} \\
					\s	\s	System.out.println(); \\
					\s	\} \\
						\\
						public static void main(String[] args) \{ \\
					\s \s		nos(1, 2, 3); \\
					\s \s		nos(10, 20, 30, 40, 50);  \\
					\s \s		nos(); \\
					\s	\} \\
					\}
				}
				\bigskip
				\outputblock{
					Arguments: 3 \\
					Numbers: 1 2 3  \\
					Arguments: 5 \\
					Numbers: 10 20 30 40 50  \\
					Arguments: 0 \\
					Numbers: 	
				}
			\end{itemize}
			
			
		\end{itemize}
	\end{itemize}

	\newpage
	
	\textbf{Method return types}
	\begin{itemize}
		\item The return type of a method specifies the type of value that the method will return when it is executed. 
		\item The return type is declared in the method syntax, immediately before the method name. 
		\item Here are some common return types in Java:
		\begin{itemize}
			\item \textbf{Primitive types}: Includes int, double, long, short, byte, char, float, 
			boolean.
			
			\codeblock{
				class Person \{ \\
				\s	String name; \\
				\s	int age; \\
				\s	long phno; \\
				\s	public void setDetails(String n, int a, long p) \{ \\
				\s \s		name = n; \\
				\s \s		age = a; \\
				\s \s		phno = p; \\
				\s	\} \\
				\s	public static void main(String[] args) \{ \\
				\s \s		Person p = new Person(); \\
				\s \s		p.setDetails("Ravi",23,987123546L); \\
				\s \s		System.out.println("Name : " + p.name); \\
				\s \s		System.out.println("Age : " + p.age); \\
				\s 	System.out.println("Phone number : " + p.phno); \\
				\s	\} \\
				\}
			}
			\bigskip
			\outputblock{
				Name : Ravi \\
				Age : 23 \\
				Phone number : 987123546
			}
			\bigskip
			\item \textbf{Reference types}: Refer to objects or classes or arrays (int[] or String[]) or interfaces or enums. Eg:
			\bigskip
			\codeblock{
				class Person \{ \\
				\s	String name; \\
				\s	int age; \\
				\s	long phno; \\
				\}
				class Test \{ \\
				\s	public static Person setPerson(Person person)\{  \\
				\s \s		person.name="Raman"; \\
				\s \s		person.age=67; \\
				\s \s		person.phno=785642312L; \\
				\s \s		return person; \\
				\s	\}
				\s	public static void main(String[] args) \{ \\
				\s \s		Person p; \\
				\s \s		p = setPerson(new Person()); \\
				\s \s		System.out.println("Name:"+p.name); \\
				\s \s		System.out.println("Age:"+p.age); \\
				\s \s		System.out.println("Phone no:"+p.phno); \\
				\s	\} \\
				\}
			}
			\bigskip
			\outputblock{
				Name:Raman \\
				Age:67 \\
				Phone no:785642312
			}
			\bigskip
			
		\end{itemize}
		
		\newpage
		\item \textbf{void}:
		\begin{itemize}
			\item Indicates that the method does not return any value. 
			\item Eg:
			\bigskip
			\codeblockfull{Test.java}{
				class Person \{ \\
				\s	String name="Raman"; \\
				\s	int age=56; \\
				\s	long phno=785642312L; \\
				\} \\
				class Test \{ \\
				\s	public static void displayPerson(Person person)\{ \\
				\s \s		System.out.println(person.name); \\
				\s \s		System.out.println(person.age); \\
				\s \s		System.out.println(person.phno); \\
				\s	\} \\
				\\
				\s	public static void main(String[] args) \{ \\
				\s \s		Person p = new Person(); \\
				\s \s		displayPerson(p); \\
				\s	\} \\
				\}
			}
		
			\bigskip
			\outputblock{
				Raman \\
				56 \\
				785642312
			}
		\end{itemize}
		
	\end{itemize}	
	
\end{flushleft}

\newpage

