\setlength{\columnsep}{3pt}
\begin{flushleft}
	
	\begin{itemize}
		\item The degree of dependency between the components is called \textbf{coupling}.
		\item If dependency is more, then it is considered as \textbf{tightly coupling}.
		\item If dependency is less, then it is considered as \textbf{lossly coupling}.
		\item Eg:
		\newpage
		\codeblockfull{A.java}{
			class A \{ \\
			\s	static int i = B.j; \\ 
			\s	public static void main(String[] args) \{ \\
			\s \s		A a = new A(); \\
			\s \s		System.out.println(a.i); \\
			\s	\} \\
			\} \\
			class B \{ \\
			\s	static int j = C.k; \\
			\} \\
			class C \{ \\
			\s	static int k = D.m1(); \\
			\} \\
			class D \{ \\
			\s	public static int m1() \{ \\
			\s \s		return 10; \\
			\s	\} \\
			\}	
		}
		\bigskip
		\outputblock{
		10
		}
		\bigskip
		\item \textbf{Tightly coupling is not good} programming practise.
		\item Problem with tighly coupling:
		\begin{itemize}
			\item Without affecting remaining components we can modify any component and hence enhaacement will become difficult.
			\item It suppresses reuseability. 
			\item It reduces mainitainabilty of the application.
			\item Hence we have to maintain dependency between the components as less as possible i.e lossly coupling is a good programming practice.
		\end{itemize}
		
	\end{itemize}	
	
\end{flushleft}
