\setlength{\columnsep}{3pt}
\begin{flushleft}

		\begin{itemize}
			\item \textbf{this()} is used to invoke one constructor from another constructor within the same class. 
			\item It allows constructors with different parameter lists to call each other and \textbf{reuse initialization code}.
			\item The this() invocation \textbf{must be the first statement in the constructor body}.
			\item It is used to \textbf{avoid code duplication when multiple constructors} in a class need to perform common initialization tasks.
			\item Eg:
			\bigskip
			\codeblockfull{Test.java}{
				class Rectangle \{ \\
				\s	float l, w, h; \\
				\s	public Rectangle() \{ \\
				\s \s		\textbf{this(0,0,0);} \\
				\} \\
				\s 	public Rectangle(int width, int height) \{ \\
				\s \s 	\textbf{this(0,width,height);} \\
				\s	\} \\
				\s	public Rectangle(int length, int width, int height) \{ \\
				\s \s 	this.l = length; \\
				\s \s 	this.w = width; \\
				\s \s 	this.h = height; \\
				\s	\} \\
				\s 	public float getArea() \{ \\
				\s \s 	float area; \\
				\s \s 	area = 2*(l*w) + 2*(l*h) + 2*(h*w); \\
				\s \s 	return area; \\
				\s	\} \\
				\} 
			}
		
			\newpage
			
			\codecontinue{
			public class TestRect \{ \\
			\s	public static void main(String[] args) \{ \\
			\s \s 	Rectangle r1 = new Rectangle(3,4,5); \\
			\s \s 	float area; \\
			\s \s 	area = r1.getArea(); \\
			\s \s 	System.out.println(area); \\
			\\
			\s \s 	Rectangle r2 = new Rectangle(); \\
			\s \s 	float area2; \\
			\s \s 	area2 = r2.getArea(); \\
			\s \s 	System.out.println(area2); \\
			\s \} \\
			\}	
			}
			\bigskip
			\outputblock{
			94.0 \\
			0.0
			}
			
		\end{itemize}
	
		\noteblock{
			\begin{itemize}
				\item The first line inside every constructor should be either super() or this() but not both at a time.
				\item We can use super() or this() only inside constructor.
			\end{itemize}	
		}
		
	
\end{flushleft}

\newpage

