\setlength{\columnsep}{3pt}
\begin{flushleft}
	
	\begin{itemize}
		\item An instance initializer block (or instance block) is a block of code defined within a class, executed when an instance of the class is created. 
		\item It initialises instance variables or perform other initialization tasks for each object of the class.
		\item It is \textbf{executed before the constructor of the class}. 
		\item It is used when you have multiple constructors or when you want to perform common initialization logic.
		\item Eg:
		\codeblockfull{Demo.java}{
			public class Demo \{ \\
			\s	int value;  \\
			\s \{	 	\\
			\s \s value = 8080;	\\
			\s \}	\\
			\s public Demo() \{\}	\\
			\s public Demo(int value) \{	\\
			\s \s this.value = value;		\\
			\s \}	\\
			\s public static void main(String[] args) \{	\\
			\s \s Demo d1 = new Demo();	\\
			\s \s Demo d2 = new Demo(10);		\\
			\s \s System.out.println(d1.value);	\\
			\s \s System.out.println(d2.value);	\\
			\s \}	\\
			\}
		}
		\bigskip
		\outputblock{
			8080 \\
			10
		}
				
	\end{itemize}	
	
	
\end{flushleft}

\newpage
