\setlength{\columnsep}{3pt}
\begin{flushleft}
	
	\begin{itemize}
		\item \textbf{super keyword} is used to refer parent class.
		\item It can be used to access members of parent class from child class.
		\item Eg:
			\codeblockfull{Car.java}{
				class Vehicle \{ \\
				\s	int maxSpeed; \\
				\s	\textbf{public void displayInfo()} \{ \\
				\s	System.out.println("Max Speed: " + maxSpeed); \\
				\s	\} \\
				\} \\
				public class Car extends Vehicle \{  \\
				\s	int numWheels; \\
				\s	\textbf{public void displayInfo()} \{ \\
				\s \s		\textbf{super.displayInfo();} \\
				\s \s		System.out.println("Wheels: "+ numWheels); \\
				\s	\} \\
				\s	public static void main(String[] args) \{ \\
				\s \s		Car c1 = new Car(); \\
				\s \s		c1.displayInfo(); \\
				\s	\} \\
				\}
			}
		
			\bigskip
			\outputblock{
				Max Speed: 0 \\
				Wheels: 0
			}
		
	\end{itemize}	
	
	
\end{flushleft}

\newpage
