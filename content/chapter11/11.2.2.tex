\setlength{\columnsep}{3pt}
\begin{flushleft}

	\begin{itemize}
		\item A constructor has the code that runs when you use \textbf{new} operator. 
		\item Main purpose of constructor is to perform \textbf{initialsation} of an object.	
		\item Rules for creation of constructor:
		\begin{itemize}
			\item Constructor has the \textbf{same name as the class.} 
			\item Access modifiers for constructors allowed are \textbf{public, private, protected and default or default (which means no access modifier at all)}.
			\item Constructor \textbf{does not have a return type, not even void.}
			\item Constructors can accept parameters to initialize the object's attributes.
			\bigskip
			\syntaxblock{
				public/private/protected/default  Classname(args) \{ \\
				\s // Code here \\
				\}
			}
		\end{itemize}

		\item Eg:
		\bigskip
		\codeblockfull{Test.java}{
			public class Test \{ \\
			\s	\textbf{public Test()} \{ \\
			\s \s		System.out.println("Constructor called"); \\
			\s	\} \\
			\\
			\s	public static void main(String[] args) \{ \\
			\s \s		\textbf{Test t1 = new Test();} \\
			\s	\} \\
			\}
		}
		
		\bigskip
		\outputblock{
			Constructor called
		}
		
		\newpage
		\item Every class has a constructor, even if you don’t write it yourself.
		
		\bigskip
		\quest{But where is the constructor? If we didn’t write it, who did?}
		{
			\begin{itemize}
				\item You can write a constructor for your class (we’re about to do
				that), but if you don’t, \textbf{the compiler writes one for you!}
				\item The default constructor is always a no-arg constructor as shown below:
				\codeblock{
					class Test \{ \\
					\s // No-arg constructor \\
					\s public Test() \{\} \\
					\}
				}
			\end{itemize}
		}
		
	
		
	\end{itemize}

	
	
	

\end{flushleft}
\newpage



