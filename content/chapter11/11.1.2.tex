\setlength{\columnsep}{3pt}
\begin{flushleft}

	\begin{itemize}
		\item Advantage of has-a relationship is \textbf{reuseability} of the code.
		\item There are 2 types of Has-A relationship:
		\begin{itemize}
					\item \textbf{Aggregation:}
			\begin{itemize}
				\item Aggregation is a weaker form of the "has-a" relationship.
				\item It is a one-way relationship and called unidirectional association. 
				\item For example, Bank can have employees but vice versa is not possible.
				\bigskip
				\codeblock{
					class Bank \{ \\\
					\s String nameOfBank; \\
					\s	Bank(String nameOfBank)	\{ \\
					\s \s		this.nameOfBank = nameOfBank; \\
					\s \} \\
					\s public void displayAllDetails(Customer customer) \{ \\
					\s \s		System.out.println("Bank = "+ nameOfBank); \\
					\s \s		System.out.println("Customer = "+ customer.nameOfCustomer); \\
					\s \} \\
					\} \\
					class Customer \{ \\
					\s	String nameOfCustomer; \\
						\s	Customer(String nameOfCustomer) \{ \\
					\s \s		this.nameOfCustomer = nameOfCustomer; \\
					\s	\} \\
					\} 
				}
				\newpage
				\codecontinue{
					class Branch \{ \\
					\s	public static void main(String arg[]) \\
					\s	\{ \\
					\s \s		Bank bank = new Bank("AXIS"); \\
					\s \s		Customer customer = new Customer("Ram"); \\
					\s \s 		bank.displayAllDetails(customer); \\
					\s	\} \\
					\}
				}
			\end{itemize}
			\bigskip
			\item \textbf{Composition:}
			\begin{itemize}
				\item Composition is a strong form of the "has-a" relationship.
				\item In composition two entities are highly dependent on each other. 
				\item One entity cannot exist without the other.
				\item It represents a \textbf{part-of} relationship.
				\item Eg: a car has an engine
				\bigskip
				\codeblock{
					class Car \{ \\
					\s	private final Engine engine;   \\
					\s	String nameOfCar; \\
					\s	String model; \\
					\s	public Car(String nameOfCar, String model) \{ \\
					\s \s		engine  = new Engine("POWERHIGH", "ABC"); \\
					\s \s		this.nameOfCar = nameOfCar; \\
					\s \s		this.model = model; \\
					\s	\}	 
				}	
			
			\newpage
			
			\codecontinue{
					\s	public void showAlldetails() \{ \\
			\s 	System.out.println("Car ="+nameOfCar); \\
			\s 		System.out.println("Model ="+model); \\
			\s 		System.out.println("Engine used ="+engine.typeOfEngine); \\
			\s 		System.out.println("Engine name ="+engine.nameOfEngine); \\
			\s	\} \\
			\} \\
			\\
			class Engine  \{ \\
			\s	String typeOfEngine; \\
			\s	String nameOfEngine; \\
			\s	Engine(String typeOfEngine, String nameOfEngine) \{ \\
			\s \s		this.typeOfEngine = typeOfEngine; \\
			\s \s		this.nameOfEngine = nameOfEngine; \\
			\s	\} \\
			\} \\
			\\
			public class Test \{ \\
			\s	public static void main(String arg[]) \{ \\
			\s \s		Car car = new Car("BMW", "5X"); \\
			\s \s		car.showAlldetails(); \\
			\s 	\} \\
			\}
			}	
				
			\end{itemize}
			
		\end{itemize}
	
	\end{itemize}


\end{flushleft}
\newpage


