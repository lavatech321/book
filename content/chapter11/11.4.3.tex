\setlength{\columnsep}{3pt}
\begin{flushleft}
	
	\begin{itemize}
		\item Allows a subclass to provide its own implementation of a method that is already defined in its superclass. 
		\item When a method in the subclass has the \textbf{same name, return type, and parameter list} as a method in the superclass, it is said to override the superclass method.
		
		\item Key points about method overriding:
		
		\begin{itemize}
			\item \textbf{Inheritance:} Based on the concept of inheritance, where a subclass inherits methods and fields from its superclass.
			
			\item \textbf{Signature:} The overriding method must have the same method signature (name, return type, and parameter list) as the method it is overriding.
			
			\item \textbf{Access modifier:} The overriding method cannot have a more restrictive access modifier than the method it is overriding. It can have the same or a more permissive access modifier.
			
			\item \textbf{@Override annotation:} It is a good practice to use the @Override annotation when overriding a method.
			
		\end{itemize}
		\newpage
		\item Eg:
		\bigskip
		\codeblockfull{Test.java}{
			class Animal \{ \\
			\s	public void makeSound() \{ \\
			\s \s		System.out.println("Animal makes a sound"); \\
			\s	\} \\
			\} \\
			\\
			class Cat extends Animal \{ \\
			\s	\textbf{@Override} \\
			\s 	\textbf{public void makeSound()} \{ \\
			\s \s		System.out.println("Cat meows"); \\
			\s	\} \\
			\} \\
			\\
			class Dog extends Animal \{ \\
			\s 	\textbf{@Override} \\
			\s	\textbf{public void makeSound()} \{ \\
			\s \s		System.out.println("Dog barks"); \\
			\s	\} \\
			\} \\
			\\
			class Test \{ \\
			\s	public static void main(String[] args) \{ \\
			\s \s		Animal animal1 = new Cat(); \\
			\s \s		Animal animal2 = new Dog(); \\
					\\
			\s \s		\textbf{animal1.makeSound();}  \\
			\s \s		\textbf{animal2.makeSound();}  \\
			\s	\} \\
			\}
		}
		\outputblock{
			Cat meows \\
			Dog barks
		}
	\end{itemize}
	
	\textbf{Method overriding with respect to var\_arg}
	
	\begin{itemize}
		\item For overriding wrt var\_arg method, the overriding method must have the \textbf{same method name, return type, and parameter types} as the overridden method.
		\item Eg:
		\bigskip
		\codeblockfull{Main.java}{
			class BaseClass \{ \\
			\s public void printValues(String... values) \{ \\
			\s \s		System.out.println("BaseClass - printValues:"); \\
			\s \s		for (String value : values) \{ \\
			\s \s \s			System.out.println(value); \\
			\s \s		\} \\
			\s	\} \} \\
			class SubClass extends BaseClass \{ \\
			\s	@Override \\
			\s	public void printValues(String... values) \{ \\
			\s \s		System.out.println("SubClass - printValues:"); \\
			\s \s		for (String value : values) \{ \\
			\s \s \s			System.out.println(value); \\
			\s \s		\} \\
			\s	\} \} \\
			\\
			class Main \{ \\
			\s	public static void main(String[] args) \{ \\
			\s \s		BaseClass base = new BaseClass(); \\
			\s \s		base.printValues("Hello", "World");  
		}
		
		\codecontinue{
			\s \s		SubClass sub = new SubClass(); \\
			\s \s		sub.printValues("Hello", "World");   \\
			\s \s		BaseClass polymorphic = new SubClass(); \\
			\s \s		polymorphic.printValues("Hello", "World"); \\  
			\s	\} \}
		}
		\bigskip
		\outputblock{
			BaseClass - printValues: \\
			Hello \\
			World \\
			SubClass - printValues: \\
			Hello \\
			World \\
			SubClass - printValues: \\
			Hello \\
			World 
		}
	\end{itemize}
	
\end{flushleft}

\newpage
