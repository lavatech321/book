\setlength{\columnsep}{3pt}
\begin{flushleft}

\begin{itemize}
	\item Inheritance means putting common code in a class\textbf{(superclass)} and tell other classes\textbf{(subclass)} that the common class is their superclass. 
	\item The subclass \textbf{extends} the superclass.
	\item The subclass inherits the members (i.e instance variables and methods) of the superclass.
	\item Eg:
	\newimage{0.34}{content/chapter11/images/new.png}

	\newpage	
	\item Instance variables are not overridden.
	\bigskip
	\item \textbf{IS-A relationship}: 
	\begin{itemize}
		\item Inheritance is also called \textbf{IS-A} relationship.
		\item Eg: FriedEggMan IS-A SuperHero
	\end{itemize}
	\bigskip
	\item The \textbf{extends} keyword is used to implement IS-A relationship.
	\bigskip
	\syntaxblock{
		class A \{\} \\
		\\
		class B extends A \{\}
	}

	\item Superclass reference can be used to hold subclass object. 	
	\item Eg:
	
	\codeblock{
		class A \{ \\
		\s	void displayA() \{ \\
		\s \s		System.out.println("This is A"); \\
		\s	\} \\
		\} \\
		class B \textbf{extends} A \{  \\
		\s	void displayB() \{  \\
		\s \s		System.out.println("This is B"); \\
		\s 	\} \\
		\} \\
		class Test \{ \\
		\s	public static void main(String[] args) \{ \\
		\s \s		A a = new A(); \\
		\s \s		a.displayA(); \\
		\s \s		a.displayB();   \xmark \s  //Results in error
	}
	\newpage
	\codecontinue{
		\s \s		A a2 = new B(); \\
	\s \s		a2.displayA(); \\
	\s \s		a2.displayB();  \xmark \s //Results in error \\
	\\		
	\s \s		B b = new B(); \\
	\s \s		b.displayA(); \\
	\s \s 		b.displayB(); \\
	\\
	\s			B b2 = new A();  \xmark \s //Results in error \\
	\s \} \\
	\}
	}

	\bigskip
	\noteblock{
	\begin{itemize}
		\item Using parent refernece, you cant call child specific methods.
		\item Parent reference can be used to hold child object, but using that refernece, we cant call child specific methods.
		\item Child reference cannot be used to hold parent object.
	\end{itemize}
	}

	
\end{itemize}
	
\end{flushleft}

\newpage

