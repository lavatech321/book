\setlength{\columnsep}{3pt}
\begin{flushleft}

\begin{itemize}
	\item Inheritance makes  putting common code in a parent class and inherit the code in child class.
	\item The child class inherits the members (i.e attributes \& methods) of the parent class.
	\newimage{0.32}{content/chapter11/images/new10.png}

	\newpage	
	\item \textbf{IS-A relationship}: 
	\begin{itemize}
		\item Inheritance is also called \textbf{IS-A} relationship.
		\item Eg: SuperMan IS-A SuperHero
	\end{itemize}
	\bigskip
	\item The \textbf{extends} keyword is used to implement IS-A relationship.
	\bigskip
	\syntaxblock{
		class A \{\} \\
		class B extends A \{\}
	}

	\item Parent class reference can hold child class object, but vice versa is not possible. Eg:
	
	\codeblock{
		class A \{ \\
		\s	void displayA() \{ \\
		\s \s		System.out.println("This is A"); \\
		\s	\} \\
		\} \\
		class B \textbf{extends} A \{  \\
		\s	void displayB() \{  \\
		\s \s		System.out.println("This is B"); \\
		\s 	\} \\
		\} \\
		class Test \{ \\
		\s	public static void main(String[] args) \{ \\
		\s \s		A a = new A(); \\
		\s \s		a.displayA(); \\
		\s \s		a.displayB();   \xmark \s  //Results in error \\
		\\
		\s \s		A a2 = new B(); \\
		\s \s		a2.displayA(); \\
		\s \s		a2.displayB();  \xmark \s //Results in error \\		
		
	}
	\newpage
	\codecontinue{
	\s \s		B b = new B(); \\
	\s \s		b.displayA(); \\
	\s \s 		b.displayB(); \\
	\\
	\s 			B b2 = new A();  \xmark \s //Results in error \\
	\s \} \\
	\}
	}


	
\end{itemize}
	
\end{flushleft}



