\setlength{\columnsep}{3pt}
\begin{flushleft}
	
	\begin{itemize}
		\item \textbf{super()} method is used to invoke a superclass constructor from a subclass constructor. 
		
		\item Key points about super():
		\begin{itemize}
			\item Must be the \textbf{first statement in the subclass} constructor body.
			\item The compiler automatically inserts a call to the superclass's default (no-argument) constructor.
			\item super() can only be used only within a constructor.
		\end{itemize}	
		\textbf{Eg: Superclass constructors with arguments}
		\bigskip
		\codeblockfull{Test.java}{
			class Animal \{ \\
			\s 	String name; \\
			\s	Animal(String name) \{ \\
			\s \s		this.name = name; \\
			\s	\} \\ 
			\} 		\\
			class Duck extends Animal \{ \\
			\s	int size; \\
			\s	public Duck(String name, int size) \{ \\
			\s \s		\textbf{super(name);} \\
			\s \s		\textbf{this.size = size;} \\
			\s	\} \\
			\} \\
			public class Test \{ \\
			\s	public static void main(String[] args) \{ \\
			\s \s		Duck d1 = new Duck("Duckling",4); \\
			\s \s		System.out.println(d1.name); // Output: Duckling \\
			\s \s		System.out.println(d1.size); // Output: 4 \\
			\s	\}	 \\
			\}
		}
		
	\end{itemize}
	
\end{flushleft}

\newpage