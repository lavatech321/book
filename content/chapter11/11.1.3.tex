\setlength{\columnsep}{3pt}
\begin{flushleft}
	\textbf{rpm}: Used to install packages with \textbf{.rpm} extension.
	\begin{tcolorbox}[breakable,notitle,boxrule=-0pt,colback=pink,colframe=pink]
		\color{black}
		\fontdimen2\font=9pt
		Syntax: rpm option package\_name
		\fontdimen2\font=4pt
	\end{tcolorbox}

	Options with \textbf{"rpm"} command:
	\begin{itemize}
		\item \textbf{-i}: Installs the package, if it isn't already installed
		\item \textbf{-v}: Provide more detailed output
		\item \textbf{-h}: Print hash marks to display progress
		\begin{tcolorbox}[breakable,notitle,boxrule=-0pt,colback=pink,colframe=pink]
			\color{black}
			\fontdimen2\font=9pt
			Syntax: rpm -ivh package\_name
			\fontdimen2\font=4pt
		\end{tcolorbox}
		Eg: Download package named \textbf{"cvs"} and install it:
		\bigskip
		\begin{tcolorbox}[breakable,notitle,boxrule=-0pt,colback=black,colframe=black]
			\color{yellow}
			\fontdimen2\font=9pt
			\# For rhel/centos7, download \& install cvs package
			\newline
			\color{green}
			\# wget https://rpmfind.net/linux/centos/7.9.2009/os/x86\_64
			/Packages/cvs-1.11.23-35.el7.x86\_64.rpm
			\newline
			\# rpm -ivh cvs-1.11.23-35.el7.x86\_64.rpm
			\newline
			\newline
			\color{yellow}
			\fontdimen2\font=9pt
			\# For rhel/centos8, download \& install cvs package
			\color{green}
			\newline
			\# wget https://rpmfind.net/linux/epel/8/Everything/x86\_64/
			Packages/c/cvs-1.11.23-52.el8.x86\_64.rpm
			\newline
			\# rpm -ivh cvs-1.11.23-52.el8.x86\_64.rpm 
			\fontdimen2\font=4pt
		\end{tcolorbox}
		\bigskip
		\bigskip
		\item \textbf{-U}: Upgrades any existing package or installs it if an earlier version
		isn't already installed.
		\begin{tcolorbox}[breakable,notitle,boxrule=-0pt,colback=pink,colframe=pink]
			\color{black}
			\fontdimen2\font=9pt
			Syntax: rpm -U package\_name
			\fontdimen2\font=4pt
		\end{tcolorbox}
		\bigskip
		\bigskip
		\item \textbf{-F}: Upgrades only existing packages. It does not install a package if it wasn't previously installed.
		\begin{tcolorbox}[breakable,notitle,boxrule=-0pt,colback=pink,colframe=pink]
			\color{black}
			\fontdimen2\font=9pt
			Syntax: rpm -F package\_name
			\fontdimen2\font=4pt
		\end{tcolorbox}
		\bigskip
		\bigskip
		\item \textbf{-e}: Un-install a package.
		\begin{tcolorbox}[breakable,notitle,boxrule=-0pt,colback=pink,colframe=pink]
			\color{black}
			\fontdimen2\font=9pt
			Syntax: rpm -e package\_name
			\fontdimen2\font=4pt
		\end{tcolorbox}
		Eg: Un-install package named \textbf{"cvs"}.
		\bigskip
		\begin{tcolorbox}[breakable,notitle,boxrule=-0pt,colback=black,colframe=black]
			\color{white}
			\fontdimen2\font=9pt
			\color{green}
			\# rpm -e cvs
			\fontdimen2\font=4pt
		\end{tcolorbox}
		\bigskip
		\begin{tcolorbox}[breakable,notitle,boxrule=-0pt,colback=yellow,colframe=yellow]
			\color{black}
			\fontdimen2\font=9pt
			Note: \textbf{rpm -e} fails with an error if there's some dependent package issues.
			\fontdimen2\font=4pt
		\end{tcolorbox}
	
		\bigskip
		\bigskip
		\item \textbf{--nodeps}: Ignore dependency error and uninstall the package (which may break the package dependent on it).
		\bigskip
		\begin{tcolorbox}[breakable,notitle,boxrule=-0pt,colback=pink,colframe=pink]
			\color{black}
			\fontdimen2\font=9pt
			Syntax: rpm -e ---nodeps package\_name
			\fontdimen2\font=4pt
		\end{tcolorbox}
		
		Eg: 
		\bigskip
		\begin{tcolorbox}[breakable,notitle,boxrule=-0pt,colback=black,colframe=black]
			\color{white}
			\fontdimen2\font=9pt
			\color{green}
			\# rpm -e ---nodeps cvs
			\fontdimen2\font=4pt
		\end{tcolorbox}
		
		\bigskip
		\bigskip
		\item \textbf{-qa}: Lists all installed packages. Here \textbf{-q} stands for \textbf{query} and \textbf{-a} stands for all.
		\bigskip
		\begin{tcolorbox}[breakable,notitle,boxrule=-0pt,colback=pink,colframe=pink]
			\color{black}
			\fontdimen2\font=9pt
			Syntax: rpm -qa
			\fontdimen2\font=4pt
		\end{tcolorbox}
		
		
		\bigskip
		\bigskip
		\item \textbf{-qf}: Identifies the package associated with a specifc file/directory.
		\bigskip
		\begin{tcolorbox}[breakable,notitle,boxrule=-0pt,colback=pink,colframe=pink]
			\color{black}
			\fontdimen2\font=9pt
			Syntax: rpm -qf /path/to/file
			\fontdimen2\font=4pt
		\end{tcolorbox}
		Eg: Find package associated with \textbf{/etc/ssh} file or \textbf{/etc} folder:
		\bigskip
		\begin{tcolorbox}[breakable,notitle,boxrule=-0pt,colback=black,colframe=black]
			\color{white}
			\fontdimen2\font=9pt
			\color{green}
			\# rpm -qf /etc/ssh
			\newline
			\# rpm -qf /etc
			\fontdimen2\font=4pt
		\end{tcolorbox}
		

		\bigskip
		\bigskip
		\item \textbf{-qc}: Lists configuration files that comes with the package installation.
		\bigskip
		\begin{tcolorbox}[breakable,notitle,boxrule=-0pt,colback=pink,colframe=pink]
			\color{black}
			\fontdimen2\font=9pt
			Syntax: rpm -qc packagename
			\fontdimen2\font=4pt
		\end{tcolorbox}
		Eg: Find all configuration files associated with package \textbf{openssh}:
		\bigskip
		\begin{tcolorbox}[breakable,notitle,boxrule=-0pt,colback=black,colframe=black]
			\color{white}
			\fontdimen2\font=9pt
			\color{green}
			\# rpm -qc openssh
			\fontdimen2\font=4pt
		\end{tcolorbox}

		\bigskip
		\bigskip
		\item \textbf{-qi}: Displays basic information for package name.
		\bigskip
		\begin{tcolorbox}[breakable,notitle,boxrule=-0pt,colback=pink,colframe=pink]
			\color{black}
			\fontdimen2\font=9pt
			Syntax: rpm -qi packagename
			\fontdimen2\font=4pt
		\end{tcolorbox}
		Eg: Find basic information associated with package \textbf{openssh}:
		\bigskip
		\begin{tcolorbox}[breakable,notitle,boxrule=-0pt,colback=black,colframe=black]
			\color{white}
			\fontdimen2\font=9pt
			\color{green}
			\# rpm -qi openssh
			\fontdimen2\font=4pt
		\end{tcolorbox}

		\bigskip
		\bigskip
		\item \textbf{-qR}: Display all package dependencies.
		\bigskip
		\begin{tcolorbox}[breakable,notitle,boxrule=-0pt,colback=pink,colframe=pink]
			\color{black}
			\fontdimen2\font=9pt
			Syntax: rpm -qR packagename
			\fontdimen2\font=4pt
		\end{tcolorbox}
		Eg: Display all dependency associated with package \textbf{openssh}:
		\bigskip
		\begin{tcolorbox}[breakable,notitle,boxrule=-0pt,colback=black,colframe=black]
			\color{white}
			\fontdimen2\font=9pt
			\color{green}
			\# rpm -qR openssh
			\fontdimen2\font=4pt
		\end{tcolorbox}


		
	
	\end{itemize}
\end{flushleft}
\newpage


