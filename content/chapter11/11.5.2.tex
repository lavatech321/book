\setlength{\columnsep}{3pt}
\begin{flushleft}
	
	\begin{itemize}
		\item Data hiding refers to encapsulating data within a class and controlling its visibility and accessibility from outside the class. 
		\item Below code shows data being directly accessible to the objects and it's not data hidiing:
		
		\codeblockfull{Test.java}{
			class Sample \{ \\
			\s int no; \\
			\s String name; \\
			\} \\
			class Test \{ \\
			\s public static void main(String[] args) \{ \\
			\s \s Sample s1 = new Sample(); \\
			\s \s s1.no = 1;   // \color{red} No Data hiding \\ \color{black}
			\s \s s1.name = "Apples"; // \color{red} No Data hiding \\
			\s	\} \\
			\}
		}
		\bigskip
		\item In order to hide the data, the instance variable should be declared as \textbf{private}.
		
		\item Eg: You can create public methods \textbf{setter()} and \textbf{getter()} to get and set private instance variables as shown below:
		
		\codeblockfull{Test.java}{
			class Sample \{ \\
			\s private int no; \\
			\s private String name; \\
			\s 	public void setter(int no, String name) \{ \\
			\s \s		this.no = no; \\
			\s \s		this.name = name; \\
			\s	\} 
		}
		\newpage
		\codecontinue{
			\s	public void getter() \{ \\
			\s \s		System.out.println(no + ": " + name); \\
			\s	\} \\
			\} \\
			class Test \{ \\
			\s	public static void main(String[] args) \{ \\
			\s \s		Sample s1 = new Sample(); \\
			\s \s		s1.setter(0,"Apple");   // Data hidden \\
			\s \s		s1.getter(); \\
			\s	\} \\
			\}	
		}
		\bigskip
		\outputblock{
		0: Apple
		}
	
		
		\item It makes variables and methods safe from outside interference and misuse.
	
		\item Data encapsulation is closely related to \textbf{data hiding}.
			
	\end{itemize}	
	
\end{flushleft}
