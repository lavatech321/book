\setlength{\columnsep}{3pt}
\begin{flushleft}
	
	\begin{itemize}
		\item Data hiding refers to binding attributes within a class.
		\item It controls data visibility and accessibility from outside the class. 
		\item Below code shows data not being hidden:
		\codeblockfull{Test.java}{
			class Sample \{ \\
			\s int no; \\
			\} \\
			class Test \{ \\
			\s public static void main(String[] args) \{ \\
			\s \s Sample s1 = new Sample(); \\
			\s \s s1.no = 1;   // \color{red} No Data hiding \\ \color{black}
			\s	\}  \}
		}
		\bigskip
		\item Data hidding is done using \textbf{private} attributes.
		
		\item Eg: You can create public methods \textbf{setter()} and \textbf{getter()} to get and set private instance variables as shown below:
		\bigskip
		\codeblockfull{Sample.java}{
			class Sample \{ \\
			\s private int no; \\
			\s 	public void setter(int no) \{ \\
			\s \s		this.no = no; \\
			\s	\} \\
			\s	public void getter() \{ \\
			\s \s		System.out.println(no); \\
			\s	\} \\
			\s	public static void main(String[] args) \{ \\
			\s \s		Sample s1 = new Sample(); \\
			\s \s		s1.setter(10);   // \textbf{Data hidden} \\
			\s \s		s1.getter(); // Output: 10 \\
			\s	\} \}
		}
			
	\end{itemize}	
	
\end{flushleft}
