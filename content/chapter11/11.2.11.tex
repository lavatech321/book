\setlength{\columnsep}{3pt}
\begin{flushleft}

	\textbf{private constructor}
	
	\tablethree{
		\hline
		Syntax & Accessibility & Uses \\
		\hline
		private Classname \{\} & Only accessible within the same class & Used to \textbf{prevent direct instantiation} of the class \\
		\hline
	}
	\begin{itemize}
			\item Eg 1:
			\codeblockfull{A.java}{
				class A \{ \\
				\s \textbf{private A()} \{ \\
				\s \s		System.out.println("A constructor called"); \\
				\s	\} \\
				\s 	public static void main(String[] args) \{ \\
				\s \s 		\textbf{A a1 = new A(); \cmark} \\
				\s	\} \\
				\}
			}
			\bigskip
			\outputblock{
				A constructor called
			}
		
			\item Eg 2:
			\codeblockfull{B.java}{
			class A \{ \\
			\s	\textbf{private A()} \{ \\
			\s \s		System.out.println("A constructor called"); \\
			\s	\} \} \\	 
			public class B extends A \{ \\
			\s	public static void main(String[] args) \{ \\
			\s \s		\textbf{B b1 = new B();} \xmark \s // Cannot instantiate \\
			\s	\} \} 
			}
		\end{itemize}

		\newpage
		\textbf{protected constructor}
		
		\tablethree{
			\hline
			Syntax & Accessibility & Uses \\
			\hline
			protected Classname \{\} & Accessible within \begin{itemize}
				\item Same class
				\item Subclasses
				\item Other classes within the same package
			\end{itemize} 
			Cannot be accessed from different package where class is not subclass. & 
			Used to allow subclasses to access the constructor but restrict direct access from unrelated classes. \\
			\hline
		}
	
		\begin{itemize}
			\item Eg 1:
			\codeblockfull{B.java}{
				class A \{ \\
				\s \textbf{protected A()} \{ \\
				\s \s		System.out.println("A constructor called"); \\
				\s	\} \\
				\} \\
				public class B extends A \{ \\
				\s	public B() \{ \\
				\s \s		System.out.println("B constructor called"); \\
				\s	\} \\
				\s	public static void main(String[] args) \{ \\
				\s \s		\textbf{B b1 = new B();} \cmark \\
				\s	\} 	\}
			}
		
			\bigskip
			\outputblock{
				A constructor called \\
				B constructor called
			}
		\end{itemize}
		
		\newpage
		
		\textbf{public constructor}

		\tablethree{
			\hline
			Syntax & Accessibility & Uses \\
			\hline
			public Classname \{\} & Accessible from anywhere like: \begin{itemize}
				\item Other classes
				\item Subclasses
				\item Different packages
			\end{itemize} 
			& 
			Used to create objects of the class from any context \\
			\hline
		}
			
		\begin{itemize}
			\item Eg 1:
			\codeblockfull{B.java}{
				class A \{ \\
				\s \textbf{public A()} \{ \\
				\s \s		System.out.println("A constructor called"); \\
				\s	\} \\
				\} \\
				public class B extends A \{ \\
				\s	public B() \{ \\
				\s \s		System.out.println("B constructor called"); \\
				\s	\} \\
				\s	public static void main(String[] args) \{ \\
				\s \s		\textbf{B b1 = new B();} \cmark \\
				\s	\} 	\}
			}
			\bigskip
			\outputblock{
				A constructor called \\
				B constructor called
			}
		\end{itemize}
	
		\newpage
		\textbf{default constructor}: Has no explicit access modifier specified.
		\newline
		\tablethree{
			\hline
			Syntax & Accessibility & Uses \\
			\hline
			Classname \{\} & Accessible within the same package but not accessible from classes outside the package.
			& 
			If no constructor is defined, a default constructor is automatically provided by the compiler. \\
			\hline
		}
		
		\begin{itemize}
			\item Eg 1:
			\codeblockfull{B.java}{
				class A \{ \\
				\s \textbf{A()} \{ \\
				\s \s		System.out.println("A constructor called"); \\
				\s	\} \\
				\} \\
				public class B extends A \{ \\
				\s	public B() \{ \\
				\s \s		System.out.println("B constructor called"); \\
				\s	\} \\
				\s	public static void main(String[] args) \{ \\
				\s \s		\textbf{B b1 = new B();} \cmark \\
				\s	\} 	\}
			}
			\bigskip
			\outputblock{
				A constructor called \\
				B constructor called
			}
		\end{itemize}
	
\end{flushleft}

\newpage
