\setlength{\columnsep}{3pt}
\begin{flushleft}
	
	All rules of method hiding are exactly same overriding except the following difference:
	
	\tabletwo{
		\hline
		Method hiding & Method overriding \\
		\hline
		Both parent and child class method should be static & Both parent and child class method should be non-static \\
		\hline
		Compiler is responsible for method resolution based on reference type & JVM is always responsible for method resolution based on runtime object \\
		\hline
		It is also known as compile-time polymorphism or static polymorphism or early binding & It is also known as runtime polymorphism or dynamic polymorphism or late binding \\
		\hline
	}
	
	Eg:
	\codeblockfull{Test.java}{
		class Parent \{ \\
		\s public static void m1() \{ \\
		\s \s	System.out.println("Parent"); \\
		\s	\} \\
		\} \\
		class Child extends Parent \{ \\
		\s	public static void m1() \{ \\
		\s \s		System.out.println("Child"); \\
		\s 	\} \\
		\}  \\
		class Test \{ \\
		\s	public static void main(String[] args) \{ \\
		\s \s		Parent p1 = new Parent(); \\
			\s \s		p1.m1(); \\
		\s \s		Child c1 = new Child(); 
	}	
	\newpage
	\codecontinue{
	\s \s		c1.m1(); \\
	\s \s		Parent p2 = new Child(); \\
	\s \s		p2.m1(); \\
	\s	\} \\
	\}
	}
	
	\bigskip
	\outputblock{
		Parent \\
		Child \\
		Parent
	}

	\bigskip
	\noteblock{
	Method hiding is like sticking poster on black board, while method overriding is like erasing board and writing new content.
	}
\end{flushleft}

\newpage
