\setlength{\columnsep}{3pt}
\begin{flushleft}

	\begin{itemize}
		\item You can have more than one constructor in your class, as long as the argument lists are different. 
		\item More than one constructor in a class means \textbf{overloaded constructors}.
		\item Eg:
		\bigskip
		\codeblockfull{Test.java}{
			class Rectangle \{ \\
			\s	float l, w, h; \\
			\s	public Rectangle() \{ \\
			\s \s		this.l=0.0f; \\
			\s \s		this.w=0.0f; \\
			\s \s		this.h=0.0f; \\
			\s	\} \\
			\s	public Rectangle(int width, int height) \{  \\
			\s \s		this.l = 0.0f; \\
			\s \s		this.w = width; \\
			\s \s		this.h = height; \\
			\s	\} \\
			\s	public Rectangle(int length, int width, int height) \{ \\
			\s \s		this.l = length; \\
			\s \s		this.w = width; \\
			\s \s		this.h = height; \\
			\s	\} \\
			\s	public float getArea() \{ \\
			\s \s		float area; \\
			\s \s	area = 2*(l*w) + 2*(l*h) + 2*(h*w); \\
			\s \s		return area; \\
			\s	\} \\
			\} 
		}
		\newpage
		\codecontinue{
			public class Test \{ \\
			\s	public static void main(String[] args) \{ \\
			\s \s		\textbf{Rectangle r1 = new Rectangle();} \\
			\s \s		float area; \\
			\s \s		area = r1.getArea(); \\
			\s \s		System.out.println(area); \\
			\\
			\s \s		\textbf{Rectangle r2 = new Rectangle(3,4);} \\
			\s \s		float area2; \\
			\s \s		area2 = r2.getArea(); \\
			\s \s		System.out.println(area2); \\
			\\
			\s \s		\textbf{Rectangle r3 = new Rectangle(3,4,5);} \\
			\s \s		float area3; \\
			\s \s		area3 = r3.getArea(); \\
			\s \s		System.out.println(area3); \\
			\s \} \\
			\}	
		}
	
		\bigskip
		\outputblock{
		0.0 \\
		24.0 \\
		94.0
		}
		
	\end{itemize}
	
\end{flushleft}
\newpage



