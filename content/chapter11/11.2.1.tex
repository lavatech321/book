\setlength{\columnsep}{3pt}
\begin{flushleft}
	\bigskip
	\paragraph{What is YUM?}
	\begin{itemize}
		\item \textbf{YUM} stands for \textbf{Yellow dog Updater, Modified}.
		\item YUM is a command-line as well as graphical-based package management tool for RPM.
		\item YUM allows to install, update, remove \& search packages easily.
	\end{itemize}

	\paragraph{What is DNF?}
	\begin{itemize}
		\item The DNF command (Dandified yum) is the next version of the YUM.
		\item It is the default package manager for Fedora 22, CentOS8, and RHEL8.
		\item It is intended to be a replacement for YUM.
	\end{itemize}

	\paragraph{Advantages of YUM over RPM}
	\begin{enumerate}
		\item Dependency resolution:
		\begin{itemize}
			\item If a package installation requires additional packages, YUM can list these dependencies and prompt the user to install them.
		\end{itemize}
		\item Locate package from multiple locations \& install them: 
		\begin{itemize}
			\item YUM can be configured to search for software packages in more than one location.
		\end{itemize}
		\item YUM allows to specify particular software versions or architectures while installing or removing the package.
		
	\end{enumerate}
\end{flushleft}
\newpage



