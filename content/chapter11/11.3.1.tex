\setlength{\columnsep}{3pt}
\begin{flushleft}
	
	\begin{itemize}
		\item Using access modifier, class can provide more information to the JVM like:
		\begin{itemize}
			\item Whether the class is accessible from anywhere or not
			\item Whether child class creation is possible or not
			\item Whether object creation is possible or not
		\end{itemize}
		\item The only applicable modifiers for top-level classes are:
		\begin{itemize}
			\item public
			\item default
			\item final
			\item abstract
			\item strictfp
		\end{itemize}
		\item But, for inner classes, the applicable modifiers are:
		\begin{itemize}
			\item private
			\item protected 
			\item static
		\end{itemize}
	
	\end{itemize}

	\newimage{0.4}{content/chapter11/images/class.png}
	
	\noteblock{
	In Java, there are only access modifiers, there is no word like access specifier.
	
	}
	
	Let's see these class level access modifiers in detail.
	
	\newpage
	
	\textbf{public classes}:
	\begin{itemize}
		\item  If a class is declared as public then we can access that class from anywhere.
		\item Eg:
		\bigskip
		\codeblockfull{Test1.java}{
			package com.lavatech.www; \\
			public class Test1 \{  \\
			\s	public void message()\{ \\
			\s \s		System.out.println("Hard work pays off!"); \\
			\s	\} \\
			\}
		}
		\bigskip
		\codeblockfull{Test2.java}{
			package com.lavatech.info; \\
			import com.lavatech.www.Test1; \\
			public class Test2 \{ \\
			\s public static void main(String[] args) \{ \\
			\s \s		Test1 t1 = new Test1(); \\
			\s \s		t1.message(); \\
			\s	\} \\
			\}	
		}
		\bigskip
		\commandblock{
		\$ javac -d . Test1.java  \\
		\$ javac -d . Test2.java  \\
		\$ java com.lavatech.info.Test2  \\
		Hard work pays off!
		}
	\end{itemize}
	
	\newpage
	\textbf{default classes}:
	
	\begin{itemize}
		\item A default class \textbf{does not have an access modifier} specified. 
		\item Also known as a \textbf{package-private class} 
		\item It is accessible \textbf{only within the same package}.
		\item If no access modifier (such as public, private, or protected) is specified, the class is considered to have default access.
		\item Eg:
		\bigskip
		\codeblockfull{Test1.java}{
			package com.lavatech.www; \\
			class Test1 \{ \\
			\s public void message()\{ \\
			\s \s		System.out.println("Hard work pays off!"); \\
			\s	\} \\
			\}
		}
		\bigskip
		\codeblockfull{Test2.java}{
			package com.lavatech.info; \\
			import com.lavatech.www.Test1;
		}
		\bigskip
		\commandblock{
			\$ javac -d . Test2.java  \\
			\color{red}
			Test2.java:2: error: Test1 is not public in com.lavatech.www; cannot be accessed from outside package \\
			import com.lavatech.www.Test1; \\
			1 error 
		}
	\end{itemize}

	\newpage
	\textbf{final class}
	\begin{itemize}
		\item Final class can’t be inherited 
		\item You can’t create child class for final class.
		\item More on final class is mentioned in inheritance section.
	\end{itemize}

	\textbf{abstract class}
	\begin{itemize}
		\item Abstract classes can be instantiated.
		\item You cannot create object of abstract class.
		\item More on abstract class is mentioned in Abstract class section.
	\end{itemize}

	\textbf{strictfp class}
	\begin{itemize}
		\item strictfp was introduced in Java1.2 version.
		\item Strictfp class ensures all methods in the class \& its subclasses, adhere to strict floating-point precision rules.
		\item strictfp is used to ensure that floating points operations give the same result on any platform. 
		\item As floating points precision may vary from one platform to another. strictfp keyword ensures the consistency across the platforms.
		\item Eg:
		\codeblockfull{Test.java}{
		\textbf{strictfp class A} \{ \\
		\s	double num1 = 10e+102; \\
		\s	double num2 = 6e+08; \\
		\s	double calculate() \{ \\
		\s \s		return num1 + num2; \\
		\s	\} \} \\
		public class Test \{  \\
		\s	public static void main(String[] args) \{  \\
		\s \s		A a1 = new A();      \\
		\s \s		System.out.println("Result: " + a1.calculate()); \\
		\s	\} \}
		}
		\newpage
		
		\bigskip
		\outputblock{
			Result: 1.0E103	
		}
		
	\end{itemize}
	
\end{flushleft}

\newpage
