
\begin{flushleft}

	\begin{tabulary}{1.0\textwidth}{|p{5em}|p{22em}|}
		\toprule
		\textbf{Function} & \textbf{Syntax \& Example} \\
		\midrule
		length() &  public int length(); \newline Eg:
		\codeblock{
			StringBuffer sb = new StringBuffer("hello world"); \\
			System.out.println(sb.length());  // Output: 11
		}   \\
		\hline
		
		capacity() & public int capacity(); \newline Eg:
		\codeblock{
			StringBuffer sb = new StringBuffer("hello world"); \\
			System.out.println(sb.capacity()); // Output: 27
		} \\
		
		\bottomrule
	\end{tabulary}
	
	\newpage
	
	\begin{tabulary}{1.0\textwidth}{|p{9em}|p{18em}|}
		\toprule
		\textbf{Function} & \textbf{Syntax \& Example} \\
		\midrule
		append() & public StringBuffer append(String s) \newline 
		public StringBuffer append(byte|int|long|float|double|char)
		\newline
		Eg:
		\codeblock{
			StringBuffer sb = new StringBuffer("hello world"); \\
			sb.append("again"); \\
			sb.append(10); \\
			sb.append(2.0f); 
		} \\
		\hline
		insert() & public StringBuffer insert(int index, String s) \newline 
		public StringBuffer insert(int index, byte|int|long|float|double|char)
		\newline
		Eg:
		\codeblock{
			StringBuffer sb = new StringBuffer("Hello"); \\
			sb.insert(5," world!"); \\
			sb.insert(0, 1); \\
			sb.insert(0,1.0f); 
		} \\
		
		\bottomrule
	\end{tabulary}
	
	
	\newpage
	
	\begin{tabulary}{1.0\textwidth}{|p{7em}|p{20em}|}
		\toprule
		\textbf{Function} & \textbf{Syntax \& Example} \\
		\midrule
		delete() &  public StringBuffer delete(int begin, int end) \newline
		This would delete characters from begin to end-1.
		\newline Eg:
		\codeblock{
			StringBuffer sb = new StringBuffer("Hello world!"); \\
			sb.delete(0,6); \\
			System.out.println(sb); // Output: world!
		}   \\
		\hline
		
		deleteCharAt() & public StringBuffer deleteCharAt(int index) \newline Eg:
		\codeblock{
			StringBuffer sb = new StringBuffer("Hello world!"); \\
			sb.deleteCharAt(5); \\
			System.out.println(sb); // Output: Helloworld!
		} \\
		\hline
		reverse() & public StringBuffer reverse() \newline Eg:
		\codeblock{
			StringBuffer sb = new StringBuffer("Hello world!");
			System.out.println(sb.reverse()); // Output: !dlrow olleH
		} \\		
		\bottomrule
	\end{tabulary}
	
	\newpage
		\begin{tabulary}{1.0\textwidth}{|p{8em}|p{19em}|}
		\toprule
		\textbf{Function} & \textbf{Syntax \& Example} \\
		\midrule
		setLength() & public void setLength(int length) \newline Eg:
		\codeblock{
			StringBuffer sb = new StringBuffer("Hello world!"); \\
			sb.setLength(5); \\
			System.out.println(sb); // Output: Hello
		} \\
		\hline
		ensureCapacity() & public void ensureCapacity(int length) \newline Eg:
		\codeblock{
			StringBuffer sb = new StringBuffer("Hello world!"); \\
			sb.ensureCapacity(50); \\
			System.out.println(sb.capacity()); // Output: 58
		} \\
		\hline
		trimToSize() & Extra allocated free memory, can you plase deallocate 
		\newline 
		public void trimToSize() \newline Eg:
		\codeblock{
			StringBuffer sb = new StringBuffer("Hello world!"); \\
			sb.ensureCapacity(50); \\
			sb.trimToSize(); \\
			System.out.println(sb.capacity()); // Output: 12
		} \\
		\bottomrule
	\end{tabulary}

	\newpage
	\begin{tabulary}{1.0\textwidth}{|p{8em}|p{19em}|}
		\toprule
		\textbf{Function} & \textbf{Syntax \& Example} \\
		\midrule
		toString() & public String toString() \newline Eg:
		\codeblock{
			StringBuffer sb = new StringBuffer("Hello world!"); \\
			String test = sb.toString(); \\
			System.out.println(test);   // Output: Hello world!
		} \\
		\hline
		setCharAt() & public char setAt(int  integer); \newline Eg:
		\codeblock{
			StringBuffer sb = new StringBuffer("hello world"); \\
			sb.setCharAt(0,'H'); \\
			System.out.println(sb);  // Output: Hello world
		} \\
		\hline
		charAt() & public char charAt(int  integer); \newline Eg:
		\codeblock{
			StringBuffer sb = new StringBuffer("hello world"); \\
			System.out.println(sb.charAt(3));  // Output: l
		} \\		
		\bottomrule
	\end{tabulary}
	
	
	
\end{flushleft}

\newpage

