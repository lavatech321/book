
\begin{flushleft}
	
	\tabletwo{
	\hline
	Unary operators &  [] , x++ , x-- \newline ++x , --x , \textbf{\~} , ! \\
	\hline
	Arithematic operators &  * , / , \%  \newline + , - \\
	\hline
	Shift operators & >>  , >>> , << \\
	\hline
	Comparison operators & < , <= , > , >= , instanceof \\
	\hline
	Equality operators &  == , != \\
	\hline
	Bitwise operators & \& , \textbf{\^} , | \\
	\hline
	Short circuit operators & \&\& , || \\
	\hline
	Conditional operator & ?: \\
	\hline
	Assignment operators &  = , += , -= , *= \\
	\hline
	}

	\textbf{Evaluation order of operands:}
	
	\begin{itemize}
		\item In java, we have only operator precedence and no operands precedence.
		\item Before applying any operator, all operands will be evaluated from left to right.
		\item Eg:
		\codeblock{
			System.out.println(1+2*3/4+5*6);  // output: 32
		}
	
		Output explaination:
		\begin{itemize}
			\item 1+2*3/4+5*6
			\item 1+6/4+5*6
			\item 1+1+5*6
			\item 1+1+30
			\item 32
		\end{itemize}
		
	\end{itemize}
	
\end{flushleft}
\newpage