
\begin{flushleft}
	
	There are 3 types of assignment operators:
	\begin{itemize}
		\item \textbf{Simple:}  The assignment operator (=) is used to assign a value to a variable.
		\bigskip
		\codeblock{
			int x = 10;
		}
		\item \textbf{Chained:} 
		\bigskip
		\codeblock{
			int a,b,c,d; \\
			a=b=c=d=20;   
		}
	
		We can’t perform chained assignment directly at the time of declaration:
		\bigskip
		\codeblock{
			int a=b=c=d=20; \xmark \\
						\\
			int b,c,d;  \\
			int a=b=c=d=20;  \cmark
		}
		
		\item \textbf{Compound:} 
		\begin{itemize}
			\item Assignment operator can be mixed with other operators.
			\item Such type of assignment operators are called compound assignment operators.
			\bigskip
			\syntaxblock{
				variable operator= expression;
			}
			\bigskip
			\codeblock{
				int a = 10;  \\
				a += 20;  \\
				System.out.println(a) ; // output: 30
			}
			
			\item In case of compound assignment operator, internal type casting will be performed automatically:
			\bigskip
			\codeblock{
				byte b=10;  \\
				b = b+1;     \xmark \\
				\\
				byte b = 10; \\
				b++;    // output: 11 \\    
				\\
				byte b=10; \\
				b+=1;    // output: 11
			}
			\bigskip
			\item Below are sample examples of compound assignment operator:
			\codeblock{
				int x = 5;
				x += 3; // x is 8 \\
				x -= 2; // x is 6 \\
				x *= 4; // x is 24 \\
				x /= 3; // x is 8  \\
				x \%= 5; // x is 3  \\
				x \&= 1; // x is 1 \\
				x |= 2; // x is 3 \\
				x \textbf{\^}= 3; // x is 0 \\
				x <<= 2; // x is 0 \\
				x >>= 1; // x is 0
			}
		\end{itemize}
		 
		
		
		
	\end{itemize}
	
	
	
	
\end{flushleft}
\newpage