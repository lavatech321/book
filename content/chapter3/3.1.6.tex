
\begin{flushleft}
	
	\begin{itemize}
		\item Equality operators can be used for every primitive type including boolean.
		\bigskip
		\codeblock{
			System.out.println(10==20);  // output: fasle \\
			System.out.println('a' == 'b'); // output: false \\
			System.out.println('a' == 97.0); // output: true \\
			System.out.println(false == false); // output: true
		}
		
		\item Equality operators can be applied for object types also. For object references, "r1" \& "r2", "r1==r2" returns true, if both reference pointing to the same object (reference comparison or address comparison)
		\bigskip
		\codeblock{
			Thread t1 = new Thread(); \\
			Thread t2 = new Thread(); \\
			Thread t3 = t1; \\
			System.out.println(t1 == t2);   // output: false \\
			System.out.println(t1 == t3);   // output: true
		}
		
		There should be some relation between argument types(either child to parent or parent to child or same type). Otherwise, it will result in compile-time error.
		\bigskip
		\codeblock{
			Thread t1 = new Thread();   \\
			Object o = new Object();   \\
			String s = new String("lava");   \\
			System.out.println(t1==o) ; // output: false    \\
			System.out.println(o==s); // output: false    \\
			//System.out.println(s==t1); \xmark
		}
		
		For any object reference "r": "r==null"  ← is always False
		\bigskip
		\codeblock{
			String s1 = new String("lava"); \\
			System.out.println(s1 == null) ; // output: false \\
			\\
			String s2 = new String(); \\
			System.out.println(s2==null); // output: false \\
			\\
			String s3 = null; \\
			System.out.println(s3==null); // output: true
		}
		
	\end{itemize}


	
\end{flushleft}
\newpage