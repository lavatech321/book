
\begin{flushleft}

	
	\begin{itemize}
		\item The increment(++) and decrement(--) operators are unary operators.
		\item They are used to increment or decrement the value of a variable by 1.
		\item \textbf{Increment Operator (++):} Used in two ways:
		\begin{itemize}
			\item \textbf{Prefix (++var)}: Variable is incremented first and then used in the expression.
			\bigskip
			\codeblock{
				int a = 5; \\
				int b = ++a; // b will be 6, a will be 6
			}
			
			\item \textbf{Postfix (var++)}: Variable is used in the expression and then incremented. 
			\bigskip
			\codeblock{
				int a = 5; \\
				int b = a++; // b will be 5, a will be 6
			}			
		\end{itemize}
		
		\item \textbf{Decrement Operator (--):} Works in a similar way and can be used in prefix and postfix forms:
		\bigskip
		\codeblock{
			int a = 5; \\
			int b = --a; // b will be 4, a will be 4  \\
			int c = a--; // c will be 4, a will be 3
		}
	\end{itemize}
	
	Summary:
	\tablefour{
		\hline
		Expression & Initial value of x & Value of y & Final value of x \\
		\hline
		y=++x; & 10 & 11 & 11 \\
		\hline
		y=++x; & 10 & 10 & 11 \\
		\hline
		y=++x; & 10 & 9 & 9 \\
		\hline
		y=++x; & 10 & 10 & 9 \\
		\hline
	}
	
	Important Points:
	\begin{itemize}
		\item Increment/decrement is \textbf{applicable only on variable and not on constant}.
		\bigskip
		\codeblock{
			System.out.println(++10); \xmark
		}
		\item \textbf{Listing} of increment/decrement operators \textbf{not allowed}.
		\bigskip
		\codeblock{
			int x=10; \\
			int y = ++(++x); \xmark
		}
	
		\item \textbf{For final variables}, increment/decrement operators \textbf{cannot} be used:
		\bigskip
		\codeblock{
			final int x=10; \\
			System.out.println(x++); \xmark
		}
	
		\item Increment/decrement is applicable on all primitive type, \textbf{except boolean datatype}:
		\begin{itemize}
			\item Integer example:
			\bigskip
			\codeblock{
				int x=10; \\
				x++; \\
				System.out.println(x);   // output: 11
			}
			\item Character example:
			\bigskip
			\codeblock{
				char ch = 'a'; \\
				ch++; \\
				System.out.println(ch); // output: 'b'
			}
			\newpage
			\item Boolean example:
			\bigskip
			\codeblock{
				boolean b=true;  \\
				b++;  \xmark  
			}
		
		
		\end{itemize}
		
	\end{itemize}
		
	\textbf{Difference between “x++” and “x=x+1”}
	
	\begin{itemize}
		\item We know that: If we apply any arithmetic operator between 2 variables “a” and “b”, the result type is always:
		\bigskip
		\codecontinue{
			maximum(int, type of a, type of b)
		}
		
		\item This is the reason why using "x=x+1" can result in compile-time error:
		\bigskip
		\codeblock{
			byte b=10; \\
			b = b+1;        \xmark 
		}
	
		\item But in case of increment/decrement operators, internal type casting will be performed automatically:
		\bigskip
		\codeblock{
			byte b=10;  \\
			b++;   \cmark
		}
		
	\end{itemize}
	
	
		
\end{flushleft}
\newpage