
\begin{flushleft}
	
	 	In Java, the short-circuit operators are:
		\begin{itemize}
			\item \textbf{\&\& (logical AND)}: Same as bitwise \&
			\item \textbf{|| (logical OR)}: Same as bitwise |
		\end{itemize}

		\tabletwo{
		\hline
		\& , |  & \&\& , || \\
		\hline
		Both arguments should be evaluated always & Second argument evaluation is optional \\
		\hline
		Relatively performance is low & Relatively performance is high \\
		\hline
		Applicable for both boolean and integral types & Applicable only for boolean but not for integral types \\
		\hline
		Eg: \newline
		For x \& y, both x and y will be evaluated. \newline
		Similarly, \newline
		For x | y, both x and y will be evaluated. 
		& 
		Eg: \newline
		For x \&\& y, y will be evaluated if and only if x is true i.e if x is false then y wont be evaluated. \newline
		Similarly, \newline
		For x || y,  y will be evaluated if and only if x is false i.e if x is true then y wont be evaluated. \\
		
		\hline
		
		}
		
		\bigskip
		Consider below code showing different behaviour of \&, \&\&, |, ||:
		\codeblock{
			int x = 10, y = 15;   \\
			if(  ++x < 10  \& ++y > 15) \{ \\
			\s	x++;   \\
			\}  \\
			else \{  \\
			\s	y++;   \\
			\}  \\
			System.out.println(x + "..." + y);  //   output: 11...17
		}
		\bigskip
		\codeblock{
			int x = 10, y = 15;   \\
			if(  ++x < 10  | ++y > 15) \{ \\
			\s	x++;   \\
			\}  \\
			else \{  \\
			\s	y++;   \\
			\}  \\
			System.out.println(x + "..." + y);  //   output: 12...16
		}
		\bigskip
		\codeblock{
			int x = 10, y = 15;   \\
			if(  ++x < 10  \&\& ++y > 15) \{ \\
			\s	x++;   \\
			\}  \\
			else \{  \\
			\s	y++;   \\
			\}  \\
			System.out.println(x + "..." + y);  //   output: 11...16
		}
		\bigskip
				\codeblock{
			int x = 10, y = 15;   \\
			if(  ++x < 10  || ++y > 15) \{ \\
			\s	x++;   \\
			\}  \\
			else \{  \\
			\s	y++;   \\
			\}  \\
			System.out.println(x + "..." + y);  //   output: 12...16
		}
		\bigskip

	
\end{flushleft}
\newpage