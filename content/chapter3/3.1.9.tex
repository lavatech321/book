
\begin{flushleft}
	
	 	In Java, the short-circuit operators are:
		\begin{itemize}
			\item \textbf{\&\& (logical AND)}: If LHS evaluates to false, RHS is not evaluated at all. Note that \&\& requires both condition to be true.
			
			\item \textbf{|| (logical OR)}: If the LHS evaluates to true, the RHS is not evaluated at all. Note that || requires atleast one condition to be true.
			
		\end{itemize}
		Eg:
		\codeblock{
			String user="Ram"; \\
			String password="ram@123"; \\
			System.out.println(user=="Ram" \&\& password=="ram@123");  // output: true \\
			System.out.println(user=="Ram" || password=="ram@123");   // output: true 
		}
		\bigskip
		Difference between Bitwise "\& |" and Logical "\&\&  ||" 
		\tabletwo{

		\& , |  & \&\& , || \\
		\hline
		Both LHS \& RHS are evaluated & RHS evaluation is optional \\
		\hline
		Performance is low & Performance is high \\
		\hline
		Applicable on boolean \& integral types & Applicable only for boolean. \\		
		}
		
	
\end{flushleft}
\newpage