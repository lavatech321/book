
\begin{flushleft}
	
	Below are available relational operator in Java:

	\newimage{0.5}{content/chapter3/images/relational.png}

	Let's see these in detail.	
	\newline
	
	\textbf{Comparison operators: > , >= , < , <=}
	\begin{itemize}
		\item Comparison operator is \textbf{applicable for every primitive datatype}, except boolean datatype.
		\item Eg:
		\bigskip
		\codeblock{
			System.out.println(10>20);  // output: false  \\
			System.out.println('a'>20);  // output: false  \\ 
			System.out.println('b'>2.0); // output: false  \\
			//System.out.println(true > false); \xmark
		}
	
		\item Comparison operator is \textbf{not applicable on object types}:
		\bigskip
		\codeblock{
			System.out.println('lava' > 'lavatech');  \xmark
		}
	
		\newpage
		\item Chaining of comparison operators is not allowed.
		\bigskip
		\codeblock{
			System.out.println(10>20>30);  \xmark
		}
		
	\end{itemize}

\textbf{Equality operators: == , !=}
	\begin{itemize}
		\item Equality operators are \textbf{applicable on all primitive type}.
		\bigskip
		\codeblock{
			System.out.println(10==20);  // output: fasle \\
			System.out.println('a' == 'b'); // output: false \\
			System.out.println('a' == 97.0); // output: true \\
			System.out.println(false == false); // output: true
		}
		
		\item Equality operators are \textbf{applicable for object types}. Returns true, if both object reference pointing to the same object.
		\bigskip
		\codeblock{
			Thread t1 = new Thread(); \\
			Thread t2 = new Thread(); \\
			Thread t3 = t1; \\
			System.out.println(t1 == t2);   // output: false \\
			System.out.println(t1 == t3);   // output: true
		}
		
		If child \& parent are not of same type, it will result in compile-time error.
		\bigskip
		\codeblock{
			Thread t1 = new Thread();   \\
			Object o = new Object();   \\
			String s = new String("lava");   \\
			System.out.println(t1==o) ; // output: false    \\
			System.out.println(o==s); // output: false    \\
			//System.out.println(s==t1); \xmark
		}
				
	\end{itemize}
	
\end{flushleft}
\newpage