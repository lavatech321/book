\setlength{\columnsep}{3pt}
\begin{flushleft}
	
	\begin{itemize}
		\item A zip file containing group of “.class” files a \textbf{jar} file.
		\item Mostly, third party software plugins are available as jar file.
		\item Create a Manifest File in jar file (Optional):
		\begin{itemize}
			\item Manifest contains information such as the entry point class and classpath. 
			\item If you don't specify a manifest, Java will automatically create a default one.
			\item Eg: Create Manifest.txt, with content like this:
			\bigskip
			\codecontinue{
				Main-Class: classname <press enter>
			}			
		\end{itemize}
		
		\item Command to create jar file without manifest:
		\bigskip
		\syntaxblock{
			jar -cvf filename.jar code1.class code2.class.... \\
			jar -cvf filename.jar *.class \\
			jar -cvf filename.jar *.*
		}
	
		\item Command to create jar file with manifest:
		\bigskip
		\syntaxblock{
			jar -cvfm filename.jar Manifest.txt code1.class code2.class .....
		}
		\newpage
		\item Extract a jar file:
		\bigskip
		\syntaxblock{
			jar -xvf filename.jar
		}
		\item Display table of content of jar file:
		\bigskip
		\syntaxblock{
			\$ jar -tvf filename.jar
		}	
		\item Execute a jar file:
		\bigskip
		\syntaxblock{
			java -jar filename.jar
		}
		\item Eg: Create 2 Java programs with manifest file, compress them in jar file \& execute the jar file:
		\codeblockfull{A.java}{
			public class A \{ \\
			\s public static void main(String[] args) \{ \\
			\s \s		B b1 = new B(); \\
			\s \s	System.out.println(b1.secret); \\
			\s	\} \\
			\}	
		}
		\bigskip
		\codeblockfull{B.java}{
			public class B \{ \\
			\s	public String secret="Choomantar"; \\
			\}	
		}
		\bigskip
		\codeblockfull{Manifest.txt}{
			Main-Class: A	
		}
		\newpage
		Create and execute the jar file:
		\bigskip
		\commandblock{
			\$ javac A.java B.java \\
			\$ jar -cvfm code.jar Manifest.txt A.class B.class  \\
			\$ java -jar code.jar  \\
			Choomantar
		}
		
		\item Eg: Create jar with ".class" files in different packages, source file and Manifest for below directory structure:
		\newimage{0.5}{content/chapter10/images/new69.png}
		\bigskip
		\commandblock{
			\$ jar -cvfm code.jar Manifest.mf A/B/One.class C/D/Two.class src/*.java \\
			\$ java -jar code.jar  
		}
		
		
	\end{itemize}
	\newpage
	\textbf{Executing a JAR file from anywhere}
	\begin{itemize}
		\item \textbf{Windows}:  \textbf{By creating a batch file.}
		\begin{itemize}
			\item A batch file contains a group of commands. 
			\item You can double click a batch file, to execute commands in it one by one.
			\item Create a batch file to execute JAR file:
			\bigskip
			\codeblockfull{file.bat}{
				java -jar full-path-of-jar-file$\backslash$code.jar
			}
		\end{itemize}
		\item \textbf{Ubuntu: By creating a shell script.}
		\begin{itemize}
			\item Create a shell script to execute JAR file.
			\bigskip
			\codeblockfull{file.sh}{
				\#!/bin/bash \\
				java -jar /full-path-of-jar-file/code.jar	
			}
			\item Set executable permission to JAR file:
			\bigskip
			\commandblock{
				chmod +x file.sh
			}
			\item Add script location in the path system variable by adding below code:
			\bigskip
			\commandblock{
				\$ vi ~/.bashrc \\
				export PATH=<code-location>:\$PATH \\
				\\
				\$ source ~/.bashrc
			}
			\item Now you can execute file.sh from anywhere:
			\bigskip
			\commandblock{
				\$ file.sh 
				Choomantar
			}
		\end{itemize}
			
	\end{itemize}
	
	
\end{flushleft}

\newpage

