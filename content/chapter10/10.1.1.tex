\setlength{\columnsep}{3pt}
\begin{flushleft}
	
	\begin{itemize}
		\item \textbf{import} keyword is used to access classes \& interfaces from \textbf{external packages or libraries}. 
		\item Import keyword is not required for following packages as they are by default available to every Java program:
		\begin{itemize}
			\item \textbf{java.lang}
			\item \textbf{default package (current working directory)}
		\end{itemize} 
		\item There are 2 types of import:
		\begin{itemize}
			\item \textbf{Explicit import}
			\item \textbf{Implicit import}
		\end{itemize}	
		\item Let's see each of these in detail.
	\end{itemize}		

	\newpage
	\textbf{Explicit class import:} 
	\begin{itemize}
		\item Used to import a single class.
			\bigskip
			\syntaxblock{
				import packageName.ClassName;	
			}
			\item Eg:
			\codeblock{
				\textbf{import java.time.LocalDate;} \\
				class demo \{ \\
				\s public static void main(String[] args) \{ \\
				\s \s	LocalDate date = LocalDate.now(); \\
				\s \s		System.out.println(date); // \textbf{Ouptut: 2023-08-15} \\
				\s	\} \}
			}
		\end{itemize}
		
		\textbf{Implicit class import:} 
		\begin{itemize}
			\item Used to import all classes from a package.
			\item Not recommended to use as reduces readability of the code.
			\bigskip
			\syntaxblock{
				import packageName.*;
			}
			\item Eg:
			\codeblock{
				\textbf{import java.util.*;} \\
				class New \{  \\
				\s	public static void main(String[] args) \{  \\
				\s \s	\textbf{Date date = new Date();} \\
				\s \s	System.out.println(date); \\
				\s \s	\textbf{Calendar cal = Calendar.getInstance();} \\
				\s \s	System.out.println(cal); \\
				\s	\} \}
			}
		
			
			\item Drawback of implicit declaration: \textbf{Ambiguity problem}. Eg:
			\codeblock{
				\textbf{import java.util.*;} \\
				\textbf{import java.sql.*;} \\
				class New \{ \\
				\s public static void main(String[] args) \{ \\
				\s \s \textbf{Date d = new Date();  \xmark \s Ambiguous error } \\
				\s	\} \\
				\} 
			}

		\end{itemize}
		
		\textbf{Note about import statement:}
		\begin{itemize}
			\item \textbf{No subpackage import}: Importing a Java package, imports only classes and interfaces but not it's subpackage classes.
			\item \textbf{No effect on execution time:} \textbf{More import statements results in more compile-time. There is no effect on execution time (runtime).}
			\item Compiler resolves class names in the below order:
			\begin{itemize}
				\item Explicit import
				\item Classes present in current working directory (default package)
				\item Implicit class import
			\end{itemize}	
		\end{itemize}
		
		Difference between C language \textbf{\#include} and Java \textbf{import} statement:
		\bigskip
		\tabletwo{
			\hline
			\#include & import statement \\
			\hline
			All I/O header files will be loaded at the beginning (at translation time) &
			No \textbf{".class"} file will be loaded at the beginning. Whenever we are using a particular class, then only corresponding \textbf{".class"} file will be loaded. \\
			\hline
			It is static include & It is "dynamic include" or "load on fly" or "load on demand" \\
			\hline
		}			
		
		
		

		
		
		
				

	
\end{flushleft}

\newpage

