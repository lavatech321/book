\setlength{\columnsep}{3pt}
\begin{flushleft}
	
	\begin{itemize}
		\item import keyword is used to access classes, interfaces, and other types from external packages or libraries. 

		\item Consider an example where you want to use \textbf{ArrayList} class from \textbf{java.util} package. For this, you will need to use fully qualified name:
		\bigskip
		\codeblock{
			class New \{ \\
			\s public static void main(String[] args) \{ \\
		    \s \textbf{java.util.ArrayList l1 = new java.util.ArrayList();}	 \\
		   	\s	\} \\
		   \} 
		}
	
		With \textbf{import} keyword, the code would look like:
		\bigskip
		\codeblock{
		 \textbf{import java.util.ArrayList;} \\
		 class New \{ \\
		 \s public static void main(String[] args) \{ \\
		 \s \s	\textbf{ArrayList l1 = new ArrayList();} \\
		 \s	\} \\
		 \}
		}
		
		\item Using import statement, it is not required to use fully qualified name everytime.
		\bigskip
		\noteblock{
			All classes and interfaces present in the following packages are by default available to every Java program. Hence we are not required to write import statement:
			\begin{itemize}
				\item java.lang
				\item default package (current working directory)
			\end{itemize}
		}
		\item 		There are 2 types of \textbf{import}. Let's see each of these in detail.	
		\end{itemize}		

		\newpage

		\textbf{Explicit class import:} 
		\begin{itemize}
			\item Used to import a single class.
			\item It is highly recommended as improves readability of the code.
			\bigskip
			\syntaxblock{
				import packageName.ClassName;	
			}
			\item Eg:
			\codeblock{
				import java.util.ArrayList; \cmark \\
				import java.util; \xmark
			}
		\end{itemize}
		
		
		\textbf{Implicit class import:} 
		\begin{itemize}
			\item Used tp import all classes from a package.
			\item Not recommended to use as reduces readability of the code.
			\bigskip
			\syntaxblock{
				import packageName.*;
			}
			\item Eg:
			\codeblock{
				import java.util.ArrayList.*; \xmark  \\
				import java.util.*; \cmark
			}
		
			
			\item Implicit declaration results in \textbf{ambiguity problem}. Eg:

			
			\codeblock{
				\textbf{import java.util.*;} \\
				\textbf{import java.sql.*;} \\
				class New \{ \\
				\s public static void main(String[] args) \{ \\
				\s \s \textbf{Date d = new Date();  \xmark \s Ambiguous error } \\
				\s	\} \\
				\} 
			}

			\newpage			
			In above code, \textbf{Date} is available in both \textbf{java.util} as well as \textbf{java.sql} package.
			
			\item While resolving class names, compiler will give precedence in the following order:
			\begin{itemize}
				\item Explicit class import
				\item Classes present in current working directory (default package)
				\item Implicit class import
			\end{itemize}
			\bigskip
			\codeblock{
				import java.util.*; \\
				import java.sql.*; \\
				class New \{ \\
				\s public static void main(String[] args) \{ \\
				\s \s \textbf{Date d = new Date();}  \\
				\s \s  System.out.println(d.getClass().getName()); \\
				\s	\} \\
				\} 
			}
			\bigskip
			\outputblock{
				java.util.Date
			}	
		\end{itemize}
		
		
		
			
		\newpage
		
		\textbf{Some more things to note about import statement:}
		\begin{itemize}
			\item \textbf{No subpackage import}: By importing a Java package, all classes and interfaces present in that package are available to Java program, but not the subpackage classes.
			\bigskip
			\item \textbf{No effect on execution time:} Import statements is totally compile-time 
			related concept. \textbf{If more number of imports, then more will be the compile-time.} But, there is no effect on execution time (runtime).
			\bigskip
			\item Difference between C language \textbf{#include} and Java \textbf{import} statement:
			\bigskip
			\bigskip
			\tabletwo{
				\hline
				\#include & import statement \\
				\hline
				All I/O header files will be loaded at the beginning (at translation time) &
				No \textbf{".class"} file will be loaded at the beginning. Whenever we are using a particular class, then only corresponding \textbf{".class"} file will be loaded. \\
				\hline
				It is static include & It is "dynamic include" or "load on fly" or "load on demand" \\
				\hline
			}	
		\end{itemize}
		
		
		
		
		

		
		
		
				

	
\end{flushleft}

\newpage

