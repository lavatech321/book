\setlength{\columnsep}{3pt}
\begin{flushleft}

	\begin{itemize}
		\item Java package groups related classes and method can be grouped into a single unit.
		\item Eg: All classes and interfaces for File I/O operations are group into \textbf{java.io} package.
		\item Advantage of package:
		\begin{itemize}
			\item \textbf{Resolve naming conflict}
			\item \textbf{Modularity:} Modularises the application
			\item \textbf{Maintainability:} Improves application maintainability.
			\item \textbf{Security:} Provides security for our components
		\end{itemize}

	\end{itemize}

	\textbf{Naming for Package}
	\begin{itemize}
		\item There is one universally accepted naming convention for packages, i.e to use internet domain name in reverse:
		\newline
		Eg:
		\codeblockfull{Test.java}{
			package com.lavatech.www; \\
			class Test \{ \\
			\s	public static void main(String[] args) \{ \\
			\s \s		int a=10; \\
			\s \s 		System.out.println(a); \\
			\s	\} \\
			\}
		}
		\bigskip
		
 		\item Below command can be used to directory structure at valid location.
 		\bigskip
 		\syntaxblock{
 			\$ javac -d <location> <filename.java>
 		}
 		\newpage
 		Eg:
 		\commandblock{
 			\# To create directory structure in current location \\
 			\$ javac -d . Test.java    \\
 			\\
 			\# To create directory structure under "D:" location \\
 			\$ javac -d D: Test.java   
 		}
 		
		Above command will create below directory structure:
		\newimage{0.8}{content/chapter10/images/mew.png}
		
		\item While running, make sure that you provide fully qualified name:
		\bigskip
		\commandblock{
		\$ java com.lavatech.www.Test  \\
		10	
		}
		
	\end{itemize}

\newpage

\textbf{Important pointers:}
\begin{itemize}
	\item There can be utmost one package statement in any Java source file.
	\newline
	Eg 1:
	\bigskip
	\codeblock{
		package pack1; \\
		package pack2;  \xmark \s \# Result in compile-time error \\
		public class A {} 	
	}
	Eg 2:
	\bigskip
	\codeblock{
		package pack1; \cmark \\
		public class A \{\} 	
	}

	\item The first non comment statement should be package statement (if it is available), otherwise we will get compile-time error.
	\bigskip
	\codeblock{
		import java.util.*; \xmark \s \# Result in compile-time error  \\
		package pack1; \\
		package pack2; \\
		public class A \{\}
	}	
	
	\item The following is valid order in any Java source file:
	\bigskip
	\codecontinue{
		package statement;   $\leftarrow$  Atmost one\\
		import statements;  $\leftarrow$ Any number		\\
		class | interface | enum declarations   $\leftarrow$ Any number
	}
	
	\item An empty source file is a valid java program. Hence the following are valid java source files:
	\bigskip
	\codeblockfull{Test.java}{
	\# Empty file	
	}

	\bigskip
	\codeblockfull{Test.java}{
	pacakge pack1;
	}	

	\bigskip
	\codeblockfull{Test.java}{
		import java.util.*;
	}

	\bigskip
	\codeblockfull{Test.java}{
		package pack1; \\
		import java.util.*;
	}

	\bigskip
	\codeblockfull{Test.java}{
		class Test \{\}
	}
\end{itemize}

	
\end{flushleft}

\newpage

