\setlength{\columnsep}{3pt}
\begin{flushleft}

	Package is group of Java ".class" files in a folder.

	\textbf{Package naming convention}
	\begin{itemize}
		\item Universally accepted naming convention for packages is \textbf{internet domain name in reverse}.
		\item Eg: For domain name \textbf{www.lavatech.com}, package name is \textbf{com.lavatech.www}.
		\item Eg: Create program with package name \textbf{com.lavatech.www}:
		\codeblockfull{Test.java}{
			package com.lavatech.www; \\
			class Test \{ \\
			\s	public static void main(String[] args) \{ \\
			\s \s		int a=10; \\
			\s \s 		System.out.println(a); \\
			\s	\} \\
			\}
		}
		\item Command to create package and place compiled ".class" file in it:
		\bigskip
		\syntaxblock{
			\$ javac -d <location> <filename.java>
		}
		Eg:
		\commandblock{
			\# Create package in current location \\
			\$ javac -d . Test.java    \\
			\\
			\# Create package under "D:" location \\
			\$ javac -d D: Test.java   
		}
		
		Above command will create below directory structure:
		\newimage{0.6}{content/chapter10/images/mew.png}
		
		\item While executing, use package name and class name:
		\bigskip
		\commandblock{
			\$ java com.lavatech.www.Test  \\
			10	
		}
		
	\end{itemize}


\textbf{Important pointers:}
\begin{itemize}
	\item There can be utmost one package statement in any Java source file.
	\newline
	Eg:
	\bigskip
	\codeblock{
		package pack1;  \cmark \\
		package pack2;  \xmark \s \# Result in compile-time error \\
		public class A \{\} 	
	}
	\item The first non comment statement should be package statement (if it is available), otherwise we will get compile-time error.
	\bigskip
	\codeblock{
		import java.util.*; \xmark \s \# Result in compile-time error  \\
		package pack1; \\
		package pack2; \\
		public class A \{\}
	}	
	
	\item The following is valid order in any Java source file:
	\bigskip
	\codecontinue{
		package statement;   $\leftarrow$  Atmost one\\
		import statements;  $\leftarrow$ Any number		\\
		class | interface | enum declarations   $\leftarrow$ Any number
	}

	
\end{itemize}

	\newpage
	
	\textbf{Example of creating and importing package}:
	\begin{itemize}
		\item Eg 1: Create "Test.class" in package \textbf{com.lavatech.www}.
		\codeblockfull{Test.java}{
			package com.lavatech.www; \\
			public class Test \{ \\
			\s	public int i=23; \\
			\}	
		}
		Create program named \textbf{Apply.java} that would import \textbf{com.lavatech.www.Test}:
		\codeblockfull{Apply.java}{
			import com.lavatech.www.Test; \\
			public class Apply \{ \\
			\s	public static void main(String[] args) \{ \\
			\s \s		Test t1 = new Test(); \\
			\s \s		System.out.println(t1.i); \\
			\s	\} \\
			\}	
		}
		Compile both \textbf{Test.java} and \textbf{Apply.java}:
		\commandblock{
			\$ javac -d . Test.java  \\
			\$ javac Apply.java
		}
		Execute the \textbf{Apply.java} file:
		\commandblock{
			\$ java Apply  \\
			23
		}
	\end{itemize}
	
	\newpage	
	\textbf{Example of Java code importing internally in package}
	\begin{itemize}
		\item Eg2: Create below directory structure with 2 empty programs \textbf{App1.java} and \textbf{App2.java} as shown  below:
		\newimage{0.8}{content/chapter10/images/new.png}
		\item Create program \textbf{App1.java}:
		\codeblockfull{App1.java}{
			package com.lavatech.www; \\
			public class App1 \{ \\
			\s	public String name="Lavatech-App1"; \\
			\}
		}
		\item Create program \textbf{App2.java}:
		\codeblockfull{App2.java}{
			package com.lavatech.www; \\
			public class App2 \{ \\
			\s	public String name="Lavatech-App2"; \\
			\s	public static void main(String[] args) \{ \\
			\s \s		App1 app1 = new App1(); \\
			\s \s		System.out.println(app1.name); \\
			\s	\} \\
			\}
		}
		\item Compile the \textbf{App1.java} \& \textbf{App2.java}
		\commandblock{
			\$ javac com/lavatech/www/App1.java  \\
			\$ javac com/lavatech/www/App2.java 
		}
		\newpage
		\item Execute \textbf{App2.java}:
		\commandblock{
			\$ java com.lavatech.www.App2 \\
			Lavatech-App1
		}
		
		\item Create \textbf{apply.java} outside the \textbf{com/lavatech/www} package:
		\newimage{0.9}{content/chapter10/images/new2.png}
		\codeblockfull{apply.java}{
			import com.lavatech.www.App1; \\
			public class apply \{ \\
			\s	public static void main(String[] args) \{ \\
			\s \s		App1 app = new App1(); \\
			\s \s		System.out.println(app.name); \\
			\s	\}  \\
			\}
		}
		\item Execute \textbf{apply.java}:
		\commandblock{
			\$ javac apply.java  \\
			\$ java apply  \\
			Lavatech-App1
		}
	\end{itemize}
	
	\newpage	
	\textbf{Example of using classpath and having Java code in different directory structure:}
	
	\begin{itemize}
		\item Create 3 empty folders and 3 java code as shown below:
		\newimage{1}{content/chapter10/images/struct.png}
		
		\item Content of A.java should be:
		\bigskip
		\codeblockfull{A.java}{
			package level1.level2; \\
			public class A \{ \\
			\s	public int a=10; \\
			\}	
		}
		
		Content of B.java should be:
		\bigskip
		\codeblockfull{B.java}{
			package level3.level4; \\
			public class B \{ \\
			\s	public int b=20; \\
			\}	
		}
		\newpage
		Content of C.java should be:
		\codeblockfull{C.java}{
			package level5.level6; \\
			import level1.level2.A; \\
			import level3.level4.B; \\
			public class C \{ \\
			\s	public static void main(String[] args)\{ \\
			\s \s		A a1 = new A(); \\
			\s \s		B b1 = new B(); \\
			\s \s		System.out.println(a1.a); \\
			\s \s		System.out.println(b1.b); \\
			\s	\}
			\}	
		}
	
		\item Compile all the 3 codes such that "A.class" file is placed under folder1, "B.class" file is placed under folder2, "C.class" file is placed under folder3.
		\bigskip
		\commandblock{
			\$ \textbf{javac -d folder1 A.java}  \\
			\$ \textbf{javac -d folder2 B.java}  \\
			\$ \textbf{javac -d folder3 -cp folder1:folder2 C.java}
		}
		Above commands should create below directory structure:
		\newimage{0.7}{content/chapter10/images/struct2.png}
		\newpage
		\item Execute class "C"  by setting classpath as shown:
		\bigskip
		\commandblock{
		\$ \textbf{java -cp folder1:folder2:folder3 level5.level6.C }\\
		10 \\
		20
		}
		
	\end{itemize}
	
\end{flushleft}
