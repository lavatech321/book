\setlength{\columnsep}{3pt}
\begin{flushleft}
	
	\begin{itemize}
		\item User-defined exception are custom exceptions created by programmer to meet programming requirements. 
		\item Eg:
		\begin{itemize}
			\item InvalidAgeException
			\item InvalidEmailException
			\item InSufficientBalanceException
		\end{itemize}
		\item Rules for creating custom exception:
		\begin{itemize}
			\item Define customised exceptions as unchecked i.e extend RuntimeException but not Exception.
			\item Inside every customised exception, super(arg) is needed to make description available to default exception handler.
			\item You will need to explicitly throw custom exception using "throw" keyword.
		\end{itemize}
		\bigskip
		\syntaxblock{
			class ExceptionName extends RuntimeException \{ \\
			\s	ExceptionName(String msg) \{  \\
			\s \s		super(msg); \\
			\s	\} \\
			\}
		}
		\item Below code create custom exception named "InvalidAgeException":
		\codeblockfull{demo.java}{
			class InvalidAgeException extends RuntimeException \{  \\
			\s	InvalidAgeException(String msg) \{ \\
			\s \s 		super(msg); \\
			\s	\} \\
			\} \\
			public class demo \{ \\
			\s	public static void main(String[] args) \{  
		}
	
		\newpage
		
		\codecontinue{
			\s \s   	try \{ \\
			\s \s    	int age = Integer.parseInt(args[0]);  \\
			\s \s		if(age < 0 | age > 100) \{ \\
			\s \s 				throw new InvalidAgeException("You seems have have entered invalid age!"); \\
			\s \s			\} \\
			\s \s			else \{ \\
			\s \s			System.out.println("Age: " + age + " is valid!"); \\
			\s \s			\} \\
			\s \s		\} \\
			\s \s		catch (InvalidAgeException e) \{ \\
			\s \s \s			System.out.println(e.toString()); \\
			\s \s		\} \\
			\s 	\} \\
			\} 
		}
		\bigskip
		\outputblock{
			\$ javac demo.java  \\
			\$ java demo 45 \\
			Age: 45 is valid! \\
			\$ java demo 450 \\
			InvalidAgeException: You seems have have entered invalid age!
		}
		
	\end{itemize}
	
\end{flushleft}

\newpage




