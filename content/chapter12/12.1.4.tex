\setlength{\columnsep}{3pt}
\begin{flushleft}

	\begin{itemize}
		\item An abstract method has no body!
		\item If you declare an abstract method, you MUST mark the class abstract as well. \item You can’t have an abstract method in a non-abstract class.
		\item Abstract methods are declared using the abstract keyword, and they end with a semicolon instead of a method body.
		\bigskip
		\syntaxblock{
			abstract class Classname() \{ \\
			\s public abstract void methodname1();  \\
			\s public abstract void methodname2();  \\
			\}
		}
		\bigskip
		
		\item These methods are intended to be overridden by the subclasses. 
		\item Subclasses \textbf{must provide the implementation} for all the abstract methods. 
		\item Eg:
		\bigskip
		\codeblockfull{Main.java}{
		abstract class Animal \{ \\
		\s	private String name; \\
		\s	public Animal(String name) \{ \\
		\s\s		this.name = name; \\
		\s	\}    \\
		\s	public String getName() \{ \\
		\s\s		return name; \\
		\s	\} \\
		\s	public abstract void sound(); \\
		\} 
		}
		\newpage
		\codecontinue{
		class Dog extends Animal \{ \\
			\s	public Dog(String name) \{ \\
			\s\s		super(name); \\
			\s	\}  \\
		\s	public void sound() \{ \\
		\s\s		System.out.println("Woof!"); \\
		\s	\} \\
		\} \\
		class Cat extends Animal \{ \\
		\s	public Cat(String name) \{ \\
		\s\s		super(name); \\
		\s	\} \\
		\s	public void sound() \{ \\
		\s\s		System.out.println("Meow!"); \\
		\s	\} \\
		\} \\
		\\
		class Main \{ \\
		\s	public static void main(String[] args) \{ \\
		\s\s		Animal dog = new Dog("Buddy"); \\
		\s\s		Animal cat = new Cat("Whiskers"); \\
		\\
		\s\s		System.out.println(dog.getName()); \\
		\s\s		dog.sound(); \\
		\\	
		\s\s		System.out.println(cat.getName()); \\
		\s\s		cat.sound(); \\
		\s	\} \\
		\}
		}
		\bigskip
		\outputblock{
			Buddy  \\
			Woof!  \\
			Whiskers \\
			Meow!
		}
	\end{itemize}

\end{flushleft}


