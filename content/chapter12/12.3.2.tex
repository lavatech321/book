\setlength{\columnsep}{3pt}
\begin{flushleft}
	
	\begin{itemize}
		\item Implementing an interface requires implementing each method of interface.
		\item Interface methods should always by declared as \textbf{public}.
		\item Interface methods are abstract whether we are declaring or not.			
		\item Hence inside interface the ofllowing method declaration are equal:
		\bigskip
		\codeblock{
			void m1(); \\
			public void m1(); \\
			abstract void m1(); \\
			public abstract void m1();
		}
		
		\item Eg:
		\codeblockfull{Circle.java}{
			public interface Drawable \{ \\
			\s void draw(); \\
			\s	double calculateArea(); \\
			\} \\
			public class Circle implements Drawable \{ \\
			\s	private double radius; \\
				\\
			\s	public Circle(double radius) \{ \\
			\s \s		this.radius = radius; \\
			\s	\}  \\
					\s	@Override \\
			\s	public void draw() \{ \\
			\s \s		System.out.println("Drawing a circle."); \\
			\s	\} \\
									\s	@Override \\
			\s	public double calculateArea() \{ 
		}
		
		\newpage
		\codecontinue{
			\s \s		return Math.PI * radius * radius; \\
			\s	\} \\
		\s public static void main(String[] args) \{ \\
		\s \s	Circle circle = new Circle(4); \\
		\s \s	circle.draw(); \\
		\s \s	System.out.println(circle.calculateArea()); \\
		\s \} \\
		\}
		}
	
		\bigskip
		\outputblock{
			Drawing a circle. \\
			50.26548245743669
		}
	\end{itemize}

	
\end{flushleft}
\newpage

