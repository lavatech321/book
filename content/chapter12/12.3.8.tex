\setlength{\columnsep}{3pt}
\begin{flushleft}
	
	\begin{itemize}
		\item Adapter class is a simple Java class that implements an interface with only empty implementation.
		\item Using adapter class, you can extend the adapter class and override only the methods they need, instead of implementing all the methods of the interface.
		\item Eg:
		\bigskip
		\codeblockfull{AudioPlayer.java}{
			public interface MediaPlayer \{ \\
			\s void play(); \\
			\s	void pause(); \\
			\s	void stop(); \\
			\} \\
			\\
			abstract class MediaPlayerAdapter implements  MediaPlayer \{ \\
			\s @Override \\
			\s	public void play() \{ \\
			\s \s		// Default implementation \\
			\s	\} \\
				\\
			\s	@Override \\
			\s	public void pause() \{ \\
			\s \s		// Default implementation \\
			\s	\} 
		}
	
		\newpage
		\codecontinue{
		\s 	@Override \\
		\s	public void stop() \{ \\
		\s \s		// Default implementation \\
		\s	\} \\
		\} \\
		\\
		public class AudioPlayer extends MediaPlayerAdapter \{ \\
		\s	@Override \\
		\s	public void play() \{ \\
		\s \s		// Implementation specific to AudioPlayer  \\
		\s \s	System.out.println("Playing audio."); \\
		\s	\} \\
		\s	public static void main(String[] args) \{ \\
		\s \s		AudioPlayer player = new AudioPlayer(); \\
		\s \s		player.play(); \\
		\s	\} \\
		\}
		}
		
		\bigskip
		\outputblock{
			Playing audio.	
		}
		
	\end{itemize}
\end{flushleft}
\newpage

