\setlength{\columnsep}{3pt}
\begin{flushleft}
	
	The return statement is used to exit a function and go back to the place from
	where it was called.
	
	\begin{tcolorbox}[breakable,notitle,boxrule=1pt,colback=pink,colframe=pink]
		\color{black}
		\fontdimen2\font=8pt
		Syntax: 
		\newline
		def function\_name(parameters):
		\newline
		\hphantom{} \hphantom{} statement(s) \newline
		\hphantom{} \hphantom{} return expression
		\fontdimen2\font=4pt
	\end{tcolorbox}			
	
	\textbf{Rules of return statement:}
	\begin{itemize}
		\item Use the return statement inside a function only.
		\item Every function returns something. If there are no return statements, then it returns None.
		\item If the return statement contains an expression, it’s evaluated first and then the value is returned.
		\item The return statement terminates the function execution.
		\item A function can have multiple return statements. When any of them is executed, the function terminates.
		\item A function can return any of the 13 datatypes.
		\item A function can return a single values in a single return statement.
	\end{itemize}
	
	\newpage
	
	Sample code:
	\begin{tcolorbox}[breakable,notitle,boxrule=-0pt,colback=black,colframe=black]
		\color{green}
		\fontdimen2\font=9pt
		def accept(no): \newline
		\hphantom{} \hphantom{} names=[] \newline
		\hphantom{} \hphantom{} for x in range(no): \newline
		\hphantom{} \hphantom{} \hphantom{} \hphantom{} name=input("Enter student name: ") \newline
		\hphantom{} \hphantom{} \hphantom{} \hphantom{} names.append(name) \newline
		return names \newline
		\newline
		ans=accept(3) \newline
		print(ans)
		\fontdimen2\font=4pt
	\end{tcolorbox}
	
	Output:
	\begin{tcolorbox}[breakable,notitle,boxrule=-0pt,colback=output,colframe=output]
		\color{black}
		Enter student name: Raman \newline
		Enter student name: Shami \newline
		Enter student name: Kavi \newline
		['Raman', 'Shami', 'Kavi']
		\fontdimen2\font=4pt
	\end{tcolorbox}
	
	\bigskip
	Sample code:
	\begin{tcolorbox}[breakable,notitle,boxrule=-0pt,colback=black,colframe=black]
		\color{green}
		\fontdimen2\font=9pt
		def accept(no): \newline
		\hphantom{} \hphantom{} info=\{\} \newline
		\hphantom{} \hphantom{} for x in range(no): \newline
		\hphantom{} \hphantom{} \hphantom{} \hphantom{} marks=[] \newline
		\hphantom{} \hphantom{} \hphantom{} \hphantom{}name=input("Enter student name: ") \newline
		\hphantom{} \hphantom{} \hphantom{} \hphantom{} for x in range(3): \newline
		\hphantom{} \hphantom{} \hphantom{} \hphantom{} \hphantom{} \hphantom{} m=int(input("Enter marks: ")) \newline
		\hphantom{} \hphantom{} \hphantom{} \hphantom{} \hphantom{} \hphantom{} marks.append(m) \newline
		\hphantom{} \hphantom{} \hphantom{} \hphantom{} info.update(\{name:marks\}) \newline
		\hphantom{} \hphantom{} return info \newline
		\newline 
		ans=accept(2) \newline
		print(ans)
		\fontdimen2\font=4pt
	\end{tcolorbox}
	
	Output:
	\begin{tcolorbox}[breakable,notitle,boxrule=-0pt,colback=output,colframe=output]
		\color{black}
		Enter student name: Ram \newline
		Enter marks: 34 \newline
		Enter marks: 45 \newline
		Enter marks: 55 \newline
		Enter student name: Raman \newline
		Enter marks: 34 \newline
		Enter marks: 23 \newline
		Enter marks: 78 \newline
		{'Ram': [34, 45, 55], 'Raman': [34, 23, 78]}
		\fontdimen2\font=4pt
	\end{tcolorbox}
		
\end{flushleft}

\newpage

