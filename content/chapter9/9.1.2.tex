\setlength{\columnsep}{3pt}
\begin{flushleft}
	\bigskip
	\begin{itemize}
		\item Runtime  errors are called exceptions.
		\item An exception is a Python object which represents an error. 
		\item If not handled properly, it prints a trackback error.
		\item Built-in exceptions are display with:
		\begin{tcolorbox}[breakable,notitle,boxrule=-0pt,colback=code,colframe=code]
			\color{white}
			\fontdimen2\font=8pt
			print(dir(\_\_builtins\_\_))
			\fontdimen2\font=4pt
		\end{tcolorbox}
	
		Output:
		\begin{tcolorbox}[breakable,notitle,boxrule=-0pt,colback=output,colframe=output]
			\color{black}
			['ArithmeticError', 'AssertionError', 'AttributeError', 'BaseException',
			\newline
			...]
			\fontdimen2\font=4pt
		\end{tcolorbox}		
	
		\item Frequently faced exceptions are:
		\begin{itemize}
			\item ModuleNotFoundError 
			\item FileNotFoundError
			\item ZeroDivisionError
			\item ImportError
			\item FileExistsError
			\item ValueError
			\item StopIteration
		\end{itemize}
	
	
	\end{itemize}
	
\end{flushleft}

\newpage

