\setlength{\columnsep}{3pt}
\begin{flushleft}
	\bigskip

	Error occurs when a certain statement is not in accordance with the prescribed usage.
	\newline
	Three kinds of errors in Python: 
	\begin{itemize}
		\item Syntax errors:
		\begin{itemize}
			\item Syntax error in your code will result in execution failure. 
			\item It is the easiest error type you can fix.
		\end{itemize}
	
		\bigskip
		\item Exceptions:
		\begin{itemize}
			\item Errors detected during program execution.
			\item They are not unconditionally fatal.
			\item Programs do not handle exceptions, and result in error messages as:
			
			\begin{tcolorbox}[breakable,notitle,boxrule=-0pt,colback=error,colframe=error]
				\color{white}
				Traceback (most recent call last):
				\newline
				File "one.py", line 2, in <module>
				newlin	
				print(a[120])
				\newline
				\textbf{IndexError: string index out of range}
				\fontdimen2\font=4pt
			\end{tcolorbox}		
			
			
		\end{itemize}
		
		\bigskip
		\item Logical errors: 
		\begin{itemize}
			\item Most difficult errors to fix.
			\item They don’t crash your code.
			\item You don’t get any error message.
			\item Your code does not run as you expected.
		\end{itemize}
		
		
	\end{itemize}
	
	

	
\end{flushleft}

\newpage

