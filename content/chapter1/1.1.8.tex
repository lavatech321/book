\begin{flushleft}
	
	\quest{Compiler and JVM battle over the question, “Who’s more important?”}{
		Go through below discussion to find the answer:\\ \\
		\textbf {JVM:} I am Java. I’m the guy who actually makes a program run. The compiler just gives you a file in bytecode after checking it’s syntax. I'm the one who run it.
		\\ \\
		\textbf{Compiler:} Excuse me? Without me, you would have to translate everything from source code and be very very slow!
		\\ \\	
		\textbf{JVM:} Your work is not important. A programmer could just write bytecode by hand. You might be out of a job soon, buddy.
		\\ \\
		\textbf{Compiler:} That’s arrogant. A programmer writing bytecode by hand is next to possible, some scholars might write, not everyone!
		\\ \\
		\textbf{JVM:} But you still didn’t answer my question, what you actually do?
		\\ \\
		\textbf{Compiler:} Remember that Java is a strongly-typed language, I can’t allow variables to hold data of the wrong type. This is a security feature, implement by ME!
		\\ \\
		\textbf{JVM:} Your type checking is not very strict! Sometimes people put the wrong type of data in an array of different type.
		\\ \\
		\textbf{Compiler:} Yes, that can emerge at runtime that only you can catch to allow dynamic binding. But my job is to stop anything that would never succeed at runtime. 
	}
	
	\newpage
	
	\anscontinue{
	
	\textbf{JVM:} OK. Sure. But what about security? Look at all the security stuff I do! You just perform silly syntax checking.
	\\ \\
	\textbf{Compiler:} Listen, I'm the first line of defense. I also prevents access violations, such as code trying to invoke a private method. I stop people from touching code they’re not meant to see.
	\\ \\
	\textbf{JVM:} Whatever. I have to do that same stuff too! 😛
		
	}
\end{flushleft}

\newpage