\setlength{\columnsep}{5pt}
\begin{flushleft}

	\begin{itemize}
		\item Introduced in Java9, JShell is a Read-Eval-Print Loop (REPL)
		\item It evaluates declarations, statements, and expressions as they are entered, and then it immediately shows the results.
		\item To use jshell, type “jshell” command in command prompt:
		\bigskip
		\commandblock{
			\$ jshell  \\
			jshell> int i=42; \\
			i ==> 42 \\
			\\
			jshell> float j=3.4f; \\
			j ==> 3.4 \\
			\\
			jshell> i+j \\
			\$3 ==> 45.4 \\
			\\
			jshell> String text = "Welcome To World of Java"; \\	
			text ==> "Welcome To World of Java" 
		}
		
		\newpage
		\item To display all variables declared:
		\bigskip
		\commandblock{
			jshell> /vars \\
			|    int i = 42 \\
			|    float j = 3.4 \\
			|    float \$3 = 45.4 \\
			|    String text = "Welcome To World of Java" \\
			|    String \$5 = "WELCOME TO WORLD OF JAVA"
		}
	
		\item To save all valid statements of Jshell to a file:
		
		\commandblock{
			jshell> /save filename.java \\
			\\
			jshell> /exit \\
			|  Goodbye
		}
		
		\item Content of filename.java:
		\codeblockfull{filename.java}{
			int i=42; \\
			float j=3.4f; \\
			i+j \\
			String text = "Welcome To World of Java"; \\
			text.toUpperCase()
		}
		
		\item You can open the file back in the jshell using open command:
			\commandblock{
				jshell> /open filename.java \\
				 \\
				jshell> i \\
				i ==> 42
			}
		
	\end{itemize}

\end{flushleft}

