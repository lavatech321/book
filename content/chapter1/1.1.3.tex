\setlength{\columnsep}{20pt}
\begin{flushleft}

	\item \quest{What year Java was invented?}{1995}

	\bigskip
	\quest{What company invented Java?}{Sun Microsystems \\ \\
		\boximage{0.3}{content/chapter0/images/sun.png}
	}

	\bigskip
	\quest{Who is founder of Java?} {
	James Gosling  \\ \\
	\boximage{0.3}{content/chapter0/images/james.jpg}
	}
  
  	\bigskip
  	\quest{What is Java mascot?}{
  	A cartoon character named Duke \\ \\
  	\boximage{0.3}{content/chapter0/images/mascot.png}
  	}
  
  	\bigskip
  	\quest{What is the original name of Java ?}{
  	"Oak" after the oak tree that was outside Gosling's office. 
  	}

	\bigskip
	\quest{What was the reason for changing original name?}{
	"Oak" was already trademarked	
	}
	
	\bigskip
	\quest{What is the inspiration behind Java's name?}{
	Java language is named after coffee grown on the Indonesian island
	}
	
	\bigskip
	\quest{What is original Java logo?}{
		Original logo: \\ \\
	\boximage{0.3}{content/chapter0/images/jlogo.png}
	}

	\bigskip
	\quest{Who has the current ownership of Java?}{Oracle acquired Java in 2009}
	
	\newpage
	
	\quest{What are the Java versions?}{
		Below are the Java version details:
		
		\tableinsidebox{
		\hline
		Version &  Year \\
		\hline
		JDK Alpha and Beta & 1995  \\
		\hline
		JDK 1.0 & Jan, 1996 \\
		\hline
		JDK 1.1 & Feb, 1997 \\
		\hline
		J2SE 1.2 or \textbf{Java2} (codename: Playground) & Dec, 1998 \\
		\hline
		J2SE 1.3 or \textbf{Java2} (codename: Kestrel) &  May, 2000 \\
		\hline
		J2SE 1.4 or \textbf{Java2} (codename: Merlin) & Feb, 2002 \\
		\hline
		J2SE 1. 5 or \textbf{Java5} (codename: Tiger) & Sep, 2004 \\
		\hline
		Java SE 1.6 or \textbf{Java6} (codename: Mustang) &  Dec, 2006 \\
		\hline
		Java SE 1.7 or \textbf{Java7} (codename: Dolphin) & July, 2011 \\
		\hline
		Java SE 1.8 or \textbf{Java8} (codename: Spider) & (18th March, 2014) \\
		\hline
		Java SE 1.9 or \textbf{Java9} & September, 2017 \\
		\hline
		\textbf{Java 10} & March, 2018 \\
		\hline
		\textbf{Java SE 11} & September 2018 \\
		\hline
		\textbf{Java SE 12} & March 2019 \\
		\hline
		\textbf{Java SE 13} & September 2019 \\
		\hline
		\textbf{Java SE 14} & March 2020 \\
		\hline
		\textbf{Java SE 15} & September 2020 \\
		\hline
		\textbf{Java SE 16} & March 2021 \\
		\hline
		\textbf{Java SE 17} & September 2021 \\
		\hline
		}
		
	}

	\newpage
	\quest{Why is Java 2 consider very significant in history of Java?}{		
		Starting \textbf{Java 2}, it is composed of three parts:
		\begin{itemize}
			\item \textbf{J2SE (Java 2 Platform, Standard Edition) or JSE}, a computing platform for the development and deployment of portable code for \textbf{desktop and server environments}.
			\item \textbf{J2EE (Java 2 Platform, Enterprise Edition) or JEE}, extending Java SE with enterprise features such as \textbf{distributed computing and web services}.
			\item \textbf{J2ME (Java 2 Platform, Micro Edition) or JME}, a computing platform for \textbf{embedded and mobile devices}.
		\end{itemize}
		Other major highlights of this release:
		\begin{itemize}
			\item \textbf{JIT compiler} became part of JVM (means turning code into executable code became a faster operation).
			\item \textbf{Swing graphical API} was introduced as alternative to AWT.
			\item Java collections framework (for working with sets of data) was introduced.
		\end{itemize}
	}

	\quest{I see Java 2 and Java 5.0, but was there a Java 3 and 4? And why is it Java 5.0 but not Java 2.0?}{
		The joys of marketing... 
		\begin{itemize}
			\item When the version of Java shifted from 1.1 to 1.2, the changes to Java were so many that the marketers decided a whole new “name”, so they started calling it Java 2, even though the actual version of Java was 1.2.
			\item But versions 1.3 and 1.4 were still considered Java 2.
			\item There never was a Java 3 or 4. 
			\item Beginning with Java version 1.5, the marketers decided a new name was needed. 
			\item The next number in the name sequence would be “3”, but calling Java 1.5 Java 3 seemed more confusing, so they decided to name it Java 5.0 to match the “5” in version “1.5”. 	
		\end{itemize}
	}

\end{flushleft}


