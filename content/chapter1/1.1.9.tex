\begin{flushleft}
	
	\newimage{0.5}{content/chapter0/images/new05.png}
	\begin{itemize}
		\item \textbf{Class Loader:} Responsible for loading the class files into the memory of the JVM.
		\item \textbf{Execution Engine:} Responsible for executing the bytecode that is loaded into the memory. It includes:
		\begin{itemize}
			\item \textbf{Interpreter:} Reads and executes the bytecode one instruction at a time. 
			\item \textbf{JIT compiler:} Compiles the bytecode into machine code for fast execution.
		\end{itemize}
		
		\item \textbf{Garbage Collector:} It periodically frees up the memory that is not used by the Java application.
		
		\item \textbf{Runtime Data Area:} It is memory space allocated by the JVM for the execution of the Java application. It includes:
		\begin{itemize}
			\item Method area
			\item Heap
			\item Stack
			\item PC registers
		\end{itemize}
		\item \textbf{Native Method Interface (JNI):} Allows Java code to call code written in other programming languages like C and C++. The JNI allows Java applications to interact with OS and hardware.
		
	\end{itemize}
	
	
\end{flushleft}

\newpage