\setlength{\columnsep}{5pt}
\begin{flushleft}

		\begin{itemize}
			\item Using simple text editor:
			\bigskip
			\codeblockfull{MyFirstApp.java}{
				public class MyFirstApp \{ \\
				\s	public static void main(String[] args) \{ \\
				\s \s		System.out.println("I Rule!"); \\
				\s \s		System.out.println("The World"); \\
				\s	\} \\
				\}	
		
			}
			\bigskip
			\commandblock{
				javac MyFirstApp.java \\
				java MyFirstApp 
			}
			\bigskip
			\outputblock{
				I Rule! \\
				The World
			}
		
			\newpage
			\item Using Eclipse:		
			\begin{enumerate}
				\item Create New Project:
				\\
				\textbf{File -> New -> Java Project -> Add “Starter”} as project name
				
				\newimage{0.35}{content/chapter0/images/new11.png}
				
				\item This will create directory structure as shown below:
				
				\newimage{0.8}{content/chapter0/images/new12.png}
				
				\newpage
				\item Right click the src -> Select “Package” -> Add “Start1” as package name as shown below:
				
				
				\newimage{0.35}{content/chapter0/images/new13.png}
				
				\item Right click the “Start1” -> Select New -> Class -> Add “MyFirstClass” as classname as shown below:
				
				\newimage{0.4}{content/chapter0/images/new14.png}
				
				\newpage
				\item This will create a class with below structure:
				\bigskip
				\codeblock{
					package Start1; \\
					public class MyFirstClass \{ \\
					\s	public static void main(String[] args) \{ \\
					\s	\s	// TODO Auto-generated method stub \\
					\s \s	System.out.println("Welcome Back!"); \\
					\s	\} \\
					\}
				}
			
				\item Execute the code by pressing the run button:
				
				\newimage{0.65}{content/chapter0/images/new15.png}
				
			\end{enumerate}
			
			
		\end{itemize}
		
\end{flushleft}

\newpage
