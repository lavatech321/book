\setlength{\columnsep}{3pt}

\begin{flushleft}

	
	The \textbf{for} loop is used to iterate over a sequence (list, tuple, string, set, frozenset, dictionary, bytes, bytearray) or
other iterable objects.

	
	\begin{tcolorbox}[breakable,notitle,boxrule=1pt,colback=pink,colframe=pink]

		\color{black}

		\fontdimen2\font=8pt

		Syntax: 

		\newline

		for x in iterable:

		\newline

		\hphantom{} \hphantom{}  statement1 \newline
		\hphantom{} \hphantom{}  statement2..

		\fontdimen2\font=4pt

	\end{tcolorbox}

	Below are the simple examples of \textbf{"for"} loop:
	\newline
	Sample code 1:
for loop with string

	\begin{tcolorbox}[breakable,notitle,boxrule=-0pt,colback=code,colframe=code]

		\color{white}

		\fontdimen2\font=8pt

		for x in range(1,2):  \newline
		\hphantom{} \hphantom{}	for y in range(1,6): \newline
		\hphantom{} \hphantom{}	\hphantom{} \hphantom{}	print(f"{x}*{y}={x*y}")  \newline
		\hphantom{} \hphantom{}	print()

		\fontdimen2\font=4pt

	\end{tcolorbox}

	
	Output:

	\begin{tcolorbox}[breakable,notitle,boxrule=-0pt,colback=output,colframe=output]

		\color{black}

		1*1=1 \newline
		1*2=2 \newline
		1*3=3 \newline
		1*4=4 \newline
		1*5=5
		\fontdimen2\font=4pt

	\end{tcolorbox}

	
\end{flushleft}


\newpage


