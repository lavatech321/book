\setlength{\columnsep}{3pt}
\begin{flushleft}

	Use if to specify a block of code to be executed, if a specified condition is true
	
	\begin{tcolorbox}[breakable,notitle,boxrule=1pt,colback=pink,colframe=pink]
		\color{black}
		\fontdimen2\font=8pt
		Syntax: 
		\newline
		if condition:
 \newline
		\hphantom{} \hphantom{}  statement1 \newline
		\hphantom{} \hphantom{}  statement2..
		\fontdimen2\font=4pt
	\end{tcolorbox}

	Sample code:
	\begin{tcolorbox}[breakable,notitle,boxrule=-0pt,colback=code,colframe=code]
		\color{white}
		\fontdimen2\font=8pt
			no1=int(input('Enter no1:')) \newline
			no2=int(input('Enter no2:')) \newline
			\newline
			if no1 > no2: \newline
			\hphantom{} \hphantom{}	print(f'Greater number is \{no1\}') \newline
			\newline
			print('Program ends here')
		\fontdimen2\font=4pt
	\end{tcolorbox}
	
	Output:
	\begin{tcolorbox}[breakable,notitle,boxrule=-0pt,colback=output,colframe=output]
		\color{black}
		Enter no1:56 \newline
		Enter no2:34 \newline
		Greater number is 56 \newline
		Program ends here
		\fontdimen2\font=4pt
	\end{tcolorbox}
			

	
	
	
\end{flushleft}

\newpage

