\setlength{\columnsep}{3pt}
\begin{flushleft}

	\begin{itemize}
		\item This is enhanced for loop introduced in Java 1.5 version.
		\item It is used to retrieve elements of arrays and collections.
		\bigskip
		\syntaxblock{
			for (type var : array) \{  \\
			\s	statements using var; \\
			\}	
		}
		\item Eg: Print elements of 1-dimensional array -
		\end{itemize}
		\tabletwo{
			\hline
			Normal for loop & Enhanced for loop \\
			\hline
			\codeblock{
				int[] x = \{10,20,30\}; \\
				for(int i=0;i<x.length;i++) \{ \\
				\s	System.out.println(x[i]); \\
				\}
			} &
			\codeblock{
				int[] x = \{10,20,30\}; \\
				for(int x1; x) \{ \\
				\s	System.out.println(x1); \\
				\}
			} \\
		}
	

	
	\begin{itemize}
		\item Eg: Print elements of 2-dimensional array -
	\end{itemize}

	\tabletwo{
		\hline
		Normal for loop & Enhanced for loop \\
		\hline

		\codeblock{
			int[][] x=\{\{10,20\},\{40,50\}\}; \\
			for(int i=0;i<x.length; i++)\{ \\
			for(int j=0;j<x[i].length;j++)\{ \\
			\s	System.out.println(x1); 	\\
			\s \} \\	
			\}
		} &
		\codeblock{
			int[][] x=\{\{10,20\},\{40,50\}\}; \\
			for(int[] x1: x) \{  \\
			for(int x2: x1) \{    \\
			\s	System.out.println(x1);  \\
			\s \}  \\
			\} 
		}  \\
	}
	
	\newpage
	
	\begin{itemize}
		
		\item Eg3: Print 3-dimensional array using for-each loop -
		\bigskip
		\codeblock{
			
			int[][][] x = \{ \\
			\s	\{ \{1, 2\}, \{3, 4\}, \{5, 6\}, \{7, 8\} \}, \\
			\s	\{ \{9, 10\}, \{11, 12\}, \{13, 14\}, \{15, 16\} \}, \\
			\s	\{ \{17, 18\}, \{19, 20\}, \{21, 22\}, \{23, 24\} \} \\
			\}; \\
			\\
			for(int[]][] x1:x) \{ \\
			\s for(int[] x2: x1 ) \{  \\
			\s \s for(int x3: x2) \{ \\
			\s \s \s System.out.println(x3); \\
			\s \s \} \\
			\s \}	 \\
			\} 
		}
		
		\item Drawback of for-each loop:
		\begin{itemize}
			\item Applicable only for arrays and collections.
			\item Using for-each loop, you can print array elements in original order but not in reverse order.	
		\end{itemize}
		
		
	\end{itemize}


	
\end{flushleft}

\newpage

