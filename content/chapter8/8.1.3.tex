\setlength{\columnsep}{3pt}
\begin{flushleft}
	
	\begin{itemize}
		\item If several options ar available, then it is not recommended to use nested if..else statement, as it reduces readability.
		
		\item Solution: switch statement
		\bigskip
		\syntaxblock{
			switch(argument) \{ \\
			\s	case arg-1:  	\\
			\s \s	action1;	\\
			\s \s	break;		\\
			\s	case arg-2:		\\
			\s \s	action2;		\\
			\s \s	break;		\\
			\s	case n:		\\
			\s \s	action-n;	\\
			\s \s	break		\\
			\s	default:		\\
			\s \s	Default action	\\
			\}
		}
		\bigskip
		\item Curly braces are mandatory.
		\item \textbf{Case and default are optional}, i.e an empty switch statement is a valid Java syntax.
		\codeblock{
			int x = 10; \\
			switch(x) \{\}  \cmark
		}
		\bigskip
		\item Allowed argument types in switch statement:
		\begin{itemize}
			\item Upto Java 1.4 version -> \textbf{byte, short, char, int}
			\item From Java 1.5 version -> byte, short, char, int, \textbf{wrapper classes (Byte, Short, Character, Integer), enum}
			\item From Java 1.7 versionbyte, short, char, int, wrapper classes (Byte, Short, Character, Integer), enum, \textbf{string}
		\end{itemize}
		
		\bigskip	
		\item Inside switch, every statement should be under some case or default.
		
		\codeblock{
			int x = 10; \\
			switch(x)\{   \\
			\s System.out.println();   \xmark \s // Compile-time error! \\
			\}
		}
		
		\item Case argument should be compile-time constant (i.e constant expression).
		\bigskip
		\codeblock{
			int x=10; \\
			int y=20; \\
			switch(x) \{ \\
			\s case 10:  \\
			\s \s System.out.println(10); \\
			\s \s	break; \\
			\s case y;    \xmark \s // Compile-time error!   \\         
			\s \s System.out.println(); \\
			\s \s	break; \\
			\} 
		}
		\bigskip
		\noteblock{
			If y is declared as "final", then there will be no compile-time error.
		}
		
		\newpage
		
		\item Switch arguments and case label can be expressions. But, case label should be constant expression.
		
		\codeblock{
			int x = 10; \\
			switch(x+1) \{ \\
			\s case 10: \\
			\s \s System.out.println(10); \\
			\s \s break; \\
			\s case 10+20+30: \\
			\s \s System.out.println(60); \\
			\s \s break; \\
			\}
		}
		\bigskip
		\item \textbf{Case label should be in range of switch arg type}, else it will result in compile-time error.
		\newline
		Eg 1:
		\codeblock{
			byte b = 10; \\
			switch(b) \{ \\			
			\s case 10: \\
			\s \s System.out.println(10); \\
			\s \s break; \\
			\s case 100: \\
			\s \s System.out.println(100); \\
			\s \s break; \\
			\s case 1000:   \xmark \s // Compille-time error! \\
			\s \s System.out.println(1000); \\
			\s \s break; \\				
			\}
		}
		\newpage
		
		Eg 2:
		\codeblock{
			byte b = 10; \\
			switch(b+1) \{  // Implicit type-caste to integer \\			 
			\s case 10: \\
			\s \s System.out.println(10); \\
			\s \s break; \\
			\s case 100: \\
			\s \s System.out.println(100); \\
			\s \s break; \\
			\s case 1000:  \\
			\s \s System.out.println(1000); \\
			\s \s break; \\				
			\}
		}
		
		\item Duplicate case labels are not allowed.
		\bigskip
		
		\codeblock{
			int x = 10; \\
			switch(x) \{  \\			 
			\s case 12: \\
			\s \s System.out.println(12); \\
			\s \s break; \\
			\s case 97: \\
			\s \s System.out.println(97); \\
			\s \s break; \\
			\s case 'a':   \xmark \s // Duplicate labels error    \\
			\s \s System.out.println(1000); \\
			\s \s break; \\				
			\}
		}
	\end{itemize}
	
	\newpage
	\textbf{Summary for case-label argument}
	\begin{itemize}
		\item It should  be constant expression.
		\item The value should be in the range of switch argument type.
		\item Duplicate case label are not allowed
	\end{itemize}
	
	\bigskip
	\bigskip
	\bigskip
	\textbf{Fall-through inside switch}
	\begin{itemize}
		\item Within the switch, if any case is matched, from that case onwards all statements will be executed until break or end of the switch.
		\item This is called fall-through inside switch.
		\item The main advantage of fall inside the switch is, we can define common action for multiple cases (code-reuseablilty).
		\bigskip
		\syntaxblock{
			switch(argument) \{  \\
			\s	case arg-1: \\
			\s	case arg-2: \\
			\s	case arg-3: \\
			\s \s		action; \\
			\s \s		break; \\
			\s	case arg-4: \\
			\s	case arg-5: \\
			\s	case arg-6: \\
			\s \s		action; \\
			\s \s		break; \\
			\}
		}
		\newpage
		Eg:
		\codeblock{
			int x = 0;
			switch(x) \{ \\
			\s case 0: \\
			\s \s System.out.println(0); \\
			\s case 1: \\
			\s \s System.out.println(1); \\
			\s \s break; \\
			\s case 2: \\
			\s \s System.out.println(2); \\
			\s default: \\
			\s \s System.out.print("default"); \\
			\}
		}
		\bigskip
		\outputblock{
			0 \\
			1
		}
		\item \textbf{Default case:}
		\begin{itemize}
			\item Within the switch, \textbf{you can take default case at most once}.
			\item Default case will be executed if and only if, there is no case matched.
			\item Within the switch, you can write default case anywhere but it is recommended to write as last case. \newline
			Eg:
			\codeblock{
				int x = 3;
				switch(x) \{ \\
				\s	default:  \\
				\s \s	System.out.println("default") \\
				\s case 0:  \\
				\s \s 	System.out.println(0) \\
				\s case 1: 
			}
			\newpage
			\codecontinue{
				\s \s	System.out.println(1) \\
				\s case 2: \\
				\s \s	System.out.println(2) \\
				\}
			}
			\bigskip
			\outputblock{
				default \\
				0
			}
		\end{itemize}
	\end{itemize}
	
	
\end{flushleft}

\newpage





