\setlength{\columnsep}{3pt}
\begin{flushleft}
	
	\begin{itemize}
		\item If several options ar available, then it is not recommended to use nested if..else statement, as it reduces readability.
		
		\item Solution: switch statement
		\bigskip
		\syntaxblock{
			switch(argument) \{ \\
			\s	case arg-1:  	\\
			\s \s	action1;	\\
			\s \s	break;		\\
			\s	case arg-2:		\\
			\s \s	action2;		\\
			\s \s	break;		\\
			\s	case n:		\\
			\s \s	action-n;	\\
			\s \s	break		\\
			\s	default:		\\
			\s \s	Default action	\\
			\}
		}
		\newpage
		\item Curly braces are mandatory.
		\item \textbf{Case and default are optional}, i.e an empty switch statement is a valid Java syntax.
		\codeblock{
			int x = 10; \\
			switch(x) \{\}  \cmark
		}
		\bigskip
		\item Allowed argument types in switch statement:
		\begin{itemize}
			\item Upto Java 1.4 version -> \textbf{byte, short, char, int}
			\item From Java 1.5 version -> byte, short, char, int, \textbf{wrapper classes (Byte, Short, Character, Integer), enum}
			\item From Java 1.7 versionbyte, short, char, int, wrapper classes (Byte, Short, Character, Integer), enum, \textbf{string}
		\end{itemize}
		
		\bigskip	
		\item Inside switch, every statement should be under some case or default.
		
		\codeblock{
			int x = 10; \\
			switch(x)\{   \\
			\s System.out.println();   \xmark \s // Compile-time error! \\
			\}
		}
		
		\item Case argument should be compile-time constant or declared as final.
		\bigskip
		\codeblock{
			int x=10; \\
			int y=20; \\
			switch(x) \{ \\
			\s case y;    \xmark \s // Compile-time error!   \\         
			\s \s System.out.println(); \\
			\s \s	break; \\
			\} 
		}
		\newpage
		
		\item Case label should be constant expression.
		\codeblock{
			int x = 10; \\
			switch(x+1) \{ \\
			\s case 10: \\
			\s \s System.out.println(10); \\
			\s \s break; \\
			\s case 10+20+30: \\
			\s \s System.out.println(60); \\
			\s \s break; \\
			\}
		}
		\bigskip
		\item \textbf{Case label should be in range of switch arg type}, else it will result in compile-time error.
		\newline
		Eg 1:
		\codeblock{
			byte b = 10; \\
			switch(b) \{ \\			
			\s case 10: \\
			\s \s System.out.println(10); \\
			\s \s break; \\
			\s case 100: \\
			\s \s System.out.println(100); \\
			\s \s break; \\
			\s case 1000:   \xmark \s // Compille-time error! \\
			\s \s System.out.println(1000); \\
			\s \s break; \\				
			\}
		}
		\newpage
		\item Duplicate case labels are not allowed.
		\bigskip
		
		\codeblock{
			int x = 10; \\
			switch(x) \{  \\			 
			\s case 97: \\
			\s \s System.out.println(97); \\
			\s \s break; \\
			\s case 'a':   \xmark \s // Duplicate labels error    \\
			\s \s System.out.println(1000); \\
			\s \s break; \\				
			\}
		}
	\end{itemize}
	
	\textbf{Summary for case-label argument}
	\begin{itemize}
		\item It should be constant expression.
		\item The value should be in the range of switch argument type.
		\item Duplicate case label are not allowed
	\end{itemize}
	
	\item \textbf{Default case:}
	\begin{itemize}
			\item Default case will be executed if and only if, there is no case matched.
			\item You can write default case anywhere but it is recommended to write as last case. \newline
			Eg:
			\codeblock{
				int x = 3; \\
				switch(x) \{ \\
				\s	default:  \\
				\s \s	System.out.println("default") \\
				\s case 0:  \\
				\s \s 	System.out.println(0); \\
				\s \s   break;
			}
			\bigskip
			\outputblock{
				default \\
				0
			}
		\end{itemize}
 
\end{flushleft}







