\setlength{\columnsep}{3pt}
\begin{flushleft}

	\begin{itemize}
		\item If you know number of iterations in advance then for loop is the best choice. 
		\newimage{0.6}{content/chapter8/images/for.png}
		
		\item Curly braces are optional, without curly braces only one statement is allowed, which should not be declarative statement.
		
		\item Egs:
		\bigskip
		\codeblock{
			for(int i = 0; true; i++) \\
			\s System.out.println("Hello");   \cmark
		}
		\bigskip
		\codeblock{
			for(int i = 0; i < 10; i++) ; \cmark
		}
		\bigskip
		\codeblock{
			for(int i = 0; i < 10; i++) \\
			\s int x = 10; \xmark
		}
		\item Let's see each section of for loop in detail:
		\begin{itemize}
			\item \textbf{Initialisation section:}
			\begin{itemize}
				\item This section will be executed only once in for loop lifecycle.
				\item Use to declare and initialise local variables.
				\item You can declare any number of variables, but should be of the same type. \item If you are trying to declare different datatype variables, then you'll get compile time error.
				\bigskip
				\codeblock{
					for(\textbf{int i = 0}; i < 10; i++) \{\} \cmark \\
					for(\textbf{int i = 0, j=0}; i < 10; i++) \{\} \cmark \\
					for(\textbf{int i = 0, String = "a"}; i < 10; i++) \{\} \xmark \\
					for(\textbf{int i = 0, int j = 0}; i < 10; i++) \{\} \xmark 
				}
				
				\item You can take any valid Java statement in this section.
				\item Eg:
				\bigskip
				\codeblock{
				int i = 0; \\
				for( \textbf{System.out.println("Hi")}; i < 3; i++ ) \{ \\
				\s System.out.println("Hello"); \\
				\}
				}
				\bigskip
				\outputblock{
					Hi \\
					Hello \\
					Hello \\
					Hello
				}
			\end{itemize}
		
			\item \textbf{Conditional section:}
			\begin{itemize}
				\item In this section you can take any valid Java expression, but it should be of the type boolean.
				\item Eg:
				\bigskip
				\codeblock{
					for(int i = 0; \textbf{true} ; i++)
				}
				\item This section is optional, if nothing is added here, the compiler will always place true.
			\end{itemize}
		
			\item \textbf{Increment/decrement section:}
			\begin{itemize}
				\item In this section, you can can take any valid Java statement.
				\item Eg:
				\bigskip
				\codeblock{
					int i = 0; \\
					for(\textbf{System.out.println("Hello")}; i < 3; System.out.println("Hi")) \{ \\
					\s i++; \\
					\}
				}
			\bigskip
			\outputblock{
				Hello
				Hi
				Hi
				Hi
			}	
			\end{itemize}
		
		\end{itemize}
		
		\bigskip
		\noteblock{
			All 3 parts of for loop are independent of each other and optional.	
		}

		\item Infinite loop examples:
		\bigskip
		\codeblock{
			for(;;) \{ \\ 
			\s System.out.println("Hello"); \\
			\}
		}
		\bigskip
		\codeblock{
			for(;;);
		}
		
		\item Unreachable statement in for loop, results in compile-time error. 
		\newline
		Eg:
		\bigskip
		\codeblock{
			for(int i = 0; \textbf{true} ; i++) \{ \\
			\s System.out.println("Hello"); \\
			\} \\
			System.out.println("Hello");  \xmark // Unreachable statement
		}
		
		
	\end{itemize}

		
	
\end{flushleft}

\newpage

