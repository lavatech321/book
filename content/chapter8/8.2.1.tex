\setlength{\columnsep}{3pt}
\begin{flushleft}
	
	
	\begin{itemize}
		\item When number of iterations is not known in advance, you should use while loop.
		\bigskip
		\syntaxblock{
			while(condition) \{ \\
			\s	Action \\
			\}	
		}
		
		\item The condition should be of boolean type.
		\bigskip
		\noteblock{
			In Java, "1" is not true or false.
			\codeblock{
				while(1) \{  \xmark  \\  
				\s	System.out.println("Hello"); \\
				\}
			}
		}
		\bigskip
		
		
		\item Curly braces are optional and without curly you can take only one statement under while, and this statement \textbf{should not be declarative}.
		
		\newpage
		
		\item Below are some valid and invalid examples of while:
		\bigskip
		\codeblock{
			while(true) \cmark \\ 
			\s	System.out.println("Hello"); 
		}
		\bigskip
		\codeblock{
			while(true); \cmark
		}
		\bigskip
		\codeblock{
			while(true) \\
			\s int x = 10; \xmark
		}
		\bigskip
		\codeblock{
			while(true) \{ \\
			\s int int x = 10; \cmark \\
			\}
		}
		
		\bigskip
		\item Unreachable statement in while loop also results in compile-time error. Below are some examples showing unreachable statement:
		\bigskip
		\codeblock{
			while(true) \{ \\
			\s	System.out.println("Hello");  \\
			\} \\
			System.out.println("Hi");   \xmark // Compile-time error 
		}
		
		\bigskip
		\codeblock{
			while(false) \{ \\
			\s	System.out.println("Hello");  \\
			\} \\
			System.out.println("Hi");   \xmark // Compile-time error 
		}
		
		\bigskip
		\codeblock{
			int a=10, b=20; \\
			while(a<b) \{  \cmark  \\     
			\s	System.out.println("Hello");  \\
			\} \\
			System.out.println("Hi");  
		}
		
	\end{itemize}
	
\end{flushleft}

