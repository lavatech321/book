\setlength{\columnsep}{3pt}
\begin{flushleft}
	
	\begin{itemize}
		\item \textbf{valueOf():}
		\begin{itemize}
			\item Create wrapper class object for the given Primitives and Strings.
			\item All wrapper classes contains the valueOf() method:
			\item Eg:
			\codeblock{
				Integer I = Integer.valueOf(10); \\
				Float F = Float.valueOf(10.5f); \\
				Double D = Double.valueOf(10.5); \\
				Boolean B = Boolean.valueOf(true); \\
				Character C = Character.valueOf('a'); 
			}
			\item Constructor V/S valueOf():
			\begin{itemize}
				\item Using constructor, \textbf{"=="} operator check for object reference equality.
				\item valueOf() creates only 1 object and references same object using == operator.
				\codeblock{
					Integer i1 = new Integer(10); \\
					Integer i2 = new Integer(10); \\
					System.out.println(i1 == i2); // false  \\
					\\
					Integer i3 = Integer.valueOf(10); \\
					Integer i4 = Integer.valueOf(10); \\
					System.out.println(i3 == i4); // true
				}
			\end{itemize}
			
			\item Forms of valueOf():
			\begin{itemize}
				\item Form 1: All wrapper classes except Character class contains static factory valueOf() method to create wrapper object for given string.
				\item Eg:
				\codeblock{
					Integer I = Integer.valueOf("10"); \\
					Float F = Float.valueOf("10.5f"); \\
					Double D = Double.valueOf("10.5"); \\
					Boolean B = Boolean.valueOf("true");
				}
			
				\bigskip
				\item Form 2: Only Integral Type wrapper classes(Byte, Short, Integer, Long)
				\bigskip
				\syntaxblock{
					public static wrapper valueOf(String s, int radix)
				}
				Allowed radix range is 2 to 36, where:
				\begin{itemize}
					\item 2 -> 0,1
					\item 8 -> 0 to 7
					\item 10 -> 0 to 9
					\item 11 -> 0 to 9, a
					\item 12 -> 0 to 9, a to b
					\item 16 -> 0 to 9, a to f
					\item 36 -> 0 to 9, a to z
				\end{itemize}
				\item Eg:
				\codeblock{
					Integer I = Integer.valueOf("1110",2);
				}
				\bigskip
				\quest{Predict the output for I in below code:
					\codeblock{
					Integer I = Integer.valueOf("1111",15);
					}
				}{
					 $(1111)_\textbf{10}$ = $15^{3}$ + $15^{2}$ + $15^{1}$ + $15^{0}$  \\
					 $(1111)_\textbf{10}$ = 3616 \\
					 Hence, I = 3616
				}
				
			\end{itemize}
		\end{itemize}


		\newpage
		\item \textbf{xxxValue():}
		\begin{itemize}
			\item Used to find primitive value for a given wrapper object.
			\item Below are available methods:
			\begin{itemize}
				\item byteValue()
				\item shortValue()
				\item intValue()
				\item longValue()
				\item floatValue()
				\item doubleValue()
			\end{itemize}
			\item Eg:
			\codeblockfull{demo.java}{
				Integer I = new Integer(130); \\
				System.out.println(I.byteValue()); // -126 \\
				System.out.println(I.shortValue()); // 130 \\
				System.out.println(I.intValue()); // 130 \\
				System.out.println(I.longValue()); // 130 \\
				System.out.println(I.floatValue()); // 130.0 \\
				System.out.println(I.doubleValue()); // 130.0 \\ \\
				Character ch = new Character('a'); \\
				char c = ch.charValue(); // c = 'a' 
			}
			\bigskip
			\quest{Why is the byte output showing -126 for 130 integer value?}{
					130(Integer size 32 bits) == 00000...010000010 \\
					Byte 8-size -- 10000010 \\
					As first digit is 1, the number is represented in 2's complement form. \\
					2's complement = 1's complement + 1  \\
					i.e 01111101 + 1 = 11111110 \\
					Hence $(11111110)_\textbf{10}$ = -126 
			}
		\end{itemize}



		\newpage
		
		\item \textbf{parseXxxx()}:
		\begin{itemize}
			\item Used to find primitive value for given String object.
			\item Available parseXxxxx() methods:
			\begin{itemize}
				\item \textbf{Integer.parseInt(String s)} - Parses the string argument as a signed decimal integer.
				\item \textbf{Long.parseLong(String s)} - Parses the string argument as a signed decimal long.
				\item \textbf{Float.parseFloat(String s)} - Parses the string argument as a floating-point number.
				\item \textbf{Double.parseDouble(String s)} - Parses the string argument as a double-precision floating-point number.
				\item \textbf{Boolean.parseBoolean(String s)} - Parses the string argument as a boolean.
			\end{itemize}
			\item Eg:
			\codeblock{
				int i = Integer.parseInt("-45"); \\
				long l = Long.parseLong("983452"); \\
				float f = Float.parseFloat("45.4"); \\
				double d = Double.parseDouble("23423.65"); \\
				boolean b = Boolean.parseBoolean("true"); \\
				System.out.println(i);  // -45 \\
				System.out.println(l);  // 983452  \\
				System.out.println(f);  // 45.4  \\
				System.out.println(d);  // 23423.65 \\
				System.out.println(b); // true
			}
			
		\end{itemize}
		\newpage
		\item \textbf{toString()} - Converts primitive datatype to String object.
		\newline
		Eg: Below code converts primitive datatype to String Object.
		\codeblock{
			int i = 13; \\
			System.out.println(Integer.toString(i)); // 13 \\
			 \\
			float f = 1.0f; \\
			System.out.println(Float.toString(f)); // 1.0 \\
			\\
			long l = 5634; \\
			System.out.println(Long.toString(l)); // 5634 \\
			\\
			double d = 783.3; \\
			System.out.println(Double.toString(d)); // 783.3
		}
		
	\end{itemize}
\end{flushleft}

\newpage