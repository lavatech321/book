\setlength{\columnsep}{3pt}
\begin{flushleft}
	
	\begin{itemize}
		\item Wrapper class constructor contain argument as shown below:
		\bigskip
		\tabletwo{
			\textbf{Wrapper class} & \textbf{Constructor arguments}	 \\
			\hline
			Byte & byte, String \\
			\hline
			Short & short, String \\
			\hline
			Integer & int, String \\
			\hline
			Long & long, String \\
			\hline
			Float & float,double, String \\
			\hline
			Double & double, String \\
			\hline
			Boolean & boolean, String \\
			\hline
			Character & char \\
		}
		
		\bigskip
		\noteblock{
			Since Java 1.9 version, above constructors are deprecated. Instead use valueOf() for better performance.
		}
	
		\item Things to note for Boolean constructor:
		\begin{itemize}
			\item If you are passing boolean primitive as argument the only allowed values are: \textbf{true} or \textbf{false}
			\item If you are passing String argument then case is not important and content is also important.
		\end{itemize}
	
		\codeblock{
			Boolean b1 = new Boolean(true);  // true \\
			Boolean b2 = new Boolean(false); // false \\
			//Boolean b3 = new Boolean(True); // invalid \\
			Boolean b4 = new Boolean("lavatech"); // false \\
			Boolean b5 = new Boolean("yes"); // false \\
			Boolean b6 = new Boolean("no"); // false \\
			Boolean b7 = new Boolean("true"); // true \\
			Boolean b8 = new Boolean("True"); // \\ true			
		}
		\newpage
		\item Eg: Converting byte to Byte object:
		\bigskip
		\codeblock{
			byte b1 = 23; \\
			Byte b2 = new Byte(b1); \\
			Byte b3 = new Byte("67"); \\
			System.out.println(b2);  // 23 \\
			System.out.println(b3); // 67
		}
		\bigskip
		\item Eg: Converting int to Integer object:
		\bigskip
		\codeblock{
			int i1 = 451; \\
			Integer I1 = new Integer(i1); \\
			Integer I2 = new Integer("451");  \\ 
			System.out.println(I1); // 451 \\
			System.out.println(I2);  // 451
		}
	\end{itemize}
\end{flushleft}

\newpage