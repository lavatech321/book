\begin{flushleft}
	
	\begin{itemize}
		\item A Java program can contain any number of classes in a single program. 
		\item However, only 1 class can be declared as \textbf{public}.
		\item If there is a public class, then \textbf{name of the program and name of public class must be matched}, otherwise the program will result in compile time error.
		\item Below are some use-cases on this:
		
		\begin{itemize}
			\item \textbf{Case I}: If there is no public class and the program contains multiple class, then the program name can be anything. 
			\bigskip
			\codeblockfull{Lava.java}{
					class A \{\} \\
					class B \{\} \\
					class C \{\} 
			}			
			\bigskip
			\commandblock{
			\$ javac Lava.java \\
			 // This will create A.class, B.class, C.class
			}
			
			\bigskip
			\item \textbf{Case II}: If \textbf{class B is public}, then name of the program should be \textbf{B.java}, otherwise we will get compile-time error saying: \textbf{class B is public should be declared in file named B.java}:
			\bigskip
			\codeblockfull{Lava.java}{
				class A \{\} \\
				public class B \{\} \xmark // Lava.java is incorrect name \\
				class C \{\}
			}
			\bigskip
			\codeblockfull{B.java}{
				class A \{\} \\
				public class B \{\} \cmark \\
				class C \{\}
			}
			
		
			\bigskip
			
			\item \textbf{Case III}: If class B and C are declared as public and name of program is “B.java”, then we will get compile time error saying: class C is public, should be declared in a file named C.java
			
			\bigskip
			\codeblockfull{B.java}{
				class A \{\} \\
				public class B \{\} \\
				public class C \{\} \xmark \s // Two class cannot be public
			}

			\item \textbf{Class IV:} If a program contains main() for multiple class, then executing those specific class will execute their respective main()
			\bigskip
			
			\codeblockfull{Lava.java}{
				class A \{ \\ 
				\s public static void main(String[] args) \{ \\
				\s \s  System.out.println("A class main"); \\
				\s	\} 			\\
				class B \{ \\ 
				\s public static void main(String[] args) \{ \\
				\s \s  System.out.println("B class main"); \\
				\s	\} \\
				class C \{ \\ 
				\s public static void main(String[] args) \{ 
			}	
			\newpage
			
			\codecontinue{
				\s \s  System.out.println("C class main"); \\
				\s	\} \\
				class D \{\} \\
				\}
			}
		
			\bigskip
			\commandblock{
			\$ javac Lava.java \\
			A.class B.class C.class D.class \\
			\\
			\$ java A \\
			A class main \\
			\\
			\$ java B \\
			B class main \\
			\\
			\$ java C \\
			C class main \\
			\\
			\$ java D \\ 
			RuntimeError: NoSuchMethodError: main() \\
			\\
			\$ java Lava \\
			RuntimeError: NoClassDefFoundError: Lava \\
			}
		\end{itemize}
	\end{itemize}

	\textbf{Conclusion:}
	\begin{itemize}
		\item While executing a java program, for every class present in that program, a separate \textbf{“.class”} will be generated.
		\item You can compile a java program (Java source file), but you can run a java \textbf{".class"} file.
		\item On executing a java class, the corresponding class main() will be executed.
		\item If the class doesn’t contain main(), then you will get runtime exception.
		\item If the corresponding .class file is not available, then you will get runtime exception.
		\item It is not recommended to declare multiple classes in a single source file. 
		\item It is recommended to declare only one class per source file and name of the program to be same as class name.
	\end{itemize}
	
\end{flushleft}
\newpage



