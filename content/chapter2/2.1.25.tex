\begin{flushleft}
	
	\begin{itemize}
		\item A Java program contains any number of classes in a single program. 
		\item Only 1 class can be declared as \textbf{public}.
		\item \textbf{Name of the program and name of public class must be matched}, otherwise it will result in compile time error.
		\item Below are some use-cases on this:
		
		\begin{itemize}
			\item \textbf{Case I}: Program name can be anything if there is no public class.
			\bigskip
			\codeblockfull{Lava.java}{
					class A \{\} \\
					class B \{\}
			}			
			\bigskip
			\commandblock{
			\$ javac Lava.java // This will create A.class, B.class
			}
			
			\bigskip
			\item \textbf{Case II}: Program name should be same as public class name.
			\bigskip
			\codeblockfull{Lava.java}{
				class A \{\} \\
				public class B \{\} \xmark // Lava.java is incorrect name 
			}
			\bigskip
			\codeblockfull{B.java}{
				class A \{\} \\
				public class B \{\} \cmark 
			}
			
			\bigskip
			\item \textbf{Case III}: There can be only 1 public class in single program.
			\bigskip
			\codeblockfull{B.java}{
				public class B \{\} \\
				public class C \{\} \xmark \s // Two class cannot be public
			}

			\item \textbf{Class IV:} Multiple class with main() means executing those specific class will execute their respective main()
			\bigskip
			\codeblockfull{Lava.java}{
				class A \{ \\ 
				\s public static void main(String[] args) \{ \\
				\s \s  System.out.println("A class main"); \\
				\s	\} 			\\
				class B \{ \\ 
				\s public static void main(String[] args) \{ \\
				\s \s  System.out.println("B class main"); \\
				\s	\} \\
				\}
			}
		
			\bigskip
			\commandblock{
			\$ javac Lava.java \\
			A.class B.class \\
			\\
			\$ java A \\
			A class main \\
			\\
			\$ java B \\
			B class main \\
			\\
			\$ java Lava \\
			RuntimeError: NoClassDefFoundError: Lava 
			}
		\end{itemize}
	\end{itemize}

	\textbf{Conclusion:}
	\begin{itemize}
		\item For every class in program, a separate \textbf{“.class”} is generated.
		\item ".java" file is compiled to generate ".class" file. ".class" file is executed.
		\item Executing a ".class" file executes it's main().
		\item Executing a ".class" file without main() results in runtime exception.
		\item Multiple classes in a single ".java" file is not recommended. 
	\end{itemize}
	
\end{flushleft}
\newpage



