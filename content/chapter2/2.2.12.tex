
\begin{flushleft}

	\begin{itemize}
		
		\item By default, floating-point numbers are represented in double form.
		\item So below declaration will result in error: 
		\bigskip		
		\codeblock{
				float f = 1.0 \xmark
		}
		\bigskip		
		
		\item Correct way to represent float data type is by suffixing \textbf{"F"} or \textbf{"f"} to the floating-point number as shown:
		\bigskip
		\codeblock{
			float f = 1.6F; \cmark \newline
			float f = 7.8f;		\cmark
		}

		\item Double data type can be represented using suffix \textbf{"D"} or \textbf{"d"} or no suffix as below:
		\bigskip
		\codeblock{
		    double a = 12.67; \cmark \newline
			double b = 13.7d; \cmark \newline
			double c = 123.456D; \cmark \newline
			double d = 0123.456; \cmark \newline
			double e = 0789.9; \cmark
		}
	
		\item Floating-point literal are only in decimal form, not in octal and hexa decimal forms. Below are \textbf{invalid} declarations:
		\bigskip
		\codeblock{
			float a = 045.8; \xmark \newline
			float b = 0X56.9; \xmark \newline
			double c = 0X56.9; \xmark
		}
		\newpage
		\item We can assign integral literal directly to floating-point variables like double and float. Below are valid declarations:
		\bigskip
		\codeblock{
			double a = 0456; \cmark \newline
			double b = 0XFace; \cmark \newline
			double c = 10; \cmark \newline
			float a = 0456; \cmark \newline
			float b = 0XFace; \cmark \newline
			float c = 10; \cmark
		}
			
		\item \textbf{Expontential format:} This is scientific notation to represent very large or small floating-point values. Use the letter “e” or “E” to indicate the exponent:
				\bigskip
		\codeblock{
			double a = 1.2e3; \cmark \newline
			float b = 1.3e4F; \cmark
		}
		
		\item \textbf{Hexadecimal floating-point literals:} You can represent double and float in hexadecimal form using the letter “p” or “P”:
		\bigskip		
		\codeblock{
			double d = 0x12.2P2; \cmark \\
			float e = 0x12.2P2f; \cmark
		}
	
		\bigskip
		\item \textbf{Usage of \_ in floating literal}:
		\begin{itemize}
			\item From Java 1.7 version, we can use "\_" in middle of big numbers to increase floating-point's readability.
			\item At the time of compilation, these "\_" symbols will be removed automatically.
			\item Eg:
					\bigskip
			\codeblock{
				float x = 78\_3.2\_34\_23f; \cmark \newline
				double y = 12\_45\_23\_\_23\_2323.90;  \cmark
			}
			
			\item "\_" symbol cannot be used in the starting or end of integer or decimal point. Below are \textbf{invalid} declarations:
			\item Eg:
					\bigskip
			\codeblock{
				float x = 78\_3.2\_34\_23f\_; \cmark \newline
				double y = \_12\_45\_23\_\_23\_2323.90; \cmark  \newline
				double z = 12\_45\_2\_.3\_\_23\_2323.90;  \cmark
			}
						
		\end{itemize}	
		
	\end{itemize}
	
\end{flushleft}

\newpage

