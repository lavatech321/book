
\begin{flushleft}

	\begin{tabulary}{1.0\textwidth}{|p{5em}|p{3em}|p{7em}|p{5em}|p{5em}|}
		\toprule
		\textbf{Data Type} & \textbf{Size} & \textbf{Range} & \textbf{Wrapper class} & \textbf{Default Value}   \\
		\midrule
		byte & 1 byte & -$2^\textbf{7}$ to $2^\textbf{7}$ -1 & Byte & 0  \\
		\hline
		short & 2 byte & -$2^\textbf{15}$ to $2^\textbf{15}$ -1 & Short & 0  \\
		\hline
		int & 4 byte & -$2^\textbf{31}$ to $2^\textbf{31}$ -1 & Integer & 0 \\
		\hline
		long & 8 byte & -$2^\textbf{63}$ to $2^\textbf{63}$ -1 & Long & 0 \\
		\hline
		float & 4 byte & -$3.4^\textbf{38}$ to $3.4^\textbf{38}$ & Float & 0.0 \\
		\hline
		double & 8 byte & -$1.7^\textbf{308}$ to $1.7^\textbf{308}$ & Double & 0.0  \\
		\hline
		boolean & 1 byte & true, false & Boolean & false  \\
		\hline
		char & 2 byte  & 0 to 65535 & Character & 0 [represents space character] \\
		\bottomrule
	\end{tabulary}
	
	\bigskip
	\begin{tcolorbox}[breakable,notitle,boxrule=-1pt,colback=yellow,colframe=yellow]
		\color{black}
		Note: \textbf{null} value
		\begin{itemize}
			\item You cannot provide "\textbf{null}" value to primitive data types.
			\item \textbf{null} is default value for object reference like String, Array etc.
			\item Trying to use \textbf{null} for primitive data types will result in compile time error.
			\newline
			Eg: char ch = null; <-- will result in error
		\end{itemize}
	\end{tcolorbox}
	
	\newpage
	\textbf{Program}: Below code gives the size of each data type in Java:
	
	\codeblock{
		package Starter; \newline
		class Test \newline
		\{ \newline
		\hphantom{} \hphantom{} public static void main (String[] args) \newline
		\hphantom{} \hphantom{}	\{ \newline
		\hphantom{} \hphantom{} \hphantom{} \hphantom{} 	System.out.println("Size of byte: " + (Byte.SIZE/8) + " bytes."); \newline
		\hphantom{} \hphantom{} \hphantom{} \hphantom{}		System.out.println("Size of short: " + (Short.SIZE/8) + " bytes."); \newline
		\hphantom{} \hphantom{} \hphantom{} \hphantom{}		System.out.println("Size of int: " + (Integer.SIZE/8) + " bytes."); \newline
		\hphantom{} \hphantom{} \hphantom{} \hphantom{}		System.out.println("Size of long: " + (Long.SIZE/8) + " bytes."); \newline
		\hphantom{} \hphantom{} \hphantom{} \hphantom{}		System.out.println("Size of char: " + (Character.SIZE/8) + " bytes."); \newline
		\hphantom{} \hphantom{} \hphantom{} \hphantom{}		System.out.println("Size of float: " + (Float.SIZE/8) + " bytes."); \newline
		\hphantom{} \hphantom{} \hphantom{} \hphantom{}		System.out.println("Size of double: " + (Double.SIZE/8) + " bytes."); \newline
		\hphantom{} \hphantom{}	\} \newline
		\}
	}
	\newpage
	\outputblock{
			Size of byte: 1 bytes. \newline
		Size of short: 2 bytes. \newline
		Size of int: 4 bytes. \newline
		Size of long: 8 bytes. \newline
		Size of char: 2 bytes. \newline
		Size of float: 4 bytes. \newline
		Size of double: 8 bytes.
	}	
	
	\textbf{Program}: Below code gives the class of each data type in Java:
	\codeblock{
			package Starter; \newline
		class Test \newline
		\{ \newline
		\hphantom{} \hphantom{} public static void main (String[] args) \newline
		\hphantom{} \hphantom{}	\{ \newline
		\hphantom{} \hphantom{} \hphantom{} \hphantom{} byte b = 56; \newline
		\hphantom{} \hphantom{} \hphantom{} \hphantom{} short s = 6789; \newline
		\hphantom{} \hphantom{} \hphantom{} \hphantom{} int x = 10; \newline
		\hphantom{} \hphantom{} \hphantom{} \hphantom{} long l = 10L; \newline
		\hphantom{} \hphantom{} \hphantom{} \hphantom{} long n = 10; \newline
		\hphantom{} \hphantom{} \hphantom{} \hphantom{} float f = 1.0F; \newline
		\hphantom{} \hphantom{} \hphantom{} \hphantom{} double d = 123.456; \newline
		\hphantom{} \hphantom{} \hphantom{} \hphantom{} System.out.println(((Object)b).getClass().getSimpleName()); \newline
		\hphantom{} \hphantom{} \hphantom{} \hphantom{} System.out.println(((Object)s).getClass().getSimpleName()); \newline
		\hphantom{} \hphantom{} \hphantom{} \hphantom{} System.out.println(((Object)x).getClass().getSimpleName()); \newline
		\hphantom{} \hphantom{} \hphantom{} \hphantom{} System.out.println(((Object)l).getClass().getSimpleName()); \newline
		\hphantom{} \hphantom{} \hphantom{} \hphantom{} System.out.println(((Object)n).getClass().getSimpleName()); \newline
		\hphantom{} \hphantom{} \hphantom{} \hphantom{} System.out.println(((Object)f).getClass().getSimpleName()); \newline
		\hphantom{} \hphantom{} \hphantom{} \hphantom{} System.out.println(((Object)d).getClass().getSimpleName()); \newline
		\hphantom{} \hphantom{}	\} \newline
		\}
	}

	\outputblock{
		Byte \newline
		Short \newline
		Integer \newline
		Long \newline
		Long \newline
		Float \newline
		Double
	}

	
	
\end{flushleft}

\newpage




% ===============================
\setlength{\columnsep}{3pt}
\begin{flushleft}
	
	\textbf{length variable}:
	\begin{itemize}
		\item length is \textbf{final} (means constant) variable is used to display \textbf{size of an array}.
		\item Value returned by length is fixed as array once created cannot change it's size.
		\bigskip
		\codeblock{
			int[] x = new int[6]; \\
			System.out.println(x.length);  
		}
		\bigskip
		\outputblock{
			6
		}
		
		\item length variable is \textbf{not} applicable on string objects.
		\bigskip
		\codeblock{
			String s="lavatech"; \\
			System.out.println(s.length); \xmark
		}
		
		\item In multi-dimensional arrays, length variable represents only base size, but not total size.
		\bigskip
		\codeblock{
			int[][] x = new int[6][3]; \\
			System.out.println(x.length);
		}
		\bigskip
		\outputblock{
			6
		}
		
	\end{itemize}
	
	\textbf{length()}:
	\begin{itemize}
		\item length() is present in \textbf{String class}.
		\item length() method is final variable applicable for string objects.
		\item It returns number of characters present in the string.
		\bigskip
		\codeblock{
			String s="lavatech"; \\
			System.out.println(s.length()); \cmark
		}
		\bigskip
		\outputblock{
			8
		}
		\bigskip
		
		\item length variable is applicable for arrays, but not for string objects.
		\item length() is applicable for string objects, but not for arrays.
		
		
		Example:
		\codeblock{
			String[] s= \{"A","AA","AAA"\}; \\
			System.out.println(s.length); \cmark \\
			System.out.println(s.length()); \xmark  \\
			System.out.println(s[0].length); \xmark \\
			System.out.println(s[0].length()); \cmark
		}
		\bigskip	
		\outputblock{
			3 \\
			error \\
			error \\
			1
		}
		
		\bigskip
		
		\item There is no direct way to find total length of multi-dimensional array. Total length of multi-dimensional array can be found as follows:
		
		\codeblock{
			int[][] x = new int[3][3]; \\
			System.out.println(x.length); \\
			System.out.println(x[0].length+x[1].length+x[2].length);
		}
		\outputblock{
			3 \\
			9
		}
				
		
	\end{itemize}
	
\end{flushleft}
\newpage
