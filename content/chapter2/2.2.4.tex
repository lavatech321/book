

\begin{flushleft}
	
	\begin{itemize}
		\item long keyword is 64-bit signed integer. 
		
		\tabletwo{
			\hline
			Size & \textbf{8 byte (64 bits)} \\
			\hline
			MAX\_VALUE & $2^\textbf{63}$-1 \\
			\hline
			MIN\_VALUE & -$2^\textbf{63}$ \\
			\hline
			Range & \textbf{-$2^\textbf{63}$} to \textbf{$2^\textbf{63}$-1}   \\
			\hline
		}
		
		\bigskip
		\item \textbf{Where is long data type used?}
		\begin{itemize}
			\item Eg 1: Amount of distance travelled by light in 1,000 days. To hold this value integer is not enough. Hence, long is used.
			\newline
			long l = 1,26,000 X 60 X 60 X 24 X 1000 miles.
			\bigskip
			\item Eg 2: The number of characters present in a big file may exceed int range. Hence, the return type of length() is long but not integer. 
			\codeblock{
				long l = f.length()
			}
			
		\end{itemize}	
	
		\bigskip
		\textbf{Long literals}
		\begin{itemize}
			\item \textbf{long} data type can be suffixed with "l" or "L".
			
			\item Below are valid \textbf{long} data type:
			\codeblock{
				long l = 10L; \cmark \\
				long b = 10; \cmark
			}
			
			\item However, below declaration will result in error:
			\codeblock{
					int x = 10L; \xmark
			}

		\end{itemize}
		
		
	\end{itemize}
	
\end{flushleft}

\newpage

