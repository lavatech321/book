

\begin{flushleft}
	
	\begin{itemize}
			\item Command line arguments are values passed to Java program when it is run from the command line. 
			\item With these command line arguments, JVM creates an array and pass it to main().
			\item Command line arguments can be accessed using the args parameter of the main(). 
			\item Args parameter is an array of String objects.
			\item You can customise behaviour of main() using command-line argument:
			
			\codecontinue{
			public static void main(String[] args) //  ← Here, \textbf{String[] args} contains command line args
			}
		
			\newimage{0.5}{content/chapter2/images/ans.png}
		
			\item Command line argument are always String[]
			
			\codeblockfull{New.java}{
				class New \{ \\
				\s	public static void main(String... args) \{  \\
				\s \s		for(int i = 0; i < args.length; i++) \{ \\
				\s \s			System.out.println(args[i]); \\
				\s \s		\} \\
				\s	\} \\
				\}
			}
		
			\commandblock{
			\$ javac New.java  \\
			\$ java New 1 23 3 \\
			1  \\
			23  \\
			3 \\
			}
			\bigskip
			\item Command line arguments are separated by space. To give one argument with space character, using "" :
			\commandblock{
				\$ java New "Note Book"
			}
			
							
	\end{itemize}
	
	
\end{flushleft}
\newpage



