\setlength{\columnsep}{3pt}
\begin{flushleft}

	\begin{itemize}
		\item ArrayList are \textbf{ordered \& indexed} collection of \textbf{hetrogenous objects} wherein \textbf{duplicate objects \& null} is allowed.
		\newimage{0.5}{content/chapter14/images/list3.png}
		\item ArrayList class implements List interface.
	\end{itemize}
	
	\textbf{Ways of initializing ArrayList}
	
	\begin{itemize}
		\item Creating am empty ArrayList:
		\bigskip
		\syntaxblock{
			// To create ArrayList of hetrogenous elements: \\
			ArrayList list = new ArrayList(); \\
			\\
			// To create ArrayList of homogenous elements: \\
			ArrayList<ClassName> list = new ArrayList<ClassName>(); 
		}
		Eg:
		\codeblock{
			ArrayList list = new ArrayList(); \\
			ArrayList<String> list = new ArrayList()<String>; 
		}
	
		\bigskip
		\item \textbf{ArrayList capacity}:
		\begin{itemize}
			\item Above syntax will create ArrayList with \textbf{default capacity of 10}.
			\item Once 10 objects are added, then a new array list object will be created with new capacity as shown below:
			\bigskip
			\codecontinue{
				\textbf{New capacity = (current capacity * 3/2) + 1}
			}
			\item Hence ArrayList capacity would be 10, 16, 25, 38,....
			\bigskip
			\noteblock{
				There is no method to check default initial capacity.
			}
			
		\end{itemize}
		
		\item Creating am empty ArrayList with initial capacity:
		\bigskip
		
		\syntaxblock{
			// To create ArrayList of hetrogenous elements: \\
			ArrayList numbers = new ArrayList(capacity); \\
			\\
			// To create ArrayList of homogenous elements: \\
			ArrayList<ClassName> list = new ArrayList<ClassName>(capacity); 
		}
		Eg:
		\codeblock{
			ArrayList numbers = new ArrayList(20); \\
			ArrayList<Integer> list = new ArrayList<Integer>(20); 
		}
			
		\newpage
		\item Initialization with data:
		\begin{itemize}
			\item Using \textbf{Arrays.asList()} method:
			\bigskip
			\codeblock{
			import java.util.*; \\ 
			String[] namesArray = \{"Jim", "Jane", "Alice"\}; \\
			ArrayList<String> namesList = new ArrayList<>(Arrays.asList(namesArray)); 
			}
			
			\item Using \textbf{List.of} method:
			\bigskip
			\codeblock{
			ArrayList<String> colors = new ArrayList<>(List.of("Red", "Green", "Blue"));
			}
			
			\item Using \textbf{"var"} and \textbf{"List.of"}: 
			\bigskip
			\codeblock{
				var fruits = new ArrayList<>(List.of("Apple", "Banana", "Orange"));
			}
			
		\end{itemize}
	
	\end{itemize}

	\textbf{Where ArrayList is best choice:}
	\begin{itemize}
		\item For retrieval operations, because ArrayList implements RandomAccess interface.
	\end{itemize}
	
	\textbf{Where ArrayList is worst choice:}
	\begin{itemize}
		\item For insertion or deletion operation in the middle.
		\item For this, LinkedList is best choice.
	\end{itemize}
	
		
	\newpage
	
	\textbf{Example:}
	
	\begin{itemize}
		\item Java program to create an empty ArrayList and perform add, remove operation and display it's size:
		
		\codeblockfull{Test.java}{
			import java.util.ArrayList; \\
			class Test \{  \\
			\s	public static void main(String[] args) \{ \\
			\s \s		\textbf{ArrayList l1 = new ArrayList();} \\
			\s \s		\textbf{l1.add("Jack");} \\
			\s \s		l1.add(10); \\
			\s \s		l1.add('A'); \\
			\s \s		\textbf{l1.add(null);} \\
			\s \s		l1.add("Jim"); \\
			\s \s		System.out.println(l1); \\
			\s \s		l1.remove("Jack"); \\
			\s \s		System.out.println(l1); \\
			\s \s		l1.add(2,"Jimmy"); \\
			\s \s		System.out.println(l1); \\
			\s \s		System.out.println("ArrayList size: "+\textbf{l1.size()}); \\	
			\s \s 		boolean ans = \textbf{l1.contains("Jack")}; \\
			\s \s		System.out.println("Jack is present in ArrayList: "+ans);	 \\
			\s	\} 	\}
		}
		\bigskip
		\outputblock{
		\textbf{[} Jack, 10, A, null, Jim \textbf{]} \\
		\textbf{[} 10, A, null, Jim \textbf{]}  \\
		\textbf{[} 10, A, Jimmy, null, Jim \textbf{]} \\
		ArrayList size: 5 \\
		Jack is present in ArrayList: false
		}
	\end{itemize}


	
	
	
\end{flushleft}

