\setlength{\columnsep}{3pt}
\begin{flushleft}
	
	\begin{itemize}
		\item ListIterator can move either to \textbf{forward direction} or to the \textbf{backward direction}. 
		\item Hence it is \textbf{bi-directional} cursor.
		\item Using ListIterator, you can perform replacement and addition of new objects in addition to read and remove operations.
	\end{itemize}
	
	\textbf{ListIterator specific methods:}
	ListIterator is the child interface of Iterator, hence all methods of Iterator interface are applicable on ListIterator.
	
	\begin{itemize}
		\item Creating ListIterator object:
		\syntaxblock{
			ListIterator l2 = collection.listIterator();
		}
		\bigskip
		\item \textbf{Forward movement:}
		\begin{itemize}
			\item Check for next element in collection: 
			\syntaxblock{
				public boolean hasNext()
			}
			\bigskip
			\item Get next element in collection: 
			\syntaxblock{
				public Object next()
			}
			\bigskip
			\item Return index of next element:
			\syntaxblock{
				public int nextIndex()
			}
		\end{itemize}
		
		\item \textbf{Backward movement:}
		\begin{itemize}
			\item Check for previous element in collection: 
			\syntaxblock{
				public boolean hasPrevious()
			}
			\bigskip
			\item Get previous element in collection: 
			\syntaxblock{
				public Object previous()
			}
			\bigskip
			\item Return index of previous element:
			\syntaxblock{
				public int previousIndex()
			}
		\end{itemize}

		\item \textbf{Extra operations:}
		\begin{itemize}
			\item Remove element in collection:
			\syntaxblock{
				public void remove()
			}
			\bigskip
			\item Add element in collection:
			\syntaxblock{
				public void add(Object o)	
			}
			\bigskip
			\item  Replaces the last element returned by \textbf{next()} or \textbf{previous()} with the specified element:
			\syntaxblock{
				void set(Object o)
			}	
		\end{itemize}	
	\end{itemize}
	\newpage
	
	\textbf{Example:}
	Java program to display ListIterator operation by retaining only those name starting with vowel character and removing rest names:
	
	\codeblockfull{Test.java}{
		import java.util.*; \\
		class Test\{ \\
		\s public static void main(String[] args) \{ \\
		\s 	ArrayList vowels = new  ArrayList(List.of("A","E","I","O","U")); \\
		\s \s		LinkedList l1 = new LinkedList(); \\
		\s \s		l1.add("Apple"); \\
		\s \s		l1.add("Banana"); \\
		\s \s		l1.add("Grapes"); \\
		\s \s		l1.add("Chikoo"); \\
		\s \s		l1.add("Eggs"); \\
		\s \s		ListIterator l2 = l1.listIterator(); \\
		\s \s		while(l2.hasNext()) \{ \\
		\s \s \s			String s = (String)l2.next(); \\
		\s \s \s			String value = Character.toString(s.charAt(0)); \\
		\s \s \s			if (! vowels.contains(value)) \{ \\
		\s \s \s \s				l2.remove(); \\
		\s \s \s			\} \\
		\s \s		\} \\
		\s \s		System.out.println(l1); \\
		\s	\} \\
		\}	
	}
	\bigskip
	\outputblock{
		\textbf{[}Apple, Eggs\textbf{]}		
	}
	
\end{flushleft}

\newpage

