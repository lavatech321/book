\setlength{\columnsep}{3pt}
\begin{flushleft}

	\begin{itemize}
		\item TreeSet are \textbf{unordered} collection of \textbf{heterogenous objects} wherein \textbf{duplicate objects \& null} \textbf{is not} allowed. 
		\item TreeSet class implements NavigableSet interface.
		\item \textbf{Objects are inserted based on some sorting order}. Order maybe  natural or customised sorting order.
		\item Underlying data structure is balanced tree.
		\newimage{0.2}{content/chapter14/images/tree.png}
			
	\end{itemize}
	
	\textbf{Ways of initializing TreeSet}
	
	\begin{itemize}
		\item Creating am empty TreeSet:
		\bigskip
		\syntaxblock{
			TreeSet t1 = new TreeSet(); \\
		}
		
		\bigskip
				
		\item Creating an empty TreeSet with custom Comparator:
		\bigskip
		
		\syntaxblock{
			TreeSet t  =  new TreeSet(Comparator c);
		}		
	\end{itemize}
	
	\newpage
	\textbf{Example Program}:
	\newline
	Java program to display TreeSet operations:
	
	\codeblockfull{Test.java}{
		import java.util.*; \\
		class Test \{ \\
		\s	public static void main(String[] args) \{ \\
		\s \s		Integer[] nos = \{ 12,3,10,15,23,30,1,3,40 \}; \\
		\s \s		\textbf{TreeSet t1 = new TreeSet(Arrays.asList(nos))}; \\
		\s \s		System.out.println(\textbf{t1}); \\
		\s \s		System.out.println(\textbf{t1.first()}); \\
		\s \s		System.out.println(\textbf{t1.last()}); \\
		\s \s		System.out.println(\textbf{t1.headSet(34)}); \\
		\s \s		System.out.println(\textbf{t1.tailSet(34)});  \\
		\s \s		System.out.println(\textbf{t1.subSet(5,34)}); \\
		\s \s 		System.out.println(\textbf{t1.descendingSet()}); \\
		\s \s 		System.out.println(\textbf{t1.lower(23)}); \\
		\s \s		System.out.println(\textbf{t1.higher(23)}); \\
		\s	\} \\
		\} 
	}
	\bigskip
	\outputblock{
	\textbf{[}1, 3, 10, 12, 15, 23, 30, 40\textbf{]} \\
	1 \\
	40 \\
	\textbf{[}1, 3, 10, 12, 15, 23, 30\textbf{]} \\
	\textbf{[}40\textbf{]} \\
	\textbf{[}10, 12, 15, 23, 30\textbf{]} \\
	\textbf{[}40, 30, 23, 15, 12, 10, 3, 1\textbf{]} \\
	15 \\
	30
	}
	
\end{flushleft}
\newpage


