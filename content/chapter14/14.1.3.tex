\setlength{\columnsep}{3pt}
\begin{flushleft}
	
	\begin{itemize}
	\item Vector are \textbf{ordered \& indexed} collection of \textbf{hetrogenous objects} wherein \textbf{duplicate objects \& null} is allowed. 

	\newimage{0.5}{content/chapter14/images/list3.png}	
	\item Like ArrayList, Vector is best choice for retrieval operation.
	\end{itemize}

	\textbf{Difference between ArrayList and Vector}
	\bigskip
	\tabletwo{
		\textbf{ArrayList} & \textbf{Vector} \\
		\hline
		Methods are \textbf{non-synchronize} & Methods are \textbf{synchronized} \\
		\hline
		At a time, multiple threads are allowed to operate on ArrayList Object and hence ArrayList is \textbf{not thread safe} & At a time, only one thread is allowed to operate on Vector Object and it is \textbf{thread safe} \\
		\hline
		Threads are not required to wait to operate on ArrayList, hence relatively \textbf{performance is \textbf{high}} & Threads are required to wait to operate on Vector Object and hence relatively \textbf{performance is low}. \\
		\hline
		Introduced in 1.2 version and it is non-legacy class & Introduced in 1.0 version and it is legacy class \\
	}

	\newpage
	\textbf{Ways of initializing Vector}
	
	\begin{itemize}
		\item Creating an empty Vector:
		\bigskip
		\syntaxblock{
			// To create Vector of hetrogenous elements: \\
			Vector list = new Vector(); \\
			\\
			// To create Vector of homogenous elements: \\
			Vector<ClassName> list = new Vector<ClassName>(); 
		}
		Eg:
		\codeblock{
			Vector list = new Vector(); \\
			Vector<String> list = new Vector<String>(); 
		}
		
		\bigskip
		\item \textbf{Vector capacity}:
		\begin{itemize}
			\item Above syntax will create Vector with \textbf{default capacity of 10}.
			\item Once 10 objects are added, then a new Vector object will be created with new capacity as shown below:
			\bigskip
			\codecontinue{
				\textbf{New capacity = (current capacity * 2)}
			}
			
		\end{itemize}
		
		\item Creating an empty Vector with initial capacity:
		\bigskip
		
		\syntaxblock{
			// To create Vector of hetrogenous elements: \\
			Vector v1 = new Vector(capacity); \\
			\\
			// To create Vector of homogenous elements: \\
			Vector<ClassName> list = new Vector<ClassName>(capacity); 
		}
		Eg:
		\codeblock{
			Vector numbers = new Vector(20); \\
			Vector<Integer> list = new Vector<Integer>(20);  			
		}
		
		\bigskip
		\item Initialization with data:
		\begin{itemize}
			\item Using \textbf{Arrays.asList()} method:
			\bigskip
			\codeblock{
				import java.util.*; \\ 
				String[] namesArray = \{"Jim", "Jane", "Alice"\}; \\
				Vector<String> namesList = new Vector<>(Arrays.asList(namesArray)); 
			}
			
			\item Using \textbf{List.of} method:
			\bigskip
			\codeblock{
				Vector<String> colors = new Vector<>(List.of("Red", "Green", "Blue"));
			}
			
			\item Using \textbf{"var"} and \textbf{"List.of"}: 
			\bigskip
			\codeblock{
				var fruits = new Vector(List.of("Apple", "Banana", "Orange"));
			}
			
		\end{itemize}
		
	\end{itemize}
	
	\textbf{Methods specific to Vector}:
	\begin{itemize}
		\item Add member to Vector:
		\syntaxblock{
			void addElement(Object o)
		}
		\bigskip
		\item Remove element at specific index of Vector:
		\syntaxblock{
			void removeElementAt(int index)
		}
		\bigskip
		\item Remove all member of Vector:
		\syntaxblock{
			void removeAllElements()
		}
		\bigskip
		\item Retrieve element at specific index of Vector:
		\syntaxblock{
			Object elementAt(int index)
		}
		\bigskip
		\item Retrieve first element of Vector:
		\syntaxblock{
			Object firstElement()
		}
		\bigskip
		\item Retrieve last element of Vector:
		\syntaxblock{
			Object lastElement()
		}
		\bigskip
		\item Get Vector size:
		\syntaxblock{
			int size()
		}
		\bigskip
		\item Get Vector capacity:
		\syntaxblock{
			int capacity()
		}
	\end{itemize}

	\newpage
	\textbf{Example Program}
	\newline
	\textbf{Program to display vector capacity functioning:}
	
	\codeblockfull{Test.java}{
		import java.util.*; \\
		public class Test \{ \\
		\s	public static void main(String[] args) \{ \\
		\s \s		\textbf{Vector v1 = new Vector(3);} \\
		\s \s		\textbf{v1.add("Apple");} \\
		\s \s		\textbf{v1.addElement("Mango");} \\
		\s \s		\textbf{v1.add(2,"Orange");} \\
		\s \s		System.out.println(\textbf{v1.capacity()}); \\
		\s \s		System.out.println(v1.capacity()); \\
		\s \s		v1.add("Banana"); \\
		\s \s		System.out.println(v1.capacity()); \\
		\s	\} \\
		\}	
	}
	\bigskip
	\outputblock{
		\textbf{[}Apple, Mango, Orange\textbf{]} \\
		3 \\
		\textbf{[}Apple, Mango, Orange, Banana\textbf{]} \\
		6
	}
	
\end{flushleft}
\newpage


