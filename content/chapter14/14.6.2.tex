\setlength{\columnsep}{3pt}
\begin{flushleft}
	
	\begin{itemize}
		\item Iterator is applicable for any collection object, hence it is universal cursor.
		\item It can perform both the read and remove operations.
		\item Iterator object is created using \textbf{iterator()} method of collection interface
	\end{itemize}
	
	\textbf{Iterator specific methods:}
	
	\begin{itemize}
		\item Create iterator object:
		\syntaxblock{
			public Iterator iterator()
		}
		\bigskip
		\item Check for next element in collection: 
		\syntaxblock{
			public boolean hasNext()
		}
		\bigskip
		\item Get next element in collection: 
		\syntaxblock{
			public Object next()
		}
		\bigskip
		\item Remove element in collection:
		\syntaxblock{
			public void remove()
		}
	\end{itemize}
	
	\noteblock{
		\begin{itemize}
			\item Enumeration and Iterator can move only in forward direction and not backward direction.
			\item These are single direction cursors but not bi-directional cursor.
			\item Iterator can perform only read \& remove operations and not replace or addition of new objects.
			\item To overcome these limitations, go for \textbf{ListIterator}
		\end{itemize}
	}
	
	\newpage
	\textbf{Example:}
	\newline
	Java program to iterate over vector object and display even numbers only:
	\codeblockfull{Test.java}{
		import java.util.*; \\
		class Test\{ \\
		\s	public static void main(String[] args) \{ \\
		\s \s		ArrayList a1 = new ArrayList(); \\
		\s \s		a1.add(123); \\
		\s \s		a1.add(788); \\
		\s \s		a1.add(545); \\
		\s \s		a1.add(642); \\
		\s \s		Iterator t1 = a1.iterator(); \\
		\s \s		while(t1.hasNext()) \{ \\
		\s \s \s			Integer i = (Integer)t1.next(); \\
		\s \s \s			if (i\%2 != 0) \\
		\s \s \s \s			t1.remove(); \\
		\s \s \s		\} \\
		\s \s		System.out.println(a1); \\
		\s	\} \\
		\}
	}
	\bigskip
	\outputblock{
		\textbf{[}788, 642\textbf{]}
	}
\end{flushleft}

\newpage

