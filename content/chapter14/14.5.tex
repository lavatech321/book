\setlength{\columnsep}{3pt}
\begin{flushleft}
	
	\begin{itemize}
		\item Map represents a group of objects as key-value pairs.	
		\newimage{0.8}{content/chapter14/images/map.png}
		
		\item Map is \textbf{not} child interface of collections.
		\newimage{0.66}{content/chapter14/images/ex6.png}
		


		\item Both keys and values are objects.
		\item Duplicate keys are not allowed, but values can be duplicated.
		\item Each key-value pair is called \textbf{entry}, hence map is considered as a collection of entry objects.
		
	\end{itemize}

	\textbf{Map interface methods:}
	
	\begin{itemize}
		\item Add one key-value pair to the map. If the key is already present then, old value will be replaced with new value and returns old value:
		\syntaxblock{
			Object put(Object key, Object value)
		}
		\newpage
		\item  Insert the specified map in the map.
		\syntaxblock{
			void putAll(Map map)	
		}
		\bigskip
		\item Returns the object that contains the value associated with the key.
		\syntaxblock{
			Value get(Object key)	
		}
		\bigskip
		\item Delete an entry for the specified key.
		\syntaxblock{
			boolean remove(Object key, Object value)	
		}
		\bigskip
		\item Returns true if some key equal to the key exists within the map, else return false.
		\syntaxblock{
			boolean containsKey(Object key)	
		}
		\bigskip
		\item Returns true if some value equal to the value exists within the map, else return false.
		\syntaxblock{
			boolean containsValue(Object value)	
		}
		\bigskip
		\item Returns true if the map is empty; returns false if it contains at least one key.
		\syntaxblock{
			boolean isEmpty()	
		}
		\bigskip
		\item Returns the number of entries in the map.
		\syntaxblock{
			int size()	
		}
		
	\end{itemize}
	
\end{flushleft}

\newpage

